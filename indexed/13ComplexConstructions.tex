\chapter{Complex constructions}
\label{chap:13}

\section{Introduction and overview}
\label{sec:13-1}


The absolute binary distinction between coordinate and subordinate clauses is not very helpful when describing complex (i.e., multi"=\isi{clause}) constructions in Palula. Instead it makes more sense to regard inter"=clausal dependence as a~continuum, as suggested by \citet[327--328]{givon2001b}, ranging from the least dependent coordinate clauses expressing cross"=event coherence through various intermediate stages~-- including chained clauses, adverbial clauses~-- to clauses with a~high degree of dependency and event integration, as exemplified in some types of complementation.



In the following sections, this gliding scale is reflected, both in the way typical functions relate to particular constructions and in the way different subtypes are being categorised within these sections. In Section \sectref{sec:13-2}, on \isi{coordination}, the constructions described are to a~large extent symmetrical, whereas Section \sectref{sec:13-3} exemplifies a~type I call \isi{clause chaining}. The latter shows a~higher degree of asymmetry but not enough to be labelled subordinate, and therefore an~intermediate category ``cosubordinate'' seems better suited. This naturally leads over to the structurally closely related Section \sectref{sec:13-4}, which deals with clauses with adverbial functions, themselves displaying a~continuum from a~low degree of asymmetry to clearly asymmetrical constructions involving nominalised clauses. Section \sectref{sec:13-5}, on complementation, contains the same kind of internal continuum, ranging from a~comparatively low degree of event integration to \isi{clause} union. Section \sectref{sec:13-6} covers relative clauses and presents a~special analytical challenge as far as the distinction between parataxis and embedding is concerned, and here too we witness a~continuum from clauses displaying mere \isi{discourse} coherence to those involving various types of \isi{nominalisation}.


\section{Coordination}
\label{sec:13-2}

Although coordinators in the traditional sense of the term are used, there seem to be stronger constraints on their use than in the more familiar European languages. Perhaps the alternative strategies for \isi{coordination}, such as \isi{juxtaposition}, \isi{discourse} particles and suffixes, represent the more native means for conjoining phrases and sentences. However, as will be seen in the next main section (\sectref{sec:13-3}), many sequences that in \ili{English} are coded with the coordinating \isi{conjunction} \textit{and}, correspond to \isi{clause} chaning in Palula.



In the subsections \sectref{subsec:13-2-1}--\sectref{subsec:13-2-4} I follow the semantic subdivisions and their associated terms as used and discussed by \citet{payne1985}: \textit{conjunction} (p and q), \textit{postsection} (p and not q), \textit{presection} (not p and q), \textit{disjunction} (p or q) and \textit{rejection} (not p and not q).


\subsection{Conjunction}
\label{subsec:13-2-1}

Palula conjunctions make use of a number of different coordinating devices. Following Haspelmath's (\citeyear{haspelmath2007}) classification, they include asyndesis (\isi{juxtaposition}), monosyndesis (using a single marker), as well as bisyndesis (using two markers).

\spitzmarke{Clitical \textit{ee}-conjoining.} A postposed \isi{conjunction} clicit \textit{ee} is primarily used to conjoin two \isi{noun} phrases, as can be seen in examples (\ref{ex:13-1})--(\ref{ex:13-3}). It is cliticized to the right of the first constituent, following on any other word class specific inflections.

\begin{exe}
\ex
\label{ex:13-1}
\gll hatés díi ba dúu putr-á yúul-a [fazelnuúr=ee hayaatnuúr] \\
\textsc{3sg.rem.acc} from \textsc{prt} two son-\textsc{pl} come.\textsc{pfv"=mpl} Fazal.Noor=\textsc{cnj} Hayat.Noor\\
\glt `He had two sons, Fazal Noor and Hayat Noor.' (B:ATI076) 

\ex
\label{ex:13-2}
\gll le bi [c̣hook-íi=ee mac̣hook-íi] aulaád \\
\textsc{3pl.dist.nom} \textsc{prt} Choke-\textsc{gn=cnj} Machoke-\textsc{gn} descendants\\
\glt `They are also the descendants of Choke and Machoke.' (A:ASC004)

\ex
\label{ex:13-3}
\gll [máa=ee so] mušqúl bh-íia de  \\
\textsc{1sg}=\textsc{cnj} \textsc{3msg.nom} in.agreement become-\textsc{1pl} \textsc{pst} \\
\glt `He and I used to be talking freely with each other.' (A:HUA008) 
\end{exe}

Infrequently the use of this \isi{conjunction} is extended to the conjoining of two constituents of other kinds, such as two verb phrases that are intimately linked and expressing something symmetrical or simultaneous, as in (\ref{ex:13-4}) and (\ref{ex:13-5}) (cf. \sectref{subsec:13-4-1} for \isi{simultaneity} expressed with a~\isi{Verbal Noun} followed by what is possibly the same marker \textit{ee}). This should, however, not be seen as a~matter of sentential \isi{coordination} in a~strict sense, as the verb phrases have a~\isi{subject} in common.

\begin{exe}
\ex
\label{ex:13-4}
\gll ɡíri dhiaáṛ [raál bh-éen-i=ee nam-éen-i]  \\
big.stone rock high become-\textsc{prs-f=cnj} get.down-\textsc{prs-f}  \\
\glt `The big stone or rock is moving up and down.' (A:DRA007)

\ex
\label{ex:13-5}
\gll táa ɡíia ta ɡíri ǰhulí xaamaár [dhreéɡ de hín-u=ee níindra-m maǰí hín-u aaraám ki hín-u] \\
there go.\textsc{pfv.pl} \textsc{prt} big.stone on dragon stretched.out  give.\textsc{cv} be.\textsc{prs"=msg=cnj} sleep-\textsc{ob} in be.\textsc{prs"=msg}  resting as be.\textsc{prs"=msg}  \\
\glt `When they came there, a~dragon was lying on top of the big stone, sleeping peacefully.' (A:DRA009-10) 
\end{exe}

\spitzmarke{Juxtaposition.} Conjunction of more than two \isi{noun} phrases is often expressed by \isi{juxtaposition}, in a~list"=like manner, (\ref{ex:13-6})--(\ref{ex:13-7}), but can occasionally be used with two \isi{noun} phrases only, as in example (\ref{ex:13-8}).

\begin{exe}
\ex
\label{ex:13-6}
\gll ɡhireé c̣hook-íi=ee mac̣hook-íi aulaád [mirmaadikoór, mulaakoór, baḍileé, zariṇeé, phaṭakeé] \\
again Choke-\textsc{gn=cnj} Machoke-\textsc{gn} descendants  Mirmadikor Mullahkor Badiley Zariney Phatakey \\
\glt `After that, the descendants of Choke and Machoke are Mirmadikor, Mullahkor, Badiley, Zariney, and Phatakey.' (A:ASC005)

\ex
\label{ex:13-7}
\gll ak zamaanée [ak ɡúu, miḍ, uṭ] doost-aán de \\
\textsc{idef} time.\textsc{gn} \textsc{ idef} ox ram camel friend-\textsc{pl}  be.\textsc{pst}  \\
\glt `Once upon a~time there where three friends: an~ox, a~ram and a~camel.' (B:SHI001)

\ex
\label{ex:13-8}
\gll díiš-a ba [baalbač-á kuṛíina] táma th-éen de \\
village-\textsc{ob} \textsc{prt} child-\textsc{pl} woman.\textsc{pl} waiting do-\textsc{3pl} \textsc{pst}  \\
\glt `In the village the children and women were waiting.' (B:AVA218) 
\end{exe}

Juxtaposition is also applied to closely interrelated verb phrases or clauses, for instance, when a~particular verbal action or event is identical for several subjects, as in (\ref{ex:13-9}), or when the verb phrases, as in (\ref{ex:13-10}), have an~identical \isi{subject} as well as an~identical recipient or beneficiary. A sequence of events with the same \isi{subject}, such as the one in (\ref{ex:13-11}), would in most cases be coded with \isi{clause chaining} (\sectref{sec:13-3}), but for some reason, maybe reflecting how upset the speaker is, \isi{juxtaposition} is used instead.

\begin{exe}
\ex
\label{ex:13-9}
\gll aní ɡhooṣṭ-á šíiṭi [ma seé hín-u, míi kuṛi seé hín-i, míi preṣ seé hín-i] \\
\textsc{prox} house-\textsc{ob} inside \textsc{1sg.nom} sleep.\textsc{cv} be.\textsc{prs"=msg}  \textsc{1sg.gn} woman sleep.\textsc{cv} be.\textsc{prs-f } \textsc{1sg.gn} mother.in.law sleep.\textsc{cv} be.\textsc{prs-f} \\
\glt `Inside this house I was asleep, my wife was asleep, and my mother"=in"=law was asleep.' (A:HUA014-5)

\ex
\label{ex:13-10}
\gll [siɡréṭ ɡaḍ-í dít-i činí phuukawéel-i]  \\
cigarette take.out-\textsc{cv} give.\textsc{pfv-f} sugar make.blow.\textsc{pfv-f} \\
\glt `We offered them cigarettes and sugar.' (A:GHA060)

\ex
\label{ex:13-11}
\gll karáaṛu yhóol-u kučúr-a khóol-a  \\
leopard come.\textsc{pfv"=msg} dog-\textsc{pl} eat.\textsc{pfv"=mpl} \\
\glt `A leopard has come and eaten the dogs.' (A:HUA020) 
\end{exe}

\spitzmarke{The \isi{conjunction} \textit{o/oór}.} A special \isi{conjunction}, \textit{o} alternatively \textit{oór}, is used sparingly and sometimes in addition to other conjuction strategies. It is not wholly unlikely that they owe something to \ili{Pashto} and \ili{Urdu} influence, respectively, being close in form and function to the most important conjunctions in each of these languages. The latter form is, in particular, found primarily in data obtained from younger persons (in A.) and in written \isi{discourse} (in B.). 


As a~\isi{conjunction} between \isi{noun} phrases, it is mainly found where more than two \isi{noun} phrases are conjoined, as in (\ref{ex:13-12}) and (\ref{ex:13-13}), and then the position is always and only between the last two phrases.

\begin{exe}
\ex
\label{ex:13-12}
\gll se míiš-a dúi dees-á [baačaá wazíir o dúi xálak-a] samaṭ-í ilaán thíil-i \\
\textsc{def} man-\textsc{ob} other day-\textsc{ob} king minister and other  people-\textsc{pl} gather-\textsc{cv} announcement do.\textsc{pfv-f}  \\
\glt `Next day the man called the king, the prime minister and other people together and made an~announcement.' (A:UXW060)

\ex
\label{ex:13-13}
\gll mheeríl-ii pahúrta [tasíi híṛu tasíi ǰhanɡaár oór tasíi aandáara] ɡaḍ-í khóol-a  \\
die.\textsc{pptc"=gn} after \textsc{3sg.gn} heart \textsc{3sg.gn}  liver and \textsc{3sg.gn} intestines take.out-\textsc{cv} eat.\textsc{pfv"=mpl}\\
\glt `After he had killed her, he took out her heart, her liver, and her intestines and ate them up.' (A:WOM651) 
\end{exe}

As we saw above, this kind of stringing together of several \isi{noun} phrases can also be done by mere \isi{juxtaposition}. In fact, when a~dialect adaptation was made (from B. to A.) of a~story that (\ref{ex:13-7}) is taken from, the same sentence was equipped with the \isi{conjunction} \textit{o} between the two last \isi{noun} phrases. 


A \isi{conjunction} word is also used sometimes to conjoin closely interrelated verb phrases or same"=\isi{subject} clauses, as the ones exemplified in (\ref{ex:13-14}) and (\ref{ex:13-15}). Often, the second proposition or \isi{phrase} provides additional information about the action or event mentioned in the first. The two are sometimes but not necessarily co"=temporal.

\begin{exe}
\ex
\label{ex:13-14}
\gll oothóon-a the nhiáaṛa phed-í ba biṣamíl-u hín-u o dun-áaṭ-u bhíl-u hín-u ki...\\
settlement-\textsc{ob} to near arrive-\textsc{cv} \textsc{prt} rest.\textsc{pfv"=msg} be.\textsc{prs"=msg} and think-\textsc{ag"=msg} become.\textsc{pfv"=msg} be.\textsc{prs"=msg } \textsc{comp} \\
\glt `When he came near the settlement, he rested and started to think...' (A:KAT056-7)

\ex
\label{ex:13-15}
\gll uth-í ba ɣusulxaaná the ba-yáan-u oór ɣusúl ɡhin-áan-u \\
stand.up-\textsc{cv} \textsc{prt} bathroom to go.\textsc{prs"=msg}  and bath take-\textsc{prs"=msg} \\
\glt `When I have got up I go to the bathroom, and I take a~bath.' (B:MOR002-3) 
\end{exe}

Rather infrequently a~\isi{conjunction} is also used to link two different"=\isi{subject} clauses, such as the ones in (\ref{ex:13-16}).

\begin{exe}
\ex
\label{ex:13-16}
\gll nhiáaṛee ba áaṇc̣-ii áak muṭ heensíl-u hín-u o áaṇc̣-a ba pač-í šiiy-í bi-áaṭ-a bhíl-a hín-a \\
near \textsc{prt} raspberry-\textsc{gn} \textsc{idef} tree dwell.\textsc{pfv"=msg}  be.\textsc{prs"=msg} and raspberry-\textsc{pl} \textsc{prt} ripen-\textsc{cv} fall-\textsc{cv} go-\textsc{ag"=mpl} become.\textsc{pfv"=mpl} be.\textsc{prs"=mpl} \\
\glt `Nearby a~raspberry bush grew and its berries had become ripe.' (A:KAT128) 
\end{exe}

However, as far as temporal sequencing is concerned, the overwhelmingly most common ways of conjoining clauses are by using cosubordinate \isi{Converb} clauses and subordinate \textit{ta}-clauses in so"=called ``\isi{clause} chains'' (as described in \sectref{sec:13-3}).


There are also examples of \isi{predicate} \isi{adjective} phrases being thus conjoined. An example can be seen in (\ref{ex:13-17}).

\begin{exe}
\ex
\label{ex:13-17}
\gll so bíiḍ-u [uxiaár o hileéṛ] de maní  \\
\textsc{3sg.nom} very-\textsc{msg} wise and brave be.\textsc{pst} \textsc{hsay} \\
\glt `He was very wise and brave.' (A:UXB004) 
\end{exe}

\spitzmarke{Conjunction with the ``separating'' particle \textit{bi}.} An added emphasis on the separateness of two or more units conjoined is given by means of a~particle \textit{bi} (approximately `too, also'), as in examples (\ref{ex:13-18})--(\ref{ex:13-20}). It is primarily added immediately after the constituents in focus, which in most cases are \isi{noun} phrases, whereas the \isi{coordination} itself is sentential rather than phrasal (which is not obvious with copular clauses lacking an~overt \isi{copula}, as in (\ref{ex:13-18})). The particle used this way is added after each of the conjoined constituents, rendering it `X as well as Y' or `both X and Y'. The use of bisyndesis for emphatic \isi{coordination} is known from a number of other languages \citep[15--17]{haspelmath2007}. 

\begin{exe}
\ex
\label{ex:13-18}
\gll ma [tarkaáṇ bi misrí bi]  \\
\textsc{1sg.nom} carpenter also mason also \\
\glt `I'm a~carpenter as well as a~mason.' (A:HOW009)

\ex
\label{ex:13-19}
\gll ma ba uth-í ba [toobaák bi heensíl-i de, khanɡaár bi heensíl-u de, ṭhónɡi bi heensíl-i de] \\
\textsc{1sg.nom} \textsc{prt} stand.up-\textsc{cv} \textsc{prt} gun also  be.present.\textsc{pfv-f} \textsc{pst} sword also be.present.\textsc{pfv"=msg } \textsc{pst} axe also be.present.\textsc{pfv-f } \textsc{pst} \\
\glt `When I got up, there was a~gun, a~sword and an~axe (with me).' (A:HUA021-2)

\ex
\label{ex:13-20}
\gll pha-íi báabu ǰhaamatreé díi [xarčá bi dawa-áan-u ǰandeé bi khéli dawa-áan-u]  \\
girl-\textsc{gn} father son.in.law.\textsc{ob} from expenses also  ask.for-\textsc{prs"=msg} goat.\textsc{pl} also numerous ask.for-\textsc{prs"=msg} \\
\glt `The girl father demands expenses (to be paid) as well as numerous goats.' (A:MAR032-3) 
\end{exe}

The particle \textit{bi} is sometimes also used in combination with coodinating \textit{ee} (\ref{ex:13-21}) or the \isi{conjunction} \textit{o} (\ref{ex:13-22}).

\begin{exe}
\ex
\label{ex:13-21}
\gll aṛé  [šišáki  bi man-áan=ee  ruikúṛi bi man-áan-a] \\
\textsc{3fsg.dist.nom} ``shishaki'' also say-\textsc{prs"=mpl=cnj}  ``ruikuli'' also say-\textsc{prs"=mpl}\\
\glt `We call her a ``shishaki'' and also a ``ruikuli''.' (A:PAS110)

\ex
\label{ex:13-22}
\gll dhrúuk-a be dac̣h-íin ta [manɡái čhoor-í hín-i o áak rumiaál bi čhoor-í hín-i]  \\
stream-\textsc{ob} go.\textsc{cv} look-\textsc{3pl} \textsc{prt} water.pot put-\textsc{cv}  be.\textsc{prs-f} and \textsc{idef } handkerchief also put-\textsc{cv} be.\textsc{prs-f } \\
\glt `Going to the stream they saw the water pot and a~handkerchief lying there.' (A:SHY059) 
\end{exe}

In \ili{Kohistani} \ili{Shina} (Ruth Laila Schmidt, pc; \citealt[252]{schmidtkohistani2008}) a~particle \textit{-ɡa} has a~function comparable to Palula \textit{bi}, but its scope is apparently much larger than that of the Palula particle, as it conjoins various types of phrases and clauses (much like Palula \textit{-ee}), not necessarily with an~additional ``separating'' semantics. 



\spitzmarke{Conjunction with the ``adversative'' construction \textit{ta{\ldots}ba}.} An adversative meaning with a~semantic contrast implied, translatable with `but, whereas, while' (but sometimes `and' is better still), is indicated with the use of a~particle \textit{ta} added immediately after the first constituent, and another particle \textit{ba} (with other \isi{discourse} functions) after the second. As is apparent from examples (\ref{ex:13-23})--(\ref{ex:13-25}), the \isi{coordination} itself is, like the use of \textit{bi,} sentential rather than phrasal.

\begin{exe}
\ex
\label{ex:13-23}
\gll eetás [míiš-a ta ṭhak-íin de kuṛíina ba čooṭ-íin de] \\
\textsc{3sg.rem.acc} man-\textsc{pl} \textsc{prt} shake.down-\textsc{3pl} \textsc{pst} woman.\textsc{pl} \textsc{prt} pluck-\textsc{3pl} \textsc{pst} \\
\glt `The men were shaking them [the walnuts] down, while the women were collecting them.' (A:JAN017)

\ex
\label{ex:13-24}
\gll aaɡhaá ta dhúura dharán ba ḍanɡ  \\
sky \textsc{prt} distant ground \textsc{prt} hard \\
\glt `The sky is far but/and the ground is hard.' (B:PRB009)

\ex
\label{ex:13-25}
\gll so ta bač bhíl-u trúu ǰáan-a ba hiimeel-í híṛ-a \\
\textsc{3msg.nom} \textsc{prt} saved become.\textsc{pfv"=msg} three person-\textsc{pl} \textsc{prt } glacier-\textsc{ob} take.away.\textsc{pfv"=mpl}  \\
\glt `He was saved, but three persons were taken by the avalanche.' (B:AVA214) 
\end{exe}

The construction \textit{ta{\ldots}ba} is sometimes, as in (\ref{ex:13-26}), used along with the coodinating marker \textit{ee}.

\begin{exe}
\ex
\label{ex:13-26}
\gll huṇḍ ta c̣híitr=ee bhun ba ɡhaawaáz de \\
above \textsc{prt} field=\textsc{cnj} below \textsc{prt} stream.bed be.\textsc{pst} \\
\glt `The field was above and the stream"=bed below.' (A:JAN082) 
\end{exe}


\spitzmarke{Conjunction with the ``adversative'' particle \textit{xu}.} Another type of adversative, with a~denial of expectation implied, is formed with a~particle \textit{xu} (also functioning as a~marker of evidentiality).

\begin{exe}
\ex
\label{ex:13-27}
\gll ɡhaḍeerá mána thíil-u xu so aahóola tool"=íi eé de \\
elder.\textsc{ob} prevention do.\textsc{pfv"=msg} but \textsc{3msg.nom} repeatedly measure-\textsc{3sg} \textsc{ampl}{\protect\footnotemark} \textsc{pst} \\
\glt `The elder one told him to stop, but he [the younger one] kept measuring.' (A:DRA019)
\end{exe}

\footnotetext{In my material the amplifier is a rarely occurring post"=verbal marker. It appears, at least in this sentence, to have a~continuative meaning. It is probably identical to the (\textit{eé}) of exclusiveness or emphasis as it occurs postposed to pronouns or numerals.}

\subsection{Presection and postsection}
\label{subsec:13-2-2}

The ``adversative'' particle \textit{xu} (see above) is also used together with the \isi{negation} \textit{na} in typical sentential pre- and \isi{postsection} (as defined in \sectref{sec:13-2}) constructions.


\spitzmarke{Presection with \textit{na{\ldots}xu}.} In this construction, exemplified in (\ref{ex:13-28}), the particle \textit{xu} occurs between the two conjoined parts, and the \isi{negation} word \textit{na} in its regular preverbal position in the first part.

\begin{exe}
\ex
\label{ex:13-28}
\gll se bi čúur páanǰ bhe ma na ṭinɡ th-áai bhóon de xu míi se preṣ phed-í ba c̣héeṇi ɡhin-í ǰ-íi seentá bhíi-um de \\
\textsc{3pl.nom} even four five become.\textsc{cv} \textsc{1sg.nom} \textsc{neg}  challenge do-\textsc{inf} be.able.\textsc{3pl} \textsc{pst} but \textsc{1sg.gn}  \textsc{def} mother.in.law arrive-\textsc{cv} \textsc{prt} stick take-\textsc{cv} hit-\textsc{3sg} \textsc{condh} be.afraid-\textsc{1sg} \textsc{pst} \\
\glt `Even those four five people could not face me, but my mother"=in"=law came and beat me with a~stick and I became afraid.' (A:HUA123-4) 
\end{exe}

\spitzmarke{Postsection with \textit{xu{\ldots}na}.} In this construction, exemplified in (\ref{ex:13-29}), the particle \textit{xu} occurs between the two conjoined clauses or propositions, and the \isi{negation} word in its regular preverbal position in the second proposition.

\begin{exe}
\ex
\label{ex:13-29}
\gll sóon-a ba so phoó koošíš th-áan-u xu waxt milaáu na bh-áan-u \\
pasture-\textsc{ob} \textsc{prt} \textsc{def.msg.nom} boy attempt   do-\textsc{prs"=msg} but time meeting \textsc{neg} become-\textsc{prs"=msg} \\
\glt `In the pastures, the boy is trying, but he doesn't find the time.' (A:SHY042) 
\end{exe}

\spitzmarke{Postsection with final \textit{na}.} Postsection can also be expressed with a~sentence"=final negative particle \textit{na} , as in (\ref{ex:13-30}) and (\ref{ex:13-31}), without the aid of \textit{xu}. 

\begin{exe}
\ex
\label{ex:13-30}
\gll ǰ-úum ma ṛas mhaar-úum, uṛ-úum ba na \\
hit-\textsc{1sg} \textsc{1sg.nom} \textsc{3sg.dist.acc} kill-\textsc{1sg} let.out-\textsc{1sg}
\textsc{prt} \textsc{neg} \\
\glt `I will kill it and not let it free.' (A:HUA058)

\ex
\label{ex:13-31}
\gll nam-í ta yhéel-i heentá ṣiṣ ḍhak-áan-u raál ba yhéel-i heentá phed-áan-a na \\
lower-\textsc{cv} \textsc{prt} come.\textsc{pfv-f} \textsc{condl} head touch-\textsc{prs"=msg}  high \textsc{prt} come.\textsc{pfv-f} \textsc{condl} reach-\textsc{prs"=msg} \textsc{neg} \\
\glt `If it is put too low people will hit their head, but if it is put too high nobody will reach up.' (A:HOW069-70)
\end{exe}

\subsection{Disjunction}
\label{subsec:13-2-3}

\spitzmarke{Juxtaposition.} The least emphasised kind of \isi{disjunction} is by mere \isi{juxtaposition}, normally of two numerals (as in the \isi{juxtaposition} of `four' and `five' in (\ref{ex:13-28}) above) or of two \isi{noun} phrases with \isi{numeral} attributes, as in (\ref{ex:13-32}), expressing an~ap\isi{proximate} figure rather than constituting an~alternative in an~absolute sense.

\begin{exe}
\ex
\label{ex:13-32}
\gll díiš-e xálak ǰamá bhe kir kareé ɡalé bíiḍ-u dít-u síinta [panǰ phuṭ-í ṣo phuṭ-í] kir dít-u síinta \\
village-\textsc{gn} people collection become.\textsc{cv} snow  when ever much-\textsc{msg} give.\textsc{pfv"=msg} \textsc{condh} five  foot-\textsc{pl} six foot-\textsc{pl} snow give.\textsc{pfv"=msg} \textsc{condh}  \\
\glt `The village people used to gather when much snow had been falling, 5 or 6 feet snow.' (B:AVA198) 
\end{exe}

\spitzmarke{Phrasal \isi{disjunction} with \textit{yaá}.} A slightly more emphasised \isi{disjunction} uses \textit{yaá} between the phrases, as seen in (\ref{ex:13-33}).

\begin{exe}
\ex
\label{ex:13-33}
\gll eesé phaíi [bhróo yaá máamu] koó eesé ṭeem-íi haazír na heensíl-u heentá, tasíi mux-íi nikh"=eeṇḍeéu bh-áan-u \\
\textsc{rem} girl.\textsc{gn} brother or uncle who \textsc{rem} time"=gn  present \textsc{neg} stay.\textsc{pfv"=msg} \textsc{condl} \textsc{3sg.gn} face-\textsc{gn}  come.out-\textsc{oblg} become-\textsc{prs"=msg}  \\
\glt `If the girl's brother or her uncle wasn't present at that time, he has to be greeted.' (A:MAR047-9) 
\end{exe}

\spitzmarke{Sentential \isi{disjunction} with \textit{yaába}{\ldots}\textit{yaába}.} Although there are few examples, it seems a~construction with a~repeated \textit{yaá} combined with the particle \textit{ba} (probably forming one phonological word), as in (\ref{ex:13-34}), is preferred for sentential \isi{disjunction}, possibly with an~added emphasis on the mutual exclusivity of the alternatives thus presented. 

\begin{exe}
\ex
\label{ex:13-34}
\gll típa šay phed-í asaám yaába khóo yaába na khóo \\
now thing arrive-\textsc{cv} \textsc{1pl.acc} or eat.\textsc{3sg} or \textsc{neg} eat.\textsc{3sg} \\
\glt `Now this thing will come here and it will either eat us or it will not eat us.' (A:GHA063) 
\end{exe}

\spitzmarke{Sentential and phrasal \isi{disjunction} \textit{ki}.} A semantically equivalent construction presenting mutual exclusivity is also found, as in (\ref{ex:13-35})--(\ref{ex:13-36}), with a~particle \textit{ki}. 

\begin{exe}
\ex
\label{ex:13-35}
\gll hasó tohfá [phedíl-u ki na phedíl-u] \\
\textsc{rem.msg.nom} gift arrive.\textsc{pfv"=msg} or \textsc{neg}  arrive.\textsc{pfv"=msg} \\
\glt `Did this gift arrive or not?' (B:FLW817)

\ex
\label{ex:13-36}
\gll buǰ-áan-u ba na ki nu ɡa šay karáaṛu ki iṇc̣ ki nu ɡa čiíz thaní \\
understand-\textsc{prs"=msg} \textsc{prt} \textsc{neg} \textsc{comp} \textsc{3msg.prox.nom}  some thing leopard or bear or \textsc{3msg.prox.nom} some thing \textsc{qt} \\
\glt `I didn't understand what sort of thing this was, whether it was a~leopard or a~bear or what sort of creature.' (A:HUA038-9) 
\end{exe}

\ili{Kohistani} \ili{Shina} uses a~word borrowed from \ili{Punjabi}, \textit{paa$\sim$wee$\sim$} `or', to express \isi{disjunction} (Ruth Laila Schmidt, pc).


\subsection{Rejection}
\label{subsec:13-2-4}

\spitzmarke{Rejection with \textit{na{\ldots}na}.} Rejection can be expressed simply with a~phrase- or \isi{clause}"=initial \isi{negation} \textit{na}, as exemplified in (\ref{ex:13-37}) and (\ref{ex:13-38}).

\begin{exe}
\ex
\label{ex:13-37}
\gll na zinaawur-á tas the asár thíil-i de na ɡhrast-á thíil-i de \\
\textsc{neg} beast-\textsc{ob} \textsc{3sg.acc} to harm do.\textsc{pfv-f} \textsc{pst}  \textsc{neg} wolf-\textsc{ob} do.\textsc{pfv-f} \textsc{pst} \\
\glt `Neither beast nor wolf had harmed him.' (A:GHA014)

\ex
\label{ex:13-38}
\gll se na kasíi xaadí dac̣h-éen-i na kasíi marɡ dac̣héeni \\
\textsc{3fsg.nom} \textsc{neg} anyone.\textsc{gn} happiness look-\textsc{prs-f}  \textsc{neg} anyone.\textsc{gn} death look-\textsc{prs-f}  \\
\glt `She cares neither for anyone's happiness, nor for anyone's death.' (A:KEE005-6) 
\end{exe}

Two other strategies available for expressing \isi{rejection} are both combinations of \isi{negation} and conjunctions already discussed above.


\spitzmarke{Rejection using ``adversative'' \isi{conjunction} and \isi{negation}, \textit{na ta{\ldots}na ba}}. It should be noted that the word order (the particle vis-à-vis the contrasted constituent) is the reverse compared with the one found in the conjunctive construction with \textit{ta{\ldots}ba}. It could also be classified as an ``emphatic negative \isi{coordination}'' \citep[17--19]{haspelmath2007}. Examples are given in (\ref{ex:13-39}) and (\ref{ex:13-40}).

\begin{exe}
\ex
\label{ex:13-39}
\gll na ta tanaám the dít-i na ba asaám the dít-i \\
\textsc{neg} \textsc{prt} \textsc{3pl.acc} to give.\textsc{pfv-f} \textsc{neg} \textsc{prt} \textsc{1pl.acc} to give.\textsc{pfv-f}  \\
\glt `Neither did they give them to them, nor to us.' (A:GHA089)

\ex
\label{ex:13-40}
\gll dúi ba na ta duaá thíil-u na ba ɡa háanǰ-a dít-a \\
other \textsc{prt} \textsc{neg} \textsc{prt} prayer do.\textsc{pfv"=ms.g} \textsc{neg} \textsc{prt} any curse-\textsc{pl} give.\textsc{pfv"=mpl}\\
\glt `Then, neither did he pray, nor did he utter any curses.' (A:PIR038-9) 
\end{exe}

As can be seen in (\ref{ex:13-41}), the strategy also occurs in combination with the conjunctive marker \textit{ee}.

\begin{exe}
\ex
\label{ex:13-41}
\gll méeǰi na ta ṣoo-íi tarapíi ɡa faaidá=ee na ba barawolxaan-íi tarapíi ɡa faaidá \\
between \textsc{neg} \textsc{prt } king-\textsc{gn} direction.\textsc{gn} any benefit=\textsc{cnj} \textsc{neg} \textsc{prt} Barawul.Khan-\textsc{gn} direction.\textsc{gn} any benefit  \\
\glt `Between them, neither was there any benefit from the king's side, nor from Barawul Khan.' (A:JAN007-8) 
\end{exe}

Rejection using ``separative'' \isi{conjunction} and \isi{negation}, \textit{bi na{\ldots}bi na}, is illustrated in (\ref{ex:13-42})--(\ref{ex:13-43}).

\begin{exe}
\ex
\label{ex:13-42}
\gll dac̣h-íin ta kaṭamúš bi náin-u iṇc̣ bi náin-u \\
look-\textsc{3pl} \textsc{prt} Katamosh also not.be.\textsc{prs"=msg} bear also not.be.\textsc{prs"=msg} \\
\glt `They saw that neither Katamosh nor the bear was there.' (A:KAT144)

\ex
\label{ex:13-43}
\gll bíiḍ-i raál bi na th"=eeṇḍeéu naam"=eeṇḍeéu bi na \\
much-\textsc{f} high also not do-\textsc{oblg} lower-\textsc{oblg} also not  \\
\glt `It should not be raised very high, neither should it be lowered [much].' (A:HOW068) 
\end{exe}

\ili{Kohistani} \ili{Shina} expresses \isi{rejection} with \textit{nee$\sim${\ldots}nee$\sim$} `no{\ldots}no', while the regular \isi{clause} \isi{negation} in this variety is \textit{na} (Ruth Laila Schmidt, pc).


\section{Clause chaining}
\label{sec:13-3}

One of the more salient features of Palula \isi{discourse} (especially of a~narrative kind) is its use of \isi{clause chaining}, thus fitting into the category ``chaining'' languages, as described by \citet[242]{thompsonetal2007} and \citet[374--376]{longacre2007}. Such \isi{clause} chains express sequences of events or actions carried out and consist of one or several non"=final clauses (note the distinction made between (non-)\textit{final} and (non)\textit{finite}), followed by one final \isi{clause}. Each non"=final \isi{clause} is relating to the one immediately preceding it and the one following it, but not necessarily to the final \isi{clause}. I have adopted the term \textit{cosubordinate} (see explanation below) to describe this type of \isi{clause} linking which is somewhat intermediate between \isi{coordination} and subordination.


\subsection{Same"=\isi{subject} chaining}
\label{subsec:13-3-1}

In examples (\ref{ex:13-44})--(\ref{ex:13-46}), the construction used in the non"=final clauses is a~\isi{nonfinite} converb (cv), a~verb form without any participant agreement, and the \isi{subject} left implicit. A literal translation of (\ref{ex:13-44}) would render something like `The father, having killed a~goat, having removed its skin, put the skin on him'. 

\begin{exe}
\ex
\label{ex:13-44}
\gll \label{bkm:Ref190770484}[báaba ba yakdám bhíiru mheer-í] [tasíi púustu ɡaḍ-í] so púustu tas ṣaawóol-u  \\
father.\textsc{ob} \textsc{prt} immediately he.goat kill-\textsc{cv} \textsc{3sg.gn} skin  remove-\textsc{cv} \textsc{def.msg.nom} skin \textsc{3sg.acc} put.on.\textsc{pfv"=msg} \\
\glt `The father killed a~he"=goat, skinned it, and put the skin on his son.' (A:DRA023)

\ex
\label{ex:13-45}
\gll tíi ba [bhun whay-í ba] [so  mhaás muṭ-í bhun wheel-í ba] [teeṇíi
  ɡhooṣṭ-á the ɡhin-í] ɡáu \\
\textsc{3sg.ob} \textsc{prt} down come.down-\textsc{cv} \textsc{prt} \textsc{def.msg.nom}  meat tree-\textsc{gn} down take.down-\textsc{cv} \textsc{prt} \textsc{refl} house-\textsc{ob}  to take-\textsc{cv} go.\textsc{pfv.msg} \\
\glt `He came down, took down the meat from the tree, and brought it to his house.' (B:SHB762)

\ex
\label{ex:13-46}
\gll ɡhaḍeerá [phed-í] laṣ=čax [kaṭéeri ɡhin-í] se taáǰ čhiníl-i  \\
elder.\textsc{ob} arrive-\textsc{cv} swiftly knife take-\textsc{cv} \textsc{def} crown cut.\textsc{pfv-f} \\
\glt `The elder (brother) came, took a~knife, and cut off the crown.' (A:DRA016) 
\end{exe}

As pointed out by \citet[327-8]{givon2001b}, although traditional grammar has made a~binary distinction between dependent/subordinate adverbial clauses, on the one hand, and independent/coordinate main clauses, on the other, it is evident that dependency as well as subordination, is a~matter of degree rather than being descrete properties of clauses (cf.  the discussion in \citet[]{cristofaro2005}). \citet[20--27]{haspelmath1995}, following the same reasoning, suggests a \textit{cosubordinate} category intermediate between \isi{coordination} and subordination, which would thus capture the status of the non"=final verbs in above"=mentioned chains, in some traditions also described as \textit{medial} verbs. Whereas the connection between a~cosubordinate \isi{clause}, and another cosubordinate \isi{clause}, or between a~cosubordinate \isi{clause} and its corresponding main/final \isi{clause}, often will have to find its translation equivalent in a~coordinate \isi{clause} (for instance, in \ili{English}), the non"=final \isi{clause} is still to a~large extent dependent on the independent \isi{clause} for its TMA specification or participant reference.



However, an~argument in favour of still viewing \isi{Converb} clauses as subordinate would be that they sometimes seem to appear inside (i.e., embedded into) the final \isi{clause}. Example (\ref{ex:13-44}) is ambiguous in this respect, since we can analyse the ergatively marked \textit{báaba} `father', either as the explicit \isi{subject} of the first non"=final \isi{transitive} \isi{clause} (as is suggested by the square brackets) or as the explicit \isi{subject} of the final \isi{transitive} \isi{clause}. 



In example (\ref{ex:13-46}) on the other hand, the final \isi{clause} is \isi{transitive} and the first converb is in\isi{transitive}, and therefore the ergatively marked \isi{subject} must be analysed as the \isi{subject} of the final (and \isi{finite}) verb, and therefore the first \isi{Converb} \isi{clause} in its entirety has to be analysed as embedded into the final \isi{clause} (as is also the second \isi{Converb} \isi{clause}, which has an~instrumental reading, see \sectref{subsec:13-4-1}). It is at this point that \isi{clause chaining} shades over into the realm of more asymmetrical adverbial functions. 



In (\ref{ex:13-45}), however, sentence"=initial \textit{ti} `he' can not be the \isi{subject} (S) of the in\isi{transitive} final verb `go'. Since it is marked for \isi{ergative alignment}, it must be the \isi{subject} (A) either of \isi{transitive} \textit{whaalé-} `take down' or \isi{transitive} \textit{ɡhin-} `take', both of these non"=final converbs in this sentence. Simple embedding is therefore not really the case here, and again cosubordination seems a~suitable categorisation. 



Some overlap between medial verbs~-- as used in chained or so"=called cosubordinate clauses~-- and converbs coding more typical adverbial functions, is indeed to be expected \citep[26]{haspelmath1995}, and we find that the very same mechanism is put to work in many clauses corresponding to distinct adverbial"=\isi{clause} types (see \sectref{subsec:13-4-1}--\sectref{subsec:13-4-5}) in the familiar European languages, with a~number of semantic relationships implied between the dependent \isi{clause} and the main \isi{clause}. 


\subsection{Different"=\isi{subject} chaining}
\label{subsec:13-3-2}


While the chaining exemplified above expresses sequencing of events or actions with one and the same \isi{subject}, another construction type is being used when there is a~\isi{subject} switch after a~non"=final \isi{clause}: a~non"=final \isi{clause} as the ones found in (\ref{ex:13-47}) and (\ref{ex:13-48}), in which a~\isi{finite} verb is followed by the particle \textit{ta}, is in turn followed by another (non"=final or final) \isi{clause}.

\begin{exe}
\ex
\label{ex:13-47}
\gll eesé míiš-a ba toopaančá ɡaḍ-í [se dhuumíiy"=ii ǰít-i de ta] se dhuumíi uḍheewíl-i de \\
\textsc{rem} man-\textsc{ob} \textsc{prt} pistol take.out-\textsc{cv} \textsc{def}  smoke-\textsc{gn} shoot.\textsc{pfv-f} \textsc{pst} \textsc{prt} \textsc{def} smoke flee.\textsc{pfv-f} \textsc{pst} \\
\glt `That man took out a~pistol, shot at the smoke, and the smoke disappeared.' (A:GHA052)

\ex
\label{ex:13-48}
\gll [áa deés táa ɡúum ta] máa=the qisá thíil-u \\
\textsc{idef} day there go.\textsc{pfv"=msg } \textsc{prt} \textsc{1sg.nom}=to story do.\textsc{pfv"=msg}  \\
\glt `One day I went there, and he told me a~story.' (A:HUA009) 
\end{exe}

Sequences of two or more non"=final clauses can include same"=\isi{subject} (SS) clauses as well as different"=\isi{subject} (DS) clauses, as can be seen in (\ref{ex:13-49}) and (\ref{ex:13-50}).

\begin{exe}
\ex
\label{ex:13-49}
\gll thée aaxeríi [áa bac̣húuṛu mheer-í]\textsc{\textsubscript{ss}} [púustu ṣaawóol-u ta]\textsc{\textsubscript{ds}} bac̣húuṛ-ii púust-a ba tas ɡhašíl-u, ṭinɡ thíil-u \\
then finally \textsc{idef} calf kill-\textsc{cv}  skin  put.on.\textsc{pfv"=msg} \textsc{prt} calf-\textsc{gn} skin-\textsc{ob} \textsc{prt} \textsc{3sg.acc}
catch.\textsc{pfv"=msg} \textsc{host} do.\textsc{pfv"=msg} \\
\glt `Then at last he killed a~calf, dressed (his son) in the skin, and the skin of the calf helped him.' (A:DRA031)

\ex
\label{ex:13-50}
\gll [thíi ak dhií paidáa bhíl-i ta]\textsc{\textsubscript{ds}} [thíi kúṛi tes ɡhin-í]\textsc{\textsubscript{ss}} [adráx ṣač-í]\textsc{\textsubscript{ss}} ɡéi \\
\textsc{2sg.gn} \textsc{idef} daughter born become.\textsc{pfv-f} \textsc{prt} \textsc{2sg.gn}  wife \textsc{3sg.acc} take-\textsc{cv} forest climb-\textsc{cv} go.\textsc{pfv.fsg} \\
\glt `A daughter of yours was born, your wife took her and went up into the forest.' (B:FOR014) 
\end{exe}

Note that, while the converbs in the SS constructions do not show agreement, the verbs in the DS constructions do. In example (\ref{ex:13-49}), the \isi{perfective} verb \textit{ṣaawóolu} agrees in \isi{gender} and number with the \isi{direct object} \textit{púustu}). 


In a~great many cases of chaining with \textit{ta,} the non"=final \isi{clause} is most naturally translatable as a `when'-\isi{clause} in \ili{English}, and a~number of other semantic relationships can be implied as well, the latter something that will become obvious when discussing different constructions expressing adverbial functions (\sectref{sec:13-4}). As was the case with the SS"=chaining using the \isi{Converb} construction, DS"=chaining by means of the \textit{ta}-construction also shades over into even more asymmetrical and more typical adverbial functions. 


\section{Clauses with adverbial functions}
\label{sec:13-4}

The type of clauses using the \isi{Converb} construction (SS) and the \textit{ta}-construction (DS) would certainly qualify as ``absolutive'' adverbial clauses according to \citeauthor{thompsonetal2007}'s (\citeyear[264--266]{thompsonetal2007}) definition, due to their wide scope and their lack of explicit signals as to the exact semantic relationship between non"=final and final clauses. Their function in the \isi{discourse} is often inferred from the context. As was pointed out above, it is not always very obvious where to draw the line between such semi"=independent chained clauses and clauses that to a~larger extent are interpretable as adverbial and dependent vis-à-vis one particular main \isi{clause}. 



Like non"=final chained clauses, clauses with adverbial functions generally precede their head. They usually precede the entire head \isi{clause} although embedding into the head \isi{clause} or head proposition also occurs quite frequently. Only rarely does the modifying \isi{clause} occur subsequent to the verb of the head \isi{clause}. All such clauses with adverbial functions are in essence dependent but, as was reflected earlier in the discussion on \isi{clause chaining} above, not equally and unquestionably subordinate in nature. 



A number of different construction types are available for clauses functioning adverbially, some of them with a~more general application, and others more restricted to certain subfunctions:


\begin{enumerate}
\item[(i)] A preposed or embedded \isi{nonfinite} \isi{Converb} \isi{clause}, in IA languages often referred to as the ``\isi{conjunctive participle} construction'' (also used in SS"=chains).

\item[(ii)] A preposed \isi{clause} with a~\isi{finite} verb followed by the grammatical particle (or general subordinator) \textit{ta} (also used in DS"=chains); infrequently together with a~\isi{clause}"=introducing subordinating \isi{conjunction} with a~more specific lexical content

\item[(iii)] A preposed \isi{clause} with a~\isi{finite} verb followed by a~conditional \isi{mood marker}

\item[(iv)] A preposed \isi{Perfective Participle} \isi{clause} governed by a~lexically specific \isi{postposition}

\item[(v)] A preposed or embedded \isi{clause} with a~\isi{Verbal Noun}, either occurring on its own (sometimes with a~suffix) or with a~lexically specific \isi{postposition}

\item[(vi)] A pre- or postposed \isi{clause} with an~Agentive \isi{Verbal Noun} and a~\isi{finite} form of \textit{bhe-} `become'

\item[(vii)] An embedded \isi{nonfinite} \isi{clause} with a~\isi{Copredicative Participle}
\end{enumerate}

These constructions will be discussed as they occur in the different adverbial functions presented below. 


\subsection{Temporality and related functions}
\label{subsec:13-4-1}

\spitzmarke{Subsequence.} Many dependent clauses with an~obvious temporal reading are in fact not much different from the sequential chaining discussed above (\sectref{sec:13-3}). Especially when it is a~matter of \isi{subsequence}, and the \isi{subject} remains the same, the \isi{Converb} construction (i) is used, as in (\ref{ex:13-51})--(\ref{ex:13-52}). Sometimes, especially in narratives, an~immediately preceding \isi{finite} verb is repeated but as a~converb, thus not adding any new information but instead highlighting the temporal relationship between the two events. This latter use of the converb is more typically adverbial~-- and the \isi{Converb} \isi{clause} to a~greater extent dependent~-- than its use in sequential chaining as laid out above.

\begin{exe}
\ex
\label{ex:13-51}
\gll (eesé zanɡal-í áa baṭ-á ǰhulí harí so kuṇaák bheešóol-u.) [bheešaá] tasíi paaṇṭí ɡaḍ-íi de \\
\textsc{rem} forest-\textsc{ob} \textsc{idef} stone-\textsc{ob} on take.away.\textsc{cv}   \textsc{def.msg.nom} child seat.\textsc{pfv"=msg} seat.\textsc{cv} \textsc{3sg.gn} clothes take.off-\textsc{3sg} \textsc{pst} \\
\glt `(In that forest he took the child to a~stone and seated him.) \textit{When he had seated him,} he took his (the child's) clothes off.' (A:BRE005)

\ex
\label{ex:13-52}
\gll (tíi taayúu hiǰrát thíil-i.) [hiǰrát the] dhruú$\sim$ṣ-a yúul-u \\
\textsc{3sg.ob} from.there migration do.\textsc{pfv-f}    migration do.\textsc{cv} Drosh-\textsc{ob} come.\textsc{pfv"=msg} \\
\glt `(He migrated from there.) \textit{When he had migrated}, he came to Drosh.' (B:ATI003-4) 
\end{exe}

As was the case in \isi{clause chaining}, the corresponding construction when the \isi{subject} in the dependent
\isi{clause} is different from the \isi{subject} in the main \isi{clause} is a~\isi{finite} \textit{ta}-\isi{clause}
(construction~(ii)), as can be seen in example (\ref{ex:13-53}) and (\ref{ex:13-54}). Again, the verb in the preceding \isi{finite} \isi{clause} is often found repeated.

\begin{exe}
\ex
\label{ex:13-53}
\gll (aḍaphará wháil-u.) [aḍaphará wháil-u ta] tasíi uǰut-í maǰí xaaráx paidóo bhíl-i \\
halfways take.down.\textsc{pfv"=msg} halfways  take.down.\textsc{pfv"=msg } \textsc{prt} \textsc{3sg.gn} body-\textsc{ob} in itching born become.\textsc{pfv-f} \\
\glt `(He brought him down halfways.) \textit{When he had brought him down halfways}, his body started to itch.' (A:DRA020)

\ex
\label{ex:13-54}
\gll [raaǰaá múṛ-u ta] putr-óom tasíi hukumát bulooṣṭéel-i \\
king die.\textsc{pfv"=msg} \textsc{prt} son-\textsc{pl.ob} \textsc{3sg.gn}  government snatch.\textsc{pfv-f} \\
\glt `When the king died, the sons seized the power.' (A:MAB003)
\end{exe}

It appears to be possible to leave out the particle \textit{ta}, as in (\ref{ex:13-55}), and retain a~different"=\isi{subject} reading, although that is not particularly common.

\begin{exe}
\ex
\label{ex:13-55}
\gll [naɡarǰúti yhéel-i] tas díi khooǰóol-u kanáa bhíl-i \\
Nagarjuti come.\textsc{pfv-f} \textsc{3sg.acc} from ask.\textsc{pfv"=msg}  what become.\textsc{pfv-f} \\
\glt `When Nagarjuti came, he asked her what happened.' (MAB029) 
\end{exe}

While the subsequential reading is contextually implicit rather than explicit in the examples given so far, a~separate temporal \isi{conjunction} \textit{kareeɡalé} or \textit{kareé ɡalé ki} `when' can also (possibly more readily so in B.), as seen in (\ref{ex:13-56}), be added to the \textit{ta}-construction to make this reading obvious, and also distinguish it more clearly from simple \isi{clause chaining}.

\begin{exe}
\ex
\label{ex:13-56}
\gll [kareeɡalé se múṛ-im ta] asím tenaám ḍhanɡéel"=im \\
when \textsc{3pl.nom} die.\textsc{pfv"=fpl} \textsc{prt} \textsc{1pl.erg} \textsc{3pl.acc} bury.\textsc{pfv"=fpl} \\
\glt `When they died, we buried them.' (B:FOR037) 
\end{exe}

This construction has characteristics of a~\isi{relative clause} (cf. \sectref{subsec:13-6-2}), like a~number of similar temporal adverbial constructions in the world's languages \citep[246--247]{thompsonetal2007}, which is even more obvious in (\ref{ex:13-57}).

\begin{exe}
\ex
\label{ex:13-57}
\gll [hasó kareé ɡalé ki sastíil-u ta] ṣúu itlaá phedíl-i \\
\textsc{3msg.nom} when ever \textsc{comp} heal.\textsc{pfv"=msg} \textsc{comp} king message arrive.\textsc{pfv-f} \\
\glt `When he had become healthy, a~message arrived from the king.' (B:ATI059) 
\end{exe}

A \isi{finite} temporal \isi{clause} with \textit{kareé} and without \textit{ta} (\ref{ex:13-58}) is also possible.

\begin{exe}
\ex
\label{ex:13-58}
\gll [kareé ǰéel-i] c̣hiír taap-áan-a \\
when give.birth.\textsc{pfv-f} milk heat.up-\textsc{prs"=mpl} \\
\glt `When it [goat] has given birth, we heat up the milk.' (A:KEE025) 
\end{exe}

\isi{Perfective Participle} clauses with a~\isi{postposition}, such as \textit{pahúrta, baád}, \textit{patú,} all glossed as `after', are also used in temporal \isi{subsequence} (construction~(iv)), thereby giving them an explicit ``temporal anteriority'' reading \citep[159]{cristofaro2005}. \textit{pahúrta} (in (\ref{ex:13-59})) and \textit{baád} (in (\ref{ex:13-60})) each governs the \isi{genitive} case of the participle, whereas \textit{patú} (in (\ref{ex:13-61})) takes the \isi{nominative}. The latter seems to occur only in B. 

\begin{exe}
\ex
\label{ex:13-59}
\gll [iẓḍúur-a ǰe ɡúum"=ii pahúrta] tas yhéel-i audás \\
Ijdur-\textsc{ob} up.to go.\textsc{pptc.msg"=gn} after \textsc{3sg.acc} come.\textsc{pfv-f} urge \\
\glt `When he had gone up to Ijdur, he felt a~need to urinate.' (A:AYA)

\ex
\label{ex:13-60}
\gll [ṣaawóol"=ii baád] dúi bhíiru mheeríl-u \\
put.on.\textsc{pptc.msg"=gn} after another he.goat kill.\textsc{pfv"=msg}  \\
\glt `When he had put it on, he killed another he"=goat.' (A:DRA024)

\ex
\label{ex:13-61}
\gll [ɡhíin-u patú] ǰumeet-í the ba-yáan-u  \\
take.\textsc{pptc.msg} after mosque-\textsc{ob} to go-\textsc{prs"=msg} \\
\glt `\textit{When I have taken it (a bath),} I go to the mosque.' (B:MOR004) 
\end{exe}

\spitzmarke{Subsequence cum instrument.} The temporal \isi{subsequence} reading of the \isi{Converb} construction shades off into what can be seen as expressing instrument or supplying the main verb with a~manner specification and is especially relevant for converbs formed from \isi{transitive} verbs of motion, as in examples (\ref{ex:13-62})--(\ref{ex:13-65}), such as \textit{ɡhín-} `take', \textit{de-} `put, give', \textit{ɡaḍé-} `take out', \textit{ɡhašé-} `take hold of, grab', \textit{ɡalé-} `throw'. With some of these converbs we may very well witness a~further grammaticalisation into postpositions, alternatively, the development of compound verbs (see \sectref{subsec:8-6-2}). 

\begin{exe}
\ex
\label{ex:13-62}
\gll ɡhaḍeerá phed-í laṣ=čax [kaṭéeri ɡhin-í] se taáǰ čhiníl-i. čhin-í [rumeel-í maǰí de] ǰeep-í=ee dít-i \\
elder.\textsc{ob} arrive-\textsc{cv} swiftly knife take-\textsc{cv} \textsc{def} crown cut.\textsc{pfv-f} cut-\textsc{cv} handkerchief-\textsc{ob} in put.\textsc{cv} pocket-\textsc{ob}=into put.\textsc{pfv-f} \\
\glt `The elder (brother) came, took a~knife, and cut off the crown. Having cut it off, he wrapped it in a~handkerchief and then put it into his pocket' (A:DRA016)

\ex
\label{ex:13-63}
\gll [maačís ɡaḍ-í] tasíi laméeṭi anɡóor ṣaawóol-u ta so uḍheewíl-u \\
matches take.out-\textsc{cv} \textsc{3sg.gn} tail fire put.on.\textsc{pfv"=msg}  \textsc{prt } \textsc{3sg.nom} flee.\textsc{pfv"=msg} \\
\glt `He took matches, and when he had put fire to its tail, it fled.' (A:HUB009)

\ex
\label{ex:13-64}
\gll yhayí [tasíi háat"=ii ɡhaš-í] ḍúkur-a šíiṭi the [ɡhin-í] ɡíia hín-a  \\
come.\textsc{cv} \textsc{3sg.gn} hand-\textsc{gn} grab-\textsc{cv} hut-\textsc{ob} inside to  take-\textsc{cv} go.\textsc{pfv.pl} be.\textsc{prs"=mpl} \\
\glt `Having come, the took him by the hand and brought him into the hut.' (A:KAT062)

\ex
\label{ex:13-65}
\gll [so darwóoza kaná wée ɡal-í] dharéndi nikháat-u \\
\textsc{def.msg.nom} door shoulder.\textsc{ob} on throw-\textsc{cv} outside appear.\textsc{pfv"=msg} \\
\glt `He came out, with the door on his shoulder.' (A:GHU028) 
\end{exe}

\spitzmarke{Accompanying circumstance.} The \isi{Converb} construction can also express an~\isi{accompanying circumstance}, or a~means specification with reference to the same \isi{subject} and the same event. In (\ref{ex:13-66})--(\ref{ex:13-68}) a~subsequential interpretation is even less an~issue.

\begin{exe}
\ex
\label{ex:13-66}
\gll máa=the [míi nóo de] maníit-u ki asíi ráaǰ-am"=ii zimawaár tu  \\
\textsc{1sg.nom}=to \textsc{1sg.gn} name give.\textsc{cv} say.\textsc{pfv"=msg} \textsc{comp}  \textsc{1pl.gn} rope-\textsc{pl.ob"=gn} responsible \textsc{2sg.nom}  \\
\glt `Calling my name, they told me: You are responsible for the ropes.' (A:ACR008)

\ex
\label{ex:13-67}
\gll paturaá nuuṭíl-u ta khayí zhay-í wée asím ǰinaazá [khaṣeel-í] wheelíl-u de... \\
back return.\textsc{pfv"=msg} \textsc{prt} which place-\textsc{ob} in \textsc{1pl.erg}  corpse drag-\textsc{cv} take.down.\textsc{pfv"=msg} \textsc{pst} \\
\glt `When I turned back to the place to where we had dragged the corpse{\ldots}.' (A:GHA043)

\ex
\label{ex:13-68}
\gll [tas wheel-í] atshareet-á phedóol-u \\
\textsc{3sg.acc} bring.down-\textsc{cv} Ashret-\textsc{ob} bring.\textsc{pfv"=msg} \\
\glt `We brought him down to Ashret' (A:GHA081) 
\end{exe}

However, an~\isi{accompanying circumstance}, or further specification of the action carried out by the \isi{subject}, is sometimes, as in (\ref{ex:13-69}) and (\ref{ex:13-70}), expressed with the more restricted \isi{Copredicative Participle} (construction~(vii)).

\begin{exe}
\ex
\label{ex:13-69}
\gll se oóra ṣaá kaṭamúš paš-í ba [utrap-íim] yhóol-a hín-a \\
\textsc{3pl.nom} over.here side Katamosh see-\textsc{cv} \textsc{prt} run-\textsc{cprd} come.\textsc{pfv"=mpl} be.\textsc{prs"=mpl} \\
\glt `When they saw Katamosh over here they came running.' (A:KAT061)

\ex
\label{ex:13-70}
\gll phoó [patuɡiraá dac̣h-íim dac̣h-íim] áa dand-á patú [haát čula-íim] ac̣híi-am díi fanaá bhíl-u hín-u \\
boy back look-\textsc{cprd} look-\textsc{cprd} \textsc{idef}  ridge-\textsc{ob} behind hand wave-\textsc{cprd} eye-\textsc{pl.ob} from  annihilation become.\textsc{pfv"=msg} be.\textsc{prs"=msg} \\
\glt `The boy kept looking back and disappeared behind a~ridge still waving his hand.' (A:SHY035) 
\end{exe}

\spitzmarke{Simultaneity.} Simultaneity involves two overlapping events and includes what
\citet[330]{givon2001b} refers to as ``point coincidence''. The backgrounded event
\citep[254--255]{thompsonetal2007} is usually expressed with a~\isi{Verbal Noun} (construction~(v)) in
Palula. If the \isi{subject} is the same in the foregrounded as well as in the backgrounded event, the
\isi{Verbal Noun} occurs alone, as in (\ref{ex:13-71}), or with a~\isi{postposition} \textit{maǰí} `in,
among' (\ref{ex:13-72}) or \textit{sanɡí} `with', whereas if there is a~\isi{subject} shift between
the foregrounded and the backgrounded event, the \isi{Verbal Noun} (or the \isi{postposition}) occurs, as in
(\ref{ex:13-73}) and (\ref{ex:13-74}), with a~suffix \textsc{-eé} (probably
\isi{accent}"=bearing and possibly identical to the suffix coding inclusivity, used with numerals, see \sectref{subsec:6-4-1}), which functions as a~switch"=reference marker similar to the contrast between
\textit{ta}-constructions vis-à-vis \isi{Converb} constructions described above.

\begin{exe}
\ex
\label{ex:13-71}
\gll bac̣húuṛu be ba, [patuɡiraá yh"=ainií ba], páand na léed-i, se bac̣húuṛ-a \\
calf go.\textsc{cv} \textsc{prt } back come-\textsc{vn }  \textsc{prt} path \textsc{neg} find.\textsc{pfv-f} \textsc{def} calf-\textsc{ob}  \\
\glt `The calf went (in), but \textit{as it was coming back}, it didn't find its way, the calf.' (A:CAV015-6)

\ex
\label{ex:13-72}
\gll so [kráam th"=ainií maǰí] rhoó d-áan-u  \\
\textsc{3sg.nom} work do-\textsc{vn} in song do-\textsc{prs"=msg}  \\
\glt `He sings \textit{while working}.' (A:Q6.34.03)

\ex
\label{ex:13-73}
\gll yhayí [se ṭék-a d-aini-eé] huṇḍ-íi utrapíl-i hín-i se míiš kh"=ainií the \\
come.\textsc{cv} \textsc{def} peak-\textsc{ob} give-\textsc{vn"=incl} above-\textsc{gn} run.\textsc{pfv-f}  be.\textsc{prs-f } \textsc{def} man eat-\textsc{vn} to  \\
\glt `\textit{As he reached the peak}, she came running from above to eat the man.' (A:PAS121-2)

\ex
\label{ex:13-74}
\gll [aǰdahaá katoolíi-a wée ač-aníi sanɡi-eé] lhooméea se míiš-a ḍáḍi išaará thíil-u, thanaáu dhrak-é \\
dragon fodder.sack-\textsc{ob} in enter-\textsc{vn} with-\textsc{incl}  fox.\textsc{ob} \textsc{def} man-\textsc{ob} toward hint do.\textsc{pfv"=msg} string pull-\textsc{imp.sg}  \\
\glt `Just as the dragon went into the sack, the fox signaled to the man to pull the string.' (B:DRB036)
\end{exe}

A somewhat parallel strategy is the use of an~agentive verbal \isi{noun} and \textit{bhe-} `become', (\ref{ex:13-75})--(\ref{ex:13-76}), lit. `became an~onlooker, became goers', etc. (construction~(vi)). With these constructions, it is not always obvious which event is backgrounded and which one is foregrounded. 

\begin{exe}
\ex
\label{ex:13-75}
\gll deerá šíiṭi siɡréṭ dhrak-í ba [tasíi maxadúši wée tasíi putr tas the dac̣h-áaṭ-u bhíl-u hín-u] \\
room inside cigarette pull-\textsc{cv} \textsc{prt} \textsc{3sg.gn} front.of in \textsc{3sg.gn}  son \textsc{3sg.acc} to look-\textsc{ag"=msg} become.\textsc{pfv"=msg} be.\textsc{prs"=msg}  \\
\glt `As he sat smoking in the room, his son was looking at him.' (A:SMO001-2)

\ex
\label{ex:13-76}
\gll áak deés [se zanɡal-í pharé báaiṭ-a bhíl-a hín-a] naaɡhaaní áak amzarái nikháat-u hín-u \\
\textsc{idef} day \textsc{3pl.nom} forest-\textsc{ob} through go.\textsc{ag"=mpl}  become.\textsc{pfv"=mpl} be.\textsc{prs"=mpl} suddenly \textsc{idef} lion appear.\textsc{pfv"=msg} be.\textsc{prs"=msg}  \\
\glt `One day, as they were walking through a~forest, suddenly a~lion appeared.' (A:UNF005-6) 
\end{exe}

Less commonly, \isi{simultaneity} or temporal overlap is expressed with a~\isi{perfective} verb form with a~postposed \textit{ee}-\isi{conjunction}, as in (\ref{ex:13-77}). Whether the \isi{perfective} verb form is \isi{finite} or \isi{nonfinite} in this case, I do not have a~conclusive answer to at the moment.

\begin{exe}
\ex
\label{ex:13-77}
\gll [so ta ɡúum=ee] ma ba patuɡiraá yhóol-u \\
\textsc{3sg.nom} \textsc{prt} go.\textsc{pfv=cnj} \textsc{1sg.nom} \textsc{prt} back  come.\textsc{pfv"=msg} \\
\glt `As he left, I went back (home).' (A:HUA043) 
\end{exe}

\spitzmarke{Precedence.} In the temporal clauses mentioned so far, there is either an~implied \isi{simultaneity} or a~sequence of events where the subordinate (or cosubordinate) `when'-\isi{clause} refers to an~event taking place before the event named in the main \isi{clause}. In so"=called \isi{precedence} clauses \citep[327]{givon2001b}, on the other hand, the event in the `before'-\isi{clause} has not yet been realised in relation to the event mentioned in the main \isi{clause} \citep[247--248]{thompsonetal2007}. This is also referred to as ``temporal posteriority'' \citep[159]{cristofaro2005}. For such clauses a~\isi{Verbal Noun} with the complex \isi{postposition} \textit{díi muṣṭú} `before' (in B. \textit{díi muxáak}) is used, as can be seen in (\ref{ex:13-78})--(\ref{ex:13-80}).

\begin{exe}
\ex
\label{ex:13-78}
\gll ṭhaaṭáak-a ba [bheš-ainií díi muṣṭú] tas díi nóo khooǰóol-u maní \\
monster-\textsc{ob} \textsc{prt} sit.down-\textsc{vn} from before \textsc{3sg.acc} from name ask.\textsc{pfv"=msg} \textsc{hsay} \\
\glt `Before sitting down, we have been told that the monster asked for his name.' (A:UXB012)

\ex
\label{ex:13-79}
\gll [khinǰ-ainií díi muṣṭú] thíi su"=eeṇḍeéeu  \\
become.tired-\textsc{vn} from before \textsc{2sg.gn} fall.asleep-\textsc{oblg} \\
\glt `You must go to bed before you get tired.' (A:TAQ131)

\ex
\label{ex:13-80}
\gll [har kram theníi díi muxáak] bíiḍ-u čitaiṇḍeéu  \\
every work do.\textsc{vn} from before much-\textsc{msg} think.\textsc{oblg} \\
\glt `It is necessary to think properly before a~work is carried out.' (B:FOX035)
\end{exe}

\subsection{Purpose}
\label{subsec:13-4-2}

Dependent clauses expressing the \isi{purpose} for an~event to be carried out usually have an~implicit human \isi{subject} coreferential with the (normally) explicit \isi{subject} of the main \isi{clause} \citep[337]{givon2001b}. The construction used, a~\isi{Verbal Noun} followed by the \isi{postposition} `to', parallels such ``dative'' \isi{purpose} clauses found in many other languages \citep[251--252]{thompsonetal2007}. The \isi{finite} verb in sentences, such as (\ref{ex:13-81})--(\ref{ex:13-82}), is almost invariably an~in\isi{transitive} motion verb `come', `go', etc.

\begin{exe}
\ex
\label{ex:13-81}
\gll nu ba [asaám mhaar"=ainií the] ukháat-u de thaní \\
\textsc{3msg.prox.nom} \textsc{prt} \textsc{1pl.acc} kill-\textsc{vn} to come.down.\textsc{pfv"=msg} \textsc{pst} \textsc{qt}  \\
\glt `He has come down to kill us.' (A:HUA071)

\ex
\label{ex:13-82}
\gll phoo-íi ɡhooṣṭ-íi tarapíi tasíi axpul-aán kuṛíina míiš-a [teeṇíi se bhoói paš-ainií the] bi-áan-a \\
boy-\textsc{gn} house-\textsc{gn} direction \textsc{3sg.gn} relative-\textsc{pl} woman.\textsc{pl}  man-\textsc{pl} \textsc{refl} \textsc{def} daughter"=in"=law see-\textsc{vn} to go-\textsc{prs"=mpl} \\
\glt `The relatives, men and women from the boy's house, are going (there), to see their daughter"=in law.' (A:MAR104) 
\end{exe}

Not infrequently the verb itself is implicit in the \isi{purpose} \isi{clause}, and instead the \isi{direct object} argument, `bowls' in (\ref{ex:13-83}), receives the ``dative'' marking.

\begin{exe}
\ex
\label{ex:13-83}
\gll theé ba se iṇc̣ kaṭamuš-á the óol thawaá [looṛíi-am the] ɡíia hín-a \\
do.\textsc{cv} \textsc{prt} \textsc{3pl.nom} bear Katamosh-\textsc{ob} to watch  make.do.\textsc{cv} bowl-\textsc{pl.ob} to go.\textsc{pfv.mpl} be.\textsc{prs"=mpl} \\
\glt `Then they made the bear watch Katamosh and went to get bowls [lit. for bowls].' (A:KAT127) 
\end{exe}

Purpose is also quite often explicitly expressed by the \isi{postposition} \textit{dapáara} `for, for the sake of' (a \ili{Pashto} loan) following the \isi{Verbal Noun}, as in (\ref{ex:13-84}) or (in B.), the \isi{postposition} \textit{pándee} `for, for the sake of' following the \isi{Verbal Noun} in the \isi{genitive}, as in (\ref{ex:13-85}).

\begin{exe}
\ex
\label{ex:13-84}
\gll [bhraawéeli th"=ainií dapáara] tanaám maǰí áak qabúl th-áan-u ki ma bhíiru ɡhin-í thíi ɡhooṣṭ-á the yh-úum \\
brotherhood do-\textsc{vn} for \textsc{3pl.acc} among one consent do-\textsc{prs"=msg} \textsc{comp} \textsc{1sg.nom} he.goat take-\textsc{cv} \textsc{2sg.gn}  house-\textsc{ob} to come-\textsc{1sg} \\
\glt `In order to form a~brotherhood, one of them agrees to bring a~he"=goat to the other one's house.' (A:MIT013)

\ex
\label{ex:13-85}
\gll ɡhooṣṭ-á baás de ba patuɡiraák [tenaám dac̣h-aníi-e pándee] ɡhireé ɡáu \\
house-\textsc{ob}  shelter give.\textsc{cv} \textsc{prt} back \textsc{3pl.acc}  look-\textsc{vn"=gn} for again go.\textsc{pfv.msg} \\
\glt `After spending the night at home, he went back again to look at them.' (B:BEL335)
\end{exe}

\subsection{Causality}
\label{subsec:13-4-3}

As noted by \citet[247]{thompsonetal2007}, it is no surprise that we find causal relationships expressed in the same way as temporal relationships, or rather that the constructions themselves are neutral between a~time and a~cause interpretation. Often in Palula, dependent clauses, such as the different"=\isi{subject} \textit{ta}-construction (see \sectref{subsec:13-4-1}), can thus imply an~immediately preceding event in time as well as a~cause for the event or action in the main \isi{clause} to happen or be carried out. 



Example (\ref{ex:13-50}), quoted above, may, for instance, be given a~sequential interpretation as well as a~causal one: `When a~daughter of yours was born, your wife took her and went up into the forest' (sequential), or: `Because a~daughter of yours was born, your wife took her and went up into the forest' (causal).



As further pointed out by \citet[335]{givon2001b}, cause (along with reason) can itself be divided into a~number of subdistinctions, such as agentive external, non"=agentive external, eventive external, non"=eventive external, eventive internal and non"=eventive internal.



For external \isi{causality}, the preferred strategy is, as can be seen in (\ref{ex:13-86}), the already mentioned \isi{finite} different"=\isi{subject} construction with \textit{ta}.

\begin{exe}
\ex
\label{ex:13-86}
\gll [ǰaanɡul-á ma bhanǰóol-u ta] ru-áan-u  \\
Jangul-\textsc{ob} \textsc{1sg.nom} beat.\textsc{pfv"=msg} \textsc{prt} cry-\textsc{prs"=msg}  \\
\glt `Jangul beat me, therefore I'm weeping.' (A:HUA104)
\end{exe}

A construction involving a~\isi{Verbal Noun},\footnote{To what degree any of the case suffixes in the examples from Biori correspond to the (possibly conjunctive) clitic \textit{ee} in the Ashret example remains at this point an~unanswered question.} also used to imply \isi{simultaneity} (see above), can also be used to signal \isi{causality}, as in (\ref{ex:13-87})--(\ref{ex:13-89}).

\begin{exe}
\ex
\label{ex:13-87}
\gll [panaahí dawainíi-a] ɣeyratxaan-á ba tes teeṇíi ɡhúuṛu tés=te de ba tes har-í har-í raulée phará the ba nawaab-í ḍáḍi lanɡúul-u \\
shelter ask.\textsc{vn"=ob} Ghairat.Khan-\textsc{ob} \textsc{prt} \textsc{3sg.acc} \textsc{refl}  horse \textsc{3sg.acc}=to give-\textsc{cv} \textsc{prt} \textsc{3sg.acc} take.away-\textsc{cv} take.away-\textsc{cv} Lowari.\textsc{ob} through do.\textsc{cv} \textsc{prt} prince-\textsc{gn} side take.across.\textsc{pfv"=msg} \\
\glt `Asking him for shelter, Ghairat Khan gave him his own horse and took him across Lowari to the territory of the prince (of Dir).' (B:ATI026-7)

\ex
\label{ex:13-88}
\gll ǰeníi-e iṇc̣ zaxmát bhíl-u \\
hit.\textsc{vn"=gn} bear wouded become.\textsc{pfv"=msg} \\
\glt `The bear was wounded from the firing.' (B:BEL320)

\ex
\label{ex:13-89}
\gll [so ma pharé ḍhak"=ainíi=ee] piaalá ma díi lháast-u \\
\textsc{3sg.nom} \textsc{1sg.nom} toward touch-\textsc{vn=cnj} cup \textsc{1sg.nom} from drop.\textsc{pfv"=msg}  \\
\glt `He bumped me, so I dropped the cup.' (A:CHE071107) 
\end{exe}

For most cases of internal \isi{causality}, where the same"=\isi{subject} condition holds, the \isi{Converb} construction is used, as in (\ref{ex:13-90}) and (\ref{ex:13-91}).

\begin{exe}
\ex
\label{ex:13-90}
\gll [ma boór bhe] táai ɡúum  \\
\textsc{1sg.nom} bored become.\textsc{cv} from.there go.\textsc{pfv.msg}  \\
\glt `I was bored, so I left.' (A:CHE071107)

\ex
\label{ex:13-91}
\gll [se xálaka qalánɡ na d-áa bhaá ba] uḍheew-í ɡíia \\
 \textsc{def} people tax \textsc{neg} give-\textsc{inf} be.able.to.\textsc{cv} \textsc{prt}  flee-\textsc{cv}  go.\textsc{pfv.mpl}   \\
\glt `The people were not able to pay the taxes, so they left, fleeing.' (A:MAB030) 
\end{exe}

Causality (perhaps internal in particular) can also be expressed in a~more explicit way, for instance, as in (\ref{ex:13-92}), through a~relativisation of \textit{wáǰa} `reason', a~\isi{noun} borrowed from \ili{Pashto}, or a~dependent \isi{clause} with a~\isi{Verbal Noun} and a~\isi{postposition}, the latter exemplified in (\ref{ex:13-93}).

\begin{exe}
\ex
\label{ex:13-92}
\gll [so wáǰa ba eeṛó ki so lhookeer-ó brhóo so kaná pharé ɡal-í
  wheél-u de ḍóo the] eetasíi asár tas the phedíl-i \\
\textsc{def.msg.nom} reason \textsc{prt} \textsc{3sg.dist.nom} \textsc{comp} \textsc{def.msg.nom} younger-\textsc{msg} brother \textsc{def}{\protect\footnotemark} shoulder.\textsc{ob} along throw-\textsc{cv} bring.down.\textsc{pfv"=msg} \textsc{pst} \textsc{host} do.\textsc{cv}  \textsc{3sg.gn} effect \textsc{3sg.rem.acc} to arrive.\textsc{pfv-f} \\
\glt `Because [lit. for the reason that] he had carried his younger brother on his shoulder, he was affected.' (A:DRA028-9)

\ex
\label{ex:13-93}
\gll čhéeli [na čit"=aníi ǰhulí] ɡhrast-í iṣkáar bhíl-i \\
goat \textsc{neg} think-\textsc{vn} on wolf-\textsc{gn} prey become.\textsc{pfv-f}  \\
\glt `Since the goat didn't really think, she fell prey to the wolf.' (B:FOX)
\end{exe}

\footnotetext{I do not have any explanation for this form to appear. If it indeed is a~\isi{determiner} of the \isi{oblique} case"=marked \textit{kaná}, the form \textit{se} is expected.}

\subsection{Conditionality}
\label{subsec:13-4-4}

As is the case in many languages \citep[257--258]{thompsonetal2007}, there is in Palula no absolute distinction made between temporality and \isi{conditionality}. Although there are distinct morphological markings available to signal \isi{conditionality}, there is, as we shall see, some overlap in coding between clauses with an~implied temporality (and to some extent \isi{causality}) vis-à-vis clauses with conditional semantics, an observation lining up with more general, cross"=linguistic, tendencies \citep[161]{cristofaro2005}.



It is also, as noted by \citet[255--260]{thompsonetal2007}, as well as by \citet[333--334]{givon2001b}, possible for languages to make explicit finer gradations along a~scale of verisimilitude, an~observation relevant for the analysis of Palula conditionals. We can discern at least three degrees of verisimilitude: a) generally true, b) possibly or probably true, and c) not true. These will be discussed and exemplified under the following headings: assumed \isi{conditionality}, hypothetical \isi{conditionality}, and \isi{counterfactuality}.



\spitzmarke{Assumed \isi{conditionality}.} Conditionals under this heading partly match what \citet[255--256]{thompsonetal2007} refer to as reality conditionals. Here we find the conditionally marked clauses that are closest related to temporality and \isi{causality} as described above. Often the conditionally marked \isi{clause} is equally translatable with `when{\ldots}' and `if{\ldots}' (and especially with past"=time reference, `whenever'). The typical construction is a~\isi{finite} (often \isi{perfective}) `if'-\isi{clause} with the \isi{conditionality} marker \textit{seentá (B. síinta)}, followed by an~im\isi{perfective} `then'-\isi{clause}. They describe a~condition that usually holds, especially in a~more generic sense, or used to do so in the past. Examples are given in (\ref{ex:13-94})--(\ref{ex:13-97}).

\begin{exe}
\ex
\label{ex:13-94}
\gll [misrí yhóol-u seentá] misrí díi tsaṭák hóons-a \\
mason come.\textsc{pfv"=msg} \textsc{condh} mason from hammer stay-\textsc{3sg}  \\
\glt `When the mason comes, he [usually] has a~hammer with him.' (A:HOW010)

\ex
\label{ex:13-95}
\gll [bíiḍ-u táru bi dít-u seentá] xaraáp  bh-éen-i \\
much-\textsc{msg} quickly also give.\textsc{pfv"=msg} \textsc{condh} bad  become-\textsc{prs-f}   \\
\glt `But also if it [salt] is given too often, it will harm her [the goat].' (A:KEE019)

\ex
\label{ex:13-96}
\gll [paačhambeé uthíit-u seentá] so bi  uth-íi de maní \\
Pashambi stand.up.\textsc{pfv"=msg} \textsc{condh} \textsc{3sg.nom} also stand.up-\textsc{3sg} \textsc{pst} \textsc{hsay}  \\
\glt `If/Whenever Pashambi stood up, he [the monster] would also stand up.' (A:UXB016)

\ex
\label{ex:13-97}
\gll [ac̣húuṛi-m ṭhak"=ainií waxt bhíl-u seentá]  yh-íin de \\
walnut.tree-\textsc{pl} shake.down-\textsc{vn} time become.\textsc{pfv"=msg} \textsc{condh} come-\textsc{3pl} \textsc{ pst}    \\
\glt `Whenever it was the time of walnut shedding, they used to come.' (A:JAN015) 
\end{exe}

Sometimes the \isi{clause}"=final conditional marker also co"=occurs, as in (\ref{ex:13-98}), with a~temporal \isi{conjunction} \textit{kareé} `when'. 

\begin{exe}
\ex
\label{ex:13-98}
\gll [míiš-ii putr kareé zuaán bhíl-u seentá]  tas the kúṛi dawainií
bandubás  th-áan-u \\
man-\textsc{gn} son when young.adult become.\textsc{pfv"=msg} \textsc{condh} \textsc{3sg.acc} to woman ask.\textsc{vn} arrangement do-\textsc{prs"=msg}  \\
\glt `When/If someone's son becomes of age, he then arranges for his marriage.' (A:MAR003-4) 
\end{exe}

Some clauses with a~predictive semantics are coded in the same manner, either with the `if'-\isi{clause} in the \isi{perfective} (\ref{ex:13-99}) or in the future (\ref{ex:13-100}), although most predictions tend to be coded as hypothetical conditionals (see below). Again, the choice between the two is most likely to do with the degree of predictability, although in some predictive contexts the two codings are used more or less interchangeably.

\begin{exe}
\ex
\label{ex:13-99}
\gll [típa čhéeli míi thaní maníit-u síinta ba]  xálak les mhaar-éen \\
now goat \textsc{1sg.gn} \textsc{qt} say.\textsc{pfv"=msg} \textsc{condh} \textsc{prt} people \textsc{3sg.dist.acc} kill-\textsc{3pl} \\
\glt `If I now say ``This is my sheep'', people will kill him.' (B:THI)

\ex
\label{ex:13-100}
\gll [ma róot-a ɡaɡalá ṭeem-íi yhayí ma  tu the ṭaak-úum seentá] tu
nikh-á  ta kráam-a the b-áaya \\
\textsc{1sg.nom} night-\textsc{ob} whenever time-\textsc{gn} come.\textsc{cv} \textsc{1sg.nom}  \textsc{2sg.nom} to call-\textsc{1sg} \textsc{condh} \textsc{2sg.nom} come.out--\textsc{imp.sg}  \textsc{prt} work-\textsc{ob} to come-\textsc{1pl} \\
\glt `If/When I come, whatever time during the night, and knock, you should come out, and we will go to work.' (A:WOM628) 
\end{exe}

In some clauses, such as in (\ref{ex:13-101}), \textit{síinta} clearly implies cause rather than condition, particularly in B. it seems.

\begin{exe}
\ex
\label{ex:13-101}
\gll [koó se muṭ-á túuri-e thíi baṭ-áam  ǰ-áan-a šíinta ba] kúuk-a
na bheš-ái  bhay-áan-a \\
someone \textsc{def} tree-\textsc{ob} below-\textsc{gn} from stone-\textsc{ins}  hit-\textsc{prs"=mpl} \textsc{condh} \textsc{prt} crow-\textsc{pl} \textsc{neg} sit.down-\textsc{inf} be.able.to-\textsc{prs.mpl} \\
\glt `Because someone is throwing stones from under the tree, the crows cannot sit down.' (B:SHB741) 
\end{exe}

In some pieces of \isi{discourse}, \textit{seentá/síinta} is also used \isi{clause}"=initially, as a~single element, either without any obvious reference to any preceding \isi{clause} and with a~rather vague semantics, glossed as `then' or `well', or with a~very general causal reference to a~larger chunk of \isi{discourse}, functioning as a~\isi{conjunction} with the meaning `so', `therefore' or `it means'. An example of the latter is (\ref{ex:13-102}).

\begin{exe}
\ex
\label{ex:13-102}
\gll seentá asíi iskuúl bi asaám the bíiḍ-i dhúura  hín-i \\
so \textsc{1pl.gn} school also \textsc{1pl.acc} to much-\textsc{f} distant be.\textsc{prs-f} \\
\glt `So (as I said), we also have a~long way to school.' (A:OUR016) 
\end{exe}

\spitzmarke{Hypothetical \isi{conditionality}.} These conditionals belong mainly under the imaginative type, following \citeauthor{thompsonetal2007}'s (\citeyear[259--260]{thompsonetal2007}) classification. Examples (\ref{ex:13-103})--(\ref{ex:13-104}) express conditions of a~lower verisimilitude than the assumed conditionals above. The conditionally"=marked \isi{clause} of this kind, often translatable with `if{\ldots}', is a~\isi{finite} (often \isi{perfective}) \isi{clause} with the \isi{conditionality} marker \textit{heentá}, followed by an~im\isi{perfective} `then'-\isi{clause}. They describe a~condition that is probable or possible, although to a~certain extent imagined. Most conditionals referring to the future are also coded this way.

\begin{exe}
\ex
\label{ex:13-103}
\gll [inšaalaá heenṣúka paás bhíl-a heentá]  ɡhróom-a iskuul-í the wh-áaya \\
 God.willing this.year  pass become.\textsc{pfv"=mpl} \textsc{condl}  village-\textsc{ob} school-\textsc{ob} to come.down-\textsc{1pl} \\
\glt `If we pass the test this year, we will go to the village school.' (A:OUR013)

\ex
\label{ex:13-104}
\gll [dúu ǰáan-a akaadúi xox bhíl-a  heentá] se bhraawú bh-áan-a,
bhraawéeli  th-áan-a \\
two person-\textsc{pl} \textsc{recp} liking become.\textsc{pfv"=mpl}  \textsc{condl } \textsc{3pl.nom} brother.\textsc{pl} become-\textsc{prs"=mpl} brotherhood do-\textsc{prs"=mpl} \\
\glt `If two persons like one another, they become brothers, they form a~brotherhood.' (A:MIT010-1) 
\end{exe}

The same structure is found in negative Conditionals (\ref{ex:13-105}) as well.

\begin{exe}
\ex
\label{ex:13-105}
\gll [thíi ninaám na phedúul-a heentá]  qeaamatée-e dees-á ma tu díi khooǰ-áam \\
\textsc{2sg.gn} \textsc{3pl.prox.acc} \textsc{neg} take.\textsc{pfv"=mpl} \textsc{condl} judgement-\textsc{gn} day-\textsc{ob} \textsc{1sg.nom} \textsc{2sg.nom} from ask-\textsc{1sg}  \\
\glt `If you don't take these (to her), I will ask you on the day of judgement.' (B:FLW800) 
\end{exe}

As with assumed \isi{conditionality}, the \isi{clause}"=final conditional marker also co"=occurs with a~temporal \isi{conjunction} in some instances of hypothetical \isi{conditionality}, as can be seen in (\ref{ex:13-106}). 

\begin{exe}
\ex
\label{ex:13-106}
\gll [maṇḍáu kareé thíil-u heentá] maṇḍáu wée  ɡhareé ḍhínɡar ɡhooṣṭ-á díi tuúš muxtalíf bh-áan-u  \\
veranda when do.\textsc{pfv"=msg} \textsc{condl} veranda in  again wood house-\textsc{ob} from a.little different become-\textsc{prs"=msg} \\
\glt `Whenever/if you were to make a~veranda, the wood of the veranda would be different from that of the house.' (A:HOW073) 
\end{exe}

Besides the `then'-clauses in (\ref{ex:13-103})--(\ref{ex:13-106}) with \isi{Future} and \isi{Present}, respectively, the \isi{Imperative} as well as the \isi{Simple Past} are possible to use in future"=referring `then'-clauses, both occurring in (\ref{ex:13-107}).

\begin{exe}
\ex
\label{ex:13-107}
\gll [ki anú múṛ-u heentá] ḍhanɡ-á  [anú ǰáand-u ba dharíit-u
  heentá]  asím tu the baxíl-u \\
\textsc{comp} \textsc{3msg.prox.nom} die.\textsc{pfv"=msg} \textsc{condl} bury-\textsc{imp.sg}  \textsc{3msg.prox.nom} alive-\textsc{msg} \textsc{prt} remain.\textsc{pfv"=msg} \textsc{condl}  \textsc{1pl.erg} \textsc{1sg.nom} to donate.\textsc{pfv"=msg} \\
\glt `Were he to die, then bury him, were he, however, to stay alive, then we have given him to you.' (B:ATI052) 
\end{exe}

Quite infrequently, the `if'-\isi{clause} occurs in a~TMA category other than the \isi{perfective} categories. The utterance in (\ref{ex:13-108}) is in effect a~threat, while the utterance in (\ref{ex:13-109}) is a~polite request.

\begin{exe}
\ex
\label{ex:13-108}
\gll [aṛé phoó na d-áan-a heentá] ma  pirsaahíb haṛáyuu traṭáa th-áan-u \\
 \textsc{dist}{\protect\footnotemark} boy \textsc{neg} give-\textsc{prs"=mpl} \textsc{condl} \textsc{1sg.nom}  Pir.Sahib  from.there expulsion do-\textsc{prs"=msg} \\
\glt `Unless they hand over the boy, I will expel Pir Sahib from there.' (B:ATI064-5)

\ex
\label{ex:13-109}
\gll [tu mheerabeení the asíi móoṇuṣ asaám d-íiṛ  heentá] šóo bh-íi  \\
\textsc{2sg.nom} thanks to \textsc{1pl.gn} person \textsc{1pl.acc} give-\textsc{2sg}  \textsc{condl} good.\textsc{msg} become-\textsc{3sg} \\
\glt `Would you be so kind as to give us our person [the bride].' (A:MAR059) 
\end{exe}

\footnotetext{This is not the expected form of a~\isi{determiner}, since the head is a~masculine singular \isi{nominative}.}


Apart from the explicitly marked \isi{Conditional} clauses examplified above, we also find \isi{conditionality} implied in \isi{Converb} clauses, as in (\ref{ex:13-110}), as well as in \textit{ta}-clauses, as in (\ref{ex:13-111}). The Past Im\isi{perfective} in the `then'-\isi{clause} in (\ref{ex:13-111}) implies unfulfilment by itself.

\begin{exe}
\ex
\label{ex:13-110}
\gll [ma tu na khaá ba] kaseé the  uṛ-áan-u \\
\textsc{1sg.nom} \textsc{2sg.nom} \textsc{neg} eat.\textsc{cv} \textsc{prt} someone.\textsc{acc} to let.out-\textsc{prs"=msg} \\
\glt `If I don't eat you, I will let you be eaten by someone else.' (A:KAT021)

\ex
\label{ex:13-111}
\gll \label{bkm:Ref190830313}[malɡíri-m the man-úum de ta] ǰinaazá  ɡal-í uḍhíiw"=an de \\
companion-\textsc{pl.ob} to say-\textsc{1sg} \textsc{ pst} \textsc{prt} corpse throw-\textsc{cv} flee-\textsc{3pl} \textsc{ pst} \\
\glt `If I were to tell my companions, they would leave the corpse and run away.' (A:GHA047) 
\end{exe}

Also the coordinate adversative construction \textit{ta{\ldots}ba} (see \sectref{subsec:13-2-1}), as in (\ref{ex:13-112}), can by itself imply \isi{conditionality}.

\begin{exe}
\ex
\label{ex:13-112}
\gll [ma nis aáǰ kh-úum ta rhootašíi-a  ba kanáa bh-úum \\
\textsc{1sg.nom} \textsc{3sg.prox.acc} today eat-\textsc{1sg} \textsc{prt} morning-\textsc{ob} \textsc{prt} like.what become-\textsc{1sg} \\
\glt `If I eat this today, what will then become of me tomorrow?' (A:HUB005) 
\end{exe}

\spitzmarke{Counterfactuality.} In counterfactual expressions, as in (\ref{ex:13-113})--(\ref{ex:13-114}), both the `if'-\isi{clause} and the `then'-\isi{clause} occur with the \isi{conditionality} marker of low verisimilitude, \textit{heentá}, i.e., the one also used in the hypothetical expressions above. The word \textit{áɡar} in (\ref{ex:13-114}) is a~subordinating expression also occurring in \ili{Urdu} \isi{conditionality} \citep[101--103]{schmidt1999}.

\begin{exe}
\ex
\label{ex:13-113}
\gll [yáraa ma islaamabaad-á ɡáu de  heentá] šuy bhíl-u de heentá,
islaamabaád  dhríiṣṭ-u de heentá \\
oh! \textsc{1sg.nom} Islamabad-\textsc{ob} go.\textsc{pfv.msg} \textsc{pst}  \textsc{condl} good become.\textsc{pfv"=msg} \textsc{pst} \textsc{condl} Islamabad see.\textsc{pfv"=msg} \textsc{pst} \textsc{condl} \\
\glt `If only I had gone to Islamabad, it would have been good, I would have seen Islamabad.' (B:ISH001-2)

\ex
\label{ex:13-114}
\gll \label{bkm:Ref190830564}[áɡar thíi dóodu ǰáand-u heensíl-u  heentá]
tasíi úmur típa čuurbhišá kaal-á de  heentá \\
if \textsc{2sg.gn} grandfather alive-\textsc{msg} stay.\textsc{pfv"=msg}  \textsc{condl} 
\textsc{3sg.gn} age now 80 year-\textsc{pl} be.\textsc{pst} \textsc{condl} \\
\glt `If your grandfather were still alive, he would by now be 80 years old.' (A:HLE3086) 
\end{exe}

The proposition in the `if'-\isi{clause} of (\ref{ex:13-114}) is contrary to fact, i.e., the person speaking knows for sure that the person referred to is dead, and therefore the `then'-\isi{clause} here expresses what would have been the case if that had been otherwise, whereas if the verisimilitude value is pending \citep[332]{givon2001b}, as is the case with the hypothetical \isi{conditionality} exemplified already, the proposition of the `then'-\isi{clause} may still be true. A hypothetical \isi{conditionality} contrasting with the \isi{counterfactuality} in (\ref{ex:13-114}) can be seen in the almost parallel sentence in (\ref{ex:13-115}).

\begin{exe}
\ex
\label{ex:13-115}
\gll [áɡar tasíi dóodu ǰáand-u hín-u  heentá] tasíi úmur čuurbhišá kaal-á hóons-a \\
if \textsc{3sg.gn} grandfather alive-\textsc{msg} be.\textsc{prs"=msg}  \textsc{condl} \textsc{3sg.gn} age 80 year-\textsc{pl} stay-\textsc{3sg} \\
\glt `If his grandfather is still alive, he would be 80 years old.' (A:HLE3086) 
\end{exe}

\spitzmarke{Concessive \isi{conditionality}.} Concessive \isi{conditionality} is basically expressed by the same means as hypothetical \isi{conditionality}, i.e., with a~\isi{finite} (often \isi{Simple Past}) `if'-\isi{clause} with the \isi{conditionality}"=marker \textit{heentá}. However, as seen in (\ref{ex:13-116}) and (\ref{ex:13-117}), the particle \textit{bi} `also, even' is added to emphasise the contrary"=to"=expectation reading. 

\begin{exe}
\ex
\label{ex:13-116}
\gll [tíi áɡar šúi paaṇṭí ṣéel-i de heentá  bi] so bač na bh"=eeṇḍeéu de  \\
\textsc{3sg.ob} if good.\textsc{f} clothes put.on.\textsc{pfv-f} \textsc{pst} \textsc{condl}  even \textsc{3msg.nom} saved \textsc{neg} become-\textsc{oblg} \textsc{pst} \\
\glt `Even if he had been wearing good clothes, he would not have been saved.' (A:CHE071114)

\ex
\label{ex:13-117}
\gll [tíi máa=the ḍábal tanxaá bi dít-u  heentá] ma eesé nookarí na th-úum \\
\textsc{3sg.ob} \textsc{1sg.nom}=to double salary even give.\textsc{pfv"=msg}  \textsc{condl } \textsc{1sg.nom} \textsc{rem} job \textsc{neg} do-\textsc{1sg} \\
\glt `Even if he were offering me the double salary, I would not take the job.' (A:CHE071114)
\end{exe}

\subsection{Clauses with other adverbial functions}
\label{subsec:13-4-5}

\spitzmarke{Concessive clauses.} There is probably no unique strategy for forming \isi{concessive} clauses. A \isi{conjunction} with ``adversative'' reading (see \sectref{subsec:13-2-1}--\sectref{subsec:13-2-2}) can, for instance, be used for this \isi{purpose}, with (\ref{ex:13-118}) or without (\ref{ex:13-119}) the contrary"=to"=expectation particle \textit{bi} `also, even'. 

\begin{exe}
\ex
\label{ex:13-118}
\gll [so šúi paaṇṭí bi ṣaá  heensíl-u de] xu bač na bhíl-u  \\
\textsc{3sg.nom} good.\textsc{f} clothes even put.on.\textsc{cv}  stay.\textsc{pfv"=msg } \textsc{pst} but saved \textsc{neg} become.\textsc{pfv"=msg} \\
\glt `Even though he was wearing good clothes, he did not survive.' (A:CHE071114)

\ex
\label{ex:13-119}
\gll [míi parčá ta dít-u xu ma paás  ba na bh"=eeṇḍeéu \\
\textsc{1sg.gn} paper \textsc{prt} give.\textsc{pfv"=msg} but \textsc{1sg.nom} passed \textsc{prt} \textsc{neg} become-\textsc{oblg} \\
\glt `Although I turned in my paper, I will not pass.' (A:CHE071114) 
\end{exe}

\spitzmarke{Substitutive clauses.} As with \isi{concessive} clauses there is no single dedicated \isi{substitutive} construction. Instead a~few different constructions are used to express the semantics of a~\isi{substitutive}.


First, there are constructions involving the \isi{noun} \textit{zhaáy} `place'. The one exemplified in (\ref{ex:13-120}), has a~subordinate \isi{clause} almost analogous to the \ili{English} `instead of{\ldots}', with a~\isi{Verbal Noun} qualifying the \isi{oblique} form of \textit{zhaáy}. The `place'-\isi{noun} in (\ref{ex:13-121}) is on the other hand only qualified by a~possessive \isi{phrase} and is therefore not a~subordinate \isi{clause} at all. 

\begin{exe}
\ex
\label{ex:13-120}
\gll [karaačí the bainií zhay-í] ma ba laahur-á  the ɡúum \\
Karachi to go.\textsc{vn} place-\textsc{ob} \textsc{1sg.nom} \textsc{prt} Lahore-\textsc{ob} to go.\textsc{pfv.msg} \\
\glt `Instead of going to Karachi, I went to Lahore.' (A:CHE071114)

\ex
\label{ex:13-121}
\gll ibrahiím aleehisalaam-íi putr ismaaíil-e zhay-í  míi dúudu qurbaán bhe hín-u \\
Ibrahim peace"=be"=upon"=him-\textsc{gn} son Ismail-\textsc{gn} place-\textsc{ob}  \textsc{1sg.gn} forefather sacrifice become.\textsc{cv} be.\textsc{prs"=msg} \\
\glt `My forefather was sacrificed instead [lit. in the place] of Ibrahim's (PBUH) son Ismail.' (B:SHI013) 
\end{exe}

It is also possible to use a~comparative construction `X-ing a, is better than X-ing b'. In this case, as seen in (\ref{ex:13-122}), the \isi{subject} head as well as the head of the \isi{predicate} \isi{noun} \isi{phrase} are Verbal Nouns.

\begin{exe}
\ex
\label{ex:13-122}
\gll [pulusá the man"=ainií díi] máa=the man"=ainií šóo de \\
police to say-\textsc{vn} from \textsc{1sg.nom=}to say-\textsc{vn} good.\textsc{msg} be.\textsc{pst} \\
\glt `Rather than telling the police, you should have told me [Lit. The telling to me is better than the telling to the police].' (A:CHE071114) 
\end{exe}

Although implicit to a~larger extent, a~\isi{substitutive} reading of an~utterance"=\isi{complement} with the direct"=quote marker \textit{thaní} (cf. \sectref{subsec:13-5-1}), is also possible ((\ref{ex:13-123})--(\ref{ex:13-124})) and may very well be the preferred way of expressing substitution. In (\ref{ex:13-124}) this interpretation is supported by the word \textit{ulṭá} `upside"=down'.

\begin{exe}
\ex
\label{ex:13-123}
\gll [ma karaačí the béem thaní] laahur-á the  ɡúum \\
\textsc{1sg.nom} Karachi to go.\textsc{1sg} \textsc{qt} Lahore-\textsc{ob} to go.\textsc{pfv.msg} \\
\glt `Instead of going to Karachi [as I said], I went to Lahore.' (A:CHE071114)

\ex
\label{ex:13-124}
\gll [díiš-a ba baalbač-á kuṛíina táma  th-éen de asée bharíiw-a be
  hín-a neečíir the whaal-éen mhaas-á whaal-éen thaní] hasé	dees-á ulṭá hasé trúu ǰinaazeé ɡhin-í wháat-a xálak-a \\
village-\textsc{ob} \textsc{prt} child-\textsc{pl} woman.\textsc{pl} waiting do-\textsc{3pl}
\textsc{pst} \textsc{1pl.gn} husband-\textsc{pl} go.\textsc{cv} be.\textsc{prs"=mpl} game to
bring.down-\textsc{3pl} meat-\textsc{pl} bring.down.\textsc{3pl} \textsc{qt}  \textsc{rem} day-\textsc{ob} upside.down \textsc{rem} three corpse.\textsc{pl} take.\textsc{cv} come.down.\textsc{pfv"=mpl} people-\textsc{pl} \\
\glt `In the village the children and women were waiting, but instead of the husbands coming back from the hunting with game and meat, those three corpses were brought down by people on that day.' (B:AVA218-20) 
\end{exe}

\spitzmarke{Additive clauses.} There is not much evidence of any entirely unique construction expressing \isi{additive} semantics. Instead the \isi{conjunction} with \textit{bi{\ldots}bi} (see \sectref{subsec:13-2-1}) seems to cover some occurrences of ``addition''. Possibly the construction found in (\ref{ex:13-125}), with a~subordinate \isi{Verbal Noun} followed by the \isi{postposition} \textit{sanɡí} `with' along with \textit{bi} in the main \isi{clause}, represents a~more explicit expression of ``addition''. 

\begin{exe}
\ex
\label{ex:13-125}
\gll [ma bhanǰ-ainíi-a sanɡí] áak the tíi míi  paiseé bi palóol-a \\
\textsc{1sg.nom} beat-\textsc{vn"=ob} with one to \textsc{3sg.ob} \textsc{1sg.gn} money.\textsc{pl}  also steal.\textsc{pfv"=mpl} \\
\glt `Besides beating me up, he stole my money.' (A:CHE071114) 
\end{exe}

It is uncertain what role the adverbial \isi{phrase} \textit{áak the} plays in signalling ``addition''. It also occurs in (\ref{ex:13-126}), where a~similar reading is possible, although the two clauses here are symmetric rather than one of them being subordinate the other.

\begin{exe}
\ex
\label{ex:13-126}
\gll haṛó hateeṇ-ú deés de ki [ak  the kir bíiḍ-u de] [dhuíime xálak  adrax-í heensíl-a] \\
\textsc{3msg.rem.nom} such-\textsc{msg}  day be.\textsc{pst} \textsc{comp} one  to snow much-\textsc{msg} be.\textsc{pst} second people  forest"=\textsc{ob} stay.\textsc{pfv"=mpl} \\
\glt `It was such a~day, that besides there being a~lot of snow, people were in the forest (or It was such a~day, that on the one hand there was a~lot of snow, and on the other there were people in the forest).' (B:AVA216)
\end{exe}

\section{Complement clauses}
\label{sec:13-5}

Palula utilises a~number of different strategies for sentential complementation, mainly determined by the \isi{complement}"=taking verb. However, due to the degree of semantic bond between the \isi{complement}"=taking \isi{predicate} and the verbal element of the \isi{complement} \citep[39-40]{givon2001b}, most of them fall, quite clearly, into one of two main types: a) sentence"=like complements, and b) \isi{nonfinite} complements. While the complements of the first type usually (but not exclusively) are extraposed and syntactically are relatively independent from the main \isi{clause}, the complements of the second type are almost invariably embedded into the main \isi{clause} and have a~time reference entirely determined by the main \isi{clause}.

The specific strategies found in the S-like main type are:

\begin{enumerate}
\item[(i)] A zero strategy, or \isi{juxtaposition}, whereby the \isi{complement}, always a~direct"=quote utterance, occurs after the main \isi{clause} and its \isi{complement}"=taking \isi{predicate} without any overt \isi{complementiser} or other linking device
\item[(ii)] A very broadly applied \textit{ki}-strategy, whereby a~particle or \isi{complementiser} \textit{ki} follows the \isi{complement}"=taking \isi{predicate} and precedes the extraposed (heavy"=shifted) \isi{finite} \isi{complement clause} (basically an~instance of relativisation, see \sectref{subsec:13-6-7})
\item[(iii)] A \textit{thané}-strategy, whereby a~reported \isi{discourse} \isi{complement} precedes a~form of \textit{thané-} `say' (often in the converb form \textit{thaní})
\item[(iv)] A minor \textit{ta}-strategy, whereby a~particle \textit{ta} follows the \isi{complement}"=taking \isi{predicate} and precedes the extraposed \isi{finite} \isi{complement clause} (basically an~instance of \isi{clause chaining}, see \sectref{sec:13-3})
\end{enumerate}

An important feature of the S-like complements with the strategies (i)--(iii) is that they always take a~quoted"=speaker's/quoted"=experiencer's perspective as far as \isi{pronominal} reference and deixis are concerned. This also means that there are hardly any constructions that are indirect; probably no reported speech ever occurs as indirect.


Strategy (ii) used with reported \isi{discourse} is often combined with strategy (iii), in which case \textit{thané-} mainly functions as a (end"=of) quotation marker (see discussion in \sectref{subsec:9-2-4}).


The specific strategies found in the \isi{nonfinite} type are:

\begin{enumerate}
\item[(v)] A \isi{complement} in the form of a~\isi{Verbal Noun} embedded into the main \isi{clause}, a~broadly applied strategy
\item[(vi)] A \isi{complement} in the form of an~\isi{Infinitive}, showing a~high degree of integration between the \isi{complement} and the main \isi{clause}
\item[(vii)] A \isi{causative} construction in which the \isi{complement}"=taking \isi{predicate} is entirely co"=lexicalised with the \isi{complement}
\end{enumerate}
Strategy~(vii) represents the top of the complementation scale \citep[74]{givon2001b} and is at best only underlyingly analysable as two clauses, but is included here as there is no other natural context for it to be discussed.


A few additional minor strategies are examplified below, but due to their isolated occurrence in the data, they have not been listed above.



In the following we will exemplify and discuss these strategies as they occur within some broadly and functionally defined classes or categories of \isi{complement}"=taking predicates (following \citealt[40--59]{givon2001b} and \citealt[120--145]{noonan2007}). However, as will be apparent in the discussion, the strategies outlined above are not randomly applied, but represent a~language"=internal differentiation along a~continuum. This is a~continuum that, on the one hand, stretches from \isi{complement}"=taking \isi{PCU} (Perception"=Cognition"=Utterance) predicates with a~weak bond between the two events to modality predicates with a~strong bond between them. On the other hand, it is a~continuum stretching between \isi{PCU} predicates with a~weak bond to manipulation predicates with a~strong bond between the two events.


\subsection{Complement"=taking \isi{PCU} predicates}
\label{subsec:13-5-1}

\spitzmarke{Utterance predicates.} Only with utterance predicates is the zero strategy ((i), above) alone used, and even then sparingly. According to my data, it is only the verb \textit{mané-} `say, call, tell', as in (\ref{ex:13-127})--(\ref{ex:13-128}), that seems to allow it.

\begin{exe}
\ex
\label{ex:13-127}
\gll uṭ-á maníit-u [índa násel izát pakaár  na bh-éen-i]  \\
camel-\textsc{ob}  say.\textsc{pfv"=msg} here lineage honour need  \textsc{neg} become-\textsc{prs-f} \\
\glt `The camel said: ``There is no need here for lineage or honour.''' (B:SHI015)

\ex
\label{ex:13-128}
\gll ak kúṛi tes the maníit-u [thíi ak  dhií paidáa bhíl-i ta, thíi
  kúṛi tes  ɡhin-í adráx ṣač-í ɡéi] \\
\textsc{idef} woman \textsc{3sg.acc} to say.\textsc{pfv"=msg} \textsc{2sg.gn} \textsc{idef}  daughter born become.\textsc{pfv-f} \textsc{prt} \textsc{2sg.gn} woman \textsc{3sg.acc}  take-\textsc{cv} forest climb-\textsc{cv} go.\textsc{pfv.fsg}  \\
\glt `A woman told him: ``A daughter was born, and your wife took her with her and went up into the forest.''' (B:FOR013-4) 
\end{exe}

More frequently, the \textit{ki}-strategy ((ii) above) is used, with \textit{mané-} as well as with a~number of other utterance predicates (or predicates that in the context imply an~utterance): (\ref{ex:13-129})--(\ref{ex:13-131}). Always when \textit{ki} is used as a~\isi{complementiser}, the \isi{complement} is heavy"=shifted to the end.

\begin{exe}
\ex
\label{ex:13-129}
\gll míi maníit-u ki [šóo tus b-óoi]  \\
\textsc{1sg.gn} say.\textsc{pfv"=msg} \textsc{comp} good.\textsc{msg} \textsc{2pl.nom} go.\textsc{imp.pl} \\
\glt `I said: ``Good, you may go!''' (A:ACR010)

\ex
\label{ex:13-130}
\gll karáaṛ-a kaṭamuš-íi méemi díi  khooǰóol-u hín-u ki [o méeš  lhéṇḍ-u ba] \\
leopard-\textsc{ob} Katamosh-\textsc{ gn} grandmother from  ask.\textsc{pfv"=msg} be.\textsc{prs"=msg} \textsc{comp} oh! aunt.\textsc{voc}  bald-\textsc{msg} \textsc{prt } \\
\glt `The leopard asked Katamosh' grandmother: ``Oh auntie, what about the bald one?''' (A:KAT096-7)

\ex
\label{ex:13-131}
\gll luumái čéɣi dít-i hín-i ki [ée amzarái,  ée karáaṛu,
  ée iṇc̣ táru wh-óoi, míi  lhéṇḍ-u láad-u] \\
fox cry give.\textsc{pfv-f} be.\textsc{prs-f} \textsc{comp} oh! lion  oh! leopard oh! bear quickly come.down-\textsc{imp.pl} \textsc{1sg.gn} bald-\textsc{msg} find.\textsc{pfv"=msg}  \\
\glt `The fox shouted: ``Oh lion, oh leopard, oh bear, come quickly, I've found the bald one.''' (A:KAT118-9) 
\end{exe}

Related utterances in Palula always occur as direct quotations, as can be seen in the examples (\ref{ex:13-127})--(\ref{ex:13-131}), or more correctly, they are always presented from a~quoted"=speaker's or experiencer's (i.e., the \isi{subject} of the matrix \isi{clause}) perspective, as far as deixis, \isi{pronominal} and time reference is concerned. Non"=verbal ``utterances''\footnote{It is probably equally possible to classify the \isi{predicate} in a~sentence like this as desiderative~-- i.e., expressing a~desire that the proposition be realised~-- rather than as an~uttererance \isi{predicate}.} are treated in this way too, i.e., thoughts, as in (\ref{ex:13-132}), are presented from the thinking agent's perspective, as if they were uttered verbally (a marginal exception to this is found with what I refer to below as ``immediate perception predicates''). 

\begin{exe}
\ex
\label{ex:13-132}
\gll áa deés se phoó xiaál thíil-i híni ki  [tuúš čhoót míi
  se yaar-íi rumeel-í  maǰí ɡhaṇḍ-í wíi-a keé-na ɡal-úum] \\
one day \textsc{def} boy opinion do.\textsc{pfv-f} be.\textsc{prs-f} \textsc{comp}   some cheese \textsc{1sg.gn} \textsc{def} friend-\textsc{gn} handkerchief-\textsc{ob}  in tie-\textsc{cv} water-\textsc{ob} why-\textsc{neg} throw-\textsc{1sg} \\
\glt `One day the boy thought: ``Why don't I put some cheese in my friend's handkerchief and throw it in the water.''' (A:SHY043)
\end{exe}

While the \isi{complement} of the utterance predicates mentioned so far are all extraposed (vis-à-vis the basic SOV word order) and have to be so when used this way, that is never allowed with any form of the verb \textit{thané-} `say, call' (strategy~(iii) above), as in (\ref{ex:13-133})--(\ref{ex:13-134}). Note that, in example (\ref{ex:13-134}), the sentence with \textit{thaní} is part of a~longer \isi{clause} chain (see \sectref{sec:13-3}).

\begin{exe}
\ex
\label{ex:13-133}
\gll sum huṇḍ the ṣuɡal-í ba [čo ba]  thaníit-u \\
dirt up to throw.away-\textsc{cv} \textsc{prt} now go.\textsc{imp.sg} say.\textsc{pfv"=msg} \\
\glt `He threw dirt up [in the air], and said: ``Go on!''' (A:PIR037)

\ex
\label{ex:13-134}
\gll c̣hatróol"=ii xálak-a xušaán bhe,  [aníi asíi nóo zindá thíil-u]
than-í,  taním bi baxšíš dít-i maṭaaíi-m   ɡeél-i \\
\ili{Chitral}-\textsc{gn} people-\textsc{pl} happy become.\textsc{cv}  \textsc{3sg.prox.ob } \textsc{1pl.gn} name alive do.\textsc{pfv"=msg} say-\textsc{cv}  \textsc{3pl.erg} also award give.\textsc{pfv-f} sweets-\textsc{pl} throw.\textsc{pfv-f}  \\
\glt `The Chitrali people became very happy, saying ``He has saved our reputation [lit. made our name live]'', and they also gave awards and distributed sweets' (A:BEW007-9) 
\end{exe}

The verb \textit{thané-} refers straightforwardly and literally back to what was actually uttered,
and especially when combined with another \isi{utterance predicate} and/or \textit{ki} preceding the
utterance, which is the case in (\ref{ex:13-135}) and (\ref{ex:13-136}), it becomes by effect
an~end"=of"=quotation marker, particularly in its converb shape \textit{thaní}.\footnote{That
  \textit{thaní} is becoming grammaticalised to a~large extent is reflected in its being used rather
  often, contrary to expectation even when there is a~\isi{subject} switch, while a~converb normally
  signals same"=\isi{subject} reference across clauses (as described elsewhere) without having any clear
  connection with the following \isi{discourse}.}

\begin{exe}
\ex
\label{ex:13-135}
\gll deeúlii xálak"=am daawát dít-i  atsharíit"=am the ki [muqaabilá
  th-íia]  thaníit-u ta atshareet-íi paalawaaṇ-aán ɡíia \\
Dir.\textsc{gn} people-\textsc{pl.ob} invitation give.\textsc{pfv-f}  Ashreti-\textsc{pl.ob} to \textsc{comp} contest do-\textsc{1pl}  say.\textsc{pfv"=msg} \textsc{comp} Ashret-\textsc{gn} wrestler-\textsc{pl} go.\textsc{pfv.pl} \\
\glt `The people of Dir invited the Ashretis: ``Let's have a~contest!'' And [having said that] the wrestlers from Ashret left.' (A:CHA001-2)

\ex
\label{ex:13-136}
\gll míi ba ṭeekíl-u, [kaaka=oó=kaakaa ínc̣-a čhéeli  khéel-i]
thaní, ta bhraáš pašambeé so  bháaru čhúuṇ-u \\
\textsc{1sg.gn} \textsc{prt} call.\textsc{pfv"=msg} uncle=oh!=uncle bear-\textsc{ob} goat  eat.\textsc{pfv-f} \textsc{qt } \textsc{prt} slowly Pashambi \textsc{def.msg.nom} load put.\textsc{pfv"=msg}  \\
\glt `I called out: ``Oh uncle, uncle, the bear has eaten the goat'', whereby Pashambi slowly put down his load.' (A:PAS056-7) 
\end{exe}

\spitzmarke{Propositional attitude predicates.} Closely related to and sometimes difficult to distinguish from utterance predicates are those predicates that express an~attitude regarding the \isi{complement} proposition, especially concerning its verisimilitude value. The particle \textit{xu} (in examples (\ref{ex:13-137}) and (\ref{ex:13-138})) expresses certainty on the part of the \isi{subject} of the matrix \isi{clause} concerning the immediately preceding argument or \isi{phrase}.


It seems also that the range of strategies available is much the same as for utterance predicates. The \isi{complement} is, like the former, presented in direct (reported) \isi{discourse}, often signalled by the end"=of"=quotation marker \textit{thaní}. 

\begin{exe}
\ex
\label{ex:13-137}
\gll se míiš-a soóč thíil-i ki [anú xu  típa ma páta kh-óo] (thaní ba) \\
\textsc{def } man-\textsc{ob} thinking do.\textsc{pfv-f} \textsc{comp} \textsc{3msg.prox.nom} \textsc{prt}  now \textsc{1sg.nom} surely eat-\textsc{3sg} \textsc{qt} \textsc{prt} \\
\glt `The man thought that he would surely eat him too [lit. The man thought: ``He will surely eat me too'']. (Having said that{\ldots})' (A:THA009-10)

\ex
\label{ex:13-138}
\gll [ée míi xudaayaá ni xu ux-íi  rhaíi hín-i xéer béem ni
  rhaíi  ɡhaš-í ɡubáa ta] thaní \\
oh! \textsc{1sg.gn} God.voc \textsc{3pl.prox.nom} \textsc{prt} camel-\textsc{gn}  foot.prints be.\textsc{prs-f} well go.\textsc{1sg} \textsc{prox} foot.prints catch-\textsc{cv} what \textsc{prt} \textsc{qt } \\
\glt `I thought: ``Oh my God! These must be the foot prints of a~camel. Well, in any case I'll go ahead and follow them.''' (A:HUA061-2) 
\end{exe}

Propositional attitudes may also, as in (\ref{ex:13-139}), be expressed as \isi{noun} complementation, the general structure remaining the same.

\begin{exe}
\ex
\label{ex:13-139}
\gll asíi xiaál ki [ɡóo mheeríl-u heentá  khóol-u] thaní \\
\textsc{1pl}.\textsc{gn}  opinion \textsc{comp} perhaps kill.\textsc{pfv"=msg}  \textsc{condl} eat.\textsc{pfv"=msg} \textsc{qt} \\
\glt `We thought that he had been killed and eaten [lit. Our opinion was that: ``If he has been killed, he has been eaten''].' (A:GHA011) 
\end{exe}

\spitzmarke{Commentative predicates.} Commentative (or factive) predicates provide a~comment on a~\isi{complement} proposition that is assumed to be real \citep[127--129]{noonan2007}. Again, the distinction between these and utterance predicates is not always very obvious, as the \isi{complement} often occurs as reported \isi{discourse}, and there is probably quite some overlap. Typically, the \isi{complement clause} is formed as a~question, with \textit{keé} `why' in (\ref{ex:13-140}) and \textit{kateeṇí} `what kind' in (\ref{ex:13-141}).

\begin{exe}
\ex
\label{ex:13-140}
\gll so hairán bhíl-u ki [aṛé čhéelí  du=khur-áam ǰhulí keé uth-í hín-i] thaní \\
\textsc{3msg.nom}  surprised become.\textsc{pfv"=msg} \textsc{comp} \textsc{dist} goat  two=leg-\textsc{pl.ob} on why stand.up-\textsc{cv} be.\textsc{prs-f} \textsc{qt} \\
\glt `He was surprised that the goat was standing up on two feet.' (B:SHB724)

\ex
\label{ex:13-141}
\gll dun-áaṭ-u bhíl-u hín-u ki  [aní ba kateeṇ-í ǰuánd] \\
think-\textsc{ag"=msg} become.\textsc{pfv"=msg} be.\textsc{prs"=msg} \textsc{comp}  \textsc{3fsg.prox.nom} \textsc{prt} what.kind-\textsc{f} life  \\
\glt `He started wondering what kind of life this was.' (A:KAT057) 
\end{exe}

When the \isi{subject} of the \isi{complement clause} and the matrix \isi{clause} are coreferent, as in (\ref{ex:13-142}), the \isi{complement} can occur as a~\isi{Verbal Noun} (strategy~(v)).

\begin{exe}
\ex
\label{ex:13-142}
\gll lhoomée [teeṇíi taalím man"=ainíi] šárum  dac̣h-íi de \\
fox.\textsc{ob} \textsc{ refl} education say-\textsc{vn} shame look-\textsc{3sg} \textsc{pst} \\
\glt `The fox was ashamed of talking about his own education.' (B:FOY006) 
\end{exe}

\spitzmarke{Predicates of knowledge acquisition.} Many \isi{complement}"=taking predicates having to do with acquisition of knowledge, use the same strategies as for utterances, and again the \isi{complement} is presented as an~utterance (either by the agent of the matrix \isi{clause}, as in (\ref{ex:13-143}), or by some other, explicit or implicit, source participant, as in (\ref{ex:13-144}) and (\ref{ex:13-145})), whether it is literally and verbally uttered or is merely cognitively processed, i.e., ``uttered'' mind"=internally. 

\begin{exe}
\ex
\label{ex:13-143}
\gll \label{bkm:Ref190835538}búd-a hín-a ki [phaí wíi-a  ɡíi] thaní \\
understand.\textsc{pfv"=mpl} be.\textsc{prs"=mpl} \textsc{comp} girl water-\textsc{ob} go.\textsc{pfv.fsg} \textsc{qt} \\
\glt `They understood that the girl must have thrown herself into the water.' (A:SHY060)

\ex
\label{ex:13-144}
\gll ak yúun padúši so xabár bhíl-u  ki [šišíi-e hakim-á thíi so
  duṣmán nawaab-á  ḍáḍi phreyíl-u] thaní \\
one month after \textsc{3msg.nom} informed become.\textsc{pfv"=msg}  \textsc{comp} Shishi-\textsc{gn} ruler-\textsc{ob } \textsc{2sg.gn} \textsc{def.msg.nom} enemy prince-\textsc{ob} side send.\textsc{pfv"=msg} \textsc{qt}  \\
\glt `A month later he [the king] was informed that the ruler of Shishi had helped his [the king's] enemy to the territory of the prince [of Dir].' (B:ATI030-1)

\ex
\label{ex:13-145}
\gll \label{bkm:Ref190835565}tíi ṣúunt-u de ki, zanɡal-íi zinaawur-óom  díi [insaán thaní áa šay hín-u\textbf{...}] thaní \\
\textsc{3sg.ob} hear.\textsc{pfv"=msg} \textsc{pst} \textsc{comp} forest-\textsc{gn} beast-\textsc{pl.ob}  from human \textsc{qt} \textsc{idef} thing be.\textsc{prs"=msg} \textsc{qt} \\
\glt `He heard from the beasts of the forest, that a~thing called ``man''....' (A:KIN002-4) 
\end{exe}

In examples (\ref{ex:13-143})--(\ref{ex:13-145}) the \isi{complement} is presented as an~already uttered proposition, but even when the~-- assumedly verbal or cognitive~-- knowledge is still to be obtained, as perceived by the \isi{subject} of the matrix \isi{clause}, the same structure is being utilised, which can be seen in (\ref{ex:13-146}) but now without the final \textit{thaní}. 

\begin{exe}
\ex
\label{ex:13-146}
\gll so insaán bulaḍ-íim bíi de páand-a  pharé ki [so insaán kateeṇ-ú
  šay  hín-u] \\
\textsc{def.msg.nom} human find.out-\textsc{cprd} go-\textsc{3sg} \textsc{pst} road-\textsc{ob}  along \textsc{comp} \textsc{def.msg.nom} human what.kind-\textsc{msg} thing be.\textsc{prs"=msg} \\
\glt `He went ahead to find out what kind of thing man is.' (A:KIN004) 
\end{exe}

\spitzmarke{Immediate perception predicates.} With at least two verbs coding directly perceived events, \textit{dac̣h-} `look, see' (in (\ref{ex:13-147})--(\ref{ex:13-148})) and \textit{dhay-} `watch, note' (the latter only in B., (\ref{ex:13-149})), another construction is regularly used (strategy~(iv)), in which the \isi{finite} \isi{complement}"=taking \isi{predicate} occurs with a~particle \textit{ta}, followed by a~\isi{finite} \isi{complement clause}, i.e., the same construction used in the chaining of different"=\isi{subject} clauses (see \sectref{sec:13-4}). 

\begin{exe}
\ex
\label{ex:13-147}
\gll huṇḍɡiraá dac̣h-íin ta [iṇc̣ muṭ-íi phúṭi ǰhulí  bheš-í áaṇc̣-a kha-áan-u] \\
upward look-\textsc{3pl} \textsc{prt} bear tree-\textsc{gn} top on sit-\textsc{cv} raspberry-\textsc{pl} eat-\textsc{prs"=msg}  \\
\glt `They were looking up and saw the bear sitting in the top of the tree eating rasberries.' (A:KAT145)

\ex
\label{ex:13-148}
\gll dac̣híl-u ta [so ba rištaá  xuščhá-ii so máamu] \\
look.\textsc{pfv"=msg} \textsc{prt} \textsc{3msg.nom} \textsc{prt} really Khosh.Shah-\textsc{gn} \textsc{def.msg.nom} uncle  \\
\glt `They saw that he was indeed Khosh Shah's uncle.' (A:JAN056)

\ex
\label{ex:13-149}
\gll káakeṛ dhay-íi ta [dúu peereṇíi-a hasé ziaarat-í  pičh-aaníi the yéel"=im] \\
Kakel note-\textsc{3sg} \textsc{prt} two fairy-\textsc{pl}  \textsc{rem} shrine-\textsc{pl} sweep-\textsc{vn} to come.\textsc{pfv"=fpl} \\
\glt `Kakel noted that two fairies came to sweep those shrines.' (B:FOR030) 
\end{exe}

It should be especially noted that while all the complements with \isi{PCU}"=predicates exemplified in the previous sections represented direct \isi{discourse}, the perceived event presented with this particular construction is indirect. Had it been presented as from the view of the experiencer in the matrix \isi{clause}, the \isi{pronoun} used in (\ref{ex:13-150}) would have been \textit{míi} `my' and not \textit{tasíi} `her'.

\begin{exe}
\ex
\label{ex:13-150}
\gll so šay ɡaḍ-í dac̣h-íi ta [tasíi  dit-i=whaáu rumiaál hín-i] \\
\textsc{def.msg.nom} thing take.off-\textsc{cv} look-\textsc{3sg} \textsc{prt} \textsc{3sg.gn} give.\textsc{pptc-f}=\textsc{adj} handkerchief be.\textsc{prs-f} \\
\glt `Taking up the thing, she saw that it was the handkerchief she had given.' (A:SHY054)
\end{exe}

\subsection{Complement"=taking modality predicates}
\label{subsec:13-5-2}

\spitzmarke{Modal predicates.} As modality is part of the TMA system as a~whole and is expressed morphologically, there are few examples of \isi{complement}"=taking predicates with a~purely modal meaning. However, one such \isi{predicate} is \textit{bha-} `be able to, can', illustrated in (\ref{ex:13-151}). This verb regularly takes an~infinitival \isi{complement}, a~strategy (vi), as far as my data is concerned, confined to this modal verb and the phasal verb \textit{ṣáat-} (see \textbf{Phasal predicates} down below).

\begin{exe}
\ex
\label{ex:13-151}
\gll se har deés akaadúi paš-áai bhóon de  \\
\textsc{3pl.nom} every day \textsc{recp} see-\textsc{inf} be.able.to.\textsc{3pl} \textsc{pst} \\
\glt `They were able to meet every day.' (A:SHY007) 
\end{exe}

That this verb is relatively morphologised as a~potentialis marker is indicated by its loss of inherent transitivity (otherwise highly unusual in the language). It is instead the \isi{complement} verb that makes the whole sentence \isi{transitive} or in\isi{transitive} and also governs case assignment, whereas morphological \isi{verb agreement} is part of the modal segment. In the \isi{perfective} \isi{transitive} sentence (\ref{ex:13-152}), the agent of `eating' is coded in the \isi{ergative}, and \textit{bhá-} agrees in \isi{gender} and number with `meat' (masculine singular) as if it were a~\isi{transitive} verb. In (\ref{ex:13-153}), which is also \isi{perfective} \isi{transitive}, the verb agrees in \isi{gender} and number with the agreeing \isi{host element} `endurance' (feminine singular). In (\ref{ex:13-154}), however, the sentence as a~whole is (\isi{perfective}) in\isi{transitive}, and the \isi{subject} of `going' is coded in the \isi{nominative} and agrees intransitively with \textit{bhá-} in \isi{gender} and number.

\begin{exe}
\ex
\label{ex:13-152}
\gll asím pileeṭ-íi buṭheé mhaás kháai bhóol-u \\
\textsc{1pl.erg} plate-\textsc{gn} all meat eat.\textsc{inf} be.able.\textsc{pfv"=msg} \\
\glt `We were able to eat all the meat on the plates.' (A:CHE070920)

\ex
\label{ex:13-153}
\gll súun-a be ba teeṇíi doost-íi  ǰuda-í bardaáš na th-ái bhéel-i \\
pasture-\textsc{ob} go.\textsc{cv} \textsc{prt} \textsc{refl} friend-\textsc{gn} separation-\textsc{gn} endurance \textsc{neg} do-\textsc{inf} be.able.\textsc{pfv-f}  \\
\glt `When he had gone to the high pasture, he could not bear the separation from his friend.' (B:FLW777)

\ex
\label{ex:13-154}
\gll ma atshareet-á the na báai bhóol-u  \\
\textsc{1sg.nom} Ashret-\textsc{ob} to \textsc{neg} go.\textsc{inf} be.able.\textsc{pfv"=msg}  \\
\glt `I was not able to go to Ashret.' (A:CHN070104) 
\end{exe}

That this exemplifies \isi{clause} union rather than a~full lexical union is supported by the fact that it is still possibile to split the two predicates and insert other constituents between them, as in (\ref{ex:13-155}).

\begin{exe}
\ex
\label{ex:13-155}
\gll ɡa manáa ba koó na bhóon de  \\
anything say.\textsc{inf} \textsc{prt} who \textsc{neg} be.able.\textsc{3pl} \textsc{pst}  \\
\glt `Noone could tell them.' (A:JAN021) 
\end{exe}

Necessity or ablility can also be expressed with \isi{complement} clauses of \isi{complement}"=taking nouns. In those cases, the \isi{Verbal Noun} strategy (v) is used. This is exemplified in (\ref{ex:13-156})--(\ref{ex:13-159}).

\begin{exe}
\ex
\label{ex:13-156}
\gll bíiḍ-i čitíl-i, xu [kuhée wée yúu   nikh"=aníi] ɡa čal na léed-i  \\
much-\textsc{f} think.\textsc{pfv-f} but well.\textsc{ob} in from  come.out-\textsc{vn} any trick \textsc{neg} find.\textsc{pfv-f}  \\
\glt `She thought a~lot, but there was no way of getting out of the well.' (B:FOX013)

\ex
\label{ex:13-157}
\gll [putr-íi ǰhaní thainií] zarurí bh-áan-u \\
son-\textsc{gn} marriage do.\textsc{vn} necessity become-\textsc{prs"=msg}  \\
\glt `It is necessary to get one's son married [lit. It becomes a~necessity to make a~son's marriage].' (A:MAR009)

\ex
\label{ex:13-158}
\gll [muloó haans"=ainií] bíiḍ-u zarurí \\
mullah stay-\textsc{vn}  much-\textsc{msg} necessity  \\
\glt `A mullah must be present.' (A:MAR042)

\ex
\label{ex:13-159}
\gll típa [las haár dainií] asíi mooqá  hín-u \\
now \textsc{3sg.dist.acc} defeat give.\textsc{vn} \textsc{1pl.gn} opportunity be.\textsc{prs"=msg}  \\
\glt `Now we can defeat him [lit. Now is our opportunity to defeat him].' (A:BEZ070) 
\end{exe}

\spitzmarke{Achievement predicates.} Although it is sometimes difficult to draw a~de\isi{finite} line between modal predicates of the ability type presented above and achievement predicates, the main difference in the typical cases lies in implicativity. While the modals above are non"=implicative, achievement predicates are implicative, i.e., signalling successful vs. failed performance/realisation. In any case, the \isi{Verbal Noun} strategy seems to be, if not the only possible, at least the preferred one. Examples are given in (\ref{ex:13-160})--(\ref{ex:13-161}). Some achievement predicates require a~\isi{postposition} with the \isi{Verbal Noun}.

\begin{exe}
\ex
\label{ex:13-160}
\gll tíi [dhoóṛ yhainií] koošíš thíil-i \\
\textsc{3sg.ob} yesterday come.\textsc{vn} attempt do.\textsc{pfv-f} \\
\glt `He tried to come yesterday.' (A:Q9.1102)

\ex
\label{ex:13-161}
\gll pirsaahib-á ba inkaár thíil-i [so phoó  deníi díi] \\
Pir.Sahib-\textsc{ob} \textsc{prt} refusal do.\textsc{pfv-f} \textsc{def.msg.nom} boy  give.\textsc{vn} from \\
\glt `Pir Sahib refused to hand over the boy.' (B:ATI062)

\ex
\label{ex:13-162}
\gll eesé mehfil-í wée [rhoó dainií] ma  aamúuṣ-um de \\
\textsc{rem} gathering-\textsc{ob} in song give.\textsc{vn} \textsc{1sg.nom} forget-\textsc{1sg} \textsc{pst} \\
\glt `In that crowd I forgot to sing a~song.' (A:PHN5101.20) 
\end{exe}

Normally the \isi{Verbal Noun} \isi{complement} is embedded into the matrix \isi{clause}. The \isi{complement} in (\ref{ex:13-161}), however, occurs after the matrix \isi{clause} and is thus added explanatorily rather than being extraposed in a~strict sense.


As is clear from examples (\ref{ex:13-163})--(\ref{ex:13-165}), achievement (or non"=achievement) can also be expressed with \isi{complement} clauses of \isi{complement}"=taking adjectives.

\begin{exe}
\ex
\label{ex:13-163}
\gll [nis phus"=ainií] askóon \\
\textsc{3sg.prox.acc} rid-\textsc{vn} easy \\
\glt `It will be easy to get rid of him.' (A:UXB019)

\ex
\label{ex:13-164}
\gll ɡhrast-á the [kuhée díi nikh"=aníi] askáan de  \\
wolf-\textsc{ob} to well.\textsc{ob} from come.out-\textsc{vn} easy be.\textsc{pst}  \\
\glt `It was easy for the wolf to get out of the well.' (B:FOX031)

\ex
\label{ex:13-165}
\gll heewand-á tanaám the [akaadúi paš-ainií] naawás  na de \\
winter-\textsc{ob} \textsc{3pl.acc} to \textsc{recp} see-\textsc{vn} difficult \textsc{neg} be.\textsc{pst} \\
\glt `It was not hard for them to meet each other throughout the winter.' (A:SHY006) 
\end{exe}

\spitzmarke{Phasal predicates.} As was also pointed out above concerning modals, many phasal notions, having to do with termination and continuation, are indicated by means other than complementation, and probably mainly so. Inception, however, can be expressed with at least two different complementation strategies.


One of those strategies is infinitival (vi), whereby the \isi{perfective} form \textit{ṣáat-} of the polysemous verb \textit{ṣač-} `climb, etc.' is utilised (\ref{ex:13-166}). 

\begin{exe}
\ex
\label{ex:13-166}
\gll káakeṛ teeṇíi kúṛi-e pándee kaantar-í ba  [biabeen-íim ɡir-áa] ṣáat-u \\
Kakel \textsc{ refl} woman-\textsc{gn} cause go.mad-\textsc{cv} \textsc{prt}  wilderness-\textsc{pl.ob} turn-\textsc{inf} begin.\textsc{pfv"=msg} \\
\glt `Kakel went mad due to this with his wife and began wandering about in the wilderness.' (B:FOR020) 
\end{exe}

Unlike the \isi{Infinitive}"=taking modal \textit{bha-} (see \textbf{Modal predicates} above), \textit{ṣáat-} is confined to the \isi{perfective} and remains in\isi{transitive}, regardless of the transitivity of the \isi{complement} \isi{predicate}. In (\ref{ex:13-167}), with an~in\isi{transitive} \isi{complement}, as well as in (\ref{ex:13-168}), with a~\isi{transitive} \isi{complement}, the phasal verb agrees in \isi{gender} and number with the \isi{subject} of the matrix \isi{clause}.

\begin{exe}
\ex
\label{ex:13-167}
\gll ɣalí bhe [padúši padúši bay-áa] ṣéet-i  \\
silent become.\textsc{cv} behind behind go-\textsc{inf} begin.\textsc{pfv-f} \\
\glt `Silently she started to lag far behind.' (B:FOY012)

\ex
\label{ex:13-168}
\gll whaid-í ba [haṛé ziaarat-íim díi  muaaf-í daway-áa] ṣáat-u \\
fall.down-\textsc{cv} \textsc{prt} \textsc{dist} shrine-\textsc{pl.ob} from forgiveness-\textsc{pl} ask-\textsc{inf} begin.\textsc{pfv"=msg}  \\
\glt `Having fallen down, he started to beg forgiveness from those shrines.' (B:FOR039) 
\end{exe}

Although the construction is a~possible example of \isi{clause} union, \textit{ṣáat-} seems not to have morphologised to the same extent as \textit{bha-}, evidenced also in the parallel use of a~\isi{Verbal Noun} (\ref{ex:13-169}) with it instead of an~\isi{Infinitive}. 

\begin{exe}
\ex
\label{ex:13-169}
\gll ak čoór tesée ɡhooṣṭ-á ačíit-u. [ɡhoóṣṭ  laṭ-ainíi] ṣáat-u \\
\textsc{idef} thief \textsc{3sg.gn} house-\textsc{ob} enter.\textsc{pfv"=msg} house search-\textsc{vn} begin.\textsc{pfv"=msg} \\
\glt `A thief entered his house. He started to search through the house.' (B:THI002-3) 
\end{exe}

With another nearly synonymous phasal \isi{predicate} \textit{široó the-}, exemplified in (\ref{ex:13-170}), the \isi{Verbal Noun} strategy is the only possible one.

\begin{exe}
\ex
\label{ex:13-170}
\gll se ṭhaaṭáak-a bi [tas sanɡí khainií]  široó thíil-u \\
\textsc{def} monster-\textsc{ob} also \textsc{3sg.acc} with eat.\textsc{vn} start do.\textsc{pfv"=msg} \\
\glt `The monster also started to eat together with him.' (A:THA007) 
\end{exe}

There is evidence, although scanty, that the phasal notion of continuation can be expressed through two (for the \isi{purpose}) less usual strategies with the \isi{complement}"=taking \isi{predicate} \textit{dhar-} `remain', one being a~converb (\ref{ex:13-171}) and the other a~\isi{Copredicative Participle} (\ref{ex:13-172}).

\begin{exe}
\ex
\label{ex:13-171}
\gll míḍ=ee ɡúu [dac̣h-í] dharíit-a \\
ram=\textsc{cnj} bull look-\textsc{cv} remain.\textsc{pfv"=mpl} \\
\glt `The ram and the bull remained (=stood for a~long time) looking.' (B:SHI018)

\ex
\label{ex:13-172}
\gll [mée$\sim$ mée$\sim$ the-íim] dharíit-i  \\
baa baa do-\textsc{cprd} remain.\textsc{pfv-f} \\
\glt `She kept bleating.' (B:FOX022) 
\end{exe}

\spitzmarke{Desiderative predicates.} The \isi{Verbal Noun} strategy is also put to use in
\isi{subject}"=controlled complements of desiderative predicates, as in
(\ref{ex:13-173})--(\ref{ex:13-177}). The \isi{Verbal Noun} may also be inflected
(\ref{ex:13-175}) or followed by
a~\isi{postposition} (\ref{ex:13-177}).

\begin{exe}
\ex
\label{ex:13-173}
\gll se [har deés akaadúi paš-ainií] dawóon de \\
\textsc{3pl.nom} every day \textsc{recp} see-\textsc{vn} ask.\textsc{3pl} \textsc{pst} \\
\glt `They wanted to meet each other every day.' (A:SHY004)

\ex
\label{ex:13-174}
\gll eeṛé wáǰa ǰhulí alataalaá [tas dubaará dunia-í  the phray"=ainií] iraadá thíil-u \\
\textsc{dist} reason on Allah.Almighty \textsc{3sg.acc} again world-\textsc{ob}  to send-\textsc{vn} decision do.\textsc{pfv"=msg}  \\
\glt `Therefore, Allah Almighty decided to send him another time to the world.' (A:ABO030)

\ex
\label{ex:13-175}
\gll [neečíir theníi-a] díiš-a xalk-íim xwaaíš thíil-i \\
hunt do.\textsc{vn"=pl} village-\textsc{ob} people-\textsc{pl.ob} desire do.\textsc{pfv-f} \\
\glt `People in the village wanted to go hunting.' (B:AVA200) \\

\ex
\label{ex:13-176}
\gll [tas the kúṛi dawainií] bandubás th-áan-u \\
\textsc{3sg.acc} to woman ask.\textsc{vn}  arrangement do-\textsc{prs"=msg} \\
\glt `He arranges to have his son married.' (A:MAR004)

\ex
\label{ex:13-177}
\gll be [teeṇíi se neečíir bhaaɡ-aníi ǰhulí]  raazí bhíl-a \\
\textsc{1pl.nom} \textsc{refl} \textsc{def} hunt divide-\textsc{vn} on agreeable become.\textsc{pfv"=mpl} \\
\glt `We have agreed on the division of our game.' (B:FOY065) 
\end{exe}

Finally, as pointed out already, some utterance predicates, using the \textit{ki}-strategy, especially those concerning ``thinking'', can equally well be considered desiderative.


\subsection{Complement"=taking manipulation predicates}
\label{subsec:13-5-3}

\spitzmarke{Permissive predicates.} When the verb \textit{uṛí-} `pour, let out' is used as the \isi{complement}"=taking verb `let', it occurs in two different constructions, both with permissive semantics. One is the \isi{Verbal Noun} strategy ((v)), also occurring in various other \isi{complement} clauses (see \sectref{subsec:13-5-1}), followed by the \isi{postposition} \textit{the} `to', as shown in (\ref{ex:13-178}) and (\ref{ex:13-179}). This is used in the indicative.

\begin{exe}
\ex
\label{ex:13-178}
\gll uc̣hí ba se čúti-m-e zaríia baándi  so baṭ húṇṭraak c̣huɡal-áan-u [se kúuk-a se muṭ-á bheš-aníi the] na  uṛ-áan-u \textsc{[}mhaás
  khainíi the] na uṛ-áan-u \\
lift.up.\textsc{cv} \textsc{prt} \textsc{def} paw-\textsc{pl"=gn} means by  \textsc{def.msg.nom} stone upward hurl-\textsc{prs"=msg}  \textsc{def} crow-\textsc{pl} \textsc{def} tree-\textsc{ob} sit.down-\textsc{vn} to \textsc{neg} let-\textsc{prs"=msg} meat eat.\textsc{vn} to \textsc{neg} let-\textsc{prs"=msg} \\
\glt `After picking up he is throwing up the stone with the help of his paws, not letting the crows sit down in the tree, or eat the meat.' (B:SHB752-6)

\ex
\label{ex:13-179}
\gll ɡhaḍeeró miṭínɡ xátum bhíl-ii pahúrta [asaám  bainií the] uṛ-íi=ee \\
elder meeting finished become.\textsc{pptc"=gn} after \textsc{1pl.acc} go.\textsc{vn} to let-\textsc{3sg=q} \\
\glt `Will the elder let us leave after the meeting is over?' (A:Q6.29.07) 
\end{exe}

Note that the manipulee of the main verb is coreferent with the agent of the \isi{complement} verb but is assigned case by the main verb, thus coded accusatively in (\ref{ex:13-179}).


The other strategy, involving the particle \textit{ta} (iv), is used in the \isi{Imperative}, with the literal rendering `let x (out), and x will do y'. Whereas in the strategy above the \isi{complement clause} is embedded, the \isi{complement clause} in this construction, seen in (\ref{ex:13-180}), is S-like and extraposed. The main verb does not assign case to the manipulee/\isi{complement clause} agent in this construction.

\begin{exe}
\ex
\label{ex:13-180}
\gll ṛo uṛí ta [rhoó d-íi] \\
\textsc{3sg.dist.nom} let.\textsc{imp.sg} \textsc{ prt} song give-\textsc{3sg} \\
\glt `Let him sing!' (A:Q6.15.02) 
\end{exe}

\spitzmarke{Causative predicates.} While the most important means of expressing causativity is morphological~-- a~matter we will return to shortly~-- manipulation that involves a~lower degree of agentive control \citep[45]{givon2001b} is often expressed through what is basically utterance predicates, also using the same strategies. Examples are given in (\ref{ex:13-181})--(\ref{ex:13-183}). Hereby the manner of causation or persuation is also made explicit \citep[136]{noonan2007}. The \isi{complement clause} is either imperative or hortative.

\begin{exe}
\ex
\label{ex:13-181}
\gll tas the ṭeekíl-u ki [nikhá c̣híitr-a  kráam-a the báaya] thaní \\
\textsc{3sg.acc} to call.out.\textsc{pfv"=msg} \textsc{comp} come.out.\textsc{imp.sg} field-\textsc{ob}  work-\textsc{ob} to go.\textsc{1pl} \textsc{qt} \\
\glt `He called out to her to come out and go with him to work in the field.' (A:WOM638)

\ex
\label{ex:13-182}
\gll yheyí se kuṇaak-á the išaará thíil-u  ki [ma khúna yha] thaní \\
come.\textsc{cv} \textsc{def} child-\textsc{ob} to signal do.\textsc{pfv"=msg} \textsc{comp} \textsc{1sg.nom} near come.\textsc{imp.sg} \textsc{qt} \\
\glt `Coming there, he signalled to the child to come to him.' (A:BRE003)

\ex
\label{ex:13-183}
\gll deeúl-ii xálak"=am daawát dít-i  atsharíit"=am the ki [muqaabilá
  th-íia]  (thaníit-u ta) \\
Dir-\textsc{gn}  people-\textsc{pl.ob} invitation give.\textsc{pfv-f} Ashreti-\textsc{pl.ob} to \textsc{comp} contest do-\textsc{1pl} say.\textsc{pfv"=msg} \textsc{prt} \\
\glt `The people of Dir invited the Ashretis to compete with them.' (A:CHA001) 
\end{exe}

When, however, agent control is strong, the preferred strategy is to utilise the morphological \isi{causative} construction \textit{per se}. This can either be in the form of direct causation, in essence deriving a~\isi{transitive} verb from an~in\isi{transitive} (such as `make sit down, seat' from `sit down' in (\ref{ex:13-184}) and `hang (something)' from `hang (by itself)' in (\ref{ex:13-185})), or in the form of indirect causation (such as `have (someone) drink' from `drink' in (\ref{ex:13-185}), `have someone tell' from `tell something' in (\ref{ex:13-186}) and `have someone educated' from `educate (oneself)' in (\ref{ex:13-187})). With indirect causation, an~animate manipulee is marked by \textit{ṣaawaá}, a~grammaticalisation of the converb of \textit{ṣaawá-} `put on, dress, turn on'.

\begin{exe}
\ex
\label{ex:13-184}
\gll eesé zanɡal-í áa baṭ-á ǰhulí har-í  so kuṇaák bhešóol-u \\
\textsc{3sg.rem.ob} forest-\textsc{ob} \textsc{idef} stone-\textsc{ob} on take.away-\textsc{cv} \textsc{def.msg.nom} child sit.down.\textsc{caus}.\textsc{pfv"=msg} \\
\glt `He took him to a~stone in the forest and put him down (i.e., had him sit).' (A:BRE005)

\ex
\label{ex:13-185}
\gll heewand-á tas the róot-a c̣hóoṇ  lam"=a-áan-a dees-á har-í wíi
pil"=a-áan-a tas šišáwi the šúwi zhay-íim  ɡhin-í ɡir-áan-a \\
winter-\textsc{ob}  \textsc{3sg.acc}  to night-\textsc{ob} oak.branches  hang-\textsc{caus}-\textsc{prs"=mpl} day-\textsc{ob} take.away-\textsc{cv} water  drink-\textsc{caus}-\textsc{prs"=mpl} \textsc{3sg.acc} beautiful.\textsc{f} do.\textsc{cv} good.\textsc{f} place-\textsc{pl.ob} take-\textsc{cv} turn-\textsc{prs"=mpl} \\
\glt `In the winter we hang oak branches for her [the goat] during the night, and in the day we take her to drink plenty of water in beautiful places.' (A:KEE091)

\ex
\label{ex:13-186}
\gll xalk-íim ṣaawaá teér bhíl-a qiseé  tha-wóol-a \\
people-\textsc{pl.ob} \textsc{manip} passed become.\textsc{pptc"=mpl} story.\textsc{pl} do-\textsc{caus}.\textsc{pfv"=mpl} \\
\glt `He had the people tell him all that had happened.' (A:UXW057)

\ex
\label{ex:13-187}
\gll áa putr-á ṣaawaá ma taalím tha-wa-áan-u  \\
\textsc{idef} son-\textsc{ob} \textsc{manip} \textsc{1sg.nom} education do-\textsc{caus}-\textsc{prs"=msg} \\
\glt `I'm having one of my sons educated' (A:KEE082) 
\end{exe}

In this construction, \isi{clause} union is complete, and we can also define it as a~total lexical union \citep[86]{noonan2007} or co"=lexicalisation of the ``cause'' and the ``caused event''. Note that this is therefore not in a~strict sense an~instance of complementation. It is only included here for comparative reasons.


\section{Relative clauses}
\label{sec:13-6}

Relativisation in Palula offers an~unusual analytical challenge, partly due to the many different strategies that seem to be available, partly due to their relative scarcity in natural \isi{discourse}, whether spoken or written. 



Generally speaking, the unmarked \isi{relative clause} (or its functional equivalent) is preposed to the modified NP as well as the main \isi{clause} in its entirety. Although there is a~preference for co"=relative (also referred to as relative"=correlative) constructions, as described by \citet{downing1974}, most of them are not of the kind typical of major IA languages \citep[410--415]{masica1991}, which is why I have chosen not to use the term other than in comparison with the more ``typical'' IA pattern. Neither does Palula share the general \ili{Shina} preference for \isi{participial} relative clauses (Carla Radloff, pc), possibly a~result of the synchronic non"=distinctiveness in Palula between many participles and the \isi{finite} verb forms historically derived from the former.



There is no attempt on my side to reflect the relative frequency of occurrence by the order with which the various strategies and types of relativisation are outlined below. The number of examples of relative clauses is basically too low to present any reliable statistics. Extraposition with \textit{ki} (\sectref{subsec:13-6-7}) is not strictly a~separate type of \isi{relative clause} but rather an~additional, but nevertheless frequently used, strategy, primarily related to presentational structure and other pragmatic functions.



Restrictive and non"=restrictive relative clauses overlap to a~large extent, although it seems that extraposed clauses are favoured with non"=restrictive clauses.


\subsection{Relative clauses with a~full NP}
\label{subsec:13-6-1}


Most similar to the typical \ili{Urdu}"=Hindi relative"=correlative construction are those sentences that have a~full NP in a~preposed modifying \isi{clause} corresponding to a~full NP in the main \isi{clause}. In (\ref{ex:13-188})--(\ref{ex:13-189}), the otherwise interrogatively used \textit{khayí} `which' is used as an~adnominal \isi{relative pronoun}, correlating with a~\isi{demonstrative} from the \isi{remote} set. Although the \isi{noun} can be repeated in full in the main \isi{clause} along with a~\isi{demonstrative}, it can equally well be coreferenced by a~\isi{demonstrative} alone.



Whereas the modifying \isi{clause} is dependent on the main \isi{clause} for its interpretation, the main (indicative and \isi{finite}) \isi{clause} can in most cases stand alone as a~complete sentence. The modifying (relative) \isi{clause} occurring on its own (which is not always possible due to a~non"=basic word order), however, would be interpretable as a~content question rather than as a~statement. Following \citet[182]{givon2001b}, a~construction like this is essentially paratactic, the linking device between the two clauses being \isi{deictic} rather than a~matter of subordination or embedding.

\begin{exe}
\ex
\label{ex:13-188}
\gll [báaba khayí xarčá dawóol-u hín-u]  eesó xarčá ɡhin-í{\ldots}  \\
father.\textsc{ob} which expenditure ask.for.\textsc{pfv"=msg} be.\textsc{prs"=msg} \textsc{rem.msg.nom} expenditure take-\textsc{cv} \\
\glt `Taking the reimbursement, which the father has demanded{\ldots}' (A:MAR037-8)

\ex
\label{ex:13-189}
\gll [se baaqimaandá mehmaan-aán khayí hín-a  práač-a khayí hín-a]
eetanaám the ɡúuli d-áan-a \\
\textsc{def} additional guest-\textsc{pl} which be.\textsc{prs"=mpl} guest-\textsc{pl} which be.\textsc{prs"=mpl} \textsc{3pl.acc} to bread give-\textsc{prs"=mpl} \\
\glt `We feed those other guests who are present.' (A:MIT024) 
\end{exe}

Some modifying clauses with adnominal \textit{khayí} are possibly embedded rather than preposed, as in (\ref{ex:13-190}), where we may further suggest that a~\isi{noun} \textit{xálak} `people' is the implicit head. However, the analysis is still pending in anticipation of more natural data of a~similar kind. It should also be noted that a~\isi{pronoun} \textit{ni} `they (here)' from the \isi{proximate} set is being used and not one from the usual \isi{remote} set.

\begin{exe}
\ex
\label{ex:13-190}
\gll tipúka ni ba khayí hín-a  dhii-á díi khooǰainií zarurí bh-áan-u \\
nowadays \textsc{3pl.prox.nom} \textsc{prt} which be.\textsc{prs"=mpl}  daughter-\textsc{ob} from ask.\textsc{vn} necessity become-\textsc{prs"=msg} \\
\glt `Nowadays there are those, for whom it is necessary to ask their daughter [who she wants to be married to]' (A:MAR019) 
\end{exe}

In (\ref{ex:13-191}), the \isi{relative clause} is probably internal. Note that the construction as a~whole is extraposed to the temporal \isi{clause}.

\begin{exe}
\ex
\label{ex:13-191}
\gll paturaá nuuṭíl-u ta [khayí zhay-í wée] asím  ǰinaazá khaṣeel-í wheelíl-u de \\
back return.\textsc{pfv"=msg} \textsc{prt} which place-\textsc{ob} in \textsc{1pl.erg} corpse drag-\textsc{cv}  bring.down.\textsc{pfv"=msg} \textsc{pst}   \\
\glt `When I returned to the place to where we had dragged him down{\ldots}' (A:GHA044)
\end{exe}

\subsection{Inde\isi{finite}"=conditional relative clauses}
\label{subsec:13-6-2}

Quite similar to, and possibly not altogether distinct from, the relative clauses above are those that contain an~\isi{interrogative} \isi{pronoun} in combination with (or even fused with) an~inde\isi{finite} word \textit{ɡalá} \textit{(B. ɡalé)} meaning `ever'. The resulting meaning is something like `whoever, whatever', referring to one in a~group of potential referents having a~non"=specific reference. The \isi{relative clause} is often a~\isi{Conditional} \isi{clause}, and is therefore more clearly subordinate than the above described construction, and the correlative in the main \isi{clause} is referred to with a~\isi{demonstrative} or personal \isi{pronoun} from the \isi{remote} set (if at all explicit). Examples are given in (\ref{ex:13-192})--(\ref{ex:13-194}).

\begin{exe}
\ex
\label{ex:13-192}
\gll [kasée ɡalé maáṭu dhraǰaá ba šúul-a the  phedíl-u heentá] hasó kh-úu \\
who.\textsc{gn} ever neck stretch.\textsc{cv} \textsc{prt} sheaf-\textsc{ob} to  reach.\textsc{pfv"=msg} \textsc{condl} \textsc{3msg.rem.nom} eat-\textsc{3sg} \\
\glt `The one (of us) whose neck can reach the sheaf shall eat.' (B:SHI016)

\ex
\label{ex:13-193}
\gll [anú íṇc̣-a sanɡí mháala kií ɡalá ɡhašíl-i  heentá] ma tas
the páanǰ sóo rupeé baxšíš  d-áan-u \\
\textsc{prox.msg.nom}{\protect\footnotemark} bear-\textsc{ob} with wrestling who.\textsc{ob} ever grab.\textsc{pfv-f} \textsc{condl} \textsc{1sg.nom} \textsc{3sg.acc} to five hundred rupee.\textsc{pl} reward give-\textsc{prs"=msg} \\
\glt `I will give 500 rupees to whoever wrestles with this bear' (A:BEW003)

\ex
\label{ex:13-194}
\gll [ṣiúul-a ɡubáa ɡalé maníit-u ta] lhooméi kaṇ  th-íi de \\
jackal-\textsc{ob} what ever say.\textsc{pfv"=msg} \textsc{prt} fox \textsc{host} do-\textsc{3sg} \textsc{pst} \\
\glt `Whatever the jackal told him, the fox gave heed to.' (B:FOY018) 
\end{exe}

\footnotetext{The form of the \isi{determiner} is not the expected one, since the head is non"=\isi{nominative}.}


The relativised NP can thus be a~possessor, an~\isi{ergative} \isi{subject} or a~\isi{direct object}. It does not even need to be a~\isi{noun} \isi{phrase} thus being relativised; the same basic construction is used in some temporal clauses with \textit{kareé ɡalá} `whenever' (see \sectref{subsec:13-4-1}), then always without an~explicit correlative.


When the \isi{noun} \isi{phrase} being relativised is a~\isi{nominative} \isi{subject}, as in (\ref{ex:13-195}), \textit{ɡalá} does not seem to be necessary to include. 

\begin{exe}
\ex
\label{ex:13-195}
\gll [koó eesé ṭeem-íi haazír na heensíl-u  heentá] tasíi mux-íi nikh"=aaṇḍeéu bh-áan-u \\
who.\textsc{nom} \textsc{dist} time-\textsc{gn} presence \textsc{neg} stay.\textsc{pfv"=msg}  \textsc{condl } \textsc{3sg.gn} face-\textsc{gn} come.out-\textsc{oblg} become-\textsc{prs"=msg} \\
\glt `The one [of them] who was not present at that time, will have to come [later] and greet./If one of them was not present at that time, he will have to come [later] and greet.' (A:MAR048-9)
\end{exe}

The same construction (without \textit{ɡalá}) can also be used with manner adverbials, as exemplified in (\ref{ex:13-195b}).

\begin{exe}
\ex
\label{ex:13-195b}
\gll [kanáa húuši ziaát bh-íi de] eendáa so musaafár šukhaáu teeṇíi huǰut-í pharé pail-óo de\\
like.what wind most become-\textsc{3sg} \textsc{pst} like.that \textsc{def.msg.nom} traveller coat \textsc{refl} body-\textsc{ob} toward fold-\textsc{3sg} \textsc{pst} \\
\glt `The stronger the wind blew, the more closely did the traveller fold his cloak around him.' (A:NOR006)
\end{exe}


\subsection{Gapped relative clauses}
\label{subsec:13-6-3}


What seems to be a~major strategy for relativisation is a~preposed gapped modifying \isi{clause}, i.e., one in which the relativised NP is not explicitly present at all. Even here the correlative NP in the main \isi{clause} often occurs with a~\isi{demonstrative} from the \isi{remote} set, thus in essence being of the same kind as the relative"=correlative construction we have seen examples of already.


The modifying \isi{clause} may be indicative \isi{finite}, as in examples (\ref{ex:13-196})--(\ref{ex:13-199}), with the positions \isi{direct object}, \isi{oblique} object coding source and \isi{transitive} \isi{subject}, as well as the \isi{oblique} object coding recipient being relativised.

\begin{exe}
\ex
\label{ex:13-196}
\gll [míi ɡhíin-i] eesé paaṇṭí tíi qabúl  thíil-i \\
\textsc{1sg.gn} take.\textsc{pfv-f} \textsc{rem} clothes \textsc{3sg.ob} acceptance do.\textsc{pfv-f} \\
\glt `He accepted the clothes that I had bought.' (A:Q9.0016)

\ex
\label{ex:13-197}
\gll [míi kitaáb ɡhíin-i de] eesó phoó  míi pitríi putr \\
\textsc{1sg.gn} book take.\textsc{pfv-f} \textsc{pst} \textsc{rem.msg.nom} boy  \textsc{1sg.gn} father's.brother.\textsc{gn} son \\
\glt `The boy I got this book from is my cousin.' (A:HLE3054)

\ex
\label{ex:13-198}
\gll [na ǰéel-i] eetás the ba  waarɣaléti man-áan-a \\
\textsc{neg} give.birth.\textsc{pfv-f} \textsc{3sg.rem.acc} to \textsc{prt}  ``warghaleti'' say-\textsc{prs"=mpl} \\
\glt `One, which hasn't given birth, we call ``warghaleti''.' (A:KEE104)

\ex
\label{ex:13-199}
\gll [míi teeṇíi kili-á dít-i de] eesó  phoó phus bhíl-u hín-u  \\
\textsc{1sg.gn} \textsc{refl} key-\textsc{pl} give.\textsc{pfv-f} \textsc{pst} \textsc{rem.msg.nom}  boy diappeared become.\textsc{pfv"=msg} be.\textsc{prs"=msg} \\
\glt `The boy whom I gave my keys has disappeared.' (A:HLE3053) 
\end{exe}

It may also occur as a~preposed \isi{Conditional} \isi{clause}, in (\ref{ex:13-200}) relativising the possessor NP.

\begin{exe}
\ex
\label{ex:13-200}
\gll [áaṛu heensíl-u heentá] eesó ḍhínɡar aaindá  dapáara nakaám bh-áan-u  \\
knot stay.\textsc{pfv"=msg} \textsc{condl} \textsc{rem.msg.nom} timber future  for failed become-\textsc{prs"=msg} \\
\glt `Timber, which has knots will be worthless in the future/If it is knotty, the timber will fail in the future.' (A:HOW032) 
\end{exe}

Another possibility, seen in (\ref{ex:13-201}) and (\ref{ex:13-202}), is to use preposed dependent \isi{Converb} clauses for relativisation.

\begin{exe}
\ex
\label{ex:13-201}
\gll [preeṣaá ba] eetás the ta man-áan-a  kaašméeṇi \\
squeeze.\textsc{cv} \textsc{prt} \textsc{3sg.rem.acc} to \textsc{prt} say-\textsc{prs"=mpl}  ``kashmeeni'' \\
\glt `That, which is squeezed out, is called ``kashmeeni''' (A:KEE055)

\ex
\label{ex:13-202}
\gll [tas mheer-í ɡal-í] zaalim"=aan-óom dhút-a  pharé ɡúuli bi de ɡíia de \\
\textsc{3sg.acc} kill-\textsc{cv} throw-\textsc{cv} brute-\textsc{pl"=ob} mouth-\textsc{ob}  toward bread also put.\textsc{cv} go.\textsc{pfv.pl} \textsc{pst} \\
\glt `The brutes, who had killed him, had also put bread in his mouth and left.' (A:GHA076-7) 
\end{exe}

A \isi{reflexive pronoun}, as in (\ref{ex:13-203}), can also occur in the modifying \isi{clause}. 

\begin{exe}
\ex
\label{ex:13-203}
\gll [teṇ teeṇíi násel pašaá ba] muxáak hasó  tshaak-íi \\
\textsc{red} \textsc{refl} descent show.\textsc{cv} \textsc{prt} before \textsc{3msg.rem.nom}  taste-\textsc{3sg} \\
\glt `The one who is able to prove his noble heritage may taste first.' (B:SHI006) 
\end{exe}

Some constructions that technically are relativisations with a~preposed \isi{Converb} \isi{clause} are lexicalised to a~large extent, and function as standard presentational constructions when new participants or places are being introduced into a~\isi{discourse}. Note in (\ref{ex:13-204}) that the coreferential \isi{noun} phrases in the main \isi{clause} are inde\isi{finite}.

\begin{exe}
\ex
\label{ex:13-204}
\gll muxáak zamanée [čakaaḍhám than-í] ak díiš-a  [kaakeṛ than-í] ak méeš heensíl-u de  \\
before time.\textsc{gn} Chakadham call-\textsc{cv} \textsc{idef} village-\textsc{ob}  Kakel call-\textsc{cv} \textsc{idef} man stay.\textsc{pfv"=msg} \textsc{pst} \\
\glt `Once there lived in the village Chakadham a~man called Kakel.' (B:FOR001)
\end{exe}

\subsection{Gapped relative clauses with a~complementiser}
\label{subsec:13-6-4}


Sometimes in gapped relative clauses, a~particle \textit{ɡa} is used as a~\isi{complementiser} or relativiser, otherwise occurring as an~inde\isi{finite} (`some', `any') or an~adnominal \isi{interrogative} (`what', `what kind of') \isi{pronoun}. This particle or word is invariable and thus is not coreferential with the modified NP. Instead it seems to merely mark the modifying \isi{clause} vis-à-vis the main \isi{clause}. Examples are given in (\ref{ex:13-205})--(\ref{ex:13-208}).


There does not seem to be any significant covariation between the presence of this marker and any particular position being relativised, although it possibly is used more readily (and maybe even non"=optionally) when an~\isi{oblique} constituent is relativised, such as the location in (\ref{ex:13-207}) and (\ref{ex:13-208}). 

\begin{exe}
\ex
\label{ex:13-205}
\gll [aṛé íṇc̣-a ǰhulí ɡa hín-u] eeṛó  míiš=ee \\
\textsc{dist} bear-\textsc{ob} on \textsc{rel} be.\textsc{prs"=msg} \textsc{dist.msg.nom}   man=\textsc{q} \\
\glt `Is it the man who is sitting on the bear?' (A:BEZ012)

\ex
\label{ex:13-206}
\gll [míi dhií ɡa saat-éen-i] eení kúṛi  aní buṭheé šaak-á aṭíl-a \\
\textsc{1sg.gn} daughter \textsc{rel} take.care.of-\textsc{prs-f} \textsc{prox} woman \textsc{prox} all wood-\textsc{pl} bring.\textsc{pfv"=mpl} \\
\glt `This woman, who is taking care of my daughter, brought all this firewood' (A:REQ17)

\ex
\label{ex:13-207}
\gll [máa=ee tu ɡa bheš-í hín-a]  aní ɡhooṣṭ-á šíiṭi \\
\textsc{1sg.nom=cnj} \textsc{2sg.nom} \textsc{rel} sit.down-\textsc{cv} be.\textsc{prs"=mpl} \textsc{prox} house-\textsc{ob} inside  \\
\glt `In the house where you and I are sitting...' (A:HUA014)

\ex
\label{ex:13-208}
\gll \label{bkm:Ref190845659}[henrík de ɡa ɡíia] eeṛáa rhalá ba  qilaá de \\
Henrik put.\textsc{cv} \textsc{rel} go.\textsc{pfv.pl}  there above \textsc{prt} fort be.\textsc{pst } \\
\glt `Above [the place] where you took Henrik, there was a~fort.' (A:CAV004) 
\end{exe}

The position of \textit{ɡa} is probably preverbal, but further investigation is needed to say with more certainty. In (\ref{ex:13-209}), the modifying \isi{clause} is postposed, and here \textit{ɡa} occurs \isi{clause}"=initially.

\begin{exe}
\ex
\label{ex:13-209}
\gll ée kučúru [ɡa so wíi ǰhuṭá thíil-u]  \\
oh! dog \textsc{rel} \textsc{def.msg.nom} water dirty do.\textsc{pfv"=msg} \\
\glt `It must be the dog that has contaminated the water' (A:PAS038) 
\end{exe}

In (\ref{ex:13-210}), the modifying string `who used to walk with the millstone' seems to function simultaneously as a~postposed non"=restrictive \isi{relative clause} to `the \ili{Dameli} strong man' and as a~preposed restrictive relative"=cor\isi{relative clause} to `that (one) didn't take up the challenge'.

\begin{exe}
\ex
\label{ex:13-210}
\gll so ɡiḍúuču paalawaáṇ [so yambaáṭ  ɡa ɡhin-í b-íi de] eesó na  ṣandíl-u maní \\
\textsc{def.msg.nom} \ili{Dameli}.man strong.man \textsc{def} mill.stone \textsc{rel} take-\textsc{cv} go-\textsc{3sg} \textsc{pst} \textsc{3msg.rem.nom} \textsc{neg} take.up.challenge.\textsc{pfv"=msg} \textsc{hsay} \\
\glt `The strong man of Damel, who used to walk with the millstone, apparently didn't take up the
challenge' (A:MAH054)
\end{exe}

The modifying \isi{clause} in (\ref{ex:13-211}), beginning with \textit{ɡa}, takes as a~whole the place of the \isi{transitive} \isi{subject}. Tentatively it is analysed as a (headless) internal \isi{relative clause}.

\begin{exe}
\ex
\label{ex:13-211}
\gll bhunaṛíi wée [ɡa šáali ṭópa dít-i]  šáali khéel-i \\
down.below.\textsc{ob} in \textsc{rel} Shali downside give.\textsc{pfv-f} Shali eat.\textsc{pfv-f} \\
\glt `Down there, the one that caught Shali ate her.' (A:PAS059)
\end{exe}

\subsection{Pronominal relative clauses}
\label{subsec:13-6-5}


Some data (from B., and mostly elicited: (\ref{ex:13-212})--(\ref{ex:13-214})) suggests the possibility of regular \isi{anaphoric} (personal or \isi{demonstrative}) pronouns being used in the modifying \isi{clause}, whether preposed or postposed. The evidence, however, is too scanty to draw any de\isi{finite} conclusions. Suffice to say that the positions being thus relativised are low on the accessibility hierarchy (postpositional object and possessor, respectively), which is fully in line with typological predictions (\citealt[147--148]{keenan1985}; \citealt[226]{andrews_relative2007}). On the other hand, we have examples of relativisation of recipient (\ref{ex:13-199}) as well as possessor (\ref{ex:13-200}) by gapping (although from A.).

\begin{exe}
\ex
\label{ex:13-212}
\gll anú [dhoóṛ míi putr-á hanís the  kitaáb dít-i] hasó phoó  \\
\textsc{3msg.prox.nom} yesterday \textsc{1sg.gn} son-\textsc{ob} \textsc{3sg.prox.acc} to  book give.\textsc{pfv-f} \textsc{rem.msg.nom} boy \\
\glt `This is the boy who my son gave a~book to yesterday.' (B:DHE5367)

\ex
\label{ex:13-213}
\gll anú míi doóst [anisée ɡhooṣṭ-á the  dhoóṛ anú méeš ɡáu] \\
\textsc{3msg.prox.nom} \textsc{1sg.gn} friend \textsc{3sg.prox.gn} house"=ob to  yesterday \textsc{prox.msg.nom} man go.\textsc{pfv.msg} \\
\glt `This is my friend, whose house this man went to yesterday.' (B:DHE5373)

\ex
\label{ex:13-214}
\gll asím [so méeš ɡhašiǰíl-u baazúur-a  phará] so kuṇaák ba atesée
ɡhooṣṭ-á  the phreyíl-u \\
\textsc{1pl.erg} \textsc{def.msg.nom} man be.caught.\textsc{pfv"=msg} bazaar-\textsc{ob}  along \textsc{ def.msg.nom} child \textsc{prt} \textsc{3sg.rem.gn} house-\textsc{ob} to send.\textsc{pfv"=msg} \\
\glt `We sent the child to the house of the man who had been caught in the bazaar.' (B:ANG018) 
\end{exe}

Pronominals used in the modifying \isi{clause} are otherwise, as we shall see, the normal case with so"=called extraposed \textit{ki}-constructions (see \sectref{subsec:13-6-7}). 


\subsection{Nominalisation and the use of participials}
\label{subsec:13-6-6}

A less"=accessible \isi{noun} \isi{phrase} may be expressed through \isi{nominalisation}, whereby a~\isi{Verbal Noun} becomes the possessor of the relativised entity. This seems to be especially common with time, location and means, as seen in (\ref{ex:13-215})--(\ref{ex:13-218}). The \isi{genitive} case of the \isi{Verbal Noun} is only explicitly present in the B. variety (\textit{-e}), whereas the \isi{genitive} has been levelled in A.

\begin{exe}
\ex
\label{ex:13-215}
\gll rhootašíi-a [ǰhambréeṛi har"=ainií] waxt  yhóol-u ta \\
morning-\textsc{ob} bride take.away-\textsc{vn} time come.\textsc{pfv"=msg} \textsc{prt} \\
\glt `In the morning, at the time of taking the bride{\ldots}' (A:GHU010)

\ex
\label{ex:13-216}
\gll [máa=the bašéš deníi-e] zhay-í so  ma mhaar-áan-u \\
\textsc{1sg.nom}=to reward give.\textsc{inf"=gn} place-\textsc{ob} \textsc{3msg.nom} \textsc{1sg.nom} kill-\textsc{prs"=msg} \\
\glt `He is going to kill me instead of rewarding me.' (B:DRB020)

\ex
\label{ex:13-217}
\gll [bhíiru uṛ-ainií] tartíb ba eeṛé  \\
he.goat let.out-\textsc{inf} method \textsc{prt} \textsc{dist} \\
\glt `The method of letting out the he"=goat [to the she"=goats] is the following...' (A:KEE073)

\ex
\label{ex:13-218}
\gll ma tu the ɣeer-í šíiṭi be [neečíir  aṭaníi-e] baát d-áam \\
\textsc{1sg.nom} \textsc{2sg.nom} to cave-\textsc{ob} inside go.\textsc{cv} game  bring.\textsc{inf"=gn} talk give-\textsc{1sg} \\
\glt `Once we have got into the cave, I will tell you about the catching of game.' (B:FOY048) 
\end{exe}

Less commonly, Perfective Participials may be used attributively, as in (\ref{ex:13-219})--(\ref{ex:13-221}), and in those cases relativise the \isi{direct object}. As an~additional means to distinguish the \isi{participial} from the formally identical \isi{finite} \isi{perfective} verb, an~explicit (but optional) \isi{participial} marker (also referred to as an~adjectiviser in this work), such as \textit{bhaáu} in (\ref{ex:13-221}), can be used.

\begin{exe}
\ex
\label{ex:13-219}
\gll phaí [teeṇíi háat"=am čooṇṭéel-i] rumiaál dít-i  hín-i \\
girl \textsc{refl} hand-\textsc{ins} embroider.\textsc{pptc-f} handkerchief give.\textsc{pfv-f} be.\textsc{prs-f} \\
\glt `The girl gave [him] the handkerchief which she herself had embroidered.' (A:SHY031)

\ex
\label{ex:13-220}
\gll xalk-íim ṣaawaá [teér bhíl-a] qiseé  thawóol-a \\
people-\textsc{pl.ob} \textsc{manip} passed become.\textsc{pptc"=mpl} story.\textsc{pl} make.do.\textsc{pfv"=mpl} \\
\glt `He had the people tell him all that had happened.' (A:UXW057)

\ex
\label{ex:13-221}
\gll amzarái [muṛ-u-bhaáu] insaán na  kha-áan-u \\
lion die.\textsc{pptc"=msg"=adj} human.being \textsc{neg} eat-\textsc{prs"=msg} \\
\glt `A lion doesn't eat a~human being which has died.' (A:UNF012) 
\end{exe}

A corresponding (headless) construction for relativising an~agent \isi{subject}, is by using the Agentive \isi{Verbal Noun}, as seen in (\ref{ex:13-222}), but neither does that seem to be a~very common strategy. 

\begin{exe}
\ex
\label{ex:13-222}
\gll [teewiz-í th-áaṭ-u] [ṛe pairaán ɡaḍ-í  ṣeeka-áaṭ-u] \\
amulet-\textsc{pl} do-\textsc{ag"=msg} \textsc{dist} djinn.\textsc{pl} take.out-\textsc{cv} lead.out-\textsc{ag"=msg} \\
\glt `He was an~expert in making amulets and a~saviour from djinns.' (A:HUA129) 
\end{exe}

Alternatively, the present"=\isi{tense} verb is itself used without an~overt head, as in (\ref{ex:13-223}). This has an~implicit habituality read into it, whereas the corresponding non"=habitual meaning would have to be expressed with an~explicit correlative head (\ref{ex:13-224}).

\begin{exe}
\ex
\label{ex:13-223}
\gll [šaak-á aṭ-áan-u] čiiríit-u  \\
wood-\textsc{pl} bring-\textsc{prs"=msg} be.delayed.\textsc{pfv"=msg} \\
\glt `The person who [usually] brings the wood has been delayed' (A:HLE3051)

\ex
\label{ex:13-224}
\gll [šaak-á aṭ-áan-u] eesó phoó čiiríit-u  \\
wood-\textsc{pl} bring-\textsc{prs"=msg} \textsc{rem.ms}\textsc{g.}\textsc{nom} boy be.delayed.\textsc{pfv"=msg}  \\
\glt `The boy who is bringing the wood has been delayed.' (A:HLE3051)
\end{exe}

\subsection{Extraposed \textit{ki}-constructions}
\label{subsec:13-6-7}

The possibility of extraposing a~modifying \isi{clause} is facilitated by the particle or \isi{complementiser} \textit{ki} (see \sectref{subsec:13-5-1}). The sentences in (\ref{ex:13-225})--(\ref{ex:13-227}) are all elicited, and are thus not necessarily the most natural way of expressing these ideas. However, regarding them as a~general indication of grammaticality, we can conclude that the construction works for a~number of different positions. Note also that the emphatic or \isi{demonstrative} \isi{pronoun} is used in the main \isi{clause}, whereas a~non"=emphatic \isi{anaphoric} \isi{pronoun} is used in the modifying \textit{ki-}\isi{clause}, although the latter is optional for the \isi{subject} position.

\begin{exe}
\ex
\label{ex:13-225}
\gll anú eesó míiš ki [níi  se kúṛi mheeríl-i] \\
\textsc{3msg.prox.nom} \textsc{rem.msg.nom} man \textsc{comp} \textsc{3sg.prox.ob} \textsc{def} woman kill.\textsc{pfv-f} \\
\glt `This is the man who killed the woman.' (A:HLE2617)

\ex
\label{ex:13-226}
\gll anú eesó míiš ki [nisíi  ɡhooṣṭ-á ma hín-u] \\
\textsc{3msg.prox.nom} \textsc{rem.msg.nom} man \textsc{comp} \textsc{3sg.prox.gn} house-\textsc{ob} \textsc{1sg.nom} be.\textsc{prs"=msg} \\
\glt `This is the man whose house I am [living] in.' (A:HLE2618)

\ex
\label{ex:13-227}
\gll anú eesó míiš ki [(nu)  dhoóṛ yhóol-u de] \\
\textsc{3msg.prox.nom} \textsc{rem.msg.nom} man \textsc{comp} \textsc{3msg.prox.nom} yesterday come.\textsc{pfv"=msg} \textsc{pst} \\
\glt `This is the man who came yesterday.' (A:HLE2618) 
\end{exe}

It seems that the construction is particularly favoured in presentational \isi{discourse}, and a~number of correlative expressions, belonging to various parts of speech, can be linked to an~extraposed \textit{ki}-\isi{clause}, expressing a~variety of pragmatic- or \isi{discourse}"=related functions (some of them touched upon when discussing clauses with adverbial functions (\sectref{sec:13-4}) and complementation (\sectref{sec:13-5})). In most cases, however, the \textit{ki}-\isi{clause} as a~whole corresponds to the correlative in the main \isi{clause}, rather than to a~particular relative word in it.

\clearpage
\spitzmarke{Nominal.} Primarily heavy"=shifting:

\begin{exe}
\ex
\label{ex:13-228}
\gll ṣíiṛ-u khúṭu ba eetás the  man-áan-a ki [trók-i čhéeli trók-u čhaál] \\
blind-\textsc{msg} knee \textsc{prt} \textsc{3sg.rem.acc} to say-\textsc{prs"=mpl} \textsc{comp} weak-\textsc{f} goat weak-\textsc{msg} kid \\
\glt ` ``A blind knee'' is what we call a~goat or a~goat kid which is weak.' (A:PAS068)

\ex
\label{ex:13-229}
\gll tíi eeṛé baát dít-i ki  \\
\textsc{3sg.ob} \textsc{dist} talk give.\textsc{pfv-f} \textsc{comp}  \\
\glt `He said that{\ldots} [lit. His gave this talk]' (A:BEW003)

\ex
\label{ex:13-230}
\gll íṇc̣i-e xiaál haṛé bhíl-i  ki [karáaṛu máa=the uṭik-í de hín-u thaní] \\
bear-\textsc{gn} opinion \textsc{dist} become.\textsc{pfv-f} \textsc{comp} leopard \textsc{1sg.nom=}to jump-\textsc{cv} give.\textsc{cv} be.\textsc{prs"=msg} \textsc{qt} \\
\glt `The bear thought that the leopard had attacked him.' (B:BEL320) 
\end{exe}

\spitzmarke{Adjectival.} Comparison, exemplification or specification:

\begin{exe}
\ex
\label{ex:13-231}
\gll anú míiš eetí maldaár ki [nisíi  ɡhooṣṭ-á čailúṭii c̣hiír bi lhayiǰ-áan-u] \\
\textsc{prox.msg.nom} man so.much wealthy \textsc{comp} \textsc{3sg.prox.gn }  house-\textsc{ob} sparrow.\textsc{gn} milk also be.found-\textsc{prs"=msg} \\
\glt `This man is so rich that you even find sparrow's milk in his house.' (A:DHN6691)

\ex
\label{ex:13-232}
\gll dhríɡ-u ba eetí ki [loomuṭ-íi aḍaphaár  tií phed-í hín-u] \\
tall-\textsc{msg} \textsc{prt} so.much \textsc{comp} deodar-\textsc{gn} halfways up.to reach-\textsc{cv} be.\textsc{prs"=msg} \\
\glt `It was so tall, that it reached halfways up the length of a~deodar tree.' (A:HUA076)

\ex
\label{ex:13-233}
\gll baazí xálak-a hateeṇ-á hín-a ki  [se bakareelí th-áan-a]  \\
a.few people-\textsc{pl}  like.that-\textsc{mpl} be.\textsc{prs"=mpl} \textsc{comp}  \textsc{3pl.nom}  shepherding do-\textsc{prs"=mpl} \\
\glt `A few people are engaged in goat and shepherding.' (B:DHN5263)

\ex
\label{ex:13-234}
\gll áak eeteeṇ-ú waxt yhóol-u ki  [atshareet-á wée xálak biǰóol-a bh-íl-a] \\
\textsc{idef} like.that-\textsc{msg} time come.\textsc{pfv"=msg} \textsc{comp} Ashret-\textsc{ob} in people several-\textsc{mpl} become-\textsc{pfv"=mpl} \\
\glt `This was the time when the population in Ashret had started to increase.' (A:GHU001) 
\end{exe}

Although not examples of relativisation in a~strict sense, it should be noted that the same construction is used with \isi{clause} adverbials (as has already been illustrated many times before). In (\ref{ex:13-235}) the \textit{ki}-\isi{clause} corresponds to a~reason, and in (\ref{ex:13-236}), it serves as an~explanation.

\begin{exe}
\ex
\label{ex:13-235}
\gll súun-a hatáwuu pándee haans-áan-a ki  [béeriṣ-a súun-a bakáara
  xušán haans-éen-i]  šidaloó haans-áan-u \\
pasture-\textsc{ob} from.there for stay-\textsc{prs"=mpl} \textsc{comp}  summer-\textsc{ob} pasture-\textsc{ob} flock happy stay-\textsc{prs-f} coldness stay-\textsc{prs"=msg} \\
\glt `They stay in the pasture, since their flock is happy there, where it's cold.' (B:DHN5266)

\ex
\label{ex:13-236}
\gll kareé eeṛé pruɡraám bhíl-i pahúrta eendáa  th-áan-a ki [théeba
  ǰhaníi deés muqarár  bh-áan-u] \\
when \textsc{dist} programme become.\textsc{pfv-f} after like.that  do-\textsc{prs"=mpl} \textsc{comp} then wedding.\textsc{gn} day fixing become-\textsc{prs"=msg} \\
\glt `After that programme has taken place, then we do the following: we fix a~day [for the wedding]' (A:MAR072-3)
\end{exe}