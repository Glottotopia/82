\chapter{Parts of speech and the lexical profile}
\label{chap:3b}
\section{Part"=of"=speech categories}
\label{sec:3b-1}


The primary criteria for the part"=of"=speech classification applied in this work (as well as in \citealt{liljegrenhaider2011}) are grammatical, although there is an obvious semantic core to each of the classes thus established (\citealt[49-54]{givon2001a}; \citealt[47-54, 102-106]{dixon2010}; \citealt[183-188]{croft2003}). These criteria include language"=particular distribution, functional range and morphological behaviour (\citealt[1-3]{schachtershopen2007}). Palula has four open classes and another nine clearly defined closed classes, but it should be stressed that this division is by no means to be seen as an entirely discrete one. Even within the open classes (particularly among the adverbs), there are closed subclasses, and even some of the closed classes are indeed open to occasional additions, through loans or \isi{derivation}:


\begin{table}[H]
\begin{tabularx}{\textwidth}{ l@{\hspace{30pt}} Q }
\textit{Open classes} &
Nouns\\
&
Verbs\\
&
Adjectives\\
&
Adverbs\\
\textit{Closed classes} &
Pronouns\\
&
Determiners\\
&
Quantifiers\\
&
Postpositions\\
&
Auxiliaries\\
&
Mood markers\\
&
Conjunctions\\
&
Discourse markers\\
&
Interjections\\
\end{tabularx}
\end{table}


In addition, there is a small number of other words or word"=like elements that has been given a somewhat more tentative classification as something other than one of the above"=mentioned categories. Some of them constitute a very small category; others are special for other reasons. Below follows a brief summary of the main characteristic features of each category.


The four open part"=of"=speech classes are treated in depth, each in a separate chapter (nouns in \chapref{chap:4}, verbs in \chapref{chap:8}, adjectives in \chapref{chap:6}, and adverbs in \chapref{chap:7}), as is also the closed class of pronouns (\chapref{chap:5}). The remaining closed classes are either treated alongside functionally or form"=wise closely related categories, or as part of the discussion of higher"=level structures. Determiners, in particular \isi{demonstrative} determiners, are dealt with along with \isi{pronominal} demonstratives (\sectref{sec:5-2}), as there is an obvious diachronic relationship as well as plenty of paradigm"=sharing between the two categories. Quantifiers, in particular numerals, are due to their shared function as \isi{noun} modifiers, treated along with adjectives (\sectref{sec:6-4}). Postpositions are due to their functional overlap with adverbs, especially in the spatial"=temporal realm, treated in the same chapter (\sectref{sec:7-2}). Auxiliaries are due to their role in expressing TMA distinctions mentioned and further exemplified in the chapters dealing with verbs (\chapref{chap:8}) and verbal categories (\chapref{chap:9}). Examples of the use of the size"=wise limited set of \isi{mood} markers appear e.g., in the treatment of imperative sentences (\sectref{subsec:9-2-1}), hearsay (\sectref{subsec:9-2-4}) and \isi{interrogative} constructions (\sectref{sec:14-2}). Conjunctions and \isi{discourse} markers are exemplified throughout \chapref{chap:12} and \chapref{chap:13}. Most of the rest of the strictly limited part"=of"=speech categories are apart from the examples and brief characterisation given below, not given any further treatment in this grammar.


\section{Nouns}
\label{sec:3b-2}
 
Examples: \textit{báaṭ} (\textsc{m}) `stone', \textit{ac̣híi} (\textsc{f}) `eye', \textit{ṭíinčuk} (\textsc{m}) `scorpion', \textit{kúṛi} (\textsc{f}) `woman',  \textit{biaabaán} (\textsc{f}) `wilderness', \textit{rúuṣ} (\textsc{f}) `anger'.


Semantically characterised by relative temporal stability of its referent, a Palula \isi{noun} typically, and unsurprisingly, denotes concepts such as things, places, animals and people, but a great deal of abstract entities are also encoded as nouns. They may be further specified for animacy or humanness. A \isi{noun} primarily functions as the head of an argument (\isi{subject}, object, etc.), but can also be the \isi{predicate}, in the latter case often without an accompanying overt \isi{copula}. The most important subclassification of nouns is the one between masculine and feminine \isi{gender} nouns. Gender assignment is almost exclusively inherent and part of the lexical specification. Only for a smaller group of nouns is \isi{gender} assigned contextually only.


The typical \isi{noun} is inflected for number (singular vs. plural) and case (\isi{nominative} vs. \isi{oblique} vs. \isi{genitive}), although the realisation of and formal expression of each category are \isi{subject} to declensional differences. There are three main declensions, two minor ones, and a smaller number of nouns that display idiosyncratic \isi{inflectional} behaviour.


\subsection{Proper nouns}
\label{subsec:3b-2-1}
Examples: \textit{atshareét} (\textsc{m}) `Ashret (the name of a valley and its main settlement)', \textit{deeúli} (\textsc{f}) `Dir (the name of a district and a former principality)', \textit{mac̣hoók} (\textsc{m}) `Machoke (the name of a tribal ancestor)'.


A distinction can be made between common nouns and proper nouns. The latter are used to refer to specific persons or places. They are normally not pluralised, and only rarely occur with any preceding modifiers. In most cases, however, they can be identified as either masculine or feminine, and as belonging to one of the above mentioned declensions.

\section{Verbs}
\label{sec:3b-3}

Examples: \textit{utráp-} (\textsc{itr}) `to run', \textit{mar-} (\textsc{itr}) `to die', \textit{ɡhin-} (\textsc{tr}) `to take, buy', \textit{phaalé-} (\textsc{tr}) `to break, tear', \textit{bhe-} (\isi{copula}, \textsc{itr}) `to become; to come into existence'.\footnote{The citation forms with a final hyphen are underlying \isi{imperfective} verb stems.}


References to less"=stable experiences or transitory states rather than to a particular entity, are typically, and again, unsurprisingly, encoded as verbs in Palula. The characteristic function of a Palula verb is as a \isi{predicate}, with the most important subclassification being one between \isi{transitive} and \isi{intransitive} verbs. This is a strict distinction, and almost without exception, a particular verb stem is either \isi{intransitive} or \isi{transitive} and cannot (without further \isi{derivation}, see below) be ambivalent or polyvalent. In addition to those two main classes, there is a subclass of copulative verbs, some of which overlap functionally with \isi{intransitive} verbs on the one hand, and with auxiliaries on the other.


Verbs are primarily inflected for \isi{tense}, aspect and argument agreement, and in addition to that, a few \isi{tense}"=\isi{mood}"=aspect (TMA) categories (such as \isi{Perfect} and Past \isi{Imperfective}) are expressed periphrastically by means of auxiliaries. Two different kinds of agreement are part of the paradigm, \isi{person agreement} and \isi{gender/number agreement}. The former is confined to the non"=\isi{tense} marked categories \isi{Future} and the Past \isi{Imperfective}, and the latter with \isi{Present} and the \isi{perfective}"=based categories (\isi{Simple Past}, \isi{Perfect} and \isi{Pluperfect}). Apart from \isi{finite} \isi{inflectional} categories, there are a number of important non"=\isi{finite} forms.


As far as \isi{inflectional} morphology is concerned, there are two main morphological verb classes, here referred to as L"=verbs (an open, productive and large class) and T"=verbs (a~closed, non"=productive, small and rather heterogeneous class). Additionally, there are a few verbs with stems that to a varying degree are suppletive. Within the class of L"=verbs there are predictable variations in the \isi{inflectional} paradigms due to \isi{accent} position and the quality of stem vowels. Many T"=verbs form their perfectives with a \isi{plosive} segment (in the clear cases a \textit{t}"=suffix), but often this has been assimilated with preceding stem segments, and it makes sense to identify a \isi{perfective} stem as distinctly different from an \isi{imperfective} stem.


\subsection{Secondary stems}
\label{subsec:3b-3-1}
Examples: \textit{pašíǰ-} (\isi{passive}) `to be seen' (from \textit{paš-} `to see'), \textit{thawá-} (\isi{causative}) `to have someone do' (from \textit{the-} `to do').


There is a fairly productive valence changing morphology by which “new” stems can be derived: a secondary one"=argument verb stem can be derived morphologically from a corresponding primary \isi{transitive} verb, and in the reverse, many two"=argument verb stems are (at least in a historical sense) derived morphologically from corresponding primary one"=argument verb stems. Similarly, a secondary three"=argument verb can be derived morphologically from a corresponding primary \isi{transitive} verb stem. 


\subsection{Conjunct verbs}
\label{subsec:3b-3-2}
Examples: \textit{ǰhaní the-} (non"=incorporating) `to marry', \textit{káaṇ the-} (incorporating) `to listen, give heed to', \textit{milaáu bhe-} (incorporating) `to meet', \textit{póo de-} (non"=incorporating) `to step on'.


Conjunct verbs are frequently occurring complex \isi{predicate} constructions that, albeit phonologically existing as a combination of two words, function as lexical units. Usually they consist of a simplex verb preceded by a \isi{noun} or an \isi{adjective}; words, or rather lexical elements, that cannot easily be identified as belonging to either of these part"=of"=speech categories may also occur in this position (see \sectref{sec:3b-15}). The verb in such a construction comes from a small set of verb stems (mostly \textit{bhe-} `become', \textit{the-} `do', \textit{de-} `give; fall'), and it is the non"=verb element that contributes the main semantic content to the complex.


There are two main types of conjunct verbs: incorporating and non"=incorporating. In the non"=incorporating conjunct, the non"=verb element functions as the \isi{direct object}, whereas in the incorporating conjunct, the non"=verb element is never treated as an argument of the \isi{clause}.

\largerpage[-1]
\section{Adjectives}
\label{sec:3b-4}
Examples: \textit{piṇḍúuru} `round', \textit{ṣíiṛu} `blind', \textit{ḍanɡ} `hard', \textit{purá} `full, complete'.


The typical \isi{adjective} functions as an attributive modifier of a \isi{noun}, or as a \isi{predicate}, in the latter case often without an accompanying overt \isi{copula}. The only formally substantiated subclassification is one based on agreement properties. On the one hand, there are those adjectives that inflectionally indicate \isi{gender}, number and case of the nouns they modify or, when they function as predicates, the nouns that are their subjects. On the other hand, there are those adjectives that are invariable in form. Adjectives pertaining to dimensions, age and human propensity show a strong tendency to be substantivised.


The great majority of inflecting adjectives occur in three forms ending in \textit{–u} (\textsc{msg.nom}),  \textit{-a} (\textsc{mpl.nom/m.obl}) and \textit{–i} (\textsc{f}), respectively, the latter with an additional \isi{umlaut} for those stems that have an accented \textit{á} or \textit{áa} in its underlying form. There is also a marginal feminine plural in \textit{-im}, largely limited to predicative use.  


\section{Adverbs}
\label{sec:3b-5}
The fourth, and only remaining, open class is adverbs. It is of only moderate size compared to the other three open classes, and some of its rather disparate subclasses are closed rather than open. Adverbs function as modifiers of constituents other than nouns on various grammatical levels. A fair number of adverbs are also part of the cross"=cutting category pro"=forms, in this case belonging to the subcategory pro"=adverbs. At least five subclasses of adverbs can be identified: spatial adverbs, temporal adverbs, manner adverbs, degree adverbs, and sentence adverbs. It should be noted that many adverbial meanings are expressed by words primarily belonging to other categories, or by entire phrases (often postpositional or \isi{noun} phrases).


\subsection{Spatial adverbs}
\label{subsec:3b-5-1}
Examples: \textit{bhun} `down, down below', \textit{aǰá} `up, up there', \textit{nhiáaṛa} `near, nearby'.


Spatial adverbs usually modify verbs or verb phrases, and as such, specify the direction of a movement or the location of an event expressed by a verb. Some of these adverbs are closely related to, yet in most cases clearly distinct from, nouns. While these do not pluralise, nor are they assigned to a \isi{gender} category, they do occur with case inflections reminiscent of those found in the \isi{noun} paradigm. 


\subsection{Temporal adverbs}
\label{subsec:3b-5-2}
Examples: \textit{típa} `now, nowadays', \textit{heeṇṣúka} `this year', \textit{dhoóṛ} `yesterday'.


There is a certain degree of overlap between temporal and spatial adverbs, and like the spatial adverbs, a number of temporal adverbs are related to nouns. In some cases it is not altogether obvious whether a particular word is primarily a \isi{noun} or primarily an \isi{adverb}. It has been decided here to categorise such a word as a \isi{noun} when assignment to one or the other \isi{gender} can be established beyond doubt.


\subsection{Manner adverbs}
\label{subsec:3b-5-3}
Examples: \textit{ɡúči} `freely, for nothing', \textit{táru} `quickly', \textit{bhraáš} `slowly'.


Although manner mainly is expressed by non"=\isi{finite} verb forms (especially converbs and copredicative participles), there is a small class of non"=derived manner adverbs whose main function is to modify verbs or verb phrases.


\subsection{Degree adverbs}
\label{subsec:3b-5-4}
Examples: \textit{phaṣ} (with `white': \textit{phaṣ paṇáaru} `white as a sheet'), \textit{bak} (with `bright': \textit{bak práal} `shining bright'), \textit{tap} (with `dark': \textit{tap c̣hiṇ} `\isi{pitch} dark').


The function of degree adverbs is mostly to modify adjectives or other adverbs. Apart from a few quantifiers that besides their \isi{noun}"=quantifying role also function as adjectival and adverbial degree modifiers, there is a set of what is referred to here as co"=lexicalised intensifiers (some examples given above). These are highly specialised (or idiomatic) elements having a degree"=modifying or intensifying function when preceding a certain \isi{adjective} or \isi{adverb} (with which they are co"=lexicalised). 


\subsection{Sentence adverbs}
\label{subsec:3b-5-5}
Examples: \textit{ɡóo} `maybe', \textit{inšaalaáh} `God willing', \textit{rištaá} `really, in truth'.
Sentence adverbs is a small subclass that modify entire utterances, i.e., they specify the speaker’s attitude toward the event.


\section{Pronouns}
\label{sec:3b-6}
Pronouns are in fact a subset of a cross"=cutting category of pro"=forms, including words that belong to a variety of part of speech categories as well as some that correspond to larger constituents. There are two major kinds of these, functionally"=semantically defined, pro"=forms: \isi{demonstrative} pro"=forms and \isi{indefinite}"=\isi{interrogative} pro"=forms, the former mostly recognised by an initial \textit{ee}"=element, the latter by an initial \textit{k}"= or \textit{ɡ}"=element. In \tabref{tab:3b-1}, a few examples of such pro"=forms can be seen.


\begin{table}[ht]
\caption{Cross"=cutting pro"=forms}
\begin{tabularx}{\textwidth}{ Q Q Q Q Q }
\lsptoprule
&
Demonstrative &
&
Indefinite"=\isi{interrogative} &
\\\midrule
Pronouns &
\textit{eesó} &
`that one' &
\textit{koó} &
`who, anyone'\\
Pro"=adjectives &
\textit{eeteeṇú} &
`that kind of' &
\textit{kateeṇú} &
`what/any kind of' \\
Pro"=adverbs &
\textit{eetáa} &
`there' &
\textit{ɡóo} &
`where, anywhere' \\
Pro"=quantifiers &
\textit{eetí} &
`that much' &
\textit{katí} &
`how much' \\
Pro"=determiners &
\textit{eesó} &
`that (one)' &
\textit{khayú} &
`which (one)' \\
Pro"=\isi{clause} (manner) &
\textit{eendáa=bhe} &
`like that' (\textsc{itr}) &
\textit{kanáa=bhe} &
`how' (\textsc{itr}) \\
&
\textit{eendáa=the} &
`like that' (\textsc{tr}) &
\textit{kanáa=the} &
`how' (\textsc{tr}) \\\lspbottomrule
\end{tabularx}
\label{tab:3b-1}
\end{table}


However, the class of pronouns has a central position with its many members, especially of the \isi{demonstrative} kind, and is therefore deserving of being treated as a part of speech in its own right. Pronouns substitute for a \isi{noun} or an entire \isi{noun} \isi{phrase}. A number of subclasses can be identified.


\subsection{Personal pronouns}
\label{subsec:3b-6-1}
Personal pronouns are words that refer to the speaker or the person spoken to. They occur in singular and plural, respectively, with two case forms available in the singular and four in the plural. Third person, i.e., words that refer to contextually identifiable referents other than speakers or hearers, is expressed by forms belonging in the \isi{demonstrative} subcategory.


\subsection{Demonstrative pronouns}
\label{subsec:3b-6-2}
Demonstrative pronouns use a basic three"=way distance/visibility differentiation extensively: a proximal category for referents close at hand, a \isi{distal} category for referents further removed from the speaker, and a \isi{remote} category for referents out of sight. Within each subset, there is a further differentiation in number, case, and \isi{gender}, the latter restricted to the singular \isi{nominative}. It is also possible to differentiate between strong and weak forms, where strong forms with an initial \textit{ee} (\textit{eesó} corresponding to \textit{so}) tend to be used for \isi{deictic} or \isi{anaphoric} functions in order to keep track of less accessible \isi{discourse} referents, whereas the weak forms are the default choice with easily accessible \isi{discourse} referents. For the proximal and \isi{distal} sets, additional forms with an initial \textit{a} (\textit{anú} corresponding to \textit{nu}) are available, seemingly in free variation with the ``bare'' forms.


\subsection{Indefinite"={interrogative} pronouns}
\label{subsec:3b-6-3}
Another subset of pronouns directly corresponding to the demonstratives, does double duty as \isi{indefinite} and \isi{interrogative} pronouns. While the same case distinctions are made as with demonstratives, there is no differentiation in \isi{gender} or number. Different \isi{indefinite} pronouns (\textit{ɡubáa, ɡa}) must be used when referring to an inanimate referent as opposed to those \isi{indefinite}"=\isi{interrogative} pronouns that refer to animate, in particular human, referents. In addition to this particular closed set, there is a small number of other words that can be used as, and therefore labelled as, \isi{indefinite} pronouns. These often have more specialised functions, some of which in fact primarily belong in the class of quantifiers.


\subsection{Reflexive pronouns}
\label{subsec:3b-6-4}
There is one frequently used \isi{pronoun}, \textit{teeṇíi} `self’s, own', identified as reflexive, i.e., a \isi{pronoun} that is co"=referential with another nominal in the \isi{clause}. It occurs almost exclusively in this form. Usually, but not exclusively, it is the possessor in a possessive construction, and its referent is identical to the \isi{clause} \isi{subject}.


\subsection{Reciprocal pronouns}
\label{subsec:3b-6-5}
There is a single \isi{reciprocal pronoun}, \textit{akaadúi} `one another, each other'. It is used, although rarely, in a few other case forms. It is, like the \isi{reflexive pronoun}, co"=referential with another nominal, but is restricted to mutual actions.


\section{Determiners}
\label{sec:3b-7}
Examples: \textit{anú} `this', \textit{áa} `a', \textit{dúi} `another', \textit{daašúma} `the tenth'.


While attributes (of nouns) are expressed by adjectives, and quantity by quantifiers, determiners establish the reference of a particular \isi{noun} (and in some cases of a \isi{pronoun}). Almost all \isi{determiner} words have dual membership (or are polysemous) and can occur pronominally as well as adnominally. Although clearly derived from quantifiers, words that are normally described as ordinal numerals are for functional reasons included here among the determiners. The special subset of \isi{demonstrative} determiners displays agreement in \isi{gender}, number and case (a \isi{nominative} masculine singular agreement form vs. a non"=\isi{nominative}/plural/feminine agreement form), using different forms for proximal, \isi{distal} and \isi{remote} referents.


As with the closely related \isi{demonstrative} \isi{pronoun} set, a further differentiation is made between strong and weak forms, where the strong forms are used along with less accessible \isi{discourse} referents, whereas the weak forms occur when the referents are easily accessible. A further, and probably still ongoing, grammaticalisation of this distinction is the use of the weak forms of the \isi{remote} set (\textit{so} and \textit{se}, respectively) to indicate definiteness or identifiability, often systematically contrasting with the \isi{indefinite} use of \textit{áa} `a, an'.


In addition to the basic three"=way differentiation, the \isi{demonstrative} determiners can be compounded with preceding spatial adverbs to derive more specialised determiners that  indicate finer degrees of distance or vertical"=horizontal position in relation to the speaker, e.g., \textit{bhun} `down there' + \textit{aṛó} `that' > \textit{bhunaṛó} `that down there'.


\section{Quantifiers}
\label{sec:3b-8}
Examples: \textit{tróo} `three', \textit{bíiḍu} `many, much', \textit{buṭheé} `all', \textit{khéli} `quite some'.


Quantifiers are modifiers in much the same sense as adjectives, but while adjectives are descriptive, i.e., denoting qualities and attributes, quantifiers are limiting, thus indicating quantity or scope of the nouns they modify. While most quantifiers show no agreement features, there are a few that agree in \isi{gender} and case with the nouns that they modify, much like adjectives. Unlike adjectives, quantifiers do not agree in number (naturally, as they are inherently either plural or singular) with the modified \isi{noun}. The quantifiers that indicate exact quantities can combine either with other such quantifiers or with non"=exact quantifiers to form compound \isi{quantifier} expressions; a few of them can be pluralised when being modified by another \isi{quantifier}. As with some subsets of adjectives, quantifiers have a strong tendency to be substantivised.


\section{Postpositions}
\label{sec:3b-9}
Examples: \textit{the} `to, for', \textit{sanɡí} `with, at', \textit{maǰí} `among, in, inside'.


Postpositions are markers of syntactic"=semantic roles or spatial"=temporal relations that are held by the nouns or pronouns they follow. These markers also form phrasal constituents with the nouns or pronouns about which they convey some information. Under certain circumstances, some postpositions form a single phonological word with the nominal form to which they are postposed.  


With most postpositions, the preceding \isi{noun} occurs in the \isi{oblique} case. Apart from single word postpositions, there are also a number of complex postpositions, consisting either of a sequence of postpositions or a \isi{postposition} followed by an \isi{adverb}. In both cases, the \isi{phrase} functions just like any single word \isi{postposition}. Some spatial and temporal adverbs can also function as postpositions.


\section{Auxiliaries}
\label{sec:3b-10}
Examples: \textit{de} Past \isi{tense} marker, \textit{bhóo} `can, to be able to', \textit{ṣáatu} `began'.


Auxiliaries is a small set of verb"=related words which, in addition to verbal morphology, express certain TMA distinctions. Although some of them can take verbal inflections, they always occur in a \isi{clause} along with a (main) verb. Some of the auxiliaries can be combined with each other.


\section{Mood markers}
\label{sec:3b-11}
Examples: \textit{ee} Polar question marker, \textit{maní} \isi{Hearsay} marker, \textit{neé} Request marker.


Mood markers is another size"=wise very limited set of words that, one way or another, specify the relationship between an utterance as a whole and the speaker and/or hearer. A \isi{mood marker} mostly occurs in utterance"=final position, sometimes cliticised to the immediately preceding element.


\section{Conjunctions}
\label{sec:3b-12}
Examples: \textit{ee} `and', \textit{yaá} `or', \textit{heentá} `if', \textit{ki} `that'.


The function of conjunctions (some of them clitics) is to connect or signal the relationship between constituents on various levels. Primarily they indicate what kind of relationship exists between two adjacent clauses, or between a dependent unit and a larger unit that the former is a part of. With a few exceptions, the conjunctions can be characterised as postpositional, since the \isi{conjunction} forms a structural unit with the conjunct it follows.


\section{Discourse markers}
\label{sec:3b-13}
Examples: \textit{ba} Switch"=topic marker, \textit{bi} Separation marker, \textit{ta} Contrast marker, \textit{eé} Amplification marker. 


Discourse markers are words (or clitics) that specify the \isi{discourse} role of a particular (preceding) unit vis"=à"=vis adjacent units. The units that are being indicated thus are primarily phrasal in nature (mostly \isi{noun} phrases), but not exclusively so. A secondary effect of some \isi{discourse} markers is that they indicate how larger units (such as clauses) are interrelated, especially when used in pairs, or when the same marker is used repeatedly in two adjacent clauses, thus partly overlapping with the function of the \isi{conjunction} category.


\section{Interjections}
\label{sec:3b-14}
Examples: \textit{óo} `yes', \textit{ohoó} `wow!', \textit{čo} `go ahead!', \textit{ée} `hey!'.


Although the category of interjections, at least theoretically, may be an open class, there are relatively few examples included in this work. These words can in themselves be used as entire utterances, and there is in most cases no clear syntactic connection with any other co"=occurring words.


\section{Other words or word"=like elements}
\label{sec:3b-15}
A single"=word word category, at least as far as this vocabulary is concerned, consists of the high frequency negator word \textit{na}. Belonging in the cross"=cutting category pro"=forms, but not really fitting into any of the aforementioned classes, is the \isi{indefinite}"=\isi{interrogative} \textit{keé} `why', substituting for an entire \isi{clause}. Another minor category is labelled honorific; such lexical units are titles or title"=like elements prefaced to, or cliticised after, names of certain highly respected people. Closely related to that are some ritualistic expressions, such as \textit{aleehisalaám} `on whom be peace', which is a \isi{phrase} that has to be added when mentioning one of the prophets according to Islamic beliefs.


As already mentioned above, some conjunct verbs consist of a simplex verb preceded by a lexical element that only occurs as part of that particular complex. Such elements have been classified as host elements. Two other processes involving what may be termed ``semi"=words'' are echo formation (usually by repeating a word and substituting the onset with \textit{m}, as in: \textit{ɡúuli} `bread' + \textsc{echo} > \textit{ɡúuli múuli} `bread and other eatables'; \textit{nirkízi} `henna' + \textsc{echo} > \textit{nirkízi mirkízi} `henna and stuff') and \isi{reduplication} (\textit{áak} `one' + \textsc{red} > \textit{aakáak} `one each'; \textit{teeṇíi} `their, etc. (\textsc{refl})' + \textsc{red} > \textit{teeṇteeṇíi} `each their, etc.').

