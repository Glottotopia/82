\chapter{Sentence modification}
\label{chap:14}

\section{Introduction and overview}
\label{sec:14-1}


Some of the more salient sentence modifications are presented in this chapter, especially those that are related to particular markers or constructional types. They include \isi{interrogative} sentences, \isi{negation} and switch"=\isi{topicality}.


\section{Interrogative sentences}
\label{sec:14-2}


The two main types of \isi{interrogative} sentences, polar and constituent interrogatives \citep[290--303]{koenigsiemund2007}, are discussed and exemplified in the sections below. 


\subsection{Polar interrogatives}
\label{subsec:14-2-1}


Polar interrogatives, or `yes/no' questions, are formed by adding a~sentence final particle \textit{ee} (B \textit{aa}) to a~declarative sentence without any subsequent change in the word order. This sentence"=final particle is accompanied by a~slight rising intonation. Examples are shown in (\ref{ex:14-1})--(\ref{ex:14-3}).

\begin{exe}
\ex
\label{ex:14-1}
%modified
\gll baastaár ɡal-úum=\textbf{ee}  \\
bedding throw-\textsc{1sg=q} \\
\glt `Should I spread out the bedding?' (A:HLE2880)

\ex
\label{ex:14-2}
%modified
\gll búd-u=\textbf{ee} \\
understand.\textsc{pfv"=msg=q} \\
\glt `Did you understand?' (A:HLN2852)

\ex
\label{ex:14-3}
%modified
\gll tu c̣hetrúul-a the ba-yáan-u=\textbf{aa} \\
\textsc{2sg.nom} Chitral-\textsc{obl} to go-\textsc{prs"=msg=q} \\
\glt `Are you going to Chitral?' (B:QAA001)
\end{exe}

The clitical nature of the particle is evident in that it attaches to the sentence"=final word, regardless of its part of speech, and in a~phonological sense becomes part of that word; for instance a~final vowel /a/ coalesces entirely with the particle. As an~SOV language, the sentence normally ends in a~verb, and therefore the particles attach to the verb, but that is not the case in most nominal sentences lacking an~overt \isi{copula}, as can be seen in (\ref{ex:14-4})--(\ref{ex:14-6}).

\begin{exe}
\ex
\label{ex:14-4}
%modified
\gll ux-á díi khooǰóol-u ki tu insaán=\textbf{ee} \\
camel-\textsc{obl} from ask.\textsc{pfv"=msg} \textsc{comp} \textsc{2sg.nom} human.being=\textsc{q} \\
\glt `He asked the camel, ``Are you a~man?''' (A:KIN007)

\ex
\label{ex:14-5}
%modified
\gll ɡhooṣṭ-á sóor=\textbf{ee} (sóor-a=ee) \\
house-\textsc{obl} whole.\textsc{mpl-q} \\
\glt `Is your family fine?' (A:DHN3125)

\ex
\label{ex:14-6}
%modified
\gll oó méeš séer-i=\textbf{ee} \\
o aunt.\textsc{voc} whole-\textsc{f=q} \\
\glt `Are you fine, auntie?' (A:HLE2013)
\end{exe}

Apart from ordinary polar questions, there are at least two different ways of forming tag questions, one (\ref{ex:14-7}) with a~sentence"=final \textit{ɡa} `what' (see \sectref{subsec:14-2-2}), and another (\ref{ex:14-8}) with \textit{nheé} (possibly derived from the \isi{negation} \textit{na} and the question particle -\textit{ee}).

\begin{exe}
\ex
\label{ex:14-7}
%modified
\gll so ɡúum \textbf{ɡa} \\
\textsc{3sg.nom} go.\textsc{pfv.msg} what \\
\glt `He left, didn't he?' (A:HLE2776)

\ex
\label{ex:14-8}
%modified
\gll míi tu the maníit-u \textbf{nheé} \\
\textsc{1sg.gen} \textsc{2sg.nom} to say.\textsc{pfv"=msg} \textsc{tag} \\
\glt `I told you, didn't I?' (A:QAM058)
\end{exe}

Alternative questions are expressed with \textit{ki na} `or not', as in (\ref{ex:14-9}), optionally with a~repetition of the verb, as in (\ref{ex:14-10}). This, too, can be used as a~tag question, as in (\ref{ex:14-11}).

\begin{exe}
\ex
\label{ex:14-9}
%modified
\gll tu the phedíl-u \textbf{ki} \textbf{na} \\
\textsc{2sg.nom} to arrive.\textsc{pfv"=msg} or \textsc{neg} \\
\glt `Did you receive it or not?' (B:DHN5736)

\ex
\label{ex:14-10}
%modified
\gll hasó tohfá phedíl-u \textbf{ki} \textbf{na} \textbf{phedíl-u}  \\
\textsc{rem.nom} gift arrive.\textsc{pfv"=msg} or \textsc{neg}  arrive.\textsc{pfv"=msg} \\
\glt `Did this gift arrive or not?' (B:FLW817)

\ex
\label{ex:14-11}
%modified
\gll búd-u \textbf{ki} \textbf{na} \\
understand.\textsc{pfv"=msg} or \textsc{neg} \\
\glt `You understood, didn't you? [lit: Did you understand or not?]' (A:SHA033)
\end{exe}
\subsection{Constituent interrogatives}
\label{subsec:14-2-2}

Constituent interrogatives, or parametric questions, are formed by replacing a~questioned constituent with an~inde\isi{finite}"=\isi{interrogative} \isi{pronoun}. Its position is usually immediately preverbal or (in the case of adnominal constituents) occurs in the \isi{phrase} immediately preceding the verb.


Palula has a~rather large inventory of \isi{interrogative} words that may replace various constituents.


\spitzmarke{Questioning core constituents of the \isi{clause}.} Examples (\ref{ex:14-12})--(\ref{ex:14-14}) illustrate the questioning of the \isi{nominative} \isi{subject} (\textit{koó} `who'), the \isi{oblique} (i.e., \isi{ergative}) \isi{subject} (\textit{kií} `who') and the \isi{accusative} \isi{direct object} (\textit{kaseé} `whom'), respectively, when referring to human beings.

\begin{exe}
\ex
\label{ex:14-12}
%modified
\gll ma \textbf{koó} saat-íin \\
\textsc{1sg}.\textsc{nom} who.\textsc{nom} look.after-\textsc{3pl} \\
\glt `Who will look after me?' (A:MAA018)

\ex
\label{ex:14-13}
%modified
\gll aní toobaák ma díi \textbf{kií} híṛ-i\\
\textsc{prox} rifle \textsc{1sg.nom} from who.\textsc{obl} take.\textsc{pfv"=f} \\
\glt `Who took the gun from me?' (A:AYA018)

\ex
\label{ex:14-14}
%modified
\gll thíi \textbf{kaseé} mheeríl-u \\
\textsc{2sg.gen} who.\textsc{acc} kill.\textsc{pfv"=msg} \\
\glt `Who did you kill?' (A:HLE2612)
\end{exe}

When questioning non"=human (in particular inanimate) core constituents, \textit{ɡubáa} `what' is used instead. As can be seen in (\ref{ex:14-15})--(\ref{ex:14-17}), the form stays the same regardless of its syntactic role (although questioned inanimates only seem to appear either as in\isi{transitive} subjects in copular clauses or as direct objects in verbal clauses).

\begin{exe}
\ex
\label{ex:14-15}
%modified
\gll amzarée thíi káaṇ-a wée \textbf{ɡubáa} maníit-u \\
lion.\textsc{obl} \textsc{2sg.gen} ear-\textsc{obl} in what say.\textsc{pfv"=msg} \\
\glt `What did the lion whisper in your ear?' (A:UNF015)

\ex
\label{ex:14-16}
%modified
\gll índa míi bakáara \textbf{ɡubáa} kh-óon \\
here \textsc{1sg.gen} flock what eat-\textsc{3pl} \\
\glt `What would my flock eat here?' (A:SHY019)

\ex
\label{ex:14-17}
%modified
\gll thíi ṣiṣ-á ǰhulí ba \textbf{ɡubáa} hín-u \\
\textsc{2sg.gen} head-\textsc{obl} on \textsc{top} what be.\textsc{prs"=msg} \\
\glt `What have you got on your head?' (A:KAT099)
\end{exe}

\spitzmarke{Questioning postpositional constituents of the \isi{clause}.} For questioning postpositional human constituents, as in (\ref{ex:14-18})--(\ref{ex:14-20}), the \isi{accusative} form \textit{kaseé} (the same as the \isi{direct object} form) is used.

\begin{exe}
\ex
\label{ex:14-18}
%modified
\gll tu aní kitaáb \textbf{kaseé} \textbf{the} dít-i \\
\textsc{2sg.nom} \textsc{prox} book who.\textsc{acc} to give.\textsc{pfv-f} \\
\glt `Who did you give this book to?' (A:HLE2613)

\ex
\label{ex:14-19}
%modified
\gll thíi \textbf{kaseé} \textbf{díi} anzayíl-u de \\
\textsc{2sg.gen} who.\textsc{acc} from send.\textsc{pfv"=msg} \textsc{pst} \\
\glt `Through whom did you send [it]?' (B:FLW817)

\ex
\label{ex:14-20}
%modified
\gll tu \textbf{kaseé} \textbf{sanɡí} yhóol-u \\
\textsc{2sg.nom} who.\textsc{acc} with come.\textsc{pfv"=msg} \\
\glt `Who did you come together with?' (A:QAM055)
\end{exe}

Postpositional or non"=core inanimate constituents can also be questioned, as in (\ref{ex:14-21})--(\ref{ex:14-22}), again using \textit{ɡubáa} `what', supplied with any case \isi{inflection} required by the \isi{postposition} in question.

\begin{exe}
\ex
\label{ex:14-21}
%modified
\gll anú bel \textbf{ɡubáa-ii} saás bhe hín-u  \\
\textsc{prox.nom.msg} spade what-\textsc{gen} whole become.\textsc{cv}  be.\textsc{prs"=msg} \\
\glt `From what [material] was this spade made?' (B:DHE5527)

\ex
\label{ex:14-22}
%modified
\gll anú \textbf{ɡubáa-ii} pándee saás bhe  hín-u \\
\textsc{prox.msg.nom} what-\textsc{gen} for whole become.\textsc{cv}  be.\textsc{prs"=msg} \\
\glt `For what [\isi{purpose}] was this made?' (B:DHE5532)
\end{exe}

\spitzmarke{Questioning constituents of the \isi{noun} \isi{phrase}.} Various modifiers of a~\isi{noun} can be questioned,
such as a~possessor: \textit{kasíi} \textit{(B kasée)} `whose' (\ref{ex:14-23});
a~cardinal \isi{numeral} or \isi{quantifier}: \textit{katí} `how many, how much' (\ref{ex:14-24});
an~\isi{ordinal numeral}: \textit{katíma} `in which' (\ref{ex:14-25}); an~\isi{adjective}
\textit{kateeṇú} `what kind of (person)' (\ref{ex:14-26}) (adjectivally inflected) or
\textit{ɡa} `what kind of (thing)' (\ref{ex:14-27}); and a~\isi{determiner} \textit{khayú} `which,
what' (\ref{ex:14-28}) (demonstratively inflected) or \textit{khayú áak}/\textit{khayáak} `which one' (\ref{ex:14-28b}).

\begin{exe}
\ex
\label{ex:14-23}
%modified
\gll aní \textbf{kasée} ziaarat-í thaní \\
\textsc{3pl.prox.nom} whose shrine-\textsc{pl} \textsc{quot} \\
\glt `Whose graves are these?' (B:FOR026)

\ex
\label{ex:14-24}
%modified
\gll thíi \textbf{katí} kuṇaak-á hín-a \\
\textsc{2sg.gen} how.much child-\textsc{pl} be.\textsc{prs"=mpl} \\
\glt `How many children do you have?' (A:DHN2012)

\ex
\label{ex:14-25}
%modified
\gll tu \textbf{katíma} (hín-u) \\
\textsc{2sg.nom} in.which be.\textsc{prs"=msg} \\
\glt `In which class are you?' (A:HLE2789)

\ex
\label{ex:14-26}
%modified
\gll rooṭaá \textbf{kateeṇ-ú} \textbf{míiš} ɡa ranɡ-íi  míiš \\
Rota what.kind-\textsc{msg} man what.kind colour-\textsc{gen} man  \\
\glt `What kind of man is Rota, and of what complexion?' (A:BEZ052)

\ex
\label{ex:14-27}
%modified
\gll \textbf{ɡa} \textbf{haál} hín-i \\
what.kind condition be.\textsc{prs-f} \\
\glt `How are you doing? [lit: what position is there]?' (A:CHN070110)

\ex
\label{ex:14-28}
%modified
\gll baačaá so míiš bulaḍ-í khooǰóol-u ki tu míi baačaí \textbf{khay-í} \textbf{zhaáy} ɡhin-áan-u \\
king \textsc{def.nom.msg} man call.for-\textsc{cv} ask.\textsc{pfv"=msg} \textsc{comp} \textsc{2sg.nom} \textsc{1sg.gen} kingdom which-\textsc{f} place take-\textsc{prs"=msg}  \\
\glt `The king called for the man and asked him, ``Which place in my kingdom do you want?''' (A:UXW052)

\ex
\label{ex:14-28b}
%modified
\gll lanaám maǰí \textbf{khayáak} ziaát taaqatwár thaní\\
\textsc{3pl.distacc} among which.one most powerful \textsc{qout}\\
\glt `Which one of them is the strongest?' (A:NOR002)
\end{exe}




While the \isi{interrogative} words exemplified above function attributively and occur before the \isi{noun} they modify, predicatively functioning modifiers being questioned occur immediately preverbally, as can be seen in (\ref{ex:14-29})--(\ref{ex:14-31}). Note that predicatively used \textit{kateeṇí} corresponds to attributively used \textit{ɡa} and \textit{kateeṇí} alike. In (\ref{ex:14-30}), an~additional adjectival \isi{interrogative} \textit{katíiti} is exemplified, questioning the size of an~inanimate \isi{noun}.

\begin{exe}
\ex
\label{ex:14-29}
%modified
\gll míi áa ilaqaa-í búuḍ-i kúṛi díi khooǰóol-u  ki aní kakaríi \textbf{kasíi} thaní \\
\textsc{1sg.gen} \textsc{idef} area-\textsc{obl} old-\textsc{f} woman from ask.\textsc{pfv"=msg} \textsc{comp} \textsc{prox} skull whose \textsc{quot} \\
\glt `I asked an~old woman from the area, ``Whose is this skull?''' (A:WOM459)

\ex
\label{ex:14-30}
%modified
\gll thíi kitaáb \textbf{katíit-i} hín-i \\
\textsc{2sg.gen} book how.big be.\textsc{prs"=mpl} \\
\glt `How big is your book?' (B:DHE5521)

\ex
\label{ex:14-31}
%modified
\gll ṛáa moosím \textbf{kateeṇ-í} \\
there weather what.kind-\textsc{f} \\
\glt `What is the weather like over there?' (A:CHN070104)
\end{exe}

A rather non"=specific \textit{kanáa} `what, like what' is used to question a~\isi{predicate} \isi{phrase} (see also below), nominal or adjectival, in (\ref{ex:14-32})--(\ref{ex:14-35}). It may also be interpreted as questioning the host position of conjunct verbs (especially those with \textit{the-} `do'). 

\begin{exe}
\ex
\label{ex:14-32}
%modified
\gll típa ba ma \textbf{kanáa} \textbf{bh-úum} \\
now \textsc{top} \textsc{1sg.nom} like.what become-\textsc{1sg} \\
\glt `What will now happen to me? [lit: Like what will I become now?]'\newline (A:MAA017)

\ex
\label{ex:14-33}
%modified
\gll naɡarǰúti yhéel-i tas díi khooǰóol-u \textbf{kanáa} \textbf{bhíl-i}\\
Nagarjuti come.\textsc{pfv-f} \textsc{3sg.acc} from ask.\textsc{pfv- msg} like.what become.\textsc{pfv-f}\\
\glt `When Nagarjuti came, he asked her, ``What happened [to you]?'''\newline (A:MAB029)

\ex
\label{ex:14-34}
%modified
\gll tíi maníit-u ki típa \textbf{kanáa} \textbf{th-íia} \\
\textsc{3sg.obl} say.\textsc{pfv-msg} \textsc{comp} now like.what do-\textsc{1pl} \\
\glt `He said, ``What shall we do now?''' (A:BEZ118)

\ex
\label{ex:14-35}
%modified
\gll anú phoó \textbf{kanáa} \textbf{bh-áanu} \\
\textsc{prox.msg.nom} boy like.what become-\textsc{prs"=msg} \\
\glt `How is this boy [feeling]?' (B:DHE4743)
\end{exe}

\spitzmarke{Questioning adverbials.} Various types of adverbials can be questioned using inde\isi{finite}"=\isi{interrogative} words.


For questioning temporal and local adverbials, \textit{kareé} `when' and \textit{ɡóo} (B \textit{ɡúu}) `where' respectively are used, as in (\ref{ex:14-36}) and (\ref{ex:14-37}). 

\begin{exe}
\ex
\label{ex:14-36}
%modified
\gll \textbf{kareé} ukháat-u \\
when come.up.\textsc{pfv"=msg} \\
\glt `When did you come [from down"=country]?' (A:HLN2850)

\ex
\label{ex:14-37}
%modified
\gll thíi ɡhoóṣṭ \textbf{ɡóo} hín-u \\
\textsc{2sg.gen} house where be.\textsc{prs"=msg} \\
\glt `Where do you live [lit: where is your house]?' (A:DHE3144)
\end{exe}

As \textit{ɡóo} only codes location \textit{per se}, goal and source need to be specified by postpositions and case, \textit{ɡóo=the} (B \textit{ɡúu=the}) `to where' (\ref{ex:14-38}) and \textit{ɡóoii} or \textit{ɡóoii thíi} (B \textit{ɡúue thíi}) `from where' (\ref{ex:14-39})--(\ref{ex:14-40}). For questioning the goal an~alternative distinct inde\isi{finite}"=\isi{interrogative} \isi{adverb}, \textit{kíi} (in B \textit{kési}) `whither' is often used, as in (\ref{ex:14-41}).

\begin{exe}
\ex
\label{ex:14-38}
%modified
\gll \textbf{ɡóo=the} bi-áan-u \\
where=to go-\textsc{prs"=msg } \\
\glt `Where are you going?' (A:HLE2452)

\ex
\label{ex:14-39}
%modified
\gll tu \textbf{ɡóo-ii} \\
\textsc{2sg.nom} where-\textsc{gen}  \\
\glt `Where are you from?' (A:DHE3143)

\ex
\label{ex:14-40}
%modified
\gll uth-í maníit-u hín-u ki ée lhéṇḍu \textbf{ɡóo-ii} \textbf{thíi} yhóol-u \\
stand.up-\textsc{cv} say.\textsc{pfv"=msg} be.\textsc{prs"=msg} \textsc{comp} o bald.one where-\textsc{gen} from come.\textsc{pfv"=msg} \\
\glt `Having stood up, he said, ``O bald one, where did you come from?'''\newline (A:KAT030)

\ex
\label{ex:14-41}
%modified
\gll íṇc̣-a maníit-u hín-u ki ée lhéṇḍu \textbf{kíi} bi-áan-u \\
bear-\textsc{obl} say.\textsc{pfv"=msg} be.\textsc{prs"=msg} \textsc{comp} o bald.one whither go-\textsc{prs"=msg} \\
\glt `The bear said, ``O bald one, where are you going?''' (A:KAT020)
\end{exe}

Adverbials corresponding to clauses, rather than to single adverbs or adverbial phrases, can be questioned by e.g., \textit{keé} `why', as in (\ref{ex:14-42}), and \textit{kanáa=bhe/=the} `how, by what means' (see \sectref{subsec:7-1-4}), (\ref{ex:14-43})--(\ref{ex:14-44}). Note that \textit{kanáa=bhe}, which consists of \textit{kanáa} `like what' and the converb of the \isi{verbaliser} \textit{bhe-} `become', is used in in\isi{transitive} clauses, and \textit{kanáa=the}, which is \textit{kanáa} and the \isi{verbaliser} \textit{the-} `do', is used in \isi{transitive} clauses.

\begin{exe}
\ex
\label{ex:14-42}
%modified
\gll kúṛi ma díi khooǰóol-u ki \textbf{keé} ru-áan-u thaníit-u ta\\
wife \textsc{1sg.nom} from ask.\textsc{pfv"=msg} \textsc{comp} why  cry-\textsc{prs"=msg} say.\textsc{pfv"=msg} \textsc{comp}\\
\glt `My wife asked me, ``Why are you crying?''' (A:HUA101)

\ex
\label{ex:14-43}
%modified
\gll tu ateeṇ-ú takṛá íṇc̣-a díi ma \textbf{kanáa=bhe} uḍhíiw"=um? \\
\textsc{2sg.nom} such-\textsc{msg} strong bear-\textsc{obl} from \textsc{1sg.nom}  like.what=become.\textsc{cv} flee-\textsc{1sg} \\
\glt `How can I flee from a~strong bear like you?' (A:KAT136)

\ex
\label{ex:14-44}
%modified
\gll thíi báabu tu \textbf{kanáa=the} saat-íi de \\
\textsc{2sg.gen} father \textsc{2sg.nom} like.what=do.\textsc{cv} take.care.of-\textsc{3sg} \textsc{pst} \\
\glt `How did your father care for you?' (A:MAA019)
\end{exe}

\subsection{Subordinate \isi{interrogative} clauses}
\label{subsec:14-2-3}


As described in \chapref{chap:13}, indirect \isi{discourse} is extremely limited in the language. There is subsequently no category of subordinate or indirect questions clearly distinct from direct questions. When questions occur in reported speech or reported perception, they always appear as if uttered by the quoted speaker or experiencer, as in (\ref{ex:14-45}).

\begin{exe}
\ex
\label{ex:14-45}
%modified
\gll so insaán bulaḍ-íim b-íi de páand-a pharé ki \textbf{so} \textbf{insaán} \textbf{ kateeṇ-ú šay hín-u} \\
\textsc{def.msg.nom} human.being search-\textsc{cprd} go-\textsc{3sg} \textsc{pst} path-\textsc{obl} along \textsc{comp} \textsc{def.msg.nom} human.being what.kind-\textsc{msg} thing be.\textsc{prs"=msg } \\
\glt `He left in search of that man along the road to find out what sort of thing he was [lit: He left to search the man along the the road: What kind of thing is that man?]' (A:KIN004)
\end{exe}

\subsection{Interrogatives in exclamative use}
\label{subsec:14-2-4}

Some sentences that essentially are \isi{interrogative} sentences in form may also be used in an~exclamative function, with or without special interjections utterance"=initially. Some examples are provided in (\ref{ex:14-46})--(\ref{ex:14-49}). 

\ea
\label{ex:14-46}
%modified
\gll dun-áaṭ-u bhíl-u hín-u ki  aní ba \textbf{kateeṇ-í} \textbf{ǰuánd} \\
think-\textsc{ag"=msg} become.\textsc{pfv"=msg} be.\textsc{prs"=msg} \textsc{comp}  \textsc{3fsg.prox.nom} \textsc{comp} what.kind-\textsc{f} life  \\
\glt `He started thinking, ``What a life!'' ' (A:KAT057)

\ex
\label{ex:14-47}
%modified
\gll nu ba \textbf{katí} \textbf{utháal-u} táapaṛ \\
\textsc{3msg.prox.nom} \textsc{top} how.much high-\textsc{msg} hill  \\
\glt `What a~high hill!' (A:HLE3117)

\ex
\label{ex:14-48}
%modified
\gll ohoó nis \textbf{keé} phooṭóol-u \\
oh \textsc{3sg.prox.acc} why break.\textsc{pfv"=msg} \\
\glt `Oh, why did you break it?!' (A:HLE3118)

\ex
\label{ex:14-49}
%modified
\gll aṛé áaṇc̣-a \textbf{kateeṇ-á} \textbf{páak-a} hín=ee \\
\textsc{dist} raspberry-\textsc{pl} that.kind-\textsc{mpl} ripe-\textsc{mpl} be.\textsc{prs"=mpl=q} \\
\glt `Haven't these raspberries ripened nicely!' (A:KAT131)
\z

\section{Negation}
\label{sec:14-3}


The main strategy for \isi{negation} is by means of a~separate and invariable negative particle, \textit{na}. 


\subsection{Basic sentence negation}
\label{subsec:14-3-1}


The pragmatically unmarked position of the negative particle is preverbal, as is evident from (\ref{ex:14-50})--(\ref{ex:14-52}), regardless of the TMA categories reflected in the \isi{predicate}.

\begin{exe}
\ex
\label{ex:14-50}
%modified
\gll amzarái muṛ-u=bhaáu insaán \textbf{na} kha-áan-u \\
lion die.\textsc{pptc"=msg"=adj} human.being \textsc{neg} eat-\textsc{prs"=msg} \\
\glt `A lion doesn't eat a~human being that has died.' (A:UNF012)

\ex
\label{ex:14-51}
%modified
\gll phoó \textbf{na} wháat-u \\
boy \textsc{neg} come.down.\textsc{pfv"=msg } \\
\glt `The boy didn't come back down.' (A:SHY040)

\ex
\label{ex:14-52}
%modified
\gll be musibat-íi waxt-íi akaadúi \textbf{na} uṛiɡal-íia\\
\textsc{1pl.nom} trouble-\textsc{gen} time-\textsc{gen} \textsc{recp} \textsc{neg} abandon-\textsc{1pl}\\
\glt `We will not abandon one another in times of trouble' (A:UNF004)
\end{exe}

Also when there is an~\isi{auxiliary} verb present, as in the periphrastically expressed TMA categories, the negative particle precedes the main verb. This can be seen in examples (\ref{ex:14-53}) and (\ref{ex:14-54}).

\begin{exe}
\ex
\label{ex:14-53}
%modified
\gll muṣṭúk-a xálak-a dhii-á díi \textbf{na} khooǰ-óon de \\
of.past-\textsc{mpl} people-\textsc{pl} daughter-\textsc{obl} from \textsc{neg} ask-\textsc{3pl} \textsc{pst}  \\
\glt `People in the old days were not asking their daughter [who she wanted to marry].' (A:MAR018)

\ex
\label{ex:14-54}
%modified
\gll asím tu \textbf{na} bulaḍíl-u hín-u \\
\textsc{1pl.erg} \textsc{2sg.nom} \textsc{neg} call.\textsc{pfv"=msg} be.\textsc{prs"=msg} \\
\glt `We have not called you.' (A:GHU030)
\end{exe}

This is also true of passives (\ref{ex:14-55}) and non"=\isi{finite} verb forms (\ref{ex:14-56}).

\begin{exe}
\ex
\label{ex:14-55}
%modified
\gll thupíik"=am ǰeníi-e karáaṛu \textbf{na} khaṇiǰíl-u \\
gun-\textsc{ins} fire.\textsc{vn"=gen} leopard \textsc{neg} be.hit.\textsc{pfv"=msg}  \\
\glt `The leopard was not hit by the gunfire.' (B:CLE355)

\ex
\label{ex:14-56}
%modified
\gll ma tu \textbf{na} khaá kaseé the uṛ-éen-i \\
\textsc{1sg.nom} \textsc{2sg.nom} \textsc{neg} eat.\textsc{cv} some.\textsc{acc} to let.out-\textsc{prs-f} \\
\glt `If I don't eat you, I will give you [lit: let you out] to someone else.'\newline (A:KAT014)
\end{exe}

Predicate \isi{noun} phrases without an~overt \isi{copula} are negated by the negative particle alone, thus occurring \isi{clause}"=finally, as seen in (\ref{ex:14-57}) and (\ref{ex:14-58}).

\begin{exe}
\ex
\label{ex:14-57}
%modified
\gll anú míi bharíiw \textbf{na} \\
\textsc{3msg.prox.nom} \textsc{1sg.gen} husband \textsc{neg} \\
\glt `This one is not my husband.' (A:WOM646)

\ex
\label{ex:14-58}
%modified
\gll šuy \textbf{na} \\
good \textsc{neg} \\
\glt `That's not good.' (B:ANG015)
\end{exe}

As for the position being strictly preverbal or being before the entire verbal group, the data shows some variability. 


With the \isi{modality verb} \textit{bhá-} `be able to', the negative particle almost always precedes the modal as well as the main verb, as in (\ref{ex:14-59})--(\ref{ex:14-60}).

\ea
\label{ex:14-59}
%modified
\gll dúi ta ɡaṭíl-u áak ḍaaku"=aan-óom"=ii qilaá tíi \textbf{na} ɡaṭáa bhóol-u \\
other \textsc{cntr} win.\textsc{pfv"=msg} \textsc{idef} robber-\textsc{pl"=obl"=gen} fort  \textsc{3sg.obl} \textsc{neg} win.\textsc{inf} be.able.to.\textsc{pfv"=msg}  \\
\glt `When he had won everything else, there was a~den of thieves that he could not capture.' (A:PIR008)

\ex
\label{ex:14-60}
%modified
\gll taníi báaba tu \textbf{na} kháai bh-óon \\
\textsc{3pl.gen} father.\textsc{pl} \textsc{2sg.nom} \textsc{neg} eat.\textsc{inf} be.able.to-\textsc{3pl} \\
\glt `Not even their fathers will be able to eat you.' (A:KAT074)
\z

However, there are occasional exceptions, as example (\ref{ex:14-61}) shows, where the negative particle occurs between the modal and the main verb.

\begin{exe}
\ex
\label{ex:14-61}
%modified
\gll patráak nikhái \textbf{na} bháam \\
back get.out.\textsc{inf} \textsc{neg} be.able.to.\textsc{1sg} \\
\glt `I won't be able to get back out [of the well].' (B:FOX006)
\end{exe}

With conjunct verbs there is even more variability. Although in most cases the negative particle occurs right before the main verb and after the \isi{host element}, there are some (incorporating) conjuncts where the negative particle occurs sometimes before the whole conjunct, (\ref{ex:14-62})/(\ref{ex:14-64}), sometimes between the \isi{host element} and the inflected verb, (\ref{ex:14-63})/(\ref{ex:14-65}). Further research may reveal certain conditions, pragmatic or of some other kind, that must be met for each to occur. 

\ea
\label{ex:14-62}
%modified
\gll ma \textbf{na} ṭinɡ bhíl-u \\
\textsc{1sg.nom} \textsc{neg} challenged become.\textsc{pfv"=msg } \\
\glt `I could not stand up against him.' (A:HUA108)

\ex
\label{ex:14-63}
%modified
\gll nis the koó ṭinɡ \textbf{na} bh-áan-a \\
\textsc{3sg.prox.acc} to anybody challenged \textsc{neg} become-\textsc{prs"=mpl } \\
\glt `Nobody could stand up against him.' (A:JAN062)

\ex
\label{ex:14-64}
%modified
\gll karáaṛu asée baát \textbf{na} kaṇ th-íi asaám ɡhaš-í ba kh-úu \\
leopard \textsc{1pl.gen} word \textsc{neg} \textsc{host} do-\textsc{3sg} \textsc{1pl.acc} catch-\textsc{cv}  \textsc{top} eat-\textsc{3sg} \\
\glt `The leopard will not listen to us; instead he will catch us and eat us.' (B:FOY031)

\ex
\label{ex:14-65}
%modified
\gll xu eesé waxt-íi peeɣambár hazrát iliaás aleehisalaam-íi beet-í káaṇ \textbf{na} th-íi de \\
but \textsc{rem} time-\textsc{gen} prophet lord  Elijah peace.be.upon.him-\textsc{gen} word-\textsc{pl} \textsc{host} \textsc{neg} do-\textsc{3sg} \textsc{pst} \\
\glt `But he did not heed the words of Lord Elijah, PBUH, the prophet of the time.' (A:ABO011)
\z

Although it was stated at the beginning of this section that the negative particle is invariable, it may nevertheless fuse phonologically with an~adjacent morpheme, such as with the present"=\isi{tense} form of the \isi{copula} (\textit{na hínu {\textgreater} náinu} etc., in B \textit{náhinu}), as in example (\ref{ex:14-66}).

\ea
\label{ex:14-66}
%modified
\gll yéei uth-í anɡaá bhe dac̣h-íi ta kuṇaák \textbf{náin-u} darák \textbf{náin-i} índa dít-i eeṛáa dít-i \\
mother stand.up-\textsc{cv} conscious become.\textsc{cv} look-\textsc{3sg}  \textsc{sub} child \textsc{neg.}be.\textsc{prs"=msg} appearance \textsc{neg.}be.\textsc{prs-f}  here give.\textsc{pfv-f} there give.\textsc{pfv-f} \\
\glt `The mother woke up and could not see the child or any sign of him whereever she turned.' (A:BRE007)
\z

Variations resulting from referential and pragmatic factors are discussed in the following sections.


\subsection{Negative pronouns/particles}
\label{subsec:14-3-2}

Negation is generally not `permeable' \citep[563]{ramat2006}, i.e., a~negative morpheme occurs only once in a~negated \isi{clause}, as in (\ref{ex:14-67})--(\ref{ex:14-69}).

\begin{exe}
\ex
\label{ex:14-67}
%modified
\gll taním \textbf{ɡa} \textbf{na} laád-u \\
\textsc{3pl.erg} what \textsc{neg} find.\textsc{pfv"=msg}  \\
\glt `They didn't find anything.' (A:DRA003)

\ex
\label{ex:14-68}
%modified
\gll dúu oostaaz-aán hín-a dúi \textbf{ɡa} \textbf{na}  bh-áan-u \\
two teacher-\textsc{pl} be.\textsc{prs"=mpl} other what \textsc{neg}  become-\textsc{prs"=msg} \\
\glt `There are [only] two teachers, nobody/nothing else.' (A:OUR017)

\ex
\label{ex:14-69}
%modified
\gll míi yaar-íi [ɡa] xabaár \textbf{náin-i=ee} \\
\textsc{1sg.gen} friend-\textsc{gen} what.kind news \textsc{neg}.be.\textsc{prs-f=q}  \\
\glt `Is there no news of my friend?' (A:SHY047)
\end{exe}

The \isi{pronoun} \textit{ɡa} belongs to a~set of inde\isi{finite}"=\isi{interrogative} pronouns and is in itself neutral (rather than negative). It could, however, be argued that the combination \textit{ɡa na} phonologically is one word, as occurring in (\ref{ex:14-67}) and (\ref{ex:14-68}), and as such constitutes a~negative \isi{pronoun}. In any case we have only one morpheme with a~clearly negative semantics in sentences like these. 



The possible emergence of a~set of negative pronouns is even more obvious with combinations inde\isi{finite}"=\isi{interrogative} \isi{pronoun} + \textit{bi} `also' + \textit{na}, as in examples (\ref{ex:14-70})--(\ref{ex:14-72}). Neither do these ``negative compounds'' occur with an~additionally negated verb; instead the entire predication lies in the scope of this negative \isi{pronoun}, itself in preverbal position in the \isi{clause}. Probably the morpheme \textit{bi} contributes an~added emphasis to the \isi{negation}, approximately corresponding to `at all, else'.

\begin{exe}
\ex
\label{ex:14-70}
%modified
\gll kaṭamuš-á \textbf{ɡá=bi=na} khóol-u hín-u \\
Katamosh-\textsc{obl} what=also=\textsc{neg} eat.\textsc{pfv"=msg } be.\textsc{prs"=msg} \\
\glt `Katamosh ate nothing.' (A:KAT065)

\ex
\label{ex:14-71}
%modified
\gll aaxeeríi waxt-íi tas sanɡí \textbf{koó=bi=na} heensíl-a  de \\
last time-\textsc{gen} \textsc{3sg.acc} with who=also=\textsc{neg} stay.\textsc{pfv"=mpl}  \textsc{pst} \\
\glt `In the end there was nobody along with him.' (A:ABO022)

\ex
\label{ex:14-72}
%modified
\gll tu díi ma \textbf{ɡóo=bi=na} lhéest-i \\
\textsc{2sg.nom} from \textsc{1sg.nom} where=also=\textsc{neg} escape.\textsc{pfv-f}  \\
\glt `Nowhere am I safe from you.' (A:PAS126)
\end{exe}

A marginal case where double \isi{negation} may be argued to occur is when a~particle \textit{hiǰ} `at all' (derived from \iliPersian or \iliPashto where it has a~clearly negative value) is added to the inde\isi{finite}"=\isi{interrogative} \textit{ɡa} in a~negated \isi{clause}, such as in (\ref{ex:14-73}) and (\ref{ex:14-74}). We may on the other hand say that it is used in a~way very similar to \textit{bi}, thus primarily adding emphasis to the already negative expression.

\begin{exe}
\ex
\label{ex:14-73}
%modified
\gll [hiǰ ɡa] xabaár \textbf{náin-i} \\
at.all what news \textsc{neg}.be.\textsc{prs-f}  \\
\glt `There is no news at all.' (A:SHY049)

\ex
\label{ex:14-74}
%modified
\gll [hiǰ ɡa] maalumaát \textbf{na} bhíl-i \\
at.all what information \textsc{neg} become.\textsc{pfv-f } \\
\glt `She didn't get any information at all.' (A:BRE008)
\end{exe}

\subsection{The scope of negation}
\label{subsec:14-3-3}


\spitzmarke{Negation of subunits.} In the examples, so far, the entire \isi{predicate} lies in the scope of the \isi{negation}. But it is also possible to negate only a~\isi{phrase} or a~subunit of a~\isi{clause}, as in (\ref{ex:14-75})--(\ref{ex:14-77}). Here, however, the negated unit is especially marked or extraposed, and the \isi{negation} does not occur in the ``regular'' immediately preverbal position, and therefore the scope of the \isi{negation} has to be interpreted as narrowed down. 

\begin{exe}
\ex
\label{ex:14-75}
%modified
\gll ɡúum táa the róot-a \textbf{dees-á} \textbf{na} róot-a  ɡúum \\
go.\textsc{pfv.msg} there to night-\textsc{obl} day-\textsc{obl} \textsc{neg} night-\textsc{obl}  go.\textsc{pfv.msg}  \\
\glt `He went there, during the night, not during the day.' (A:PIR015-7)

\ex
\label{ex:14-76}
%modified
\gll ḍaaku"=aan-óom"=ii qilaá ǰhulí tándar dít-u, xu ée iskandár \textbf{thíi} \textbf{dúšii} \textbf{ba} \textbf{na} \\
robber-\textsc{pl"=obl"=gen} fort on thunder give.\textsc{pfv"=msg} but o Alexander \textsc{2sg.gen} direction.\textsc{gen} \textsc{top} \textsc{neg} \\
\glt `A thunder fell on the den of thieves, but not from you, o Alexander.'\newline (A:PIR045-6)

\ex
\label{ex:14-77}
%modified
\gll eetás matíl-u seentá tasíi bi ɡhiíṛ  bh-áan-u \textbf{xu} \textbf{na} \textbf{aksár} dhruus-áan-a\\
\textsc{3sg.rem.acc} churn.\textsc{pfv"=msg} \textsc{condh} \textsc{3sg.gen} also ghee  become-\textsc{prs"=msg} but \textsc{neg} often sip-\textsc{prs"=mpl}  \\
\glt `If that [liquid] is churned, it becomes ghee, but often that is not done, and [people actually] drink it.' (A:KEE041-2)
\end{exe}

The exact nature of and the mechanisms available for negating subunits is an~area needing further research.


\spitzmarke{Negation in complex constructions.} With one complex construction involving the modal \textit{bha-} `be able to', already touched upon briefly above (see examples (\ref{ex:14-59})--(\ref{ex:14-61})), we observed that the negative particle tends to precede both the \isi{complement}"=taking verb and the infinitival \isi{complement}. This underlines the high degree of \isi{clause} integration pointed out in \sectref{subsec:13-5-2}. Although it is in fact the ability that is negated, the \isi{negation} occurs closest to what is formally the subordinate verb, and it is not even possible to negate the subordinate verb only.


In other complex constructions, with a~\isi{Verbal Noun} in the \isi{complement} and where the bond is not quite so tight between the \isi{complement}"=taking \isi{predicate} and the verbal element of the \isi{complement}, it is obvious that either of the two clauses can be negated, either that coded by the matrix verb, as in (\ref{ex:14-78}) and (\ref{ex:14-79}), or the one coded by the \isi{Verbal Noun}, as in (\ref{ex:14-80}) and (\ref{ex:14-81}). 

\begin{exe}
\ex
\label{ex:14-78}
%modified
\gll heewand-á \textbf{tanaám} \textbf{the} \textbf{akaadúi} \textbf{paš-ainií} naawás  na de \\
winter-\textsc{obl} \textsc{3pl.acc} to \textsc{recp} see-\textsc{vn} difficult \textsc{neg}
be.\textsc{pst} \\
\glt `During the winter, it was not difficult for them to meet.' (A:SHY006)

\ex
\label{ex:14-79}
%modified
\gll uc̣hí ba se čúti-m-e zaríia baándi  so baṭ húṇṭraak c̣huɡal-áan-u  \textbf{se} \textbf{kúuk-a} \textbf{se} \textbf{muṭ-á} \textbf{bheš-aníi} \textbf{the} na uṛ-áan-u \textbf{mhaás} \textbf{khainíi} \textbf{the} na uṛ-áan-u\\
lift.up.\textsc{cv} \textsc{top} \textsc{def} paw-\textsc{pl"=gen} means by  \textsc{def.msg.nom} stone upward hurl-\textsc{prs"=msg}  \textsc{def} crow-\textsc{pl} \textsc{def} tree-\textsc{obl} sit.down-\textsc{vn} to \textsc{neg} let-\textsc{prs"=msg} meat eat.\textsc{vn} to \textsc{neg} let-\textsc{prs"=msg} \\
\glt `After picking them up, it was throwing the stones with the help of its paws, not allowing the crows to sit in the tree, not allowing them to eat the meat.' (B:SHB752-6)

\ex
\label{ex:14-80}
\gll qáburee farišteém \textbf{dunia-í} \textbf{wée} \textbf{xudaá-ii} \textbf{húkum} \textbf{na} \textbf{man"=ainíi} \textbf{wáǰa} \textbf{ǰhulí} tas bíiḍ-u ziaát  ɡoor-íi azaáb dít-i\\
in.grave angel.\textsc{pl.obl} world-\textsc{obl} in God-\textsc{gen} order \textsc{neg} say-\textsc{vn} cause on \textsc{3sg.acc} very-\textsc{msg} much grave-\textsc{gen} punishment give.\textsc{pfv-f}\\
\glt `In the grave the angels punished him severely for not obeying God's commands.' (A:ABO026)

\ex
\label{ex:14-81}
%modified
\gll čhéeli \textbf{na} \textbf{čit"=aníi} \textbf{ǰhulí} ɡhrast-íi iṣkáar bhíl-i \\
goat \textsc{neg} think-\textsc{vn} on wolf-\textsc{gen} prey become.\textsc{pfv-f} \\
\glt `Because of not thinking [clearly], the goat fell prey to the wolf.' (B:FOX034)
\end{exe}

As far as \isi{Conditional} constructions are concerned, I only have clear examples of the `if'-\isi{clause} being negated, as in (\ref{ex:14-82}), in which case the negative particle occurs immediately preverbally.

\begin{exe}
\ex
\label{ex:14-82}
%modified
\gll \textbf{thíi} \textbf{ninaám} \textbf{na} \textbf{phedúul-a} \textbf{heentá}  qeaamatée-e dees-á ma tu díi khooǰ-áam \\
\textsc{2sg.gen} \textsc{3pl.prox.acc} \textsc{neg} take.\textsc{pfv"=mpl} \textsc{condl}  judgement-\textsc{gen} day-\textsc{obl} \textsc{1sg.nom} \textsc{2sg.nom} from ask-\textsc{1sg}  \\
\glt `If you don't take these [to her], I will ask you on the day of judgment.' (B:FLW800)
\end{exe}

In a~number of coordinate constructions where one or more elements are negated, the negative particle does not occur in preverbal position, but instead appears in more or less fixed positions according to the particular construction in question, including sentence"=finally in \isi{postsection} constructions (see \sectref{subsec:13-2-2}) and \isi{clause} (or \isi{phrase}) initially or as parts of the strings \textit{bi na...bi na} and \textit{na ta...na ba} in \isi{rejection} constructions (see \sectref{subsec:13-2-4}). 


\subsection{The pragmatics of negation}
\label{subsec:14-3-4}


A couple of observations on the pragmatics of \isi{negation} should be pointed out in particular.


The first concerns possessive \isi{negation}. Just as one main strategy of expressing possession is by means of an~\isi{existential} construction, the negated counterpart is in the form of a~denial of existence, whether alienable as in (\ref{ex:14-83}) or inalienable as in (\ref{ex:14-84}).

\begin{exe}
\ex
\label{ex:14-83}
\gll ma díi paiseé náhin-a \\
\textsc{1sg.nom} from money.\textsc{pl} \textsc{neg}.be.\textsc{prs"=mpl}  \\
\glt `I don't have any money [lit: Money is not from me].' (B:ANG008)

\ex
\label{ex:14-84}
\gll lesée putr-á na heensíl-a de \\
\textsc{3sg.dist.gen} son-\textsc{pl} \textsc{neg} stay.\textsc{pfv"=mpl} \textsc{pst}  \\
\glt `He had no sons [lit: His sons were not].' (B:FOR003)
\end{exe}

Such a~possessive \isi{clause} can also, as in (\ref{ex:14-85}), include an~inde\isi{finite}"=\isi{interrogative} \isi{pronoun}.

\begin{exe}
\ex
\label{ex:14-85}
\gll tasíi ba ɡa wasá na heensíl-u \\
\textsc{3sg.gen} \textsc{top} what.kind capacity \textsc{neg} stay.\textsc{pfv"=msg } \\
\glt `He had no strength [lit: His any capacity was not].' (A:GHA017)
\end{exe}

The other comment concerns so"=called \isi{Obligative} constructions. The positive (non"=negated) \isi{Obligative} codes necessity or obligation (see \sectref{subsec:9-2-3}), especially with \isi{transitive} verbs. The obligative verb form negated, however, has a~primarily prohibitive reading, corresponding to `it is not advisable, one should not, one should avoid'. This is particularly the case with \isi{transitive} verbs, as in (\ref{ex:14-86}) and (\ref{ex:14-87}), whereas a~negated in\isi{transitive} \isi{Obligative} (\ref{ex:14-88}) can imply non"=ability.

\ea
\label{ex:14-86}
\gll kháač-u kráam kuṇaak-íi maxadúši wée na  th"=eeṇḍeéu \\
bad-\textsc{msg} work child-\textsc{gen} front.of in \textsc{neg} do-\textsc{oblg } \\
\glt `One should not display bad manners in front of children.' (A:SMO024) 
\ex
\label{ex:14-87}
\gll anú phoó axsaá nis ɡhooṣṭ-á the na har"=eeṇḍeéu\\
\textsc{prox.msg.nom} boy dirty \textsc{3sg.prox.acc} house-\textsc{obl} to \textsc{neg} take.away-\textsc{oblg}\\
\glt `This boy is dirty; he should not be brought into the house.' (A:Q6.09.16)

\ex
\label{ex:14-88}
\gll so na yh-eeṇḍeéu\\
\textsc{3msg}.\textsc{nom} \textsc{neg} come-\textsc{oblg}\\
\glt `He was not able to come.' (A:Q6.12.02)
\z

A simple denial of an~obligation, on the other hand, is expressed by other constructions, such as those in (\ref{ex:14-89}) and (\ref{ex:14-90}).

\begin{exe}
\ex
\label{ex:14-89}
\gll muniir-íi ɡáaḍ-u ɡhoóṣṭ sam-ainií ɡa zarurát  náin-i \\
Munir-\textsc{gen} big-\textsc{msg} house build-\textsc{vn} some necessity  \textsc{neg.}be.\textsc{prs}-\textsc{f } \\
\glt `Munir does not have to build a~big house [contrary to the obligation that he first assumed he was under].' (A:CHE080304)

\ex
\label{ex:14-90}
\gll tasíi bhróo peexawur-á the bíi de ta so ba na ɡúum \\
\textsc{3msg}.\textsc{gen} brother Peshawar-\textsc{obl} to go.\textsc{3sg} \textsc{pst} \textsc{sub}  \textsc{3msg.nom} \textsc{top} \textsc{neg} go.\textsc{pfv.msg } \\
\glt `Since his brother was going to Peshawar, he didn't go [i.e., it was not necessary any longer].' (A:CHE080304)
\end{exe}

\subsection{Prohibitive negation}
\label{subsec:14-3-5}


As was mentioned in \sectref{subsec:9-2-1}, there is no prohibitive category morphologically distinct from the \isi{Imperative}. Instead \isi{prohibitive negation} is formed by the same means as indicative \isi{negation}, i.e., by the imperative verb form being immediately preceded by the negative particle \textit{na}, as shown in (\ref{ex:14-91}) and (\ref{ex:14-92}).

\begin{exe}
\ex
\label{ex:14-91}
%modified
\gll teeṇíi kuṇaák anú qísum \textbf{na} \textbf{bhanǰé} \\
\textsc{refl} child \textsc{prox.msg.nom} kind \textsc{neg} beat.\textsc{imp.sg}  \\
\glt `Don't beat your own child like this!' (B:ANG015)

\ex
\label{ex:14-92}
%modified
\gll ée iṇc̣ ma típa \textbf{na} \textbf{kha} \\
o bear \textsc{1sg.nom} now \textsc{neg} eat.\textsc{imp.sg } \\
\glt `O bear, don't eat me now!' (A:KAT023)
\end{exe}

\section{Switch"=topicality}
\label{sec:14-4}


Although pragmatic- and \isi{discourse}"=related functions are only marginally part of this work, at least one very frequently occurring particle, \textit{ba}, which has a~rather wide scope, will need some brief and tentative comments.



We have already come across \textit{ba} as it occurs together with another particle, \textit{ta}, in coordinate contrasting or adversative expressions (see \sectref{subsec:13-2-1}), but it also occurs alone as an~expression of \isi{topicality} or emphasis. While a~clearly identifiable or recently referred to \isi{subject} (with its expected \isi{topicality}) normally is not particularly marked for \isi{topicality}, it seems most other entities need to be identified as such by the specific postposed switch"=topic \citep[149]{andrews2007} marker \textit{ba}.



In example (\ref{ex:14-93}), the \isi{subject} of the first sentence is a~particular witch; then in the next sentence the man Pashambi, who, as the main character of this historical account, has been previously introduced but not recently referred to, is reintroduced as the topic \isi{noun} \isi{phrase} and is thus marked with the switch"=topic marker \textit{ba}.

\ea
\label{ex:14-93}
%modified
\gll úuč-a se  be heensíl-i  hín-i \textbf{pašambeé} \textbf{ba} bakáara ɡhin-í úuč-a the ɡúum hín-u \\
Uch-\textsc{obl} \textsc{3fsg.nom} go.\textsc{cv} stay.\textsc{pfv-f} be.\textsc{prs-f} Pashambi  \textsc{top} flock take-\textsc{cv} Uch-\textsc{obl} to go.\textsc{pfv.msg} be.\textsc{prs"=msg} \\
\glt `She had gone to live in Uch. [Now it so happened that] Pashambi was going with his flock to Uch.' (A:PAS113-4)
\z

Many times when using the marker \textit{ba}, an~explicit contrast with an~immediately preceding \isi{subject} is obtained, as in (\ref{ex:14-94}) and (\ref{ex:14-95}), which is not very different from the use of `while' or `however'/`though' in \iliEnglish.

\begin{exe}
\ex
\label{ex:14-94}
%modified
\gll míi ɡhoóṣṭ lookúṛi hín-u \textbf{iskuúl} \textbf{ba} asíi kaṇeeɡhaá hín-i\\
\textsc{1sg.gen} house Lokuri be.\textsc{prs"=msg} school \textsc{top} \textsc{1pl.gen} Kanegha be.\textsc{prs-f}\\
\glt `My house is in Lokuri, while our school is in Kanegha.' (A:OUR004)

\ex
\label{ex:14-95}
%modified
\gll tus aakáak looṛíi-a aṭ-óoi \textbf{iṇc̣} \textbf{ba}  kaṭamuš-á the óol bh-íi \\
\textsc{2pl.nom} one.each bowl-\textsc{pl} bring-\textsc{imp.pl} bear \textsc{top} Katamosh-\textsc{obl} to watch become-\textsc{3sg } \\
\glt `Go and get a~bowl each! The bear, though, will stay here and watch Katamosh.' (A:KAT125-6)
\end{exe}

Sometimes, although still being a~contrast of sorts, \textit{ba} serves primarily as a~signal that similar"=looking or otherwise somehow related topics are non"=identical, for instance in lists, as the ones in (\ref{ex:14-96}) and (\ref{ex:14-97}), or genealogical accounts, as in (\ref{ex:14-98}). 

\ea
\label{ex:14-96}
%modified
\gll \textbf{koó} \textbf{ba} paiseé d-áan-a \textbf{koó} \textbf{ba} toobaák d-áan-a \textbf{koó} \textbf{ba} ṭeép d-áan-a\\
who \textsc{top} money.\textsc{pl} give-\textsc{prs"=mpl} who \textsc{top} gun  give-\textsc{prs"=mpl} who \textsc{top} tape.recorder give-\textsc{prs"=mpl} \\
\glt `Some give money, others give guns, others tape"=recorders.' (A:MAR091-3) 
\ex
\label{ex:14-97}
\gll tus hakim-í buṭheé putr-á mhaar-úuy-a tes bi mheer-í \textbf{kuṛíina} \textbf{ba} ɡhaš-í ukaal-úuy=a \textbf{ɡhooṣṭ-áam} \textbf{ba} anɡáar ṣaa-wúuy=a\\
\textsc{2pl.nom} ruler-\textsc{gen} all son-\textsc{pl} kill-\textsc{imp.pl-q} \textsc{3sg.acc} also kill-\textsc{cv} woman.\textsc{pl} \textsc{top} take-\textsc{cv} bring.up-\textsc{imp.pl=q}  house-\textsc{pl.obl} \textsc{top} fire put.on-\textsc{imp.pl=q}\\
\glt `Kill all of the ruler's sons, kill him, [then] take all the women up here, and [then] set their houses on fire!' (B:ATI033-6)

\ex
\label{ex:14-98}
\gll míi nóo aaxuunseéd \textbf{míi} \textbf{báabii} \textbf{nóo} \textbf{ba}  ɡulseéd \textbf{míi} \textbf{dóodii} \textbf{nóo} \textbf{ba} ɣulaamseedmalák \textbf{ɣulaamseedmalak-íi} \textbf{báabii} \textbf{nóo} \textbf{ba} sahibǰií... \\
\textsc{1sg.gen} name Akhund.Seyd \textsc{1sg.gen} father.\textsc{gen} name \textsc{top} Gul.Seyd \textsc{1sg.gen} grandfather.\textsc{gen} name \textsc{top} Ghulam.Seyd.Malak Ghulam.Seyd.Malak-\textsc{gen} father.\textsc{gen} name \textsc{top} Sahib.Jee  \\
\glt `My name is Akhund Seyd, my father's name was Gul Seyd, my grandfather's name was Ghulam Seyd Malak, and Ghulam Seyd Malak's father's name was Sahib Jee{\ldots}' (A:ASH019-20)
\z

This particle, along with its topic"=switching function, may also be seen as a~device for signalling natural continuity, `and then{\ldots} and then', itself having a~conjunctive function, connecting one piece of \isi{discourse} with the next, which is obvious when looking at example (\ref{ex:14-99}).

\begin{exe}
\ex
\label{ex:14-99}
%modified
\gll tarkaáṇ teeṇíi the bheénš ɡal-íi  \textbf{rhalá} \textbf{bheenš-á} \textbf{ǰhulí} \textbf{ba} čauráts ɡal-íi \textbf{čaurats-í} \textbf{ǰhulí} \textbf{ba} bhít-a ɡal-íi\\
carpenter \textsc{refl} do.\textsc{cv} main.beam put.in-\textsc{3sg} on.top main.beam-\textsc{obl} on \textsc{top} cross.beam put.in-\textsc{3sg} cross.beam-\textsc{obl} on \textsc{top} plank-\textsc{pl} put.in-\textsc{3sg}\\
\glt `The carpenter himself puts up the main beam, and then on top of the main beam he puts in the cross"=beams, and then on the cross"=beams he puts in planks.' (A:HOW016-7)
\end{exe}

If not explicitly contrasted with any particular or easily definable entity in the preceding utterances, the reading is rather one of special emphasis put on the \isi{phrase} thus marked by \textit{ba} in (\ref{ex:14-100})--(\ref{ex:14-103}), sometimes corresponding to \iliEnglish `as for', other times corresponding to what would be intonationally signalled as somehow outstanding. Example (\ref{ex:14-104}) may be described as a~cleft construction, where \textit{ba} marks focus that precedes the background.

\begin{exe}
\ex
\label{ex:14-100}
%modified
\gll \textbf{ma} \textbf{ba} ɡáaḍ-u zuaán míiš de \\
\textsc{1sg.nom} \textsc{top} grown-\textsc{msg} young man be.\textsc{pst}  \\
\glt `As for me, I was a~strong young man.' (A:PAS004)

\ex
\label{ex:14-101}
%modified
\gll \textbf{tu} \textbf{ba} kanáa=the las sanɡí  mháala ɡhaš-áan-u\\
\textsc{2sg.nom} \textsc{top} like.what=do.\textsc{cv} \textsc{3sg.dist.acc} with wrestling take-\textsc{prs"=msg}\\
\glt `How can \textit{you} wrestle with him?' (A:MAH060)

\ex
\label{ex:14-102}
%modified
\gll \textbf{neečíir} \textbf{ba} eesé waxt-íi bíiḍ-i \\
hunting \textsc{top} \textsc{rem} time-\textsc{gen} much-\textsc{f } \\
\glt `As for hunting, there was a~lot of it in those days.' (A:HUA046)

\ex
\label{ex:14-103}
%modified
\gll \textbf{aní} \textbf{ba} kateeṇ-í ǰuánd \\
\textsc{3fsg.prox.nom} \textsc{top} what.kind-\textsc{f} life  \\
\glt `What a life \textit{this} is!' (A:KAT057)

\ex
\label{ex:14-104}
%modified
\gll \textbf{kháač-a} \textbf{kráam-a} \textbf{díi} \textbf{ba} teeṇíi zaán bač th"=eeṇḍeéu \\
bad-\textsc{obl} work-\textsc{obl } from \textsc{top} \textsc{refl} self safe do-\textsc{oblg} \\
\glt `Bad manners [in particular] is what you must avoid.' (A:SMO023)
\end{exe}

This can be used also in questioning about the general whereabouts of a~particular person, as in (\ref{ex:14-105}).

\begin{exe}
\ex
\label{ex:14-105}
%modified
\gll o méeš \textbf{kaṭamúš} \textbf{ba}\\
o aunt.\textsc{voc} Katamosh \textsc{top}\\
\glt `O auntie, what about Katamosh?' (A:KAT112)
\end{exe}

Sometimes, it is difficult to see exactly what \textit{ba} does other than signal a~switch in referentiality. That can be seen in how the entity pronominally referred to by the first \textit{tasíi} `his' is not the same as that referred to by the second \textit{tasíi} `his' in example (\ref{ex:14-106}).

\begin{exe}
\ex
\label{ex:14-106}
%modified
\gll tasíi áak putr de \textbf{ɣaazisamadxaán} \textbf{ba}  tasíi nóo de \\
\textsc{3sg.gen} \textsc{idef} son be.\textsc{pst} Ghazi.Samad.Khan \textsc{top}  \textsc{3sg.gen} name be.\textsc{pst} \\
\glt `He had a~son whose name was Ghazi Samad Khan.' (A:GHA004)
\end{exe}

A topic"=marked entity, as in (\ref{ex:14-107}) and (\ref{ex:14-108}), can also be further expanded in an~extraposed \textit{ki}-construction.

\ea
\label{ex:14-107}
%modified
\gll \textbf{míi} \textbf{šiǰrá} \textbf{ba} eeteeṇ-ú ki \textbf{míi} \textbf{putr-íi} \textbf{nóo} \textbf{umarseéd...}\\
\textsc{1sg.gen} line \textsc{top} like.this-\textsc{msg } \textsc{comp} \textsc{1sg.gen} son-\textsc{gen} name Umar.Said\\
\glt `As for my line, it looks like this: My son's name is Umar Said{\ldots}' (A:ASH019) 
\ex
\label{ex:14-108}
\gll \textbf{paš-ainií} \textbf{dasturá} \textbf{ba} eeṛó ki \textbf{phoo-íi} \textbf{ɡhooṣṭ-íi} \textbf{tarapíi} \textbf{tasíi} \textbf{axpul-aán} \textbf{kuṛíina} \textbf{míiš-a} \textbf{teeṇíi} \textbf{se} \textbf{bhoóy} \textbf{paš-ainií} \textbf{the} \textbf{bi-áan-a}\\
see-\textsc{vn} custom \textsc{top} \textsc{dist.nom} \textsc{comp} boy-\textsc{gen} house-\textsc{gen} direction \textsc{3sg.gen} relative-\textsc{pl} woman.\textsc{pl}  man-\textsc{pl} \textsc{refl} \textsc{def} daughter"=in"=law see-\textsc{vn} to go-\textsc{prs"=mpl}\\
\glt `The custom of bride"=inspection, then, is the following: The relatives, men and women from the boy's house, go there, to see their daughter"=in law.' (A:MAR104)
\z

Apart from the cross"=referencing between a~topic marked with \textit{ba} and the content of a \textit{ki}-\isi{clause}, it seems that the marking of a~non"=\isi{subject} entity with \textit{ba}, allows for one of the other arguments to be extraposed to a~postverbal position, as is the case in (\ref{ex:14-109}) and (\ref{ex:14-110}). Although (\ref{ex:14-109}) corresponds to a~\isi{passive} construction in \iliEnglish, it is not \isi{passive} in Palula, which is seen in the otherwise regular \isi{ergative} case marking of the extraposed agent \isi{subject}. The exact information status of the extraposed argument is a~matter for further research.

\begin{exe}
\ex
\label{ex:14-109}
%modified
\gll \textbf{islaám} \textbf{ba} aṭíl-i hín-i \textbf{ɡabarúuṭ-ii} \textbf{putr-óom}\\
Islam \textsc{top} bring.\textsc{pfv-f} be.\textsc{prs-f} Gabaroot-\textsc{gen} son-\textsc{pl.obl }\\
\glt `It is Gabaroot's sons who brought Islam [here].' (A:ASH054)

\ex
\label{ex:14-110}
%modified
\gll deeúli yhaí \textbf{áak} \textbf{bhróo} \textbf{ba} tíi phrayíl-u \textbf{saaw-á} \textbf{the}\\
Dir come.\textsc{cv} one brother \textsc{top} \textsc{3sg.obl} send.\textsc{pfv"=msg} Sau-\textsc{obl} to\\
\glt `When he had come to Dir, there was one brother that he sent to Sau.' (A:ASH036-7)
\end{exe}

Switch"=topic marked entities are by no means confined to participants in the \isi{clause} or even \isi{noun} phrases. Almost any word or \isi{phrase} can be highlighted and brought to the foreground by \textit{ba}: a~\isi{noun} modifier, as in (\ref{ex:14-111}) and (\ref{ex:14-114}), an~adverbial (\ref{ex:14-112}) or a~\isi{locative} expression (\ref{ex:14-113}). 

\begin{exe}
\ex
\label{ex:14-111}
\gll áa kúṛi ǰabá wée teeṇíi biǰéel-i dhi-á  tasíi heensíl-im de \textbf{áa} \textbf{phalúuṛ-u} \textbf{ba} putr de\\
\textsc{idef} woman grass in \textsc{refl} several-\textsc{f} daughter-\textsc{pl} \textsc{3sg.gen} stay.\textsc{pfv"=fpl} \textsc{pst} \textsc{idef} single-\textsc{msg} \textsc{top} son be.\textsc{pst}\\
\glt `A woman had with her on the lawn all her daughters and a \textit{single} son.' (A:BRE001)

\ex
\label{ex:14-112}
%modified
\gll \textbf{típa} \textbf{ba} ma kanáa bh-úum \\
now \textsc{top} \textsc{1sg.nom} like.what become-\textsc{1sg } \\
\glt `Now then, what will become of me.' (A:MAA017)

\ex
\label{ex:14-113}
%modified
\gll \textbf{díiš-a} \textbf{ba} baalbač-á kuṛíina táma  th-éen de \\
village-\textsc{obl} \textsc{top} child-\textsc{pl} woman.\textsc{pl} waiting do-\textsc{3pl} \textsc{pst } \\
\glt `[Back] in the village, children and women were waiting.' (B:AVA218)

\ex
\label{ex:14-114}
%modified
\gll dúu oostaaz-aán hín-a o \textbf{čuurbhišá} \textbf{ ba} kuṇaak-á hín-a asíi iskuúl \\
two teacher-\textsc{pl} be.\textsc{prs"=mpl} and forty  \textsc{top} child-\textsc{pl} be.\textsc{prs"=mpl} \textsc{1pl.gen} school  \\
\glt `There are two teachers, and \textit{forty} children in our school.' (A:OUR011)
\end{exe}

This includes clauses in complex constructions. Same"=\isi{subject} clauses, as in (\ref{ex:14-115})--(\ref{ex:14-116}), as well as different"=\isi{subject} clauses with adverbial functions, as in (\ref{ex:14-117})--(\ref{ex:14-118}), can be marked with \textit{ba} (see \sectref{sec:13-4}).

\ea
\label{ex:14-115}
%modified
\gll \textbf{teewiz-í} \textbf{the} \textbf{ba} se bhalaa-ɡaán ma  díi ṣeekóol-u \\
amulet-\textsc{pl} do.\textsc{cv} \textsc{top} \textsc{def} evil.spirit-\textsc{pl} \textsc{1sg.nom} from lead.out.\textsc{pfv"=msg}{\protect\footnotemark} \\
\glt `Once he had made amulets, he saved me from these evil spirits.' (A:HUA131)

\ex
\label{ex:14-116}
%modified
\gll \textbf{aḍaphará} \textbf{whaí} \textbf{ba} damá thíil-u \\
halfways come.down.\textsc{cv} \textsc{top} rest do.\textsc{pfv"=msg } \\
\glt `Halfway down then, we rested.' (A:GHA057)

\ex
\label{ex:14-117}
\gll \textbf{phooṭóol-u} \textbf{ta} \textbf{ba} ɡhueeṇíi-am maníit-u ki ni bíiḍ-a zinaawúr
xálaka hín-a\\
break.\textsc{pfv"=msg} \textsc{sub} \textsc{top} Pashtun-\textsc{pl.obl} say.\textsc{pfv"=msg}  \textsc{comp} \textsc{3pl.prox.nom} much-\textsc{mpl} wild people be.\textsc{prs"=mpl}\\
\glt `After [the Ashretis had been] breaking [the beam], the Pashtuns said, ``These people are very wild.''' (A:CHA008)

\ex
\label{ex:14-118}
%modified
\gll \textbf{phedóol"=ii} \textbf{pahúrta} \textbf{ba} hukumát xabaár  bhíl-u\\
arrive.with.\textsc{pptc"=gen} after \textsc{top} government informed become.\textsc{pfv"=msg}\\
\glt `After that [they had brought it there], the government found out about it.' (A:GHA08)
\z

\footnotetext{It is not entirely clear why the verb shows singular agreement here even though the \isi{direct object} is in the plural.}