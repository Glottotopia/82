\chapter{Adverbs and postpositions}
\label{chap:7}

\section{Adverbs}
\label{sec:7-1}


Hardly surprising, adverbs and adverbial expressions comprise a~heterogenous category in Palula. Only a~few adverbs are entirely non"=derived (whether synchronically or diachronically speaking), and it is a~relatively small word class, consisting of a~number of subclasses, many of which are made up of closed rather than open sets. Most spatial and temporal adverbs are \isi{pronominal} or nominal, most manner adverbs are verbal and some members of the small subclass of degree adverbs are also used as modifiers of nouns. Many spatial adverbs are also used as postpositions. Still, as described in \sectref{subsec:6-3-4}, some adjectives are derived from the small group of non"=derived (primarily calendrical) adverbs.


\subsection{Symmetrical \isi{adverb} sets}
\label{subsec:7-1-1}

The systematic differentiation between the categories proximal, \isi{distal}, \isi{remote} and inde\isi{finite}"=\isi{interrogative} among pronouns (and other pro"=forms; see \sectref{subsec:5-2-1}) is to some extent carried over to and partly overlapping with certain sets of adverbial pro"=forms. Although the symmetry is not complete, the relevant forms are given in \tabref{tab:7-1}, to be easily identified when referred to in the sections to follow.



For the spatial adverbs (also referred to as ``adverbial demonstratives'' or ``locational deictics'', \citealt[431]{diessel2006}), all four categories are represented, and as with the pronouns, there are in most cases two forms available in each category: a~neutral (or highly accessible) form, and an~emphasised form (used for referring back to something less accessible in the \isi{discourse}, or correlatively, see \sectref{sec:13-6}), the latter with an~initial \textit{ee-}. As not to clutter the table more than necessary, only the emphasised forms are included (e.g., emphasised \textit{eetáa} corresponds to neutral \textit{táa}), and only the relevant A dialect forms of these adverbs (there is a~regular correspondence between A \textit{ee-} and B \textit{ha-}; A \textit{ɡóo}~-- B \textit{ɡúu}; ablative forms A \textit{-áai}~-- and B \textit{-áauu}.)



Used adverbially, the \isi{oblique} forms (also functioning as \textsc{3sg} \isi{ergative} pronouns) are always used
along with a~positional or directional \isi{postposition} (hence the hyphen), see \sectref{subsec:7-2-2}, the only
exception being the inde\isi{finite}"=\isi{interrogative} \textit{kíi}, which can be used alone meaning
`where to'. The \isi{locative} and ablative can also be followed by a~directional \isi{postposition}, but can
also stand alone as spatial adverbs. As will be further discussed and exemplified in \sectref{sec:7-2}, the
members of the spatial set may also take on a~temporal interpretation along with certain
postpositions.


\begin{table}[ht]
\caption{Symmetrical \isi{adverb} sets}
\begin{tabularx}{\textwidth}{ l Q Q Q Q }
\lsptoprule
&
Proximal &
Distal &
Remote &
Inde\isi{finite}"=\isi{interrogative}\\\midrule
Spatial &
&
&
&
\\
--~\isi{locative} &
\textit{índa} &
\textit{eeṛáa} &
\textit{eetáa} &
\textit{ɡóo} \\
&
`here' &
`there (visible)' &
`there (invisible)' &
`where'\\
--~\isi{oblique} &
\textit{aníi-} &
\textit{eeṛíi-} &
\textit{eetíi-} &
\textit{kíi-} \\
&
`here' &
`there (visible)' &
`there (invisible)' &
`where'\\
--~ablative &
\textit{andóoii, indóoii} &
\textit{eeṛáai} &
\textit{eetáai} &
\textit{ɡóoii} \\
&
`from here' &
`from there (visible)' &
`from there (invisible)' &
`from where'\\
Degree &
&
&
\textit{eetí} &
\textit{katí} \\
&
&
&
`so, such' &
`how, how much'\\
Manner &
&
&
&
\\
--~in\isi{transitive} &
&
&
\textit{eendáa=bhe} &
\textit{kanáa=bhe} \\
&
&
&
`like that' &
`like what, how'\\
--~\isi{transitive} &
&
&
\textit{eendáa=the} &
\textit{kanáa=the} \\
&
&
&
`like that' &
`like what, how'\\\lspbottomrule
\end{tabularx}
\label{tab:7-1}
\end{table}

The degree adverbs are primarily modifiers of adjectives, whereas the manner adverbs modify entire clauses. The form \textit{eendáa} with the converb form of \textit{bhe-} `become' is used with in\isi{transitive} clauses, while that with \textit{the-} `do' is used with \isi{transitive} clauses. There is no obvious symmetrical correspondence to the inde\isi{finite}"=\isi{interrogative} \isi{temporal adverb} \textit{kareé} `when', but the nearest functional equivalent is \textit{eetheél} `then, at that time'. 

\subsection{Spatial adverbs}
\label{subsec:7-1-2}

Most adverbial expressions with spatial semantics are either \isi{pronominal}/\isi{deictic} or nominal in nature or derived from those word classes, often playing a~syntactic role similar to that of \isi{noun} phrases and postpositional phrases.

\subsubsection*{The \isi{deictic} set \textit{índa~-- eeṛáa~-- eetáa~-- ɡóo}}

As mentioned above, most of these adverbial deictics are related to demonstratives, and some may be
used pronominally as well as adverbially. Typically they describe location, such as \textit{índa} `here', \textit{eeṛáa} `there (further away but still visible),
\textit{eetáa} `there' (not visible), \textit{ɡóo} `where', the latter primarily
interrogatively. It may, beside its primary static reading, as in (\ref{ex:7-1}), also imply goal, as in
(\ref{ex:7-2}).

\begin{exe}
\ex
\label{ex:7-1}
%modified
\gll áaḍ-a \textbf{índa} bhíl-a áaḍ-a ba naaṛéy-a  the ɡíia \\
half-\textsc{mpl} here become.\textsc{pfv"=mpl} half-\textsc{mpl} \textsc{top} Narey-\textsc{obl}  to  go.\textsc{pfv.pl}  \\
\glt `Some remained here, while others went to Narey.' (A:ANC005)

\ex
\label{ex:7-2}
%modified
\gll \textbf{táa} yhaí ba ɡóo hín-u bhraapútr  thaníit-u \\
there come.\textsc{cv} \textsc{top} where be.\textsc{prs"=msg} nephew  say.\textsc{pfv"=msg} \\
\glt `When he came there, he said, ``Where is he, nephew?''' (A:PAS058)
\end{exe}
The members of the series \textit{andóoii~-- eeṛáai~-- eetáai~-- ɡóoii} have an~ablative function,
expressing a~movement away from a~location, as in example (\ref{ex:7-3}), or an~origin in a~certain
location.

\begin{exe}
\ex
\label{ex:7-3}
%modified
\gll \textbf{eetáai} wheel-í ɡhróom-a phedóol-u \\
from.there take.down-\textsc{cv} village-\textsc{obl} bring.\textsc{pfv"=msg} \\
\glt `Getting it [the corpse] down from there, we brought it to the village.' (A:GHA075)
\end{exe}
The members of the \isi{locative} series as well as of the ablative may be further specified (for goal or
source) or emphasised by a~following \isi{postposition}, and for some postpositions, such as \textit{wée}
in (\ref{ex:7-4}), the members of the \isi{oblique} series have to be used instead of those of the
\isi{locative} or ablative series (for further examples, see \sectref{subsec:7-2-2}).

\begin{exe}
\ex
\label{ex:7-4}
%modified
\gll ṣiṣ ba \textbf{aníi=wee} bi buṭ-í \textbf{aníi=wee}  bi buṭ-í \\
head \textsc{top} \textsc{3sg.prox.obl=}in also plait-\textsc{cv} \textsc{3sg.prox.obl=}in also plait-\textsc{cv} \\
\glt `As for his hair, it was braided into many braids.' (A:JAN069)
\end{exe}

\subsubsection*{Various spatial adverbs}

\spitzmarke{\textit{huṇḍ} `up'.} This essentially nominal \isi{adverb} is used for a~location situated in a~more or less vertical upward direction (not necessarily visible) from the point of reference. Used alone it can refer to a~location `up (there)', as in (\ref{ex:7-5}), as well as a~movement to such a~location.

\begin{exe}
\ex
\label{ex:7-5}
%modified
\gll \textbf{huṇḍ} ta c̣hiítr=ee \textbf{bhun} ba ɡhaawaáz de  \\
above \textsc{cntr} field=\textsc{cnj} below \textsc{top} stream.bed be.\textsc{pst} \\
\glt `The field was up above and the sand and stream were down below.'\newline (A:JAN082)
\end{exe}
From \textit{huṇḍ}, a~number of other semantically related adverbs or adverbial phrases can be derived: \textit{huṇḍ the/húṇṭe} `up (to)', as in (\ref{ex:7-6}); \textit{huṇḍíi/huṇḍíi thíi} `from above'; \textit{huṇḍɡiraá} (A), example (\ref{ex:7-7}); or \textit{huṇṭeɡiráak/huṇṭráak} (B) `upward, uphill', \textit{huṇḍaṛáa} ({\textless} \textit{huṇḍ + eeṛáa}) `up there'.

\begin{exe}
\ex
\label{ex:7-6}
%modified
\gll sum \textbf{huṇḍ} \textbf{the} ṣuɡal-í ba čo ba  thaníit-u \\
soil up to throw-\textsc{cv} \textsc{top} go go.\textsc{imp.sg} say.\textsc{pfv"=msg} \\
\glt `He threw soil up [in the air] and said, ``Go on!''' (A:PIR037)

\ex
\label{ex:7-7}
%modified
\gll \textbf{huṇḍɡiraá} dac̣h-íin ta iṇc̣ muṭ-íi phúṭi ǰhulí bheš-í áaṇc̣-a kha-áan-u \\
upward  look-\textsc{3pl} \textsc{sub} bear tree-\textsc{gen} top on sit-\textsc{cv} raspberry-\textsc{pl} eat-\textsc{prs"=msg} \\
\glt `They were looking up and saw the bear sitting in the top of the tree eating raspberries.' (A:KAT145)
\end{exe}

\spitzmarke{\textit{bhun} `down'.} This essentially nominal \isi{adverb} is used for a~location situated downward or below the point of reference. Used alone it can refer to a~location, `down (there)' as in (\ref{ex:7-5}), as well as a~movement to such a~location, as in (\ref{ex:7-8}). The following adverbs or adverbial phrases are examples of derivations from \textit{bhun}: \textit{bhún the/bhuná} `down (to)', \textit{bhuníi thíi} `from below', \textit{bhunɡiraá} (A) or \textit{bhunteɡiráak/bhuntráak} (B) `downward', \textit{bhunimaá} `(on the way) downhill', exemplified in (\ref{ex:7-9}), \textit{bhunaṛáa} (and other \isi{deictic} forms based on it) `down there'.

\largerpage
\begin{exe}
\ex
\label{ex:7-8}
%modified
\gll so méeš hatí maǰí-e uṭik-í \textbf{bhun}  whaí ba uḍheew-í wháat-u\\
\textsc{dem.msg.nom} man \textsc{3sg.rem.obl} in-\textsc{gen} jump-\textsc{cv} down
come.down.\textsc{cv} \textsc{top} flee-\textsc{cv} come.down.\textsc{pfv"=msg} \\
\glt `Meanwhile, the man jumped down [from the tree] and fled down [to the village].' (B:CLE377)

\ex
\label{ex:7-9}
%modified
\gll tuúš \textbf{bhunimaá} wh-íi ta karáaṛu  ukh-áand-u \\
some downhill  come.down-\textsc{3sg} \textsc{sub} leopard come.up-\textsc{prs"=msg} \\
\glt `Getting some ways downhill, [he meets] the leopard coming up.'\newline (A:KAT095)
\end{exe}

\spitzmarke{\textit{aǰá} (B \textit{aǰé}) `up (there)'.} This essentially \isi{deictic} \isi{adverb} is used for a~location situated upstream or in a~slightly upward location from the point of reference (compare with \textit{huṇḍ} which implies a~place more or less straight above the point of reference). Used alone it can refer to a~location, `up/over (there), upstream', as well as a~movement to such a~location, as in (\ref{ex:7-10}). Although not quite as productive as \textit{huṇḍ} and \textit{bhun}, this \isi{adverb} is similarly used as a~building block for other adverbs and adverbial expressions, such as \textit{aǰimaá} `upward', \textit{aǰaṛáa} `up/over there', \textit{aǰí/aǰí thíi} (B) `from upstreams/up"=country'.

\begin{exe}
\ex
\label{ex:7-10}
%modified
\gll yáab ɡhaš-í \textbf{aǰá} ɡúum ta  \\
canal take-\textsc{cv} up go.\textsc{pfv.msg} \textsc{sub} \\
\glt `I went up along the [irrigation] canal.' (A:HUA053)
\end{exe}

\spitzmarke{\textit{túuri} `down (there)'.} This \isi{adverb}, otherwise mostly used as a~\isi{postposition} (see \sectref{subsec:7-2-2}), is only occasionally used without a~\isi{noun} or \isi{pronoun} as an~argument. It seems to be contrasting with the aforementioned \isi{adverb} in describing a~location or movement downstream from the point of reference, such as in (\ref{ex:7-11}). It may also be followed by the \isi{postposition} \textit{wée} (see \sectref{subsec:7-2-2}). 

\begin{exe}
\ex
\label{ex:7-11}
%modified
\gll dúe xálaka ba \textbf{túuri} dhrúuk-a ǰe ɡéa  \\
other people \textsc{top} down gorge-\textsc{obl}  down.to go.\textsc{pfv.pl} \\
\glt `The other people went down into the valley.' (B:AVA205)
\end{exe}

\spitzmarke{\textit{rhalá} `up (above, on top)'.} This adverbial expression, related to the \isi{adjective} \textit{raál} (primarily occurring as as a~\isi{host element} in conjunct verbs with meanings such as `raise, lift up'), is used for describing the position on top of a~structure or object, as in example (\ref{ex:7-12}). In some sentences its function may rather be analysed as that of a~\isi{manner adverb} or a~\isi{postposition}.

\largerpage
\ea
\label{ex:7-12}
%modified
\gll dúi hiimaál whaí ṭópa traác̣ de asaám  \textbf{rhalá} ɡaḍíl-a\\
other glacier come.down.\textsc{cv} down \textsc{host} give.\textsc{cv} \textsc{1pl.acc} up take.out.\textsc{pfv"=mpl}\\
\glt `Another avalanche struck came and brought us out [from under the snow].' (A:ACR015)
\z

\spitzmarke{\textit{awaaɡír} `up high, at a~high elevation'.} This adverbial expression (possibly also adjectival) seems to be exclusively used for referring to a~high elevation in the mountains, as in (\ref{ex:7-12b}).

\begin{exe}
\ex
\label{ex:7-12b}
%modified
\gll paṇáar-u táapaṛ thaní \textbf{awaaɡír} qaribán dúu thiáaḍ-a míil-a táapaṛ hín-u\\
white-\textsc{msg} hill \textsc{quot} high.up about two and.half-\textsc{mpl} mile-\textsc{pl} hill be.\textsc{prs"=msg}\\
\glt `There is a hill called the White Hill about two and a half kilometers up.' (A:HUA117)
\end{exe}

\spitzmarke{\textit{phará} `yonder, over there'.} This \isi{deictic} \isi{adverb} (probably based on a~spatial
\isi{noun} *\textit{phaár} or *\textit{paár}, compare with \textit{aḍaphaár} `halfways' below) is used
for a~location some distance away but still visible (and in a~straight line) from the point of
reference, as in (\ref{ex:7-13}), often in contrast with \textit{oóra} `on this side, over here'
(see below). Like some of the other spatial adverbs above, it is used as a~building block for other
adverbs and adverbial expressions: such as \textit{pharimaá} (A)/\textit{pehrimaá} (B)
`(some ways) forward, onward', \textit{pharaṛáa} `over there, yonder', \textit{pharíi
  thíi} (A)/\textit{parí thíi} (B) `from some distance away', \textit{pharaɡiraá}
`at some distance'. The \isi{deictic} expression \textit{pharaṛáa} may be further modified by vowel
lengthening: \textit{phaaraṛáa}, implying a~location even further away than
\textit{pharaṛáa}. (Compare with the \isi{postposition} \textit{pharé}, \sectref{subsec:7-2-2}).

\begin{exe}
\ex
\label{ex:7-13}
%modified
\gll ɡhoortsí bi so áa kučúru bi uḍheew-í [oóra]    šan-á so šay bi tanaám sanɡí yhaí šan-á, ma ba \textbf{phará} khilaí dharíit-u \\
Ghortsi also \textsc{def.msg.nom} one dog also flee.\textsc{cv} over.here roof-\textsc{obl} \textsc{dem.msg.nom} thing also \textsc{3pl.acc} with come.\textsc{cv} roof-\textsc{obl} \textsc{1sg.nom} \textsc{top} yonder alone remain.\textsc{pfv"=msg}\\
\glt `Both Ghortsi [the name of a dog] and the other dog fled from that thing and went back to the roof over here. The thing followed them onto the roof, but I stayed by myself, away from there.' (A:HUA031-3)
\end{exe}

\spitzmarke{\textit{oóra} `on this side, over here'.} As already mentioned and exemplified in (\ref{ex:7-13}), this \isi{deictic} \isi{adverb} is used in contrast with \textit{phará}, implying a~location close to the point of reference as opposed to a~location further away. I am not aware of any derivations from this \isi{adverb}. 

\spitzmarke{\textit{patú} `behind, in back'.} This basic form is, according to my material, only used as a~postpostion (in A) or \isi{conjunction} (in B), see \sectref{subsec:7-2-2}; however, a~couple of (partly spatial, partly temporal) adverbial expressions are clearly derived from it: \textit{patuɡiraá} (exemplified in (\ref{ex:7-14}))\textit{/patuɡiróo/paturaá} (A)/\textit{patuɡiráak} (B) `back', \textit{padúši ({\textless} patú + dúši)} `behind', as in (\ref{ex:7-15}), the latter almost exclusively used in a~temporal sense in B (compare with \sectref{subsec:7-1-3}).

\ea
\label{ex:7-14}
%modified
\gll se páand \textbf{patuɡiraá} na lháay-a de  \\
\textsc{3fsg.nom} path back  \textsc{neg} find-\textsc{3sg} \textsc{pst}\\
\glt `It did not find its way back.' (A:CAV020)

\ex
\label{ex:7-15}
%modified
\gll bakáara ta muṣṭú bhe whéet-i=ee  be ba \textbf{padúši} wh-áand-a \\
flock \textsc{cntr} in.front become.\textsc{cv} come.down.\textsc{pfv-f=cnj} \textsc{1pl.nom} \textsc{top} behind come.down-\textsc{prs"=mpl}\\
\glt `The flock got down before us, and then we came down behind.' (A:PAS052)
\z

\spitzmarke{\textit{muṣṭú} (A only) `in front, ahead, forward'.} As with the aforementioned adverbial group, \textit{muṣṭú} has in addition to its basic spatial meaning, as in (\ref{ex:7-16}), acquired some largely temporal aspects (see \sectref{subsec:7-1-3}). A nearly synonymous \textit{muṣṭuɡiraá} is also derived from it. 

\begin{exe}
\ex
\label{ex:7-16}
%modified
\gll tuúš \textbf{muṣṭú} bíi ta áak luumái ḍhoó dít-i hín-i \\
some in.front go.\textsc{3sg} \textsc{sub} \textsc{idef} fox sight give.\textsc{pfv-f} be.\textsc{prs-f}\\
\glt `Going some distance forward, he saw a~fox.' (A:KAT011)
\end{exe}

\spitzmarke{\textit{nhiáaṛa} (B \textit{niháaṛa}) `near, nearby'.} This \isi{adverb} is often used in a~so"=called compound \isi{postposition} (\sectref{subsec:7-2-3}) but may also be used alone, see (\ref{ex:7-17}), as a~\isi{spatial adverb}.
\begin{exe}
\ex
\label{ex:7-17}
%modified
\gll tasíi ba páanǰ ṣo bhraawú \textbf{nhiáaṛa} hóons"=an de \\
\textsc{3sg.gen} \textsc{top} five six brother.\textsc{pl} near stay-\textsc{3pl} \textsc{pst}\\
\glt `Five or six of her brothers were living nearby.' (A:HUA119)
\end{exe}

\spitzmarke{\textit{dhúura} `far away'.} Like the aforementioned \isi{adverb}, it is often used in a~compound \isi{postposition} (\sectref{subsec:7-2-3}), but may also be used independently (\ref{ex:7-17b}).

\begin{exe}
\ex
\label{ex:7-17b}
%modified
\gll \textbf{dhúura} áa ɣar-í učát ɣar-í eeṛáa piír hín-u \\
far.away \textsc{idef} peak-\textsc{obl} high peak-\textsc{obl} there.\textsc{rem} pir be.\textsc{prs"=msg}\\
\glt `Far away, on the top of a high mountain, there is a pir [a holy person].' (A:PIR014)
\end{exe}


\spitzmarke{\textit{šíiṭi} `inside'.} This \isi{adverb}, also used as a~simple \isi{postposition}
(\sectref{subsec:7-2-2}), as well as in postpositional sequences (\sectref{subsec:7-2-4}), may be derived diachronically from
a~\isi{noun} formerly denoting `house' or `home'\footnote{For `house' as well as `home',
  \textit{ɡhoóṣṭ} is used in today's Palula.} and a~\isi{postposition} `to' (compare with \iliKalkoti
\textit{šíi} `house', \textit{šíiti} `inside, into the house'), or it may have been borrowed
as a~whole or partly from a~variety of Kohistani (compare with \iliGawri \textit{šiṭ} `house, home',
\textit{šiṭši} `inside', \textit{ši} `in, into', \citealt[119]{baart1997}; \citeyear[76]{baart1999a}). It is mostly used
dynamically, as in example (\ref{ex:7-18}), as a~\isi{spatial adverb}.

\begin{exe}
\ex
\label{ex:7-18}
%modified
\gll ma \textbf{šíiṭi} be tes sanɡí madád th-áam \\
\textsc{1sg.nom} inside go.\textsc{cv} \textsc{3sg.acc} with help do-\textsc{1sg}\\
\glt `I will go inside and help him.' (B:FOY060)
\end{exe}

\spitzmarke{\textit{aḍaphaár} `halfway, midway'.} Composed of \textit{*aḍa {\textgreater} áaḍa} `half' and \textit{*phaar} (see above) `yonder', it is used in this essentially nominal form for the movement up to a~point somewhere right between the point of origin and an~expected goal, but it may occur in an~\isi{oblique} form \textit{aḍaphará} with no obvious difference in meaning, a~form to which the \isi{postposition} \textit{tií} (\sectref{subsec:7-2-2}) can be added, rendering the ap\isi{proximate} meaning `as far as halfway'. It can also occur in the \isi{genitive} \textit{aḍapharíi} `from the middle', as seen in example (\ref{ex:7-19}). 
\begin{exe}
\ex
\label{ex:7-19}
%modified
\gll \textbf{aḍapharíi} huṇḍ the ta ǰaláṣ bhun the ba lhíst-u \\
from.middle up to \textsc{cntr} hairy down to \textsc{top} bald-\textsc{msg}\\
\glt `It was half hairless and half covered in hair. The top half was hairy and the bottom half was hairless.' (A:HUA075)
\end{exe}

\subsection{Temporal adverbs}
\label{subsec:7-1-3}

Like spatial adverbs, many adverbial expressions with temporal semantics are nominal in nature, but there are also a~number of synchronically non"=derived temporal adverbs.

\subsubsection*{General \isi{deictic} adverbs}

\spitzmarke{\textit{típa} `now'.} This freqently used and synchronically non"=derived \isi{adverb} is
used for referring to the present moment, as illustrated in (\ref{ex:7-20}), as well as to
`nowadays' in general. There is also a~rather little used form \textit{tipaán tií} `until
now, even to this day'. Interestingly `from now, after this' is expressed with the proximal
ablative member of the spatial set, \textit{andóoii pahúrta}.

\begin{exe}
\ex
\label{ex:7-20}
%modified
\gll \textbf{típa} ba ma tasíi paalaweeṇíi qiseé tháan-u \\
now \textsc{top} \textsc{1sg.nom} \textsc{3sg.gen} strongmanship.\textsc{gen} story.\textsc{pl} do.\textsc{prs"=msg}\\
\glt `Now I am telling the stories about his strongmanship.' (A:PAS029)
\end{exe}

\spitzmarke{\textit{muṣṭú} (A)/\textit{muxáak} (B) `before, in the past, once'.} The A \isi{adverb} has spatial functions (see \sectref{subsec:7-1-2}) along with its temporal ones. Both the A and the B \isi{adverb} may take additional modifiers, such as degree adverbs (\textit{bíiḍu muṣṭú} `long before, a~long time ago') or calendrical expressions (\textit{dúu yúuna muxáak} `two months ago'). The meaning `since then, for a~long time' can be expressed by \textit{muṣṭúi niiɡiraá}. A sentence with the B \isi{adverb} \textit{muxáak} is given in (\ref{ex:7-21}).

\begin{exe}
\ex
\label{ex:7-21}
%modified
\gll \textbf{muxáak} be iskuul-í the b-íia de \\
before  \textsc{1pl.nom} school-\textsc{obl} to go-\textsc{1pl} \textsc{pst}\\
\glt `Once, we were going to school...' (B:ANG001)
\end{exe}

\spitzmarke{\textit{eetheél/(eesé) waxtíi} `at that time, in those days, then'.} Usually, for reference to a~specific time (usually in the past), as in (\ref{ex:7-22}), the \isi{genitive} of the \isi{noun} \textit{waxt} (or \textit{waqt}) `time' is used with a~preceding \isi{demonstrative} (if referred to contextually) or another identifying modifier (\textit{kufurdóore waxtíe} `in the pagan era' B), but also the synchronically non"=derived \isi{adverb} \textit{eetheél}, possibly more widely used in the past, is still in use, as can be seen in example (\ref{ex:7-23}). The latter can refer to a~point of time in the past as well as in the future (e.g., in \textit{eetheél tií} `before that time, up to that time'). The \isi{genitive} of the \iliEnglish loan \textit{ṭeém} `time' is used in a~way quite similar to \textit{waxtíi}, taking various modifiers, such as \textit{basandíi ṭeemíi} `at spring time'.

\begin{exe}
\ex
\label{ex:7-22}
%modified
\gll neečíir ba eesé \textbf{waxt-íi} bíiḍ-i \\
hunting \textsc{top} \textsc{rem} time-\textsc{gen}  much-\textsc{f} \\
\glt `Hunting was a common custom at that time.' (A:HUA046)

\ex
\label{ex:7-23}
%modified
\gll \textbf{eetheél} maidaaní ǰanɡ de \\
that.time  of.field  war be.\textsc{pst } \\
\glt `In those days there used to be fighting in the fields.' (A:PIR005)
\end{exe}


\spitzmarke{\textit{aakatí waxtí (maǰí)/waxt bhe/padúši} (B) `later, after some time'.} A number of expressions, some of them containing forms of the \isi{noun} \textit{waxt}, are used when referring to a~later point in time, in (\ref{ex:7-24}) with the converb of \textit{bhe-} `become' (compare manner adverbs formed with converbs, \sectref{subsec:7-1-4}). Only in the B variety is \textit{padúši} used in a~clearly temporal sense (see \sectref{subsec:7-1-2}).

\begin{exe}
\ex
\label{ex:7-24}
%modified
\gll khéli \textbf{waxt} \textbf{bhe} dac̣h-íi ta \\
quite.some time become.\textsc{cv} look-\textsc{3sg} \textsc{sub}\\
\glt `Some time later he was looking.' (A:SMO016) 
\end{exe}

\spitzmarke{\textit{waxtíi thíi} `early'.} Again, the \isi{noun} \textit{waxt} is used, now in the \isi{genitive} and with the directional \isi{postposition} \textit{thíi}. Also the \isi{adverb} \textit{raɣáṣṭi} has been noted in B.


\spitzmarke{\textit{kareé} `when, what time'.} This is a~member of the series of
inde\isi{finite}"=\isi{interrogative} pronouns and other pro"=forms (see \sectref{subsec:13-4-1} and
\sectref{subsec:14-2-2}).



Adjectival derivations with \textit{-úk-} are also used in expressing similar temporal"=\isi{deictic} propositions, e.g., \textit{tipúku} `of now, ``nowish''', \textit{muṣṭúku} `of the past, of old, ``oldish''', as seen in (\ref{ex:7-25}), \textit{eetheelúku} `of that time' (see \tabref{tab:7-2} on p.~\pageref{tab:7-2}). 

\begin{exe}
\ex
\label{ex:7-25}
%modified
\gll \textbf{muṣṭúk-a} xálak-a dhii-á díi na khooǰ-óon de \\
of.past-\textsc{mpl} people-\textsc{pl}  daughter-\textsc{obl} from \textsc{neg} ask-\textsc{3pl} \textsc{pst}  \\
\glt `People in the old days were not asking their daughter [who she wanted to marry].' (A:MAR018)
\end{exe}

\subsubsection*{Calendrical adverbs}

The most common calendrical adverbs (\tabref{tab:7-cal}) used deictically are synchronically non"=derived. In (\ref{ex:7-26}), the use of \textit{dhoóṛ} `yesterday' is illustrated.


\begin{table}
\caption{A selection of basic calendrical adverbs}
\begin{tabularx}{\textwidth}{ l@{\hspace{20pt}} Q l@{\hspace{20pt}} Q }
\lsptoprule
\textit{aáǰ} &
`today' &
\textit{heeṇṣúka} &
`this year'\\
\textit{dhoóṛ} &
`yesterday' &
\textit{páar} (B \textit{par)} &
`last year'\\
\textit{eetríli} &
`the day before yesterday' &
\textit{triimbarúṣ} (B \textit{trimbaríṣ)} &
`two years ago'\\
\textit{trúnǰi} (B) &
`the day after tomorrow' &
\textit{bhióol} &
`last night'\\\lspbottomrule
\end{tabularx}
\label{tab:7-cal}
\end{table}


\begin{exe}
\ex
\label{ex:7-26}
%modified
\gll \textbf{dhoóṛ} índa kir dít-u de típa  bi kir hín-u \\
yesterday  here snow give.\textsc{pfv"=msg} \textsc{pst} now  also snow be.\textsc{prs"=msg} \\
\glt `It was snowing here yesterday, and even now there is snow.'\newline (A:CHE070320)
\end{exe}

These adverbs also behave morphologically more like spatial \isi{deictic} adverbs than nouns, when expressing notions such as `before, until' or `from, since'. While the basic form is used with postpositions otherwise taking \isi{oblique} arguments, an~\isi{inflection} \textit{-ii} or an~idiosyncratic \textit{-oo} (compare with B \textit{hatáwuu} `from there' and related forms) with an~ablative function is used with \textit{pahúrta} and \textit{niiɡiraá}. Some multi"=word calendrical expressions are displayed in \tabref{tab:7-calpost}.


\begin{table}
\caption{Calendrical expressions involving postpositions}
\begin{tabularx}{\textwidth}{ l@{\hspace{20pt}} Q P{30mm} Q }
\lsptoprule
\textit{aáǰ tií} &
`until today' &
\textit{aaǰíi niiɡiraá, aaǰíi pahúrta} &
`from today, after today'\\
\textit{dhoóṛ tií} &
`until yesterday' &
\textit{dhoóṛoo niiɡiraá} &
`since yesterday'\\
\textit{eetríli tií} &
`until the day before yesterday' &
\textit{eetríloo niiɡiraá} &
`since the day before yesterday'\\\lspbottomrule
\end{tabularx}
\label{tab:7-calpost}
\end{table}


Although \textit{heeṇṣúka} `this year' diachronically may be an~adjectivally derived form (with \textit{-úk-}), it is used like the other non"=derived \isi{deictic} adverbs, as in (\ref{ex:7-27}), without modifying a~\isi{noun} head. 

\begin{exe}
\ex
\label{ex:7-27}
%modified
\gll \textbf{heeṇṣúka} ma čiiríit-u \\
this.year  \textsc{1sg.nom}  be.delayed.\textsc{pfv"=msg} \\
\glt `This year I'm delayed.' (A:SHY026)
\end{exe}

Belonging semantically to this group is also \textit{rhootašíia} (with the alternative forms \textit{rhoošíia} and \textit{rhoošée}) `tomorrow', used as in (\ref{ex:7-28}), which, however, is the \isi{oblique} case of the \isi{noun} \textit{rhootašíi} `morning'.

\begin{exe}
\ex
\label{ex:7-28}
%modified
\gll ma nis aáǰ kh-úum ta \textbf{rhootašíi-a} ba kanáa bh-úum\\
\textsc{1sg.nom} \textsc{3sg.acc} today eat-\textsc{1sg} \textsc{sub} morning-\textsc{obl} \textsc{top} like.what become-\textsc{1sg}\\
\glt `If I eat this today, what will then become of me tomorrow.' (A:HUB005)
\end{exe}

As with the general \isi{deictic} adverbs, adjectival forms in \textit{-úk-} can be productively derived from most, if not all, of these calendrical adverbs: \textit{aaǰúku} `of today', today's', \textit{bhiaalúku} `of/from last night', \textit{parúku} `of/from last year, last year's', etc. 


The identity of most calendrical cyclic expressions (\tabref{tab:7-2}), on the other hand, is clearly nominal, although the \isi{oblique} case forms can be said to have been lexicalised as adverbs. In (\ref{ex:7-29}), the use of \textit{heewandá, róota} and \textit{deesá} can be seen.


\begin{table}[t]
\caption{Calendrical cyclic expressions}
\begin{tabularx}{\textwidth}{ Q l l Q }
\lsptoprule
Nominative (basic) &
&
Oblique (adverbial) &
\\\midrule
\textit{deés} &
`day' &
\textit{deesá} &
`during the day, on the day'\\
\textit{róot, raát} (B \textit{raát)} &
`night' &
\textit{róota} (B \textit{rúuta)} &
`at night'\\
\textit{basaánd} (B \textit{basán)} &
`spring' &
\textit{basandá} &
`in the spring'\\
\textit{béeriṣ} &
`summer' &
\textit{béeriṣa} &
`in the summer'\\
\textit{šaraál} (B \textit{šarál)} &
`autumn' &
\textit{šaralá} &
`in the autumn'\\
\textit{heewaánd} (B \textit{heewán)} &
`winter' &
\textit{heewandá} &
`in the winter'\\\lspbottomrule
\end{tabularx}
\label{tab:7-2}
\end{table}

 
\ea
\label{ex:7-29}
\gll \textbf{heewand-á} tas the \textbf{róot-a} c̣hóoṇ  lama-áan-a \textbf{dees-á} har-í wíi
pila-áan-a tas šišáwi=the šuí zhay-íim ɡhin-í ɡir-áan-a\\
winter-\textsc{obl } \textsc{3sg.acc} to night-\textsc{obl} oak.twigs  hang-\textsc{prs"=mpl} day-\textsc{obl} take.away-\textsc{cv} water give.to.drink-\textsc{prs"=mpl} \textsc{3sg.acc}  beautiful=do.\textsc{cv} good.\textsc{f} place-\textsc{pl.obl} take-\textsc{cv} turn-\textsc{prs"=mpl} \\
\glt `In the winter we hang oak"=branches for her [the goat] during the night, and in the day we take her to drink plenty of water in beautiful places.' (A:KEE090-1)
\z

While the above \isi{oblique} forms of the calendrical nouns express a ``temporal location'', a~couple of these and a~number of other temporal nouns (such as \textit{kaál} `year', \textit{yúun} `month', \textit{haftá} `week') may be used in various forms and derivations to express temporal quantities or frequences. The examples (\ref{ex:7-30})--(\ref{ex:7-39}) all constitute adverbial phrases in Palula.

\begin{exe}
\ex
\label{ex:7-30}
\gll har deés  \\
every day \\
\glt `every day' 

\ex
\label{ex:7-31}
\gll dees-íi \\
day-\textsc{gen} \\
\glt `daily' 

\ex
\label{ex:7-32}
\gll dáaš panǰíiš reet-íi \\
ten fifteen night-\textsc{gen} \\
\glt `every ten to fifteen days'

\ex
\label{ex:7-33}
\gll čáar reet-í padúši \\
four night-\textsc{obl} behind  \\
\glt `four days later' (B)

\ex
\label{ex:7-34}
\gll bhiš=raatúku\\
twenty=of.night    \\
\glt `for twenty days'

\ex
\label{ex:7-35}
\gll daš reet-í tií \\
ten night-\textsc{obl} until  \\
\glt `for as long as ten days' (B)

\ex
\label{ex:7-36}
\gll tróo yúun"=ii baád \\
three month-\textsc{gen} after \\
\glt `after three months'

\ex
\label{ex:7-37}
\gll bhiš kaal-á maxadúši \\
twenty year-\textsc{pl} in.front  \\
\glt `20 years earlier/ago'

\ex
\label{ex:7-38}
\gll ṣo kaal"=íi niiɡiraá \\
six year-\textsc{gen} since  \\
\glt `since six years'

\ex
\label{ex:7-39}
\gll dúi kaal-á the \\
other year-\textsc{obl} to \\
\glt `another year, next year'
\end{exe}

Most time"=of"=day expressions are nouns, quite a~few of them loans from \iliPashto. They are like the
calendrical cyclic ones above in that they occur in their \isi{oblique} case form when functioning as
time"=of"=day adverbs. Two exceptions are \textit{rhoošnaám} `morning' (in (\ref{ex:7-40})) and
\textit{dhrumanaám} `mid"=afternoon', which are synchronically non"=derived adverbs.

\begin{exe}
\ex
\label{ex:7-40}
%modified
\gll \textbf{dhrumanaám} ba pašambeé ta c̣hóoṇ-ii bháaru ɡhin-í ma ba ɡúči \\
afternoon \textsc{top} Pashambi \textsc{cntr} oak-\textsc{gen} load take-\textsc{cv} \textsc{1sg.nom} \textsc{top} free \\
\glt `In the afternoon Pashambi carried fodder [to the goats] while I was free.' (A:PAS051)
\end{exe}

\subsubsection*{Other temporal adverbs}

A few other adverbs with a temporal meaning, all poly"=morphemic, are shown in \tabref{tab:7-advx}.

\begin{table}
\caption{Other temporal adverbs}
\begin{tabularx}{\textwidth}{ l@{\hspace{20pt}} Q l@{\hspace{20pt}} Q }
\lsptoprule
\textit{aáǰkal} &
`nowadays' &
\textit{ǰim=ǰím} &
`all day'\\
\textit{luu=lúu} &
`all night' &
\textit{raat=uu=deés} (B) &
`day and night'\\\lspbottomrule
\end{tabularx}
\label{tab:7-advx}
\end{table}

\subsection{Manner adverbs}
\label{subsec:7-1-4}

Manner is mainly expressed by \isi{nonfinite} verb forms, primarily the converb and the copredicative participle. The former of those is the basis for what can be considered a~\isi{derivation} of manner adverbs (or its nearest equivalent) from other parts of speech, particularly from adjectives. 

\subsubsection*{Non"=derived manner adverbs}

There is a~small class of non"=derived manner adverbs. Some of those are shown in \tabref{tab:7-mann}. 


\begin{table}
\caption{A selection of non"=derived manner adverbs}
\begin{tabularx}{\textwidth}{ l@{\hspace{20pt}} Q l@{\hspace{20pt}} Q }
\lsptoprule
\textit{bhraáš} &
`slowly' &
\textit{lap} &
`quickly, fast'\\
\textit{táru} &
`quickly, soon' &
\textit{khilaí} &
`alone'\\\lspbottomrule
\end{tabularx}
\label{tab:7-mann}
\end{table}

\begin{exe}
\ex
\label{ex:7-41}
%modified
\gll ǰanɡibaazxáan-a \textbf{bhraáš} šukhaáu ɡaḍ-í čhúuṇ-u \\
Jangibaz.Khan-\textsc{obl} slowly coat take.off-\textsc{cv} put.\textsc{pfv"=msg}  \\
\glt `Slowly, Jangibaz Khan took off his coat.' (A:JAN066)
\end{exe}

\subsubsection*{The \isi{deictic} set \textit{eendáa=bhe~-- kanáa=bhe~-- eendáa=the~--
    kanáa-the}} %\edcomm{Please check the hierarchical rank of this heading.}\authcomm{Yes, fine}}

This series exemplifies the \isi{derivation} of manner adverbs. Here a~\isi{deictic} \isi{manner adverb} (or adverbial \isi{phrase}) is formed by adding the converb form of a~\isi{verbaliser} (\textit{the-} `do' or \textit{bhe-}
`become') to the \isi{deictic} \isi{adjective}\footnote{This can also be analysed as a~pro"=form of the host
  element in a~\isi{conjunct verb} (which does not necessarily need to be an~\isi{adjective}) or a~\isi{predicate}
  \isi{phrase} (whether nominal or adjectival).} \textit{eendáa} (B \textit{handáa} or the alternative
non"=emphatic form \textit{andáa}). For modifying an~in\isi{transitive} \isi{predicate}, as in (\ref{ex:7-42}),
\textit{bhe} is used, for modifying a~\isi{transitive} \isi{predicate}, as in (\ref{ex:7-43}), \textit{the} is
used, and for questioning the manner by which something is done, the inde\isi{finite}"=\isi{interrogative}
\textit{kanáa} and the relevant converb is used (see \sectref{subsec:14-2-2}).

\ea
\label{ex:7-42}
%modified
\gll \textbf{andáa=bhe} praš-í wée baṭ-á wh-áand-a \\
like.that=become.\textsc{cv} slope-\textsc{obl} in stone-\textsc{pl} come.down-\textsc{prs"=mpl} \\
\glt `Just like that, stones came down the slope.' (A:AYB008)
\z

\ea
\label{ex:7-43}
%modified
\gll dharéndi mháalu dac̣h-áaṭ-u bhíl-u hín-u  siɡréṭ dhrak-í ba \textbf{andáa=the} \\
outside father look-\textsc{ag"=msg} become.\textsc{pfv"=msg} be.\textsc{prs"=msg} cigarette pull-\textsc{cv} \textsc{top} like.that=do.\textsc{cv}  \\
\glt `While looking at his father outside, he was smoking the cigarette like that.' (A:SMO009)
\z

The clue to the interpretation of the \isi{deictic} \isi{manner adverb} is usually found in the immediate context of the utterance, either explicitly in the wider \isi{discourse}, sometimes rendering `thus' a~good translation equivalent, or by extralinguistic means, such as gestures by the speaker. 


\subsubsection*{Derivation of manner adverbs}

The \isi{derivational} process described above is quite productively applied to words from various parts
of speech (primarily adjectives and pronouns) in forming manner adverbial expressions, as
illustrated in \tabref{tab:7-3}. To what extent these derivations are adverbs, adverbial phrases or
adverbial (mini-)clauses is still open to further analysis, but considering their embedded status
and the fact that they form a~phonological word, we tentatively define them as derived
adverbs. Neither is the relationship between \isi{Converb} clauses with conjunct verbs and these manner
adverbials entirely clearcut. An exact parallel to this formation is described by
\citet[219]{schmidtkohistani2008} for Kohistani \iliShina.


\begin{table}[ht]
\caption{Examples of manner adverbial derivation}

\begin{tabularx}{\textwidth}{ P{17mm} l l Q Q }
\lsptoprule
Adjective, etc. &
&
&
Derived manner adverbial &
\\\midrule
\textit{šúi} &
`good' &
{\textgreater} &
\textit{šúi=the, šúi=bhe} &
`well'\\
\textit{šišáwi} &
`beautiful' &
{\textgreater} &
\textit{šišáwi=the, šišáwi=bhe} &
`completely'\\
\textit{teéz} &
`strong' &
{\textgreater} &
\textit{teéz=bhe} &
`hard, with force'\\
\textit{tíiṇu} &
`sharp' &
{\textgreater} &
\textit{tíiṇu=bhe} &
`carefully, intently'\\
\textit{buṭheé} &
`all' &
{\textgreater} &
\textit{buṭheé=bhe, buṭheé=the} &
`all of them, together'\\\lspbottomrule
\end{tabularx}
\label{tab:7-3}
\end{table}


Also ideophonic expressions with no specific or standardised lexical source, such as those in (\ref{ex:7-44}), can be made manner adverbs this way. (The adverbs in the \iliEnglish translation in (\ref{ex:7-44}) are only ap\isi{proximate} equivalents of the \isi{Converb} cum reduplicated ideophone complexes.\footnote{A widespread covariation of so"=called periphrastic `do'-constructions and reduplicated stems has been noted by \citet{jaeger2006}.})

\begin{exe}
\ex
\label{ex:7-44}
\gll se insaan-á ráaǰ mučaá o muṭ-á sanɡí  so amzarái seéb
     \textbf{ram"=raám=the} \textbf{kaš-kaáš=the} ɡhaṇḍ-í ɡaíl-u\\
\textsc{def} man-\textsc{obl} rope open.\textsc{cv} and tree-\textsc{obl} with \textsc{dem.msg.nom} lion Sir \textsc{red}-``firm''[\textsc{idph}]=do.\textsc{cv} \textsc{red}-``tight''[\textsc{idph}]=do.\textsc{cv} tie-\textsc{cv} throw.\textsc{pfv"=msg}\\
\glt `The man opened a~rope and tied Lord Lion firmly and tightly to the tree.' (A:KIN023)
\end{exe}

\subsection{Degree adverbs}
\label{subsec:7-1-5}
The function of degree adverbs is mostly to modify adjectives, but some of them also function as modifiers of other adverbs as well as of nouns. 

\subsubsection*{Non"=derived degree adverbs}

Forming a~small class, non"=derived degree adverbs are mainly used as modifiers of other adverbs but also as quantifiers of nouns, especially non"=count nouns. The list in \tabref{tab:7-deg} is not meant to be exhaustive. See (\ref{ex:7-9}) for an~example with \textit{tuúš}. 


\begin{table}
\caption{Non"=derived degree adverbs}
\begin{tabularx}{\textwidth}{ l@{\hspace{30pt}} Q l@{\hspace{30pt}} Q }
\lsptoprule
\textit{tuúš} &
`some, little' &
\textit{khéli} &
`several, somewhat'\\\lspbottomrule
\end{tabularx}
\label{tab:7-deg}
\end{table}

\subsubsection*{The \isi{deictic} pair \textit{eetí} (B \textit{hatí})~-- \textit{katí}}%\edcomm{Please    check the hierarchical rank of this heading.}\authcomm{Yes, fine}

These \isi{deictic} degree adverbs intensify adjectives, meaning `such, so', as in (\ref{ex:7-45})\textit{,} and `what, how', as in (\ref{ex:7-46}). (Note that \textit{katí} in particular also functions as a~direct quantifying modifier of a~\isi{noun} head, meaning `how much, how many'.)

\begin{exe}
\ex
\label{ex:7-45}
%modified
\gll baṣ \textbf{hatí} teéz bhíl-u ooḍháal whéet-i \\
rain such strong become.\textsc{pfv"=msg} flood come.down.\textsc{pfv-f} \\
\glt `The rain became so strong that it flooded.' (B:FLO169)
\end{exe}
\begin{exe}
\ex
\label{ex:7-46}
%modified
\gll nu ba \textbf{katí} utháal-u táapaṛ \\
\textsc{3msg.prox.nom} \textsc{top} how.much high-\textsc{msg}  hill  \\
\glt `What a~high hill!' (A:HLE3117)
\end{exe}

\subsubsection*{Reduplication}

Reduplication is one strategy applied for degree modification of manner adverbs. The process can be
full \isi{reduplication}, as seen in (\ref{ex:7-47}), or \isi{reduplication} of the first \isi{syllable}. In example
(\ref{ex:7-48}) both full and first"=\isi{syllable} \isi{reduplication} are being used.

\begin{exe}
\ex
\label{ex:7-47}
%modified
\gll wháat-a andáa=bhe \textbf{bhraáš} \textbf{bhraáš} \\
come.down.\textsc{pfv"=mpl} like.that=become.\textsc{cv} \textsc{red} slowly  \\
\glt `We came down like that, very slowly.' (A:GHA056)
\end{exe}
\begin{exe}
\ex
\label{ex:7-48}
%modified
\gll mhaamaǰaán ba \textbf{la-láp} \textbf{la-láp} khóo de \\
Mahmad.Jan \textsc{top} \textsc{red}-fast \textsc{red}-fast eat.\textsc{3sg} \textsc{pst } \\
\glt `Mahmad Jan was eating very fast.' (A:MAH044)
\end{exe}

\subsubsection*{Co-lexicalised intensifiers}

There is a~number of more or less standard compounds with an~\isi{adjective}/\isi{adverb} and a~matching intensifying
element, not much different from the effect
other degree adverbs have on the modified constituent. Such an~\isi{intensifier} is either uniquely occurring with a~particular \isi{adjective}/\isi{adverb}, or occurs only with a limited set of adjectives/adverbs. It seems those elements are mostly made up of a~single closed \isi{syllable}, as can be seen in \tabref{tab:7-col}.


\begin{table}
\caption{Examples of co"=lexicalised intensifiers}
\begin{tabularx}{\textwidth}{ l@{\hspace{20pt}} Q l@{\hspace{20pt}} Q }
\lsptoprule
\textit{phaṣ paṇáaru} &
`white as a~sheet' &
\textit{tap c̣hiṇ} &
`\isi{pitch} dark'\\
\textit{kham kiṣíṇu} &
`\isi{pitch} black' &
\textit{bak práal} &
`shining bright'\\
\textit{čáu lhóilu} &
`bright red' &
\textit{ḍanɡ khilayí} &
`all alone'\\
\textit{tak zeṛ} &
`bright yellow' &
\textit{čap mhoóru} &
`extremely sweet'\\
\textit{pak kaantíiru} &
`mad as a hat' &
\textit{šam šidáalu} &
`ice"=cold'\\
\textit{pak bíidri} &
`completely clear' &
\textit{šam níilu} &
`deep green/blue'\\\lspbottomrule
\end{tabularx}
\label{tab:7-col}
\end{table}


Strikingly similar compounds have been observed in several other languages in the region, some of them even involving similar or identical forms as those found in Palula: e.g., in \iliDameli \citep[163]{perder2013} and \iliKhowar (Elena Bashir, pc, and own field notes).

\subsection{Sentence adverbs}
\label{subsec:7-1-6}
Sentence adverbs (\tabref{tab:7-sadv}) modify entire utterances. They may for instance specify the speaker's attitude toward the event referred to, or emphasise its truth"=value, as in (\ref{ex:7-48b}). 


\begin{table}
\caption{A selection of sentence adverbs}
\begin{tabularx}{\textwidth}{ l@{\hspace{20pt}} Q l@{\hspace{20pt}} Q }
\lsptoprule
\textit{ɡóo} &
`perhaps, maybe' &
\textit{inšaalaáh} &
`God willing'\\
\textit{rištaá} &
`really, indeed' &
\textit{mheerabeení=the} &
`kindly'\\\lspbottomrule
\end{tabularx}
\label{tab:7-sadv}
\end{table}

\begin{exe}
\ex
\label{ex:7-48b}
%modified
\gll so ba \textbf{rištaá} xučháii so máamu \\
\textsc{3msg.nom} \textsc{top} really Khush.Shah.\textsc{gen} \textsc{def.msg.nom} uncle \\
\glt `He was indeed Khush Shah's uncle.' (A:JAN056)
\end{exe}

\section{Postpositions}
\label{sec:7-2}

All regularly used adpositions in Palula are postpositions. Most of them have spatial"=temporal functions, but a~few express some grammatical as well as spatial"=temporal functions. Whenever a~full \isi{noun} is followed by a~\isi{postposition} it occurs in the \isi{oblique} case (for those nouns that display a~distinct \isi{oblique} case), except for a~couple of postpositions that take a~full \isi{noun} \isi{genitive} argument. When a~\isi{pronoun} is followed by a~\isi{postposition}, there is a~slightly higher degree of case differentiation (this will be commented on under each individual \isi{postposition} presented) as compared to nouns followed by postpositions.


Apart from simple single word postpositions (\sectref{subsec:7-2-2}), there are two types of complex postpositions, compound postpositions (\sectref{subsec:7-2-3}) and postpositional sequences (\sectref{subsec:7-2-4}). Some of the postpositions also function as adverbs (see above) or as heads of adverbial phrases. 


\subsection{Postpositions vis-à-vis case inflection}
\label{subsec:7-2-1}


Case \isi{inflection} has been treated elsewhere (see \sectref{sec:4-5}), but since the differentiation between this category and adpositions (both following nouns) is not always obvious, particularly in IA languages, this should be commented on briefly. 


One line of argumentation is phonological, where a~string of segments with its own \isi{pitch accent} would be considered a~separate word, thus a~\isi{postposition} and not an~\isi{inflection}. This would definitely place the longer strings, i.e., those with two or more syllables, in the \isi{postposition} category. The shorter, monosyllabic ones, however, are phonologically weak, and tend to cliticise to the preceding \isi{noun} word or \isi{pronoun}. Another line of argument is morphological, by which case"=inflections occur close to the stem, whereas postpositions combine with already case"=inflected forms of the \isi{noun} (and mostly with non"=\isi{nominative} forms of pronouns), and thus are more peripheral. Using this argument, the \isi{genitive} morpheme ends up as somewhat ambiguous in this respect (see \sectref{subsec:4-5-3}).


A third line of argumentation may be more helpful, a~mainly syntactic one, with the coordinating suffix or clitic \textit{ee} as a~diagnostic tool (see \citealt[77]{baart1999a}, for his analysis on case marking in \iliGawri). While \isi{inflectional} suffixes obligatorily occur on both of two nouns coordinated with \textit{ee}, a~\isi{postposition} is attached once, to the \isi{coordination} as a~whole, and does not occur inside the \isi{coordination}. According to the latter diagnostics, the \isi{genitive} along with the other case inflections (as described in \sectref{sec:4-5}) are clearly distinguished from the postpositions (as described below). 

\largerpage[-1]
Many of the postpositional functions described below, especially those of the more central simple postpositions, are expressed by what are clearly \isi{inflectional} case"=suffixes rather than free postpositions  in closely related Kohistani \iliShina \citep[115--130]{schmidtkohistani2001}. 


\subsection{Simple postpositions}
\label{subsec:7-2-2}

The order of the following presentation, without any claim of total comprehensiveness, reflects to a~great extent the relative centrality and frequency of simple postpositions in Palula, proceding from the more central or frequent to the more specialised and less frequent ones. The postpositions \textit{the} and \textit{díi} are very common and express a~number of grammatical as well as spatial"=temporal functions. 


\spitzmarke{\textit{the} `to, for'.} This \isi{postposition} takes an~\isi{oblique} nominal argument, `house' in (\ref{ex:7-50}), or an~\isi{accusative} \isi{pronominal} argument, `them' and `us' in (\ref{ex:7-49}). There is a~wide range of meanings, all having the core semantics of marking the recipient of a~transaction or the goal of a~movement. 
\begin{exe}
\ex
\label{ex:7-49}
%modified
\gll na ta \textbf{tanaám} \textbf{the} dít-i na ba  \textbf{asaám} \textbf{the} dít-i \\
\textsc{neg} \textsc{cntr} \textsc{3pl.acc} to give.\textsc{pfv-f} \textsc{neg} \textsc{top} \textsc{1pl.acc} to give.\textsc{pfv-f} \\
\glt `Neither did they give them [the guns] to them, nor to us.' (A:GHA089)
\end{exe}
\begin{exe}
\ex
\label{ex:7-50}
%modified
\gll ma bhíiru ɡhin-í thíi \textbf{ɡhooṣṭ-á} \textbf{the} yhúum \\
\textsc{1sg.nom} he.goat take-\textsc{cv} \textsc{2sg.gen} house-\textsc{obl} to come.\textsc{1sg} \\
\glt `I'll take a~goat with me and come to your house.' (A:MIT013)
\end{exe}

Apart from marking the recipient in a~typical ``di\isi{transitive}'' \isi{clause}, \textit{the} can also identify a~beneficiary in a~\isi{transitive} \isi{clause} and the recipient of an~abstract entity, such as in the utterance in (\ref{ex:7-51}). 

\begin{exe}
\ex
\label{ex:7-51}
%modified
\gll dhii-á \textbf{mac̣hook-á} \textbf{the} maníit-u... \\
daughter-\textsc{obl} Machoke-\textsc{obl} to say.\textsc{pfv"=msg} \\
\glt `The daughter told Machoke...' (A:MAA016)
\end{exe}

If \textit{the} marks a~goal expressed as a~spatial pro"=\isi{adverb} (also functioning as an~in\-animate/abstract \isi{demonstrative} \isi{pronoun}), the \isi{locative} member of the set, \textit{táa} `there', as in (\ref{ex:7-52}), is used. However, if the goal of a~movement carried out by a~person is another person, the postpositions \textit{khúna} or \textit{kéeči} (see below) have to be used instead.


\begin{exe}
\ex
\label{ex:7-52}
%modified
\gll \textbf{táa} \textbf{the} misrí bulaḍ-eeṇḍeéu \\
there to mason call.for-\textsc{oblg}\\
\glt `A mason is called to there [to see to that].' (A:KAT009)
\end{exe}


Verbal Nouns in complements of permissive predicates as well as Verbal Nouns in \isi{purpose} clauses are also taken as arguments by \textit{the} (see \sectref{subsec:13-5-3} and \sectref{subsec:13-4-2}). Some arguments with \textit{the} are part of the valence pattern of some predicates, particularly non"=\isi{nominative} experiencers (\sectref{subsec:12-2-6}) and objects of some conjunct verbs (\sectref{subsec:12-2-8}). It is also used in some temporal expressions (see \sectref{subsec:7-1-3}) as well as in specifying the direction of spatial adverbs (see \sectref{subsec:7-1-2}). Also, see (\sectref{subsec:7-2-4}), for the use of \textit{the} in postpositional sequences. 


\spitzmarke{\textit{díi} `from, (out) of, than'.} This \isi{postposition} takes an~\isi{oblique} nominal argument, `the Damelis' in (\ref{ex:7-53}), or an~\isi{accusative} \isi{pronominal} argument. There is a~wide range of meanings, all referring to the source of a~transaction or the point of origin/reference.

\begin{exe}
\ex
\label{ex:7-53}
%modified
\gll ṛaním eeṛé riwaayát /{\ldots}/ \textbf{ɡiḍúuč-am} \textbf{díi}  ɡhíin-i hín-i \\
\textsc{3pl.dist.erg} \textsc{dist} tradition {} \iliDameli-\textsc{obl.pl} from take.\textsc{pfv-f} be.\textsc{prs-f}\\
\glt `They have received [lit: taken] this tradition from the Damelis.' \mbox{(A:MIT002-5)}
\end{exe}

A~place as the source or starting point for a~movement is almost always expressed with the \isi{genitive} (along with the \isi{postposition} \textit{thíi}). In contrast, \textit{díi} is primarily postposed to nouns denoting human sources, such as `my grandmother' and `my father' in (\ref{ex:7-54}). The transferred, or brought forth, entity, however, can be abstract, such as an~utterance, as well as concrete, for instance in expressing the animate source of reproduction, as in (\ref{ex:7-55}).

\begin{exe}
\ex
\label{ex:7-54}
\gll \textbf{míi} \textbf{déedi} \textbf{díi} míi ṣúunt-u \textbf{míi} \textbf{báaba} \textbf{díi} míi ṣúunt-u\\
\textsc{1sg.gen} grandmother from \textsc{1sg.gen} hear.\textsc{pfv"=msg} \textsc{1sg.gen} father.\textsc{obl} from \textsc{1sg.gen} hear.\textsc{pfv"=msg}\\
\glt `I heard it from my grandmother, and I heard it from my father.' (A:PAS005)

\ex
\label{ex:7-55}
%modified
\gll \textbf{khar-á} \textbf{díi} khar ǰa-yáan-u \\
donkey-\textsc{obl} from donkey be.born\textsf{-}\textsc{prs"=msg} \\
\glt `A donkey is born by a~donkey.' (B:PRB006)
\end{exe}

Some arguments with \textit{díi} are part of the valence pattern of some complex predicates (\sectref{subsec:12-2-8}), and Verbal Nouns in complements of negative implicative predicates are also taken as arguments by \textit{díi} (\sectref{subsec:13-5-2}). The possessor in one main (primarily alienable, as the `mason's hammer' in (\ref{ex:7-56})) possessional construction is also marked with \textit{díi}.

\begin{exe}
\ex
\label{ex:7-56}
%modified
\gll misrí yhóol-u seentá \textbf{misrí} \textbf{díi} tsaṭák hóons-a \\
mason come.\textsc{pfv"=msg} when mason from hammer stay-\textsc{3sg} \\
\glt `When the mason comes he will have a~hammer [lit: When the mason has come, from the mason a~hammer will be present].'  (A:HOW010)
\end{exe}

Inalienable possession, on the other hand, is often expressed with the \isi{genitive} case (a distribution parallel to that in Kohistani \iliShina, where the \isi{genitive} case similarly is used for inalienable possession while the addessive case \textit{-di} or \textit{-idi} is used for alienable possession, see \citealt[65, 69--70]{schmidtkohistani2008}). The \isi{postposition} \textit{díi} is also used to express the comparative degree of the standard of comparison (\sectref{subsec:6-3-3}).


\spitzmarke{\textit{sanɡí} `(along) with, at'.} This \isi{postposition} takes an~\isi{oblique} nominal argument or an~\isi{accusative} \isi{pronominal} argument. It typically expresses accompaniment, as in (\ref{ex:7-57}).

\ea
\label{ex:7-57}
%modified
\gll ḍaaku"=aan-óom yhaí \textbf{bakáara-m} \textbf{sanɡí} tas ɡhaš-í híṛ-u de \\
robber-\textsc{pl"=obl} come.\textsc{cv} flock-\textsc{obl}  with \textsc{3sg.acc} take-\textsc{cv} take.away.\textsc{pfv"=msg} \textsc{pst} \\
\glt `The robbers came and abducted him along with his flock.' (A:GHA005)
\z

Used with an~inanimate \isi{noun}, such as the `tree' in (\ref{ex:7-58}), it can have a~further connotation of being attached to.

\begin{exe}
\ex
\label{ex:7-58}
%modified
\gll ma tu ráaǰ-a de ɡhaṇḍ-í ɡal-áan-u \textbf{aní} \textbf{muṭ-á} \textbf{sanɡí} \\
\textsc{1sg.nom} \textsc{2sg.nom} rope give.\textsc{cv} tie-\textsc{cv} throw-\textsc{prs"=msg} \textsc{prox} tree-\textsc{obl} with \\
\glt `I will tie you with ropes to this tree.' (A:KIN021)
\end{exe}

Some arguments with \textit{sanɡí} are part of the valence pattern of some in\isi{transitive} verbs with a~postpositional object (\sectref{subsec:12-2-4}) as well as that of some complex predicates (\sectref{subsec:12-2-8}), typically those coding events or actions involving two participants on some sort of equal basis. The \isi{postposition} \textit{sanɡí} may also take as an~argument a~\isi{Verbal Noun} in a~\isi{simultaneity} \isi{clause} (\sectref{subsec:13-4-1}).


\spitzmarke{\textit{ǰhulí} `on (top of), on to, over, about, due to'.} This \isi{postposition} takes an~\isi{oblique} nominal argument or (in most cases) an~\isi{accusative} \isi{pronominal} argument in its essentially spatial sense `on, onto' and an~\isi{oblique} \isi{pronominal} argument in more abstract senses such as `about, concerning'. It typically expresses a~position on the immediate surface of something, such as the `stone' in (\ref{ex:7-59}), or somebody. It is also used for the movement onto the surface or into the position on top of something or somebody.

\begin{exe}
\ex
\label{ex:7-59}
%modified
\gll \textbf{se} \textbf{baṭ-á} \textbf{ǰhulí} se kuṇaak-íi paaṇṭí bi heensíl-i de \\
\textsc{def} stone-\textsc{obl} on \textsc{def}  child-\textsc{gen}  clothes also  stay.\textsc{pfv-f} \textsc{pst } \\
\glt `On the stone were also the child's clothes.' (A:BER012)
\end{exe}


A \isi{pronoun} referring to an~inanimate but still concrete entity can also (optionally it seems) occur in the \isi{oblique} form with this \isi{postposition}, the case otherwise used in \isi{pronominal} reference to abstract entities (as in example (\ref{ex:7-60})). Note that the \isi{oblique} \isi{pronoun} (which is the same as the \textsc{3sg} agent in \isi{ergative alignment}) here in fact is referring to the plural entity \textit{muṣṭookhurá} `forelegs'.

\ea
\ex
\label{ex:7-60}
%modified
\gll muṣṭookhur-á dhraǰaá ba \textbf{tíi} \textbf{ǰhulí} ṣiṣ čhoor-í ba bhéṭ-u\\
foreleg-\textsc{pl} stretch.out.\textsc{cv} \textsc{top} \textsc{3sg.obl} on head put-\textsc{cv} \textsc{top} sit.down.\textsc{pfv"=msg}\\
\glt `Stretching out its forelegs, it put its head on them/there and sat down.' (A:PAS061)
\z

Used with abstract nouns it can encode a~whole range of meanings, some of them probably shading out into idiomatic expressions. The more common ones denote the topic of an~utterance `on, about, concerning', a~reason for something to happen `due to, with that', as in (\ref{ex:7-61}), or the means or attitude by which something is carried out.

\ea
\label{ex:7-61}
%modified
\gll lhooméi \textbf{teeṇíi} \textbf{mákar-a} \textbf{ǰhulí} askúun-a baándi kuhée díi nikhéet-i\\
fox \textsc{refl} cunning-\textsc{obl} on ease-\textsc{obl} on well.\textsc{obl} from appear.\textsc{pfv-f}\\
\glt `The fox easily got out of the well due to his own cunning.' (B:FOX033)
\z

The reason reading is also the usual when \textit{ǰhulí} takes a~\isi{Verbal Noun} as its argument (\sectref{subsec:13-4-3}). In this abstract usage, another postpostion, \textit{baándi}, also in (\ref{ex:7-61}), a~loan from \iliPashto, is alternatively used, particularly in A. Some arguments with \textit{ǰhulí} are part of the valence pattern of some predicates, particularly non"=\isi{nominative} experiencers and objects of some conjunct verbs (\sectref{subsec:12-2-6} and \sectref{subsec:12-2-8} respectively).


\spitzmarke{\textit{maǰí} `among, in, inside, during'.} This \isi{postposition} takes an~\isi{oblique} nominal argument in its spatial (and temporal) sense `in, into, inside', an~animate \isi{oblique} plural nominal or \isi{accusative} plural argument in the sense `among, out of', whereas the \isi{oblique} form is used when its temporal sense is expressed with a~\isi{pronoun}. One of the basic uses of \textit{maǰí} is to single somebody out as part of a~group, as in example (\ref{ex:7-62}), `of them' or `among them'. The argument taken is always a~plural entity or a~collective expression.

\begin{exe}
\ex
\label{ex:7-62}
%modified
\gll \textbf{tanaám} \textbf{maǰí} áak míiš muṭ-á ǰe ukh-áai bhóo de \\
\textsc{3pl.acc} among one man tree-\textsc{obl} up ascend-\textsc{inf}  be.able.\textsc{3sg} \textsc{pst} \\
\glt `Of them only one man was able to climb the tree.' (A:UNF007)
\end{exe}

The other basic use of this \isi{postposition} is to express a~position inside of something, the `bazaar' in (\ref{ex:7-63}), often in a~certain geographical location.

\begin{exe}
\ex
\label{ex:7-63}
%modified
\gll a ɣaríb méeš \textbf{baazúur-a} \textbf{maǰí} teeṇíi kuṇaák bhanǰ-úu de \\
\textsc{idef} poor man bazaar-\textsc{obl} in \textsc{refl} child beat-\textsc{3sg} \textsc{pst} \\
\glt `A poor man was beating his own child in [the middle of] the bazaar.' (B:ANG002)
\end{exe}

It is also used for the movement of something `into' another something, as in (\ref{ex:7-63b}).

\begin{exe}
\ex
\label{ex:7-63b}
%modified
\gll ak muṭ-á wée \textbf{trúu} \textbf{šúunɡ-am} \textbf{maǰí} so mhaás čhúuṇ-u \\
\textsc{idef} tree-\textsc{obl} in three branch-\textsc{pl.obl} in.between \textsc{def.msg.nom} meat put.down.\textsc{pfv-msg} \\
\glt `He placed it between three branches in a tree.' (B:SHB735)
\end{exe}

As mentioned above, \textit{maǰí} can also be used in a~temporal sense, `while, at, during', with an~\isi{oblique} \isi{pronoun} such as in (\ref{ex:7-64}), a~nominal time expression or a~\isi{Verbal Noun} (\sectref{subsec:13-4-1}).

\begin{exe}
\ex
\label{ex:7-64}
%modified
\gll \textbf{tíi} \textbf{maǰí} áa ǰhaṭíl-u ṭhaaṭáaku yhóol-u \\
\textsc{3sg.obl} at \textsc{idef} hairy-\textsc{msg} monster come.\textsc{pfv"=msg} \\
\glt `Meanwhile a~hairy monster came in.' (A:THA005)
\end{exe}

\spitzmarke{\textit{túuri} `under, beneath, below'.} This \isi{postposition} takes an~\isi{oblique} nominal argument, e.g., the `deodar tree' in (\ref{ex:7-65}), or an~\isi{accusative} (alternatively \isi{oblique}) \isi{pronominal} argument. It describes a~position which is in a~purely spatial sense the opposite to that of \textit{ǰhulí}, but it is not particularly frequent in my data. It typically expresses the position in or the movement into a~position lower than or beneath something. It can also be used with abstract nouns such as `agreement' in  (\ref{ex:7-65b}).

\begin{exe}
\ex
\label{ex:7-65}
%modified
\gll karáaṛu \textbf{se} \textbf{loomuṭ-á} \textbf{túuri} yeí ba \\
leopard \textsc{def} deodar.tree-\textsc{obl} under come.\textsc{cv} \textsc{top} \\
\glt `The leopard got in under the deodar tree.' (B:CLE357)

\ex
\label{ex:7-65b}
%modified
\gll \textbf{eesé} \textbf{muaahidá} \textbf{ túuri} bhéṭ-a índi aakatí waxt heensíl-a de\\
\textsc{rem} agreement below sit.down.\textsc{pfv"=mpl} here few time remain.\textsc{pfv"=mpl} \textsc{pst}\\
\glt `They stayed under that agreement and remained here for some time.'\newline (A:MAB014)
\end{exe}

Either an~\isi{oblique} or an~\isi{accusative} \isi{pronoun} can be taken as an~argument of this \isi{postposition} referring to an~inanimate entity, the former exemplified in (\ref{ex:7-66}), and the latter in (\ref{ex:7-67}). Whether this reflects a~dialectal difference or is just an~example of free variation, is a~matter for further study.

\begin{exe}
\ex
\label{ex:7-66}
%modified
\gll karáaṛu \textbf{tíi} \textbf{túuri} bheš-í=n-u \\
leopard \textsc{3sg.obl} under sit.down-\textsc{cv}=be.\textsc{prs"=msg } \\
\glt `The leopard was sitting under it [the tree].' (B:SHB749)
\end{exe}
\begin{exe}
\ex
\label{ex:7-67}
\gll dharíit-u tasíi so wíi tas ba anɡóor-a ǰhulí čhoor-í ba khaṭúur-a d-áan-a \textbf{tas} \textbf{túuri}\\
remain.\textsc{pfv"=msg} \textsc{3sg.gen} \textsc{dem.msg.nom} water \textsc{3sg.acc} \textsc{top} fire-\textsc{obl} on put-\textsc{cv} \textsc{top} log-\textsc{pl} give-\textsc{prs"=mpl } \textsc{3sg.acc} under\\
\glt `The remaining water is then heated by building a fire under it with large pieces of wood.' (A:KEE048)
\end{exe}

\spitzmarke{\textit{wée} `in, on, into'.} This spatial \isi{postposition} takes an~\isi{oblique} nominal argument or an~\isi{oblique} \isi{pronominal} argument. It is exclusively used as a~\isi{postposition} of nouns or pronouns with inanimate referents. In most cases it signals contact with the surface of something, as the fruit put into the bag in (\ref{ex:7-68}), or the penetration into a~location or beneath the surface of an~object, such as the hunter sitting in the tree in (\ref{ex:7-69}), i.e., inside the structure of tree branches. This latter characterisation is basically what differentiates it from the \isi{locative} use of the \isi{oblique} case of a~\isi{noun} or a~\isi{locative} pro"=\isi{adverb} as well as the use of the goal"=specific \isi{postposition} \textit{the} with a~\isi{noun}. It seems, however, that many \isi{locative} nominal expressions can occur almost interchangeably as \isi{oblique} nouns and as nouns followed by \textit{wée}. Perhaps the \isi{postposition} serves a~function of further emphasising the \isi{locative} reading, especially when a~movement is implied.

\ea
\label{ex:7-68}
%modified
\gll tíi so meewá samaṭ-í ba \textbf{booǰée} \textbf{wée} de wheelíl-u\\
\textsc{3sg.obl} \textsc{def.msg.nom} fruit collect-\textsc{cv} \textsc{top} bag.\textsc{obl} in  put.\textsc{cv} take.down.\textsc{pfv"=msg }\\
\glt `He collected the fruit, put it into the bag and brought it down [home].' (A:HUB010)
\end{exe}
\begin{exe}
\ex
\label{ex:7-69}
%modified
\gll so iškaarí méeš \textbf{muṭ-á} \textbf{wée} hín-u \\
\textsc{def.msg.nom} hunter man tree-\textsc{obl} in be.\textsc{prs"=msg} \\
\glt `The man is in the tree.' (B:CLE375)
\z

The \isi{noun} taking this \isi{postposition} may also be an~abstract entity, such as `our school class' in (\ref{ex:7-70}).

\begin{exe}
\ex
\label{ex:7-70}
%modified
\gll \textbf{asíi} \textbf{ǰameet-í} \textbf{wée} áaṣṭ kuṇaak-á hín-a \\
\textsc{1pl.gen} class-\textsc{obl} in eight child-\textsc{pl}  be.\textsc{prs"=mpl} \\
\glt `There are eight children in our class.' (A:OUR010)
\end{exe}


\spitzmarke{\textit{ǰe} `up (into), up (along)'.} This rather specialised spatial \isi{postposition} takes an \isi{oblique} nominal argument or an~\isi{oblique} \isi{pronominal} argument. It is exclusively used with nouns or pronouns with inanimate referents. It is similar to \textit{wée}, in that it signals contact with the surface of something or the penetration into a~location, but it is restricted to an~upward movement, such as that of the hunter making his way up into the tree in (\ref{ex:7-71}), and it occurs mostly with verbs having a~connotation of upward movement. It seems that this \isi{postposition} primarily has a~connotation of movement, but there are examples of it being used in a~static sense as well.

\ea
\label{ex:7-71}
%modified
\gll hasó iškaarí ba bhíilam-e \textbf{loomuṭ-á} \textbf{ǰe} huṇṭráak ukháat-u\\
\textsc{rem.msg.nom} hunter \textsc{top} fear-\textsc{gen}  deodar.tree-\textsc{obl} up upward come.up.\textsc{pfv"=msg}\\
\glt `Due to fear, that hunter climbed into a~deodar tree.' (B:CLE357)
\z



\spitzmarke{\textit{thíi} `from'.} This spatial \isi{postposition} takes a~\isi{genitive} nominal argument, but there is no evidence for any \isi{pronominal} forms occurring with it. It is exclusively used with inanimate nouns (and as a~directional specifier of spatial adverbs, see \sectref{subsec:7-1-2}) and is in many ways used interchangeably with a~\isi{locative} nominal expression in the \isi{genitive} case. However, it may serve a~function of emphasising the movement away from (as it does when used with certain adverbs), whether in a~concrete spatial sense, as in (\ref{ex:7-72}), or in an~extended temporal sense, as in (\ref{ex:7-73}), although the latter is not very common.

\begin{exe}
\ex
\label{ex:7-72}
%modified
\gll ée wíi tu xu \textbf{sóon"=ii} \textbf{thíi} wh-áand-u \\
o water \textsc{2sg.nom} apparently pasture-\textsc{gen} from  come.down-\textsc{prs"=msg}  \\
\glt `O water, I see you are coming down from the high pasture.' (A:SHY047)
\end{exe}
\begin{exe}
\ex
\label{ex:7-73}
%modified
\gll rhoošnaám \textbf{waxt-íi} \textbf{thíi} be c̣híitr-a be kráam th-íia  \\
morning  time-\textsc{gen} from \textsc{1pl.nom} field-\textsc{obl} go.\textsc{cv} work do-\textsc{1pl} \\
\glt `Let's go to the field early in the morning and work.' (A:WOM474)
\end{exe}

\spitzmarke{\textit{khúna/kéeči} `(near) to, with'.} These two synomymous postpositions take an \isi{oblique} nominal argument (\ref{ex:7-75}) or an~\isi{accusative} \isi{pronominal} argument (\ref{ex:7-74}). They are almost exclusively postposed to nouns or pronouns with animate (and particularly human) referents. It can describe the position `near to, at, with' as well as the movement `(near) to'. In the directional meaning, \textit{khúna/kéeči} may also be followed by \textit{the} `to' in a~\isi{postpositional sequence} (see \sectref{subsec:7-2-4}).


\begin{exe}
\ex
\label{ex:7-74}
%modified
\gll amzarái \textbf{tas} \textbf{khúna} yhóol-u hín-u \\
lion \textsc{3sg.acc} near.to come.\textsc{pfv"=msg } be.\textsc{prs"=msg} \\
\glt `The lion approached him.' (A:UNF013)
\end{exe}

\largerpage
\ea
\label{ex:7-75}
%modified
\gll tusím \textbf{lúuɡ-a} \textbf{míiš-a} \textbf{kéeči} ma baasaá míi bheezatí thawéel-i \\
\textsc{2pl.erg} strange-\textsc{obl} man-\textsc{obl} with \textsc{1sg.nom} spend.night.\textsc{cv} \textsc{1sg.gen} disgrace cause.to.do.\textsc{pfv-f} \\
\glt `You disgraced me by making me spend the night with a~stranger.' (A:UXW051)
\z




\spitzmarke{\textit{šíiṭi} `in, inside'.} This \isi{postposition}, which is also used as a~\isi{spatial adverb} (see \sectref{subsec:7-1-2}), takes an~\isi{oblique} nominal argument or an~\isi{oblique} \isi{pronominal} argument. It is exclusively postposed to nouns or pronouns with inanimate referents, particularly those denoting enclosed areas or places, such as the `house' in (\ref{ex:7-76}). It can describe the position `in, inside' as well as the movement `into, inside'.

\begin{exe}
\ex
\label{ex:7-76}
%modified
\gll \textbf{eení} \textbf{ɡhooṣṭ-á} \textbf{šíiṭi} ma seé hín-u  \\
\textsc{prox} house-\textsc{obl}  inside \textsc{1sg.nom} sleep.\textsc{cv} be.\textsc{prs"=msg}  \\
\glt `Inside this very house I was asleep.' (A:HUA014-5)
\end{exe}

In the directional meaning, \textit{šíiṭi} may also be followed by \textit{the} in a~\isi{postpositional sequence} (\sectref{subsec:7-2-4}).


\spitzmarke{\textit{pharé} (\textit{B} \textit{phará}) `along, through, across, over'.} This \isi{postposition} takes an \isi{oblique} nominal argument or an~\isi{oblique} \isi{pronominal} argument. It is exclusively used as a~\isi{postposition} of nouns or pronouns with inanimate referents. It expresses an~outstretched contact of one entity with the surface of another entity or that the two are located parallel to one another. Typically this is the way of expressing path, as in (\ref{ex:7-77}).

\begin{exe}
\ex
\label{ex:7-77}
%modified
\gll hazrát iisa=aleehisalaám \textbf{eesé} \textbf{páand-a} \textbf{pharé} bi-áan-u\\
Lord Isa=peace.be.upon.him \textsc{rem} path-\textsc{obl} along go-\textsc{prs"=msg}\\
\glt `Lord Isa [Jesus], PBUH, came walking along that path.' (A:ABO033)
\end{exe}

However, this \isi{postposition} can capture a~number of slightly different, but still related, positions and movements, such as the bread put on the mouth of the dead man in example (\ref{ex:7-78}). 

\ea
\label{ex:7-78}
%modified
\gll tas mheer-í ɡal-í zaalim"=aan-óom \textbf{dhút-a} \textbf{pharé} ɡúuli bi de ɡíia de \\
\textsc{3sg.acc} kill-\textsc{cv} throw-\textsc{cv} brute-\textsc{pl"=obl} mouth-\textsc{obl} toward bread also put.\textsc{cv} go.\textsc{pfv.pl} \textsc{pst} \\
\glt `The brutes, who had killed him, had also put bread in his mouth and left.' (A:GHA076-7)
\z

\spitzmarke{\textit{dúši/ḍáḍi} `toward, at, in the direction of'.} This \isi{postposition} takes an \isi{oblique} nominal argument or an~\isi{accusative} \isi{pronominal} argument if animate and an \isi{oblique} argument if inanimate and \isi{locative}. While \textit{dúši} is used in A as well as in B, \textit{ḍáḍi} seems to be most common in B. It is used with nouns or pronouns with human or animate referents as well as with \isi{locative} expressions, such as the `village' in (\ref{ex:7-79}). 

\begin{exe}
\ex
\label{ex:7-79}
%modified
\gll \textbf{díiš-a} \textbf{dúši} tilíl-i hín-i \\
village-\textsc{obl} toward walk.\textsc{pfv-f} be.\textsc{prs-f} \\
\glt `She started to walk toward the village.' (A:KAT085)
\end{exe}

\spitzmarke{\textit{maxadúši} (B \textit{muxadúši}) `before, in front of'.} This diachronically complex \isi{postposition} (with the above introduced \textit{dúši} as one of its components) is also used as an~\isi{adverb}. It takes a~\isi{genitive} argument, \textit{kuṇaak-íi} in (\ref{ex:7-80}), which in most cases has a~human referent. The relation expressed is commonly of an~abstract kind. For temporal \isi{precedence} a~compound \isi{postposition} \textit{díi muṣṭú/díi muxáak} `before' is used (\sectref{subsec:7-2-3}).

\begin{exe}
\ex
\label{ex:7-80}
%modified
\gll \textbf{kuṇaak-íi} \textbf{maxadúši} šóo kráam ta th"=eeṇḍeéu \\
child-\textsc{gen}  in.front.of good.\textsc{msg} work \textsc{cntr} do-\textsc{oblg} \\
\glt `One should act properly in front of children.' (A:SMO022)
\end{exe}


\spitzmarke{\textit{patú} `after, behind'.} This \isi{postposition} takes an~\isi{oblique} nominal argument, as in (\ref{ex:7-81}), or an~\isi{accusative} \isi{pronominal} argument. While the same word is almost exclusively used as a~subordinate \isi{conjunction} in B (corresponding to \textit{pahúrta} in A), it primarily expresses a~spatial relationship in A. It is also used in the sense of being `in the pursuit of' something.

\begin{exe}
\ex
\label{ex:7-81}
%modified
\gll pal-í \textbf{áak} \textbf{ṭómb-a} \textbf{patú} \\
hide-\textsc{cv} \textsc{idef} trunk-\textsc{obl} behind \\
\glt `I hid behind a~tree trunk and...' (A:HUA072)
\end{exe}


\spitzmarke{\textit{tií} `until, up to, as far as'.} This \isi{postposition} takes an~\isi{oblique} nominal (and inanimate) argument, `knees' in (\ref{ex:7-83}), or, if referring to a~point in time or space, a~\isi{locative} \isi{spatial adverb}, as seen with \textit{eeṛáa} in (\ref{ex:7-82}). It shows that something extends to a~specific point in space or time. 

\begin{exe}
\ex
\label{ex:7-82}
%modified
\gll \textbf{eeṛáa} \textbf{tií} ta máa=the šiǰrá páta náin-u \\
there.\textsc{dist} up.to \textsc{cntr} \textsc{1sg.nom}=to line knowledge \textsc{neg}.be.\textsc{prs"=msg} \\
\glt `I don't know his line until that point.' (A:ASH015)
\end{exe}
\begin{exe}
\ex
\label{ex:7-83}
%modified
\gll heewand-á \textbf{khúṭ-am} \textbf{tií} kir d-áan-u \\
winter-\textsc{obl} knee-\textsc{obl.pl} as.far.as snow give-\textsc{prs"=msg}  \\
\glt `In the winter the snow reaches up to the knees.' (B:DHN4628)
\end{exe}

\spitzmarke{\textit{dapáara} (A)/\textit{pándee} (B) `for'.} This \isi{postposition}, which expresses \isi{purpose} (if inanimate) or beneficiary (if human), takes a~\isi{genitive} argument. The beneficiary use can be seen in (\ref{ex:7-84}).

\ea
\label{ex:7-84}
%modified
\gll ma ba aní pooštrá abaíim \textbf{thíi} \textbf{dapáara} saat-áan-u=ee\\
\textsc{1sg.nom} \textsc{top} \textsc{prox} fattened she.goat.\textsc{pl} \textsc{2sg.gen} for keep-\textsc{prs"=msg=q}\\
\glt `Do you think I have taken care of these fattened goats for you?' (A:PAS093)
\z

When a~\isi{purpose} is expressed pronominally, it is the form of the pro"=\isi{adverb} that expresses a~movement from a~location (i.e., an~ablative) that is being used, e.g., \textit{eeṛáai} (\ref{ex:7-85}) in A, or \textit{hatáawuu} (\ref{ex:7-86}) in B. The \isi{postposition} \textit{dapáara/pándee} may also take as an~argument a~\isi{Verbal Noun} in a~\isi{purpose} \isi{clause}, (see \sectref{subsec:13-4-2}).

\ea
\label{ex:7-85}
%modified
\gll \textbf{eeṛáai} \textbf{dapáara} muxtalíf muxtalíf teeṇ-teeṇíi saamaán hín-i\\
from.there.\textsc{dist} for different different \textsc{red"=refl}  things be.\textsc{prs-f}\\
\glt `For that a~lot of different things are needed.' (A:HOW018)
\z 

\begin{exe}
\ex
\label{ex:7-86}
%modified
\gll súun-a the \textbf{hatáawuu} \textbf{pándee} har-áan-a \\
pasture-\textsc{obl} to from.there for take.away-\textsc{prs"=mpl} \\
\glt `That is why we take them to the high pasture.' (B:SHC015)
\end{exe}


\subsection{Compound postpositions}
\label{subsec:7-2-3}

Compound postpositions are phrases consisting of one of the simple directional postpositions, \textit{díi} `from, of' and \textit{the} `to', followed by an~\isi{adverb}. Although this list probably is far from exhaustive, this postpositional type does not seem to be extremely common in Palula. It should be noted that they need not occur immediately next to each other.

\spitzmarke{\textit{the nhiáaṛa} `close to, next to'.}

The \isi{spatial adverb} \textit{nhiáaṛa} `near, nearby' (see \sectref{subsec:7-1-2}) occurs with the \isi{postposition} \textit{the} `to', meaning `near to, close to', as in example (\ref{ex:7-87}). 

\begin{exe}
\ex
\label{ex:7-87}
%modified
\gll malikc̣heétr thaní \textbf{ɡhróom-a} \textbf{the} \textbf{nhiáaṛa} áak c̣híitr hín-u \\
Malikchetre \textsc{quot} village-\textsc{obl} to near \textsc{idef} field be.\textsc{prs"=msg} \\
\glt `Near to the village, there is a~field called Malikchetre.' (A:JAN029)
\end{exe}

The \isi{adverb} component in the compound can be further modified, such modifications, in (\ref{ex:7-88}) \textit{tuúš bi}, occurring between the \isi{adverb} and the simple \isi{postposition}.

\begin{exe}
\ex
\label{ex:7-88}
%modified
\gll se \textbf{wíi-a} \textbf{the} \textbf{tuúš} \textbf{bi} \textbf{nhiáaṛa} ɡíi hín-i \\
\textsc{def} water-\textsc{obl} to a.little also near go.\textsc{pfv.f} be.\textsc{prs-f} \\
\glt `She went a~bit closer to the water.' (A:SHY054)
\end{exe}

If the argument is human, e.g., `me' in (\ref{ex:7-89}), a~compound \textit{keeči the nhiáaṛa}, consisting of a~\isi{postpositional sequence} (\sectref{subsec:7-2-4}) and an~\isi{adverb}, is used.

\begin{exe}
\ex
\label{ex:7-89}
%modified
\gll \textbf{ma} \textbf{kéeči} \textbf{the} \textbf{nhiáaṛa} bheš \\
\textsc{1sg.nom} with to near sit.down.\textsc{imp.sg} \\
\glt `Sit down next to me!' (A:AYB029)
\end{exe}

\spitzmarke{\textit{díi muṣṭú/muxáak} `before, prior to'.}

The \isi{temporal adverb} \textit{muṣṭú} (B \textit{muxáak}) `before, in the past' (see \sectref{subsec:7-1-3}) occurs with the \isi{postposition} \textit{díi} `from', meaning `prior to', as in example (\ref{ex:7-91}). The compound \textit{díi muṣṭú/muxáak} also occurs with Verbal Nouns in temporal subordination (\sectref{subsec:13-4-1}). 

\begin{exe}
\ex
\label{ex:7-91}
\gll asíi dóodu c̣hoók \textbf{kaṭur-á} \textbf{díi} \textbf{dúu} \textbf{sóo} \textbf{kaal-á} \textbf{muṣṭú} yhóol-u hín-u\\
\textsc{1pl.gen} grandfather Choke Kator-\textsc{obl} from two hundred year-\textsc{pl} before come.\textsc{pfv"=msg} be.\textsc{prs"=msg}\\
\glt `Our ancestor Choke arrived two hundred years prior to the Kator [dynasty].' (A:ASH047)
\end{exe}


\spitzmarke{\textit{díi dhuúra} `far from, away from'.} The \isi{spatial adverb} \textit{muṣṭú} (B \textit{dhuúra}) `far away' (see \sectref{subsec:7-1-2}) occurs with the \isi{postposition} \textit{díi} `from', meaning `far from, away from', as in example (\ref{ex:7-92}). 

\ea
\label{ex:7-92}
\gll taníi c̣híitr ba \textbf{taníi} \textbf{ɡhooṣṭ-á} \textbf{díi} \textbf{taqriibán} \textbf{tróo} \textbf{čúur} \textbf{kulumiṭer-á} \textbf{dhúura} de\\
\textsc{3pl.gen} field \textsc{top} \textsc{3pl.gen} house-\textsc{obl} from about three four kilometre-\textsc{pl} away.from be.\textsc{pst}\\
\glt `Their field was about three to four kilometres away from their house.' (A:WOM468)
\z

\subsection{Postpositional sequences}
\label{subsec:7-2-4}

Another kind of complex \isi{postposition} consists of a~sequence of two simple postpositions. The first in the sequence is the semantically more central one, whereas the second is a~further fine"=tuning or specification, usually as far as the direction is concerned. Exactly which postpositions can and which cannot be combined in this fashion is a~matter of further research; (\ref{ex:7-93})--(\ref{ex:7-98}) are only a~few illustrative examples. It should be noted that the \isi{noun} preceding the \isi{postpositional sequence} is regularly assigned case by the first component of the sequence. The (nominally derived) \isi{postposition} \textit{dúši}, itself assigning \isi{oblique} case to the preceding \isi{noun} in (\ref{ex:7-96}), receives \isi{genitive} marking from the second postposiion \textit{thíi}.
\largerpage

\begin{exe}
\ex
\label{ex:7-93}
%modified
\gll \textbf{ḍúkur-a} \textbf{šíiṭi} \textbf{the} ɡhin-í ɡíia hín-a \\
hut-\textsc{obl} inside to take-\textsc{cv} go.\textsc{pfv.pl} be.\textsc{prs"=mpl} \\
\glt `They brought him inside the hut.' (A:KAT062)
\end{exe}
\begin{exe}
\ex
\label{ex:7-94}
%modified
\gll so kuṇaák \textbf{se} \textbf{ṭhaaṭáak-a} \textbf{khúna} \textbf{the} ɡúum \\
\textsc{def.msg.nom} child \textsc{def} monster-\textsc{obl} near to go.\textsc{pfv.msg} \\
\glt `The child went over to the monster.' (A:BER003)
\end{exe}
\begin{exe}
\ex
\label{ex:7-95}
%modified
\gll \textbf{tasíi} \textbf{šan-á} \textbf{ǰhulí} \textbf{the} wháat-u \\
\textsc{3sg.gen} roof-\textsc{obl} on to come.down.\textsc{pfv"=msg}  \\
\glt `I reached [got down to] the roof of his house.' (A:HUA091)
\end{exe}
\begin{exe}
\ex
\label{ex:7-96}
%modified
\gll aakatí kasaán nikháat-a \textbf{ɡiḍ-á} \textbf{dúšii} \textbf{\textmd{thíi}} \\
some persons appear.\textsc{pfv"=mpl} Damel-\textsc{obl} toward.\textsc{gen} from \\
\glt `A few people came down, from the Damel side.' (A:JAN043)
\end{exe}
\begin{exe}
\ex
\label{ex:7-97}
\gll ɡulsambér ɡhambúri-m-e bhéṭi wíi \textbf{tesée} \textbf{muxadúši} \textbf{phará} wheelíl-i\\
forest.flower flower-\textsc{pl"=gen} bouquet water \textsc{3sg.gen} in.front.of along bring.down.\textsc{pfv-f}\\
\glt `The water brought down a~bouquet of forest flowers in front of her.'\newline (B:FLW805)
\end{exe}
\begin{exe}
\ex
\label{ex:7-98}
%modified
\gll \label{bkm:Ref193771773}\textbf{ǰeep-í} \textbf{wée} \textbf{yúu} se ɡhambúri-m ɡhaḍíl-im \\
pocket-\textsc{obl} in from \textsc{def} flower-\textsc{pl}  take.out.\textsc{pfv"=fpl}\\
\glt `He brought out the flowers from inside his pocket.' (B:FLW794)
\end{exe}


The second component in the sequence \textit{wée yúu} in (\ref{ex:7-98}) does not occur as an~independent \isi{postposition} but is certainly related to the ablative function of this particular segment \textit{(y)uu/w(uu)} occurring in a~few spatial expressions, especially in the B variety: \textit{hatáa-(w)uu} `from there', \textit{aṛáa-(y)uu} `from there', \textit{índee-(y)uu} `from here', \textit{ɡóo-(y)uu} `from where', but also as \textit{-oo} in some temporal adverbs in A (see \sectref{subsec:7-1-3} above). \citeauthor{schmidtkohistani2001} mention a~rare ablative suffix \textit{-nyuu/nuu/uu} in Kohistani \iliShina with a~very similar distribution (\citeyear[130]{schmidtkohistani2001}).