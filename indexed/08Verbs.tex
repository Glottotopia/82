\chapter{Verbs}
\label{chap:8}

\section{The verb and its properties}
\label{sec:8-1}



Distributionally, the verb normally occurs \isi{clause}"=finally, functioning as the main \isi{predicate} of the \isi{clause}. The verb in (\ref{ex:8-1}) is preceded by a~single argument, a~\isi{subject}, whereas the verb in (\ref{ex:8-2}) is preceded by an~\isi{oblique} (postpositional) object, a~\isi{subject} and a~\isi{direct object} (see \chapref{chap:11}).

\begin{exe}
\ex
\label{ex:8-1}
%modified
\gll hazratǰaán \textbf{uthíit-u} \\
Hazrat.Jan stand.up.\textsc{pfv"=msg} \\
\glt `Hazrat Jan stood up.' (A:GHU025)
\end{exe}
\begin{exe}
\ex
\label{ex:8-2}
%modified
\gll se ɣar-í the asím tas \textbf{phedóol-u} \\
\textsc{def} peak-\textsc{obl} to \textsc{1pl.erg} \textsc{3sg.acc} bring.\textsc{pfv"=msg} \\
\glt `We brought him to the peak.' (A:GHA029)
\end{exe}

Verbs can also function as the \isi{predicate} of a~dependent \isi{clause}, and depending on what type of \isi{clause} this is, the verb occurs as an~inflected \isi{finite} verb (just like in the examples above) or in one of its non"=\isi{finite} forms (see \sectref{sec:13-3}--\sectref{sec:13-6}). Usually, the verb also then is \isi{clause}"=final.


Phonologically, the vast majority of all verb stems in Palula are either mono- or disyllabic, the monosyllabic stem being the most commonly occurring non"=derived stem.\footnote{The distinction between derived and non"=derived stems is not altogether straightforward, as, from a~diachronic perspective, many \isi{transitive} stems that can be considered derived causatives do not have any synchronic non"=\isi{causative} counterparts. For the most part in this work, I consider those as stems in their own right, i.e. non"=derived. I refrain from distinguishing between root and stem, as that would raise more questions than could be answered here.} Verb stems are without exception accented on their final \isi{syllable}, either on a~short vowel or on the first mora of a~long vowel. 


The typical monosyllabic verb stem has a~CVC"=structure, but CVCC, CCVC, CCCVC and VC also occur. A small class of (about ten) monosyllabic verbs has a~CV, or a ~CCV, stem, among them some of the most frequently occurring verbs in the language. These different patterns are exemplified in \tabref{tab:8-mono}. 

\begin{table}[t]
\caption{Examples of mono"=syllabic verb stems}
\begin{tabularx}{\textwidth}{ Q l Q }
\lsptoprule
\textit{buḍ-} &
`cause pain, sting' &
\textbf{CVC}\\
\textit{krin-} &
`sell' &
CCVC\\
\textit{dhraǰ-} &
`be stretched out, grow tall' &
CCCVC\\
\textit{kamb-} &
`shiver' &
CVCC\\
\textit{ač-} &
`enter' &
VC\\
\textit{su-} &
`sleep' &
CV\\
\textit{bhe-} &
`become' &
CCV\\\lspbottomrule
\end{tabularx}
\label{tab:8-mono}
\end{table}
 
There is a~variety of structures among disyllabic verb stems (\tabref{tab:8-poly}), the most common being a~CVCV stem, and for the small group of trisyllabic verb stems, CVCVCV is the most common.

\begin{table}[t]
\caption{Examples of poly"=syllabic verb stems}
\begin{tabularx}{\textwidth}{ Q l Q }
\lsptoprule
\textit{mané-} &
`say' &
\textbf{CVCV} \\
\textit{čooṇṭá-} &
`write, embroider' &
CVCCV\\
\textit{bhanǰá-} &
`beat' &
CCVCCV\\
\textit{ḍhroonké-} &
`bray, moo' &
CCCVCCV\\
\textit{troké-} &
`become thin' &
CCVCV\\
\textit{dhraké-} &
`pull, draw' &
CCCVCV\\
\textit{uṛí-} &
`let out, pour' &
VCV\\
\textit{urbhá-} &
`fly somethinɡ' &
VCCCV\\
\textit{biṣám-} &
`rest' &
CVCVC\\
\textit{sambáṛ-} &
`watch out, defend oneself' &
CVCCVC\\
\textit{aamúuṣ-} &
`forget' &
VCVC\\
\textit{utráp-} &
`run' &
VCCVC\\
\textit{karooṛé-} &
`dig, scratch' &
\textbf{CVCVCV}\\
\textit{tramaké-} &
`fondle, fumble' &
CCVCVCV\\
\textit{badhooṛá-} &
`butt (horns)' &
CVCCVCV\\
\textit{akaṭé-} &
`gather, meet' &
VCVCV\\
\textit{aamburé-} &
`rub, wrinkle' &
VCCVCV\\
\textit{bulooṣṭá-} &
`snatch, fight over' &
CVCVCCV
\\\lspbottomrule
\end{tabularx}
\label{tab:8-poly}
\end{table}


Although verbs form a~major word class in most, if not all, of the world's languages, the way events are encoded varies a~great deal. One effect of this is a~dramatic variation in the number of verbs found in languages. While at one extreme, the main European languages, such as \iliEnglish, can boast 10,000 or more verbs, there are at the other extreme languages in other parts of the world with markedly different lexical structures that manage with minimal verbal systems of 10--40 simple verbs (\citealt[347--348]{viberg1993}, \citeyear[409]{viberg2006}). In the light of that we will try to discern whether Palula has a~verbal structure similar to European languages or should rather be included in the category of languages with minimal verbal systems or at least be said to share some lexical characteristics of either one.



Regardless of the size of the verb lexicon, the twenty most frequent verbs in any one language tend to have some characteristics in common, and a~number of basic meanings coded as verbs are more or less bound to show up here \citep[209]{viberg2006}. The presence of the verbs GO, GIVE, TAKE, MAKE, SEE and SAY, \citet[247]{viberg1993} points out as unmarked, in this sense, occurring as highly frequent verbs in \iliEnglish and in a~number of other European languages as well as in non"=European languages. 



In Palula, too, we find these verbs among the Top Twenty (\tabref{tab:8-1}), if we count \textit{har-} `take away' as roughly corresponding to TAKE and \textit{dac̣hé-} `look' as equivalent to SEE, although in the latter case there is a~more general perception verb \textit{paš-}, glossed as `see', that is 27th in the ranking. \citet[409]{viberg2006} refers to these ``universal'' verbs as \textit{nuclear}, covering such basic semantic domains as motion, possession, production, verbal communication and perception. He also mentions the verbs HIT, COME, KNOW and WANT as possible candidates for this \textit{nuclear} group, although with a~typologically slightly more marked status. In the Palula Top Twenty list we also find HIT and COME, whereas the closest equivalent of WANT shows up in the 21--40 range as \textit{dawá-} `ask for' and KNOW (as a~simple verb) with an~even lower frequency.



Not all the Top Twenty verbs of European languages, such as \iliEnglish, are \textit{nuclear} in the same sense and must therefore be defined as language- or area"=specific. This is the case with BE and HAVE, the first one being the overall most frequent in almost all European languages, and the second a~verb with few parallels outside Europe. In \iliEnglish, the modals \textit{will, can, may, shall} and \textit{must} are all among the twenty most frequent, and in many other European languages the modals CAN and MUST are found in this frequency range \citep[346--349]{viberg1993}. 



A similar tendency can be seen when studying the Palula Top Twenty list. As in European languages,
an~equivalent of BE tops the list by a~wide margin, not wholly surprising considering the solidly
Indo"=European identity of this language. The suppletive and defective verb \textit{hin-} `is', with its invariable past"=\isi{tense} form \textit{de}\footnote{The form \textit{de} is alternatively analysable as a mere past"=\isi{tense} marker, see \sectref{subsec:8-3-12} and \sectref{sec:12-1}, and as such not part of any verb"=specific paradigm.} (itself possibly a~grammaticalisation of a~participle form of GIVE) is the Palula \isi{copula} as well as an~important \isi{auxiliary} participating in the formation of a~number of periphrastic \isi{tense}"=aspect categories. The verb \textit{háans-} `live, exist' is in a~similar manner grammaticalised in one of its uses, mainly ``standing in'' for the defective \textit{hin-} in some of the TMA categories.
\largerpage[-1]

\begin{table}[t]
\caption{Palula Verbs Top Twenty. The twenty most frequent verbs. (The percentage is calculated on occurrence of {finite} verb forms in the text corpus.)}

\begin{tabularx}{\textwidth}{ l@{\hspace{20pt}} Q Q r }
\lsptoprule
&
Palula verb stem &
Ap\isi{proximate} gloss &
\% occurrence\\\midrule
1 &
\textit{hin-} &
`be' &
25.0\\
2 &
\textit{bhe-} &
`become' &
8.2\\
3 &
\textit{the-} &
`do, make' &
7.7\\
4 &
\textit{be-} &
`go' &
5.5\\
5 &
\textit{mané-} &
`say' &
5.0\\
6 &
\textit{háans-} &
`live, exist' &
3.9\\
7 &
\textit{de-} &
`give' &
3.6\\
8 &
\textit{yhe-} &
`come' &
3.6\\
9 &
\textit{thané-} &
`call, say, name' &
2.3\\
10 &
\textit{whe-} &
`get down' &
1.8\\
11 &
\textit{kha-} &
`eat' &
1.7\\
12 &
\textit{nikhé-} &
`appear, get out' &
1.4\\
13 &
\textit{dac̣hé-} &
`look' &
1.2\\
14 &
\textit{har-} &
`take away' &
1.1\\
15 &
\textit{mhaaré-} &
`kill' &
1.1\\
16 &
\textit{ǰe-} &
`hit, beat' &
1.1\\
17 &
\textit{uṛí-} &
`let out, pour' &
0.9\\
18 &
\textit{bheš-} &
`sit down' &
0.9\\
19 &
\textit{čhooré-} &
`put' &
0.8\\
20 &
\textit{khooǰá-} &
`ask' &
0.8\\
\midrule
&&& 77.7\\
\lspbottomrule
\end{tabularx}
\label{tab:8-1}
\end{table}


The two verbal communication verbs \textit{thané-} and \textit{mané-} interact in a~language"=specific way and are, beyond their simple characteristic as utterance"=verbs that take direct"=quote complements \citep[155]{givon2001a}, also grammaticalised in one or more forms as quotatives and hearsay markers, respectively, see \sectref{subsec:9-2-4} and \sectref{subsec:13-5-1}.\footnote{This, however, is not reflected to any greater extent in this frequency count where only \isi{finite} verb forms are taken into account.} The high frequency of the verbs \textit{bhe-} `become', \textit{the-} `do, make', and \textit{de-} `give' reflects a~subareal feature that we will have reason to return to in our discussion below, namely that in addition to their more literal meaning mainly are featured as verbalisers \citep[368]{masica1991} or ``dummy verbs'' in so"=called ``conjunct verbs'' (\citeyear[326]{masica1991}), see \sectref{subsec:8-6-1}. 



Another language"=specific or possibly subarea specific feature can be traced in the presence of at least four verbs coding events close to COME and GO among the Top Twenty: \textit{be-} `go', \textit{yhe-} `come', \textit{whe-} `get down' and \textit{nikhé-} `appear, get out'. While the first two of these four motion verbs are spatially (but not directionally) neutral, the two others include a~spatial specification (along with a~directional neutralisation), \textit{whe-} coding a~movement up"=to"=down, and \textit{nikhé-} a~movement inside"=to"=outside. Another verb belonging to this group, although with respect to frequency at a~much lower ranking, is \textit{ukhé-} `get up', coding a~movement reverse of \textit{whe-}, i.e. down"=to"=up. This tendency finds a~number of parallels in other areas of the lexicon (spatially ``fine"=tuned'' adverbs, demonstratives, postpositions, etc).



Leaving the individual verbs, their lexical"=grammatical characteristics, and the events they code aside, one of the more striking observations we can make has to do with the relative textual verb occurrence within certain frequency ranges. \citet[409]{viberg2006} claims that the twenty most frequent verbs tend to cover close to 50 per cent of the textual frequency of verbs in European languages. He compares that with the language Kalam in Papua New Guinea, with a~total number of simple verbs around 100, of which fifteen verbs account for 90 per cent of the textual occurrences. As indicated in \tabref{tab:8-2}, Palula is located between these two, with close to 80 per cent accounted for by the Top Twenty verbs. That makes it significantly different from the European type, but it is still quite different from languages with minimal verbal systems. 


\begin{table}[ht]
\caption{Palula textual verb occurrence related to frequency ranges}

\begin{tabularx}{\textwidth}{ Q P{50mm} }
\lsptoprule
Frequency range (number of verbs) &
\% \isi{finite} verb occurrence\\\midrule
1--5 (5) &
\phantom{1}51.5\\
6--20 (15) &
\phantom{1}26.2\\
21--40 (20) &
\phantom{1}10.0\\
41--138 (98) &
\phantom{1}12.3\\
All (138) &
100.0\\\lspbottomrule
\end{tabularx}
\label{tab:8-2}
\end{table}


Interestingly, half of the occurrences in text is accounted for by only the five topmost verbs, among them the two most productive verbalisers, \textit{bhe-} `become' and \textit{the-} `do, make', with the other fifteen verbs in the Top Twenty comprising another quarter of all verbs. The following twenty verbs account for a~tenth, while the remaining 100 or so verbs only represent twelve per cent of the total number of verbs occurring in the text corpus. This does not mean that Palula has no more than 138 simple verbs (in fact, I have elicited more or less complete paradigms for nearly 400 verbs\footnote{As simple verbs in this case, I also consider attested stems derived productively with the \isi{valency}"=increasing \textit{á/awá}-suffix and the \isi{valency}"=decreasing \textit{íǰ}-suffix respectively.}), but it suggests that the total number is likely to be in the hundreds rather than in the thousands, and that any verbs beyond these 138 are quite infrequent, although we must not make too hasty conclusions based on a~rather small corpus that is somewhat limited in terms of genre. Can the lexical structure observed for Palula in some ways be related to lexical typology in general? Is it possible to trace any particular areal or subareal features responsible for some of the properties of the lexical structure in Palula? Are we observing the effects of an~ongoing development of the lexical structure in one direction or the other? These are big and interrelated questions, and I only intend to hint at some possible explanations and give suggestions for further research.



\citet[409]{viberg2006} only contrasts two extremes as it seems; on the one hand, we have, from a~European perspective, well"=known languages with large verbal systems, comprising thousands of simple verbs, some with very specialised meanings; on the other, we have languages with minimal systems, comprising less than 100 simple verbs. In the latter case, a~small set of simple verbs are used systematically as building blocks to form complex verbs with more specialised meanings equal to those of semantically specialised simple verbs in the languages of the former type. Palula, however, does not really fit into any of those two extremes. Instead it seems to belong to an~intermediary type. Its verb inventory is quite rich, and some meanings are rather specialised, as we saw above with the spatially specified motion verbs, but we can probably count the total number in the hundreds rather than in the thousands. It shares with the ``minimal type'' the property of specialised event coding by means of complex verbs (or rather, complex predicates). \citet[348]{viberg2006} exemplifies two such strategies, one combining a~\isi{noun} or an~\isi{adjective} with a~verb, and in effect forming new lexical units equal to simple verbs: \isi{noun}+verb {\textgreater} complex \isi{predicate}, or \isi{adjective}+verb {\textgreater} complex \isi{predicate}; another by combining two verbs: verb1+verb2 {\textgreater} complex \isi{predicate}. 



The former strategy (discussed below, \sectref{subsec:8-6-1}) is well attested in Palula, and I will, in line with \citet[326]{masica1991}, refer to this as the \textit{conjunct verb} construction (while in some other traditions it is called a \textit{light} \textit{verb} construction). It is a~productive and easily applicable strategy, especially for verbalising culturally new concepts, and we find, perhaps not surprisingly, a~substantial number of loan words (primarily from \iliUrdu or \iliPashto) in the so"=called ``host'' slot of \textit{conjunct verbs}. The \textit{conjunct verb} as a~phenomenon is neither language"=specific nor area"=specific. Instead it seems to be a~feature (although not exclusively) of languages in a~larger ``macro"=area'', possibly comprising a~large part of Asia. Nevertheless, \citeauthor{masica1991} notes ``impressionistically'' that it does appear to be more common in the northern part of South Asia than in the south, possibly owing something to \iliPersian influence (\citeyear[368]{masica1991}). As already noted above, there are primarily three verbs acting as verbalisers in Palula, each representing a~particular \isi{argument structure} (see \sectref{subsec:12-2-8}). This strategy may very well be on the increase in Palula, partly due to influence from languages of wider communication. Since it is such a~productive strategy, especially for incorporating culturally new concepts \citep[85]{gambhir1993}, it is not too farfetched to assume that once a~model like this has been established even simple verbs with a~specific meaning already existing in the language may be replaced by the more easily accessible \textit{host} + \textit{verbaliser} constructions. So far in Palula, however, it seems to be primarily a~question of creating entirely new vocabulary to suit new situations, such as access to new technology, formal education and acquisition of new knowledge, and only to a~limited extent a~matter of replacing already existing basic vocabulary. 



The latter is of course not in itself a~threat to the Palula inherited lexicon, but it may eventually mean a~rather radical restructuring. An example of this we can see in \iliPersian, which in many ways stands as an ``areal model'' as far as conjunct verbs are concerned. In \iliPersian, these ``new'' verb complexes have been gradually replacing ``old'' simple verbs for the last 700 years, resulting in the existence of numerous parallel complex and simple verbs corresponding to more or less the same verbal concepts. An additional effect of this phenomenon has been to provide a~literary language or prestigious register in which the simple verb is used vis-à-vis an~everyday language in which the corresponding complex \isi{predicate} is applied to a~much higher degree \citep[1369]{follietal2005}. 


\largerpage[-1]
The second strategy mentioned above, through which new predicates can be formed by combining two simple verbs, is not as well attested in Palula (some possible examples will be discussed under \sectref{subsec:8-6-2}), while it is considered a~typical~-- although not exclusive~-- IA feature (\citealt[559]{ebert2006}; \citealt[250--252]{masica2001}). 



As to the structural properties of verbs, all the three morphological markers commonly found on verbs \citep[409]{viberg2006} are present in Palula: TMA markers, agreement markers, and \isi{valency} markers. These will be dealt with in \sectref{sec:8-4}--\sectref{sec:8-5}, but before discussing these \isi{inflectional} categories, some important classes of verb stems will be introduced and exemplified.


\section{Stems and verb classes}
\label{sec:8-2}
  
As far as \isi{inflectional} morphology is concerned, there are two main morphological verb classes in Palula, which I have chosen to refer to as L-verbs~-- an~open, productive and large class~-- and T-verbs~-- a~closed, non"=productive and rather heterogeneous class. Additionally, there is a~small group of verbs with stems that are suppletive to varying degrees (see \tabref{tab:8-3}). Also within each of the main classes there are variations in the \isi{inflectional} paradigms (described in detail below) due to \isi{accent}"=position and the quality of stem vowels.


\begin{table}[b]
\caption{Partial paradigm illustrating stems and main morphological verb classes}

\begin{tabularx}{\textwidth}{ l P{20mm} P{37mm} P{27mm} }
\lsptoprule
& \isi{L-verb}\newline `cross' &
\isi{T-verb}\newline `climb, rise, quarrel' &
Suppletive verb\newline `see'\\\midrule
\isi{Imperfective} stem &
\textit{lanɡ-} &
\textit{ṣač-} &
\textit{paš-} \\
\isi{Present} (\textsc{msg}) &
\textit{lanɡ-áan-u} &
\textit{ṣač-áan-u} &
\textit{paš-áan-u} \\
\isi{Future} (\textsc{3sg}) &
\textit{láanɡ-a}\newline (B \textit{láanɡ-e}) &
\textit{ṣáač-a}\newline (B \textit{ṣáač-e}) &
\textit{páaš-a}\newline (B \textit{páaš-e}) \\
\isi{Imperative} (\textsc{sg}) &
\textit{láanɡ} &
\textit{ṣáač} &
\textit{páaš} \\
Perfective stem &
\textit{lanɡíl-} &
\textit{ṣáat-} &
\textit{dhriṣṭ-} \\
Perfective (\textsc{msg}) &
\textit{lanɡíl-u} &
\textit{ṣáat-u} &
\textit{dhríṣṭ-u} \\
Perfective (\textsc{fsg}) &
\textit{lanɡíl-i} &
\textit{ṣéet-i} &
\textit{dhríṣṭ-i} \\\lspbottomrule
\end{tabularx}
\label{tab:8-3}
\end{table}


All of the \isi{inflectional} suffixes are each associated with either a~\isi{perfective} or an~\isi{imperfective}
stem, as illustrated in the partial paradigm (\tabref{tab:8-3}). For most verbs the \isi{perfective} stem
(stripped of its \isi{perfective} ``\isi{inflection}'') is identical to the \isi{imperfective} stem (see \sectref{subsec:8-4-3}), but
in some of the classes the two are clearly distinguished as separate stems. For a~more consistent
treatment, however, I have chosen to indicate all \isi{perfective} stems in this work inclusive of
morphological perfectivity, regardless of it being present in a~regular or predictable form or as
stem modification/alternation only. Apart from \isi{imperfective}"=\isi{perfective} stem alternations, some
\isi{imperfective} stems show vowel alternations between \isi{imperfective} verb forms with an~\isi{accent}"=bearing
suffix and those \isi{imperfective} verb forms that are formed with an~\isi{accent}"=neutral suffix (as described
in \sectref{subsec:3-4-3} and \sectref{subsec:3-5-1}). What I cite as a~stem (with a~final hyphen) is an~underlying form, mostly
corresponding to the non"=strengthened vowel quality/quantity found with \isi{accent}"=bearing suffixes.


\section{Morphological verb classes}
\label{sec:8-3}
 
The following classification of verbs into different classes is based on formal criteria, taking both stem alternation and \isi{inflectional} allomorphy into consideration. Only a few \isi{inflectional} categories will be exemplified in the present section, especially as indicators of \isi{verb class} membership, whereas a fuller description of verb \isi{inflection} follows in \sectref{sec:8-4}.


\subsection{Consonant"=ending L-verbs}
\label{subsec:8-3-1}


As mentioned above, the L-class, so named because of its \isi{perfective} ending in \textit{íl, óol}, etc., includes the majority of verbs and is also the productive formation insofar as new simple verbs are formed (which, however, is synchronically quite rare). Within this large class, we can discern three important subclasses, each including a~substantial number of verbs, primarily recognised on the basis of suffix vowels (or more correctly the vowels arising from the coalescing of final stem vowel and an~actual suffix vowel). The first is verbs with consonant"=ending stems (\tabref{tab:8-4}). The great majority of them are \isi{intransitive}, and all morphological passives formed with \textit{-íǰ} (see \sectref{subsec:8-5-2}) also belong to this subclass, but a~few non"=derived \isi{transitive} verbs are also found. For \isi{perfective} stems, the \isi{accent} always occurs on the final vowel \textit{í} (of the segment \textit{íl}), which is the single feature setting them apart from the \isi{imperfective} stems. One of the more prominent features of this subclass is the consonant"=ending singular imperative, for most of these verbs identical to the stem itself.


\begin{table} 
\caption{Partial paradigm for consonant"=ending L-verbs}
\begin{tabularx}{.5\textwidth}{ll}
\lsptoprule
&
`reach, arrive'\\\midrule
\isi{Imperfective} stem &
\textit{phed-}\\
\isi{Present} (\textsc{msg}) &
\textit{phed-áan-u}\\
\isi{Future} (\textsc{3sg}) &
\textit{phéd-a} (B \textit{phéd-e}) \\
\isi{Imperative} (\textsc{sg}) &
\textit{phéd} \\
Perfective stem &
\textit{phedíl-} \\
Perfective (\textsc{msg}) &
\textit{phedíl-u} \\
Perfective (\textsc{fsg}) &
\textit{phedíl-i} \\\lspbottomrule
\end{tabularx}
\label{tab:8-4}
\end{table}
 
A~few examples of the large subclass of consonant"=ending L-verbs are shown in \tabref{tab:8-lc}. Regular morphophonemic alternations \textit{aa--oo--ee}, \textit{ee--ii} and \textit{a--aa} (as shown by the parenthetical forms in the lists with examples in this chapter) are results of historical strengthening/lengthening and \isi{umlaut} (as described in \sectref{sec:1-4} and \sectref{sec:3-5}). 


\begin{table}[t]
\caption{A selection of consonant"=ending L"=verbs}
\begin{tabularx}{\textwidth}{ Q l Q Q }
\lsptoprule
\textit{čhin-} &
`cut' &
\textit{ɡhuáṛ- (ɡhuáaṛ-)} &
`boil' \textit{(itr)}\\
\textit{baṭ- (báaṭ-)} &
`fit' &
\textit{biṣám- (biṣáam-)} &
`rest'\\
\textit{bilíǰ-} &
`melt' &
\textit{khoṇḍ-} &
`talk'\\
\textit{háans- (hóons-, heens-)} &
`stay, live'
&
\textit{ǰháan- (ǰhóon-, ǰheen-)} &
`recognise, know'\\
\textit{buc̣húṇ-} &
`card (wool)' &
\textit{kamb- (káamb-)} &
`shiver'\\
\textit{buláḍ- (buláaḍ-)} &
`search for' &
\textit{núuṭ-} &
`return' \textit{(tr)}\\
\textit{čar- (čáar-)} &
`graze' (\textit{itr}) &
\textit{pičhíl-} &
`slip'\\
\textit{utráp- (utráap-)} &
`run' &
\textit{uḍhéew- (uḍhíiw-)} &
`flee'\\\lspbottomrule
\end{tabularx}
\label{tab:8-lc}
\end{table}


\subsection{\textit{a}-ending L-verbs}
\label{subsec:8-3-2}
 
The next subclass comprises verbs with stems ending in an~accented \textit{á} (\tabref{tab:8-5}). Many, but not all, of these verbs are \isi{transitive}, and among them we also find all \isi{causative} and \isi{transitive} verbs (productively) derived with \textit{-á} or \textit{-awá} from non"=\isi{causative} and \isi{intransitive} verbs. The final vowel of the stem has coalesced with the suffix vowel and sometimes been \isi{subject} to further historical strengthening or given rise to \isi{umlaut}. Unique for this particular subclass are thus the future verb forms with \textit{óo} and the \isi{perfective} segments \textit{óol} or \textit{éel}. 


\begin{table}
\caption{Partial paradigm for \textit{a}-ending L-verbs}
\begin{tabular}{ ll }
\lsptoprule
&
`eat'\\\midrule
\isi{Imperfective} stem &
\textit{kha-} \\
\isi{Present} (\textsc{msg}) &
\textit{kha-áan-u}, /kʰiˈaːnu/, etc.\\
\isi{Future} (\textsc{3sg}) &
\textit{khóo} (B \textit{khúu})\\
\isi{Imperative} (\textsc{sg}) &
\textit{khá} \\
Perfective stem &
\textit{kháal- ({\textless}khá- + -íl)}\\
Perfective (\textsc{msg}) &
\textit{khóol-u} (B \textit{khúul-u})\\
Perfective (\textsc{fsg}) &
\textit{khéel-i} \\\lspbottomrule
\end{tabular}
\label{tab:8-5}
\end{table}


The stem-\textit{á} followed by the suffix \textit{-áan} has resulted in a~few parallel present \isi{tense} surface realisations: /kʰaˈjaːnu/, /kʰiˈaːnu/, /ˈkʰaːjnu/, /ˈkʰajnu/. The form with the epenthetic \textit{-y-} /j/ is the dominant one in the B variety, whereas the other forms are commonly heard in A.\footnote{\citet[22]{morgenstierne1941} does not report any forms other than those with \textit{y}, which is why we can assume that to be the more conservative pronunciation and the other ones to have evolved during the last few decades.} This is one of the largest subclasses, and~-- again~-- the verbs listed in \tabref{tab:8-la} are only a few examples.


\begin{table}[h]
\caption{A selection of \textit{a}"=ending L"=verbs}
\begin{tabularx}{\textwidth}{ l@{\hspace{20pt}} Q l@{\hspace{20pt}} Q }
\lsptoprule
\textit{samá-} &
`build' &
\textit{ǰhaaná-} &
`wake up' (\textit{itr})\\
\textit{leewá-} &
`lie (tell a~lie)' &
\textit{khooǰá-} &
`ask'\\
\textit{butsá-} &
`inject, pierce' &
\textit{lanɡá-} &
`take across'\\
\textit{bha-} &
`be able to' &
\textit{bhuuǰá-} &
`wake up' (\textit{tr})\\
\textit{bhanǰá-} &
`beat' &
\textit{lišá-} &
`close'\\
\textit{čooṇṭá-} &
`write, embroider' &
\textit{mučá-} &
`open, untie'\\
\textit{čulá-} &
`rock' &
\textit{pačá-} &
`cook'\\
\textit{ḍhanɡá-} &
`bury, plant' &
\textit{ṣa-} &
`put on'\\\lspbottomrule
\end{tabularx}
\label{tab:8-la}
\end{table}

\subsection{\textit{e}-ending L-verbs}
\label{subsec:8-3-3}


The third important subclass of L-verbs (\tabref{tab:8-6}) includes both \isi{intransitive} and \isi{transitive} verbs. Here we find some \isi{transitive} verbs clearly corresponding to or derived from \isi{intransitive} verbs in other verb classes, although often through less transparent (and synchronically unproductive) processes than the ones we found in the \textit{a}-ending class.


\begin{table}[ht]
\caption{Partial paradigm for \textit{e}-ending L-verbs}
\begin{tabularx}{.66\textwidth}{ ll }
\lsptoprule
&
`kill'\\\midrule
\isi{Imperfective} stem &
\textit{mhaaré-}\\
\isi{Present} (\textsc{msg}) &
\textit{mhaar-áan-u} \\
\isi{Future} (\textsc{3sg}) &
\textit{mhaar-íi} \\
\isi{Imperative} (\textsc{sg}) &
\textit{mhaar-á} (B \textit{mhaaré})\\
Perfective stem &
\textit{mheeríl- ({\textless} mhaaré- + -íl)}\\
Perfective (\textsc{msg}) &
\textit{mheeríl-u} \\
Perfective (\textsc{fsg}) &
\textit{mheeríl-i} \\\lspbottomrule
\end{tabularx}
\label{tab:8-6}
\end{table}


A few other examples of such verbs are displayed in \tabref{tab:8-le}. An underlying (and historical) final accented \textit{é} is assumed here that also occurs in the surface form of B (but not A) imperative singular. Although the final \textit{é} has been \isi{subject} to deletion in most other forms, the future \isi{tense} \textit{-íi} results from the coalescence of this stem-\textit{é} and the suffix-\textit{e} (compare with the \textit{-ée} in the closed \isi{syllable} of the \textsc{2sg} future verb form in B, \textit{aṭ-éeṛ}, etc, where this strengthening process has not applied).


\begin{table}
\caption{A selection of \textit{e}"=ending L"=verbs}
\begin{tabularx}{\textwidth}{ Q Q Q Q }
\lsptoprule
\textit{akaṭé-} &
`gather' (\textit{tr}) &
\textit{ɡhašé-} &
`catch, take'\\
\textit{aṭé-} &
`bring' &
\textit{naamé- (neem-)} &
`bow'\\
\textit{buc̣haalé- (buc̣heel-)} &
`become hungry' &
\textit{ɡaḍé-} (B \textit{ɡhaḍé-}) &
`take off, take out'\\
\textit{niaaṭé- (nieeṭ-)} &
`shave, shear' &
\textit{whaalé- (wheel-)} &
`take down'\\
\textit{bhe-} &
`become' &
\textit{the-} &
`do'\\
\textit{čapé-} &
`chew, gnaw' &
\textit{piṭé-} &
`close'\\
\textit{dac̣hé-} &
`look' &
\textit{phaalé- (pheel-)} &
`newline, chop'\\
\textit{čuuṣé-} &
`suck' &
\textit{ɡhaṇḍé-} &
`fasten, tie' \\\lspbottomrule
\end{tabularx}
\label{tab:8-le}
\end{table}


Only some of the verbs in this class have a~long accented vowel in the \isi{perfective} forms with
\textit{íil}, while most are left with a~short \textit{íl}. I don't have any explanation for
this, but interestingly, \citet[22--23]{morgenstierne1941} has documented alternative (``older'')
forms of \textit{the-} `do': \textit{thiānu} (\textsc{prs msg}) and \textit{thīelo}
(\textsc{pfv msg}), which suggests that there may have been for all \textit{e}-ending verbs
an~intermediate form with both the vowel quality of the stem and that of the suffix preserved. For
this particular verb, the modern \isi{perfective} forms differ between the dialects, so that the A form
uses a~long \textit{íil}, whereas B has a~short \textit{íl}. Also, the converbs of
\textit{the-} and \textit{bhe-} (two extremely frequent verbs, being the two most common
verbalisers, see \sectref{subsec:8-6-1} and \sectref{subsec:12-2-8}) are formed with \textit{e}
(\textit{the} and \textit{bhe}) instead of the \textit{-í} of the other verbs of this subclass.


\subsection{Other L-verbs}
\label{subsec:8-3-4}


Two other frequent L-verbs should be mentioned that do not easily fit into any of the above"=mentioned subclasses: \textit{yhe-} `come' and \textit{ru-} `cry' (in \tabref{tab:8-7}). Both have (at least historically) stems ending in vowels that in various ways interact with the suffix"=vowels. 


\begin{table}
\caption{Partial paradigm for two vowel"=ending L-verbs}
\begin{tabular}{ lll }
\lsptoprule
&
`come' &
`cry'\\\midrule
\isi{Imperfective} stem &
\textit{yhe-} &
\textit{ru-} \\
\isi{Present} (\textsc{msg}) &
\textit{yh-áand-u} &
\textit{ru-áan-u} \\
\isi{Future} (\textsc{3sg}) &
\textit{yh-íi} (B \textit{yíi}) &
\textit{r-íi} \\
\isi{Imperative} (\textsc{sg}) &
\textit{yhá} (B \textit{yé}) &
\textit{ró} \\
Perfective stem &
\textit{yháal-} &
\textit{rúul-} \\
Perfective (\textsc{msg}) &
\textit{yhóol-u} (B \textit{yúul-u}) &
\textit{rúul-u} \\
Perfective (\textsc{fsg}) &
\textit{yhéel-i} (B \textit{yéel-i}) &
\textit{rúul-i} \\\lspbottomrule
\end{tabular}
\label{tab:8-7}
\end{table}


Although the paradigm of \textit{ru-} in many ways reminds us of that of the \textit{e}-ending L-verbs, and therefore in some sense could be seen as a~subgroup of that class, \textit{yhe-} presents a~more complicated case. For one thing, the present \isi{tense} is not like anything we have seen in the classes presented above, with its \textit{áand}-formation instead of the \textit{áan}-suffix we have seen so far. There is reason to return to this issue shortly, as this present"=formation is also a~conspicuous feature of a~group of T-verbs. The other problem has to do with the quality of the underlying stem vowel, giving rise to a~future suffix akin to the \textit{e}-ending verbs above, while the \isi{perfective} is more what would be expected with a~final \textit{á}. Although I represent the stem with a~final \textit{e} for the time being, it is merely a~vocalic placeholder; it may very well go back to a~stem ending in a~\isi{diphthong} \textit{ái} or \textit{éi}. In B there is also an~unexpected alternation between \textit{y} and \textit{yh}, stem"=initially, an~alternation that cuts right through the \isi{imperfective} realm. We will return to this issue of \isi{aspiration} when discussing the other verbs with \textit{-áand}, but for now we will only regard it as an~irregularity among the L-verbs.


While the L-verbs within their respective three subclasses present a~rather homogeneous picture, the paradigms of the T-verbs display a~much higher degree of irregularity. In reality, these verbs, together with the suppletive verbs mentioned, could be seen as a~continuum stretching from verbs with almost agglutinative morphology, through verbs that show an~increasingly irregular correspondence between \isi{perfective} and \isi{imperfective} stems, and ending up at the other extreme with entirely suppletive verbs, without any segments in common and with irregular inflections. 


\subsection{Consonant"=ending T-verbs}
\label{subsec:8-3-5}
\largerpage[-1]

\begin{table}[t]
\caption{Partial paradigm for consonant"=ending T-verbs}

\begin{tabularx}{\textwidth}{ l@{\hspace{20pt}} Q Q Q }
\lsptoprule
&
`forget' &
`understand' &
`take'\\\midrule
\textbf{\isi{Imperfective} stem} &
\textit{aamúuṣ-} &
\textit{buǰ-} &
\textit{ɡhin-} \\
\isi{Present} (\textsc{msg}) &
\textit{aamuuṣ-áan-u} &
\textit{buǰ-áan-u} &
\textit{ɡhin-áan-u} \\
\isi{Future} (\textsc{3sg}) &
\textit{aamúuṣ-a} &
\textit{búǰ-a} &
\textit{ɡhín-a} \\
\isi{Imperative} (\textsc{sg}) &
\textit{aamúuṣ} &
\textit{búǰ} &
\textit{ɡhín} \\
\textbf{Perfective stem} &
\textit{aamúuṣṭ-} &
\textit{búd-} &
\textit{ɡhíin-} \\
Perfective (\textsc{msg}) &
\textit{aamúuṣṭ-u} &
\textit{búd-u} &
\textit{ɡhíin-u} \\
Perfective (\textsc{fsg}) &
\textit{aamúuṣṭ-i} &
\textit{búd-i} &
\textit{ɡhíin-i} \\\lspbottomrule
\end{tabularx}
\label{tab:8-8}
\end{table}



The T-verbs form, together with the suppletive verbs, a~closed class of verbs, comprising between 50 and 60 verbs altogether in my database. Many of these verbs form their perfectives with a~\isi{plosive} segment, in the clear cases a \textit{t}-suffix, but often this has been assimilated with preceding stem segments, hence the name. The largest subclass of the T-verbs are those with a~stem ending in a~consonant (\tabref{tab:8-8}). Like the consonant"=ending L-verbs, the imperative singular is identical to the \isi{imperfective} stem, save for historical strengthening/lengthening. This class includes \isi{intransitive} as well as \isi{transitive} verbs, with no particular preference for one or the other.


In some cases, the \isi{imperfective} and \isi{perfective} stems are identical or nearly identical, save for a \textit{t}-segment added to the \isi{perfective}. In other cases, the stems are considerably different, involving stem vowel alternation, alternation between an~aspirated and an~unaspirated stem, metathesis or assimilation of a~final consonant with the \isi{perfective} \textit{t}-element. In the list with additional examples of such verbs (\tabref{tab:8-tc}), the \isi{imperfective} stem is listed in the first column along with an~alternating \isi{imperfective} stem form within parenthesis, followed by the \isi{perfective} stem in the second column.


\begin{table}[t]
\caption{A selection of consonant"=ending T"=verbs}
\begin{tabularx}{\textwidth}{ l@{\hspace{40pt}} l@{\hspace{40pt}} Q }
\lsptoprule
\textit{béeṣṭ-} \textit{(bíiṣṭ-)} &
\textit{bíiṣṭ-}	&
`wind up' \\
\textit{lhay-} \textit{(lháay-)}	&
\textit{láad-}	&
`find' \\
\textit{har-} &
\textit{(háar-)}	\textit{hiṛ-}	&
`take away' \\
\textit{kirn-} &
\textit{(krin-)} \textit{kríint-}	&
`sell' \\
\textit{khinǰ-}&
\textit{khind-}	&
`become tired' \\
\textit{lun-} &
\textit{lúunt-}	&
`cut, reap' \\
\textit{bheš-} &
\textit{bheṭ-}	&
`sit down' \\
\textit{muč-} &
\textit{mut-}	&
`rain' \\
\textit{\textit{pač-}} (\textit{páač-})	&
\textit{páak-}	&
`ripen, be cooked' \\
\textit{mar-} \textit{(máar-)}&
\textit{muṛ-}	&
`die' \\
\textit{péeṣ-} \textit{(píiṣ-)} &
\textit{píṣṭ-}	&
`grind' \\
\textit{pil-} &\textit{píil-}	&
`drink' \\
\textit{sil-} &\textit{síit-}	&
`sew' \\
\textit{šuǰ-} &\textit{šud-}	&
`end, finish' \textit{(itr)}\\
\textit{níiš-} &\textit{níiṣṭ-}	&
`falter' \\
\textit{šuš-} &\textit{šuk-}	&
`dry'  \textit{(itr)} \\
\textit{ṣuṇ-} &\textit{ṣúunt-}	&
`hear' \\
\textit{ṣač-} \textit{(ṣáač-)} &	\textit{ṣáat-}	&
`quarrel, climb, light' \\\lspbottomrule
\end{tabularx}
\label{tab:8-tc}
\end{table}

\subsection{\textit{e}-ending T-verbs}
\label{subsec:8-3-6}
\largerpage

The next subclass, \textit{e}-ending T-verbs (\tabref{tab:8-9}), is considerably smaller than the consonant"=ending subclass, and is similar to the \textit{e}-ending L-verbs. Even here we need to reconstruct a~stem ending in an~accented \textit{é}, and again the B variety has regularly preserved this \textit{é} in imperative singular. The \isi{perfective} and \isi{imperfective} stems are identical for all of these verbs, save for an~ending segment \textit{íit} or \textit{ít} in the \isi{perfective}.


\begin{table}[p]
\caption{Partial paradigm for \textit{e}-ending T-verbs}
\begin{tabular}{ ll }
\lsptoprule
&
`apply'\\\midrule
\isi{Imperfective} stem &
\textit{malé-}\\
\isi{Present} (\textsc{msg}) &
\textit{mal-áan-u} \\
\isi{Future} (\textsc{3sg}) &
\textit{mal-íi} \\
\isi{Imperative} (\textsc{sg}) &
\textit{malá} (B \textit{malé})\\
Perfective stem &
\textit{malíit-} \\
Perfective (\textsc{msg}) &
\textit{malíit-u} \\
Perfective (\textsc{fsg}) &
\textit{malíit-i} \\\lspbottomrule
\end{tabular}
\label{tab:8-9}
\end{table}


Some more examples are displayed in \tabref{tab:8-te}. Some of these verbs are to varying degrees grammaticalised in some of their uses, \textit{de-} (and possibly also \textit{ǰe-}) as a~\isi{verbaliser}, and \textit{thané-} as a \isi{quotative}, and \textit{mané-} as a hearsay marker. Like \textit{the-} `do' and \textit{bhe-} `become' of the \textit{e}-ending L-verbs, \textit{de-} and \textit{ǰe-} occur as converbs (see \sectref{subsec:12-2-8}) in the forms \textit{de} and \textit{ǰe} instead of with the regular suffix \textit{-í}.


\begin{table}[p] 
\caption{A selection of \textit{e}"=ending T"=verbs}
\begin{tabularx}{\textwidth}{ l@{\hspace{15pt}} l@{\hspace{15pt}} l@{\hspace{15pt}} l@{\hspace{15pt}} l@{\hspace{15pt}} l@{\hspace{15pt}} }
\lsptoprule
\textit{bhayé-} &
\textit{bhayíit-} &
`sow, cultivate' &
\textit{khal-} &
\textit{khalíit-} &
`stir'{\protect\footnotemark}\\
\textit{čiiré-} &
\textit{čiiríit-} &
`be delayed' &
\textit{mané-} &
\textit{maníit-} &
`say'\\
\textit{de-} &
\textit{dít-} &
`give' &
\textit{phrayé-} &
\textit{phrayíit-} &
`send'{\protect\footnotemark}\\
\textit{ǰe-} &
\textit{ǰít-} &
`hit, shoot' &
\textit{thané-} &
\textit{thaníit-} &
`say, call'\\\lspbottomrule
\end{tabularx}
\label{tab:8-te}
\end{table}

\addtocounter{footnote}{-2}
\stepcounter{footnote}\footnotetext{The paradigm of this verb is not entirely stable within the A dialect; it is sometimes treated as an~\isi{L-verb}.}
\stepcounter{footnote}\footnotetext{In B (and for some A-speakers) an \textit{e}-ending \isi{L-verb}.}

\subsection{Accent"=shifting T-verbs}
\label{subsec:8-3-7}


Two small groups of T-verbs can be described as hybrids between consonant"=ending and \textit{e}-ending verbs. The verbs in one of them (\tabref{tab:8-tas1}) behave like the consonant"=ending verbs in the \isi{imperfective} and like the \textit{e}-ending verbs in the \isi{perfective}.


\begin{table}[p] 
\caption{Examples of {accent}"=shifting T"=verbs, type 1}
\begin{tabularx}{\textwidth}{ Q l l l l Q }
\lsptoprule
\textit{ač- (áač-)} &
\textit{ačíit-} &
`enter' &
\textit{pal- (páal-)} &
\textit{palíit-} &
`be hidden'\\
\textit{bhiy-} &
\textit{bhiyíit-} &
`be afraid' &
\textit{šiy-} &
\textit{šiyíit-} &
`fall, be dropped'\\
\textit{dhar- (dháar-)} &
\textit{dharíit-} &
`remain' &
&
&
\\\lspbottomrule
\end{tabularx}
\label{tab:8-tas1}
\end{table}


The verbs of the other group (\tabref{tab:8-tas2}) look like \textit{e}-ending verbs in the \isi{imperfective} but like consonant"=ending verbs in the \isi{perfective}. 


\begin{table}[p] 
\caption{Examples of {accent}"=shifting T"=verbs, type 2}
\begin{tabularx}{\textwidth}{ Q l l Q Q l }
\lsptoprule
\textit{ɡalé-} &
\textit{ɡeél-/ɡaíl-} &
% `throw, leave'{\protect\footnotemark} &
`throw, leave'\footnote{In B an \textit{e}-ending \isi{L-verb}.} & 
\textit{ukualé-} &
\textit{ukuéel-} &
% `throw, leave'{\protect\footnotemark} &
`throw, leave'\footnote{In B an \textit{e}-ending \isi{L-verb}.}  
\\
\textit{čhooré-} &
\textit{čhúuṇ-} &
`put' &
&
&
\\\lspbottomrule
\end{tabularx}
\label{tab:8-tas2}
\addtocounter{footnote}{-2}
% \stepcounter{footnote}\footnotetext{In B an \textit{e}-ending \isi{L-verb}.}
% \stepcounter{footnote}\footnotetext{In B an \textit{e}-ending \isi{L-verb}.}
\end{table}


Whereas the first group is all \isi{intransitive}, with similar \isi{imperfective} and \isi{perfective} stems, the second are all \isi{transitive} with some sort of unpredictable stem alternation. Most likely, \textit{ɡalé-} and \textit{ukualé-} have developed from being \textit{e}-ending L-verbs (which they still are in B) to assimilating the two identical segments adjacent to one another and thus losing their characteristic \textit{íl}-endings. The \textit{e}-ending \isi{L-verb} \textit{whaalé-} `take down' (presented above), which is structurally similar and semantically parallel to \textit{ukualé-}, is \isi{accent}"=shifting with some speakers of the A variety. 


\subsection{\textit{aand}-verbs}
\label{subsec:8-3-8} 

As already mentioned above, there is a~group of four (\isi{intransitive}) motion verbs that all share the present"=\isi{tense} allomorph \textit{-áand} vis-à-vis the usual \textit{-áan}. One of them, \textit{yhe-} `come', an~\isi{L-verb} with a~unique paradigm, was mentioned already. The remaining three (in \tabref{tab:8-10}) are T-verbs. As with \textit{yhe-}, the final \textit{e}-element is merely a~vocalic placeholder as far as representation is concerned.\footnote{I have suggested an underlying \textit{-éi} elsewhere \citep{liljegrenhaider2011,liljegrenhaider2015}. This is also reflected in the morphological representation in the sample texts.}


\begin{table}[h]
\caption{Partial paradigm for T-verbs with -\textit{aand} (present)}
\begin{tabularx}{\textwidth}{ l Q Q Q }
\lsptoprule
&
`appear,\newline come out' &
`come/go/\newline climb up' &
`come/go down'\\\midrule
\textbf{\isi{Imperfective} stem} &
\textit{nikhé- } &
\textit{ukhé-} &
\textit{whe-} \\
\isi{Present} (\textsc{msg}) &
\textit{nikh-áand-u} &
\textit{ukh-áand-u} &
\textit{wh-áand-u} \\
\isi{Future} (\textsc{3sg}) &
\textit{nikh-íi} &
\textit{ukh-íi} &
\textit{wh-íi} \\
\isi{Imperative} (\textsc{sg}) &
\textit{nikhá} &
\textit{ukhá} &
\textit{whá} \\
\textbf{Perfective stem} &
\textit{nikháat-} &
\textit{ukháat-} &
\textit{wháat-} \\
Perfective (\textsc{msg}) &
\textit{nikháat-u} &
\textit{ukháat-u} &
\textit{whaát-u} \\
Perfective (\textsc{fsg}) &
\textit{nikhéet-i} &
\textit{ukhéet-i} &
\textit{whéet-i} \\\lspbottomrule
\end{tabularx}
\label{tab:8-10}
\end{table}


Following \citet[22]{morgenstierne1941}, the present"=\isi{tense} suffix is analysed as an~old participle \textit{-ant-}. What we do not know at this point is why it mostly occurrs in modern"=day Palula as \textit{-áan}, whereas only this small group of verbs, with its \textit{-áand}, has preserved the \isi{consonant cluster} (although with a~lengthened vowel and with the \isi{plosive} voiced in the position before the vocalic \isi{gender}/number suffix, of which both are well"=supported processes in the Palula diachronic development). One line of thought to be pursued is that \isi{aspiration}, now segmentally at the onset of the first \isi{syllable}, historically would have been located at the morpheme break, i.e. between the stem and the \isi{participial} suffix \textit{(ukháand- {\textless} uké+h+ánt)}, only later transposed leftwards (another well"=attested process in the Palula lexicon). This is supported by the lack of \isi{aspiration} in, for instance, the historical \isi{derivation} \textit{ukualé-} `bring up' from \textit{ukhé-} `come/go/climb up'.


\subsection{\textit{i}-ending T-verbs}
\label{subsec:8-3-9}


Another subclass of equal size is one ending (at least underlyingly) with the vowel \textit{i}, as seen in \tabref{tab:8-11}. They are also (\tabref{tab:8-11b}), like the previous group, in some sense motion verbs, at least three of them describing upward motion. Two of them are \isi{transitive} and two \isi{intransitive} (including the already exemplified \textit{uthí-}).


\begin{table} 
\caption{Partial paradigm for \textit{i}-ending T-verbs}
\begin{tabular}{ll }
\lsptoprule
&
`get/stand up'\\\midrule
\textbf{\isi{Imperfective} stem} &
\textit{uthí-} \\
\isi{Present} (\textsc{msg}) &
\textit{uth-áan-u} \\
\isi{Future} (\textsc{3sg}) &
\textit{uthíi} \\
\isi{Imperative} (\textsc{sg}) &
\textit{uthí} \\
\textbf{Perfective stem} &
\textit{uthíit-} \\
Perfective (\textsc{msg}) &
\textit{uthíit-u} \\
Perfective (\textsc{fsg}) &
\textit{uthíit-i} \\\lspbottomrule
\end{tabular}
\label{tab:8-11}
\end{table}


\begin{table} 
\caption{Examples of \textit{i}"=ending T"=verbs}
\begin{tabularx}{\textwidth}{ Q Q Q Q Q l@{\hspace{20pt}} }
\lsptoprule
\textit{uc̣h(í)-} &
\textit{uc̣híi-} &
`lift (up)' &
\textit{uṛ(í)-} &
\textit{uṛíi-} &
`pour, let out'\\
\textit{urbh(í)-} &
\textit{urbhíi-} &
`fly' &
&
&
\\\lspbottomrule
\end{tabularx}
\label{tab:8-11b}
\end{table}

\subsection{\textit{u}-ending T-verb}
\label{subsec:8-3-10}


In my database, there is a~single \isi{T-verb} ending in a~vowel \textit{u} (or \textit{o}), the verb \textit{su-} `sleep, fall asleep' (in \tabref{tab:8-12}).


\begin{table}[ht]
\caption{Partial paradigm for the vowel"=ending verb \textit{su}-}

\begin{tabularx}{.5\textwidth}{ l@{\hspace{20pt}} Q }
\lsptoprule
&
`sleep'\\\midrule
\textbf{\isi{Imperfective} stem} &
\textit{su-} \\
\isi{Present} (\textsc{msg}) &
\textit{su-áan-u} \\
\isi{Future} (\textsc{3sg}) &
\textit{s-íi} \\
\isi{Imperative} (\textsc{sg}) &
\textit{só} \\
\textbf{Perfective stem} &
\textit{sut-} \\
Perfective (\textsc{msg}) &
\textit{sút-u} \\
Perfective (\textsc{fsg}) &
\textit{sút-i} \\\lspbottomrule
\end{tabularx}
\label{tab:8-12}
\end{table}


\subsection{Suppletive verbs}
\label{subsec:8-3-11}


There are only three radically suppletive verbs in the language. One, \textit{paš-/dhriṣṭ-} `see', was already introduced in \tabref{tab:8-3}. The other two are the verbs \textit{be-/ɡ(a)-} `go' and the \isi{copula}/\isi{auxiliary} \textit{hin-/de}. The latter will be presented in \sectref{subsec:8-3-1}2 along with other highly irregular or defective verbs. Due to its unique forms, the verb `go' is presented in a~more comprehensive paradigm in \tabref{tab:8-13}, although we will return to the \isi{inflectional} categories themselves and their functions in the sections to follow.


\begin{table}[t]
\caption{Paradigm for suppletive \textit{be-/ɡ(a)-} `go'}

\begin{tabularx}{\textwidth}{ l@{\hspace{20pt}} l@{\hspace{20pt}} Q Q }
\lsptoprule
&
&
Singular &
Plural\\\midrule
\textbf{Imperfective stem} &
\textit{be-} &
&
\\
\isi{Present} &
\textsc{m} &
\textit{biáanu} /ˈbajnu, biˈaːnu/\newline
(B \textit{bayáanu}) &
\textit{biáana} /ˈbajna, biˈaːna/\newline
(B \textit{bayáana})\\
&
\textsc{f} &
\textit{biéeni} /ˈbejni, biˈeːni/\newline
(B \textit{bayéeni}) &
\textit{biéenim} /ˈbejnim, biˈeːnim/ \newline
(B \textit{bayéenim})\\
\isi{Future} &
1st &
\textit{béem} &
\textit{báaya} (B \textit{béea})\\
&
2nd &
\textit{bíiṛ} (B \textit{béeṛ}) &
\textit{bíit} (B \textit{béet})\\
&
3rd &
\textit{bíi} &
\textit{bíin} (B \textit{béen})\\
\isi{Imperative} &
&
\textit{ba} (B \textit{be}) &
\textit{bóoi} (B \textit{búi})\\
\\
\textbf{Perfective stem} &
\textit{ɡ(a)-} &
&
\\
Perfective &
\textsc{m} &
\textit{ɡúum} (B \textit{ɡáu}) &
\textit{ɡíia} (B \textit{ɡéea})\\
&
\textsc{f} &
\textit{ɡíi} (B \textit{ɡéi}) &
\textit{ɡíia} (B \textit{ɡéea/ɡéi})\\\lspbottomrule
\end{tabularx}
\label{tab:8-13}
\end{table}



\subsection{Irregular verbs and verbs with highly grammaticalised functions}
\label{subsec:8-3-12}

\subsubsection*{The \isi{copula}/\isi{tense} auxiliary}

The \isi{copula}/\isi{tense} \isi{auxiliary} has an~incomplete paradigm, displayed in \tabref{tab:8-14}, using forms of the verb \textit{háans-} `live, stay' in the \isi{imperfective} with non"=present reference. The \isi{perfective} (or more correctly, past, as far as this particular verb is concerned) \textit{de} is invariable and most probably a (fairly recent) grammaticalisation of the \isi{conjunctive participle} of \textit{de-} `give, put'. There is as far as I know no other \iliShina variety with a~similar past"=\isi{tense} \isi{copula}. Instead they all tend to go back to the same stem: \iliKalkoti \textit{aas}, \iliSauji \textit{al-}, Gilgiti \textit{as-} (\citeauthor{radloffshakil1998} w. Shakil \citeyear{radloffshakil1998}), Kohistani \iliShina \textit{asílo-}, Drasi \textit{asiló-}, Guresi \textit{asúlu-} \citep[44--45]{schmidt2004}. The latter is, in addition to the absence of an overt \isi{copula} with many nonverbal predicates, an argument for an alternative interpretation (as argumented for in \sectref{subsec:12-1-1}) of \textit{de} as simply a past \isi{tense} marker. The use of these forms as \isi{tense} auxiliaries will be further discussed and exemplified in \sectref{subsec:9-1-5}--\sectref{subsec:9-1-8}, and their copular use in \sectref{sec:12-1}.


\begin{table}[ht]
\caption{Paradigm for copula}
\begin{tabularx}{\textwidth}{ l@{\hspace{30pt}} l@{\hspace{30pt}} Q Q }
\lsptoprule
&
&
Singular &
Plural\\\midrule
\isi{Present} &
\textsc{m} &
\textit{hín-u} &
\textit{hín-a} \\
&
\textsc{f} &
\textit{hín-i} &
\textit{hín-im} \\
Perfective (past) &
&
\textit{de} &
\textit{de}\\\lspbottomrule
\end{tabularx}
\label{tab:8-14}
\end{table}


\subsubsection*{Modal verbs and verb forms with highly grammaticalised functions}

\spitzmarke{\textit{ṣáat-} `start, begin'.} The \isi{perfective} of the verb \textit{ṣač-} `climb, rise', functions, apart from its lexical use, as a~modal verb with a~\isi{complement} \isi{predicate} in the infinitive (see \sectref{subsec:12-2-7} and \sectref{subsec:13-5-2}). 



\spitzmarke{\textit{bha-} `can, be able to'.} This \textit{a}-ending \isi{L-verb} is an~entirely modal verb, taking \isi{complement} predicates in the infinitive (see \sectref{subsec:12-2-7} and \sectref{subsec:13-5-2}).



\spitzmarke{\textit{thaní} `called, said' (labelled \textsc{quot}).} Although the verb \textit{thané-} `call, say' does occur (but rarely) in other forms, it has become grammaticalised as a~converb, \textit{thaní}, and as such has come to function as a~quotation marker (see \sectref{subsec:9-2-4}) and may potentially develop into an~\isi{auxiliary} or clitic with an~even more restricted syntactic distribution.



\spitzmarke{\textit{maní} `it has been told' (labelled \textsc{hsay}).} Like \textit{thaní, maní} is a~converb, related to the verb \textit{mané-} `say', and has in this particular form become grammaticalised as a~hearsay marker (see \sectref{subsec:9-2-4}). It is also distributionally more restricted and even more specialised than \textit{thaní}, and can therefore alternatively be described as a~special type of \isi{auxiliary} or verb clitic. It can also be said to have become enough separate from the ``regular'' use of the verb \textit{mané-}.



Although not necessarily analysable as verbs any more, two other important modality words or markers should be mentioned:



\spitzmarke{\textit{heentá} `would, might, were' (labelled \textsc{condl}).} Invariable in this form, it occurs in constructions primarily expressing \isi{conditionality} and is therefore as much to be regarded as a~subordinate \isi{conjunction} (functioning like `if'-`then') as a~verb form. See \sectref{subsec:13-4-4}.



\spitzmarke{\textit{seentá} \textit{(B síinta)} `should, shall, will' (labelled \textsc{condh}).} Like \textit{heentá} it is formally invariable, occurring in \isi{Conditional} constructions, where the connection between cause and result essentially is factual and temporal (functioning similarly to `when'-`then'). See \sectref{subsec:13-4-4}.


\section{Inflectional categories}
\label{sec:8-4}

Verbs are primarily inflected for \isi{tense} and argument agreement, whereas aspect in the present analysis is considered as already included in a~\isi{perfective} or \isi{imperfective} stem. The stem itself may be analysed as composed of a~base and a~\isi{valency} specification (see \sectref{sec:8-5}). The focus in the rest of this chapter (\sectref{sec:8-4}-\sectref{sec:8-6}) is on verb forms (whether formed inflectionally or derivationally), whereas verbal categories as defined functionally (capitalised throughout this work) is the topic of \sectref{chap:9}. Some of the latter (such as the TMA"=categories \isi{Future} and \isi{Present}) directly correspond to \isi{inflectional} categories (\sectref{subsec:9-1-1}-\sectref{subsec:9-1-4}, \sectref{subsec:9-2-1}, \sectref{subsec:9-2-3}, and \sectref{sec:9-3}), others (such as \isi{Perfect} and Past \isi{Imperfective}) are expressed periphrastically, i.e. outside the actual system of \isi{inflectional} morphology (\sectref{subsec:9-1-5}-\sectref{subsec:9-1-8}, \sectref{subsec:9-2-2}, and \sectref{subsec:9-2-4}). 


Argument agreement occurs immediately outside the stem, except for when it occurs subsequent to \isi{inflectional} \isi{tense}. Two different kinds of agreement are part of the paradigm, \isi{person agreement}, and \isi{gender/number agreement}, the former confined to the agreement marking directly attached to the \isi{imperfective} stem. Gender/number agreement occurs twice in the case of the \isi{Perfect}, as main verb \isi{inflection} as well as \isi{auxiliary} verb \isi{inflection}. 


The singular imperative is basically an~accented form of the verb stem, whereas the plural is formed with a~unique suffix, without any direct parallels in the agreement system at large. Apart from \isi{finite} \isi{inflectional} categories, there are a~number of important non"=\isi{finite} forms. \tabref{tab:8-15}, with \textit{til-} `walk' as an~example verb, gives an~overview of verbal \isi{inflectional} categories, including \isi{finite} as well as non"=\isi{finite} forms.


\begin{table}[ht]
\caption{Verb forms (\textit{til}- `walk')}

\begin{tabular}{llll}
\lsptoprule
&
&
Singular &
Plural\\\midrule
\isi{Imperative} &
&
\textit{til} &
\textit{tíl-ooi} \\
\isi{Future} &
1st &
\textit{tíl-um} &
\textit{til-íia} \\
&
2nd &
\textit{tíl-aṛ} &
\textit{tíl-at} \\
&
3rd &
\textit{tíl-a} &
\textit{tíl-an} \\
\isi{Present} &
\textsc{m} &
\textit{til-áan-u} &
\textit{til-áan-a} \\
&
\textsc{f} &
\textit{til-éen-i} &
\textit{til-éen"=im} \\
Perfective &
\textsc{m} &
\textit{tilíl-u} &
\textit{tilíl-a} \\
&
\textsc{f} &
\textit{tilíl-i} &
\textit{tilíl-im} \\
\isi{Obligative} &
&
\textit{til"=eeṇḍeéu} &
\\
\isi{Infinitive} &
&
\textit{til-áa(i)} &
\\
\isi{Converb} &
&
\textit{til-í} &
\\
Copredicative participle &
&
\textit{til-íim} &
\\
Verbal \isi{noun} &
&
\textit{til"=ainií} &
\\
Agentive verbal \isi{noun} &
\textsc{m} &
\textit{til-áaṭ-u} &
\textit{til-áaṭ-a} \\
&
\textsc{f} &
\textit{til-éeṭ-i} &
\textit{til-éeṭ-im} 
\\\lspbottomrule
\end{tabular}
\label{tab:8-15}
\end{table}


\subsection{Agreement morphology}
\label{subsec:8-4-1}

As mentioned above, two different types of argument agreement are present in the language: a) \isi{person agreement}, and b) \isi{gender/number agreement}. A similar situation where different sets of agreement markers are being used in one and the same language is not at all uncommon in IA languages, where the ``primary'' person forms almost always descend from the \iliOIA present active~-- although not necessarily every single suffix as such~-- while the ``secondary'' \isi{gender}/number forms are adjectival and connected with verb forms built on participles, which have only later come to function as \isi{finite} verbs \citep[259--260]{masica1991}. 



While \isi{person agreement} is always with the \isi{intransitive} \isi{subject} or the \isi{transitive} agent, the \isi{gender/number agreement} follows an~\isi{ergative} pattern in the \isi{perfective} and an~\isi{accusative} pattern in the \isi{imperfective}. For details on agreement patterns, see \sectref{sec:11-1}.


\subsubsection*{Person agreement}
\largerpage[-1]
Person agreement (\tabref{tab:8-16}) occurs with the non"=\isi{tense} marked \isi{imperfective} stem, and more
specifically with the \isi{Future} and the Past \isi{Imperfective} categories. All person"=agreement suffixes
added to the stem, except the \textsc{1pl}, are \isi{accent}"=neutral.


\begin{table}[ht]
\caption{Person"=agreement suffixes}

\begin{tabularx}{\textwidth}{ l l@{\hspace{20pt}} Q l@{\hspace{20pt}} Q }
\lsptoprule
&
Singular &
&
Plural &
\\\midrule
1 &
\textit{khóṇḍ-um} &
`I will speak' &
\textit{khoṇḍ-íia} &
`we will speak'\\
2 &
\textit{khóṇḍ-aṛ} &
`you \textsc{(sg)} will speak' &
\textit{khóṇḍ-at} &
`you \textsc{(pl)} will speak'\\
3 &
\textit{khóṇḍ-a} &
`he, she will speak' &
\textit{khóṇḍ-an} &
`they will speak'\\\lspbottomrule
\end{tabularx}
\label{tab:8-16}
\end{table}


The consonant segments in \textsc{1sg}, \textsc{2pl} and \textsc{3pl} are fully recognisable from the ancient present ending (\textit{-āmi, -asi, -ati, -āmas, -atha, -anti}), and are also, including \textsc{3sg}, somewhat expected considering the general diachronic loss of final syllables. The Palula \textsc{2sg} and \textsc{1pl}, however, remain largely unexplained. Similar \isi{person agreement} suffixes are found in other \iliShina varieties, such as those of the Kohistani \iliShina subjunctive: \textit{-am, -ii/-ee, -ee, -ooṇ, -at, -an/-en} \citep[114]{schmidtkohistani2008}. 



Due to interaction with a~final stem"=vowel, however, whether synchronically realised or underlying,
the actual future endings may take a~number of different shapes, as can be seen in
\tabref{tab:8-17}, but they all go back to the same basic suffixes, and there is therefore in
a~strict sense only one ``conjugation'' \citep[261]{masica1991} in Palula.


\begin{table}[ht]
\caption{Person"=agreement allomorphs}

\begin{tabularx}{\textwidth}{ l Q Q }
\lsptoprule
&
Singular &
Plural\\\midrule
1 &
\textit{-um, -úum} (B \textit{-áam}) &
\textit{-íia, -áaya} \\
2 &
\textit{-aṛ} (B \textit{-eṛ}), \textit{-óoṛ} (B \textit{-áaṛ}), \textit{-íiṛ}\newline (B \textit{-éeṛ}) &
\textit{-at} (B \textit{-et}), \textit{-óot} (B \textit{-áat}), \textit{-íit}\newline (B \textit{-éet}) \\
3 &
\textit{-a} (B \textit{-e}), \textit{-óo} (B \textit{-úu}), \textit{-íi} &
\textit{-an} (B \textit{-en}), \textit{-óon} (B \textit{-áan}), \textit{-íin}\newline (B \textit{-éen}) \\\lspbottomrule
\end{tabularx}
\label{tab:8-17}
\end{table}
 

The first form displayed in each cell in \tabref{tab:8-17} is the unassimilated suffix following a~consonant"=ending verb stem. The second form displayed is the person"=suffix fused with the final vowel of an \textit{a}-ending stem, and the third form is used with \textit{e}-ending stems. \textsc{1sg} uses the second form for \textit{a}-ending and \textit{e}-ending verbs alike, and \textsc{1pl} uses the first form for consonant"=ending and \textit{e}-ending verbs.
 

\largerpage[-1]

Verb stems ending in \textit{u} make use of the third ending for second and third persons, and the second ending for both \textsc{1sg} and \textsc{1pl}. That is also the case with the \textit{áand}-verbs, except that there is a~special \textsc{1pl} form \textit{-éea} or \textit{-éya} in the B dialect with these verbs (as in \textit{nikh-éea} `we will appear' and \textit{y-éya} `we will come'). The verb \textit{the-} `do' shows some variation in its \textsc{1pl}, \textit{th-íia} or \textit{th-áaya}, and there is similar variability in the paradigms of the \textit{i}-ending verbs; a \textsc{1sg} form \textit{uṛ-íim} has, for instance, been noted in the B dialect for the verb \textit{uṛí-} `pour'.



The deviant pattern of suppletive \textit{be-} `go' has already been displayed in \tabref{tab:8-13}, with all its \isi{person agreement} forms included.


\subsubsection*{Gender/number agreement}
 

Gender/number agreement occurs with the \isi{perfective} stem and with present \isi{tense}. All these categories
are historically \isi{participial} categories, hence also referred to as \isi{adjectival agreement}
\citep[260]{masica1991}. The interaction of \isi{gender} and number results in four possible agreement suffixes, all familiar from \isi{noun} and \isi{adjective} morphology. The suffixes are the same, but with the present (\tabref{tab:8-18}) there is always the additional property of vowel alternation (\isi{umlaut}) occurring inside the preceding present \isi{tense} suffix, when the agreement suffix includes a~high front vowel (as in the two feminine suffixes).

\begin{table}[t]
\caption{Gender/number agreement with the present}
\begin{tabularx}{\textwidth}{ l l Q l Q }
\lsptoprule
&
Singular &
&
Plural &
\\\midrule
\textsc{m} &
\textit{pil-áan-u} &
`I, you, he, it (\textsc{msg}) am/are/is drinking' &
\textit{pil-áan-a} &
`we, you, they (\textsc{mpl}) are drinking'\\
\textsc{f} &
\textit{pil-éen-i} &
`I, you, she, it (\textsc{fsg}) am/are/is drinking' &
\textit{pil-éen"=im} &
`we, you, they (\textsc{fpl}) are drinking'\\\lspbottomrule
\end{tabularx}
\label{tab:8-18}
\end{table}




In some sense, the \textit{-m} of the feminine plural could be seen as a~plural marking added to the (for plural otherwise unmarked) feminine \textit{-i}, as a~more peripheral layer, thus being a ``tertiary'' (later added\footnote{Compare the \isi{perfective} number/\isi{gender} forms of `go' (\tabref{tab:8-13}), which lacks this four"=way contrast.}) element to use \citeauthor{masica1991}'s (\citeyear[260--261]{masica1991}) terminology. Its use is indeed also less stable or optional (as compared to the non"=optional opposition \textsc{msg} vs. \textsc{mpl} vs. \textsc{f}). It has a~direct parallel in the \isi{nasal} element~-- marking feminine plural~-- in \iliUrduHindi verb morphology. Its origin is probably to be found in \isi{noun} morphology rather than in \isi{adjective} morphology (see \sectref{subsec:4-6-3}).


With the \isi{perfective}, the occurrence of such vowel alternation in the preceding segments of the word varies between different verb classes. With the most commonly occurring L-verbs with a~consonant"=ending stem the agreement suffixes are the sole reflexes of agreement (\tabref{tab:8-19}).\clearpage


\begin{table}[p]
\caption{Gender/number agreement with the perfective}

\begin{tabularx}{\textwidth}{ l l Q l Q }
\lsptoprule
&
Singular &
&
Plural &
\\\midrule
\textsc{m} &
\textit{phedíl-u} &
`I, you, he, it (\textsc{msg}) arrived' &
\textit{phedíl-a} &
`we, you, they (\textsc{mpl}) arrived'\\
\textsc{f} &
\textit{phedíl-i} &
`I, you, she, it (\textsc{fsg}) arrived' &
\textit{phedíl-im} &
`we, you, they (\textsc{fpl}) arrived'\\\lspbottomrule
\end{tabularx}
\label{tab:8-19}
\end{table}


However, accented \textit{áa} and \textit{óo} (in B \textit{úu,} historically derived from \textit{áa}) in the \isi{perfective} stem, as in \tabref{tab:8-20}, are \isi{subject} to vowel alternation (\isi{umlaut}), just parallel to the present verb forms. 


\begin{table}[p]
\caption{Vowel alternation related to {gender}/number agreement}

\begin{tabularx}{\textwidth}{ l l@{\hspace{20pt}} l@{\hspace{20pt}} l@{\hspace{20pt}} l@{\hspace{20pt}} }
\lsptoprule
&
Singular &
&
Plural &
\\\midrule
\textsc{m} &
\textit{mučóol-u} &
`X opened (\textsc{msg})' &
\textit{mučóol-a} &
`X opened (\textsc{mpl})'\\
&
\textit{nikháat-u} &
`(\textsc{msg}) appeared' &
\textit{nikháat-a} &
`(\textsc{mpl}) appeared'\\
&
\textit{ṣáat-u} &
`(\textsc{msg}) quarreled' &
\textit{ṣáat-a} &
`(\textsc{mpl}) quarreled'\\
&
\textit{páak-u} &
`(\textsc{msg}) ripened' &
\textit{páak-a} &
`(\textsc{mpl}) ripened'\\
\textsc{f} &
\textit{mučéel-i} &
`X opened (\textsc{fsg})' &
\textit{mučéel"=im} &
`X opened (\textsc{fpl})'\\
&
\textit{nikhéet-i} &
`(\textsc{fsg}) appeared' &
\textit{nikhéet"=im} &
`(\textsc{fpl}) appeared'\\
&
\textit{ṣéet-i} &
`(\textsc{fsg}) quarreled' &
\textit{ṣéet"=im} &
`(\textsc{fpl}) quarreled'\\
&
\textit{péek-i} &
`(\textsc{fsg}) ripened' &
\textit{péek"=im} &
`(\textsc{fpl}) ripened'\\\lspbottomrule
\end{tabularx}
\label{tab:8-20}
\end{table}


In some of the periphrastic categories agreement occurs twice, shown in \tabref{tab:8-21}, first in the main verb, then also in the \isi{auxiliary} (although there is a~strong tendency for at least one \textit{-m} in the \textsc{fpl} agreement forms to be dropped).


\begin{table}[p]
\caption{Double {gender}/number agreement}

\begin{tabularx}{\textwidth}{ l P{20mm} Q P{20mm} Q }
\lsptoprule
&
Singular &
&
Plural &
\\\midrule
\textsc{m} &
\textit{so phedíl-u hín-u} &
`I, you, he, it (\textsc{msg}) have/has arrived' &
\textit{se phedíl-a hín-a} &
`we, you, they (\textsc{mpl}) have arrived'\\
\textsc{f} &
\textit{se phedíl-i hín-i} &
`I, you, she, it (\textsc{fsg}) have/has arrived' &
\textit{se phedíl-im}
\textit{hín-i(m)} &
`we, you, they (\textsc{fpl}) have arrived'\\\lspbottomrule
\end{tabularx}
\label{tab:8-21}
\end{table}


\subsection{Verb forms derived from the {imperfective} stem}
\label{subsec:8-4-2}



\begin{table}
\caption{{Imperative} formation}
\begin{tabularx}{\textwidth}{ Q l@{\hspace{20pt}} l@{\hspace{20pt}} Q }
\lsptoprule
Stem &
\isi{Imperative} singular &
\isi{Imperative} plural &
\\\midrule
\textit{čár-} &
\textit{čáar} &
\textit{čáar"=ooi} &
`graze!' (\textit{itr})\\
\textit{lamá-} &
\textit{lamá} &
\textit{lam-óoi} &
`hang!' (\textit{tr})\\
\textit{čaaré-} &
\textit{čaará (B čaaré)} &
\textit{čaar-óoi} &
`graze!' (\textit{tr})\\
\textit{uthí-} &
\textit{uthí} &
\textit{uth-óoi} &
`stand up!'\\
\textit{su-} &
\textit{só} &
\textit{s-óoi} &
`sleep!'\\\lspbottomrule
\end{tabularx}
\label{tab:8-22}
\end{table}



\textbf{\isi{Imperative}.} All verbs form distinct singular and plural imperatives, for example \textit{čhín} \textsc{(sg)}~-- \textit{čhínooi} (\textsc{pl}) `cut!' More examples are provided in \tabref{tab:8-22}. The imperative singular is in the typical case identical to the \isi{imperfective} stem, and it always carries the \isi{accent} on its last \isi{syllable}. For nouns with a~consonant"=ending stem, the imperative singular is always the same as the stem, but with additional lengthening of an~accented stem"=vowel \textit{á} to \textit{áa} (in the A dialect only), and strengthening of an~accented stem"=vowel \textit{áa} to \textit{óo} (in B \textit{úu}), and \textit{ée} to \textit{íi}. In the \textit{e}-ending verbs, the underlying final \textit{é} has (in the A dialect) become \textit{á} in imperative singular and are thus formally identical to the imperative of \textit{a}-ending verbs.

The imperative plural is invariably formed with a~suffix \textit{-ooi} (in B \textit{-uui}). This suffix receives the \isi{accent} with vowel"=ending verb stems (replacing its final vowel with the imperative suffix), whereas consonant"=ending stems keep the word"=\isi{accent} on the stem"=vowel, often with a~subsequent weakening of the suffix to \textit{-oi} /-oj/ (B \textit{-ui} /-uj/). See \sectref{subsec:9-2-1} for a description of the \isi{Imperative} as a TMA"=category.


\spitzmarke{\isi{Future}.} The \isi{tense}"=unspecified \isi{imperfective}, is always realised as the
\isi{imperfective} stem directly followed by one of six \isi{person agreement} suffixes (see
\sectref{subsec:8-4-1} and \tabref{tab:8-16}), and has without any further specification a (primarily) future meaning in Palula, while it forms the Past \isi{Imperfective} when used together with the \isi{auxiliary} \textit{de}.


Historically this is the old present \isi{tense}, which in the modern language has become confined to the non"=present \isi{imperfective} realm. While in many IA languages the old present has been driven out by newer formations and been left with less central functions, some of them with vague future or subjunctive meanings, it is especially in the north"=western IA languages where it has come to function as the future \textit{per se} \citep[288]{masica1991}. It is interesting, however, that these forms seem rather marginal even in the closest relatives of Palula, namely \iliSauji and \iliKalkoti. In \iliSauji, they are used as subjunctives, but very infrequently, and only the \textsc{1sg} forms \textit{-um/-om/-aam} correspond with any greater precision with the Palula forms. The form \textit{-iyee} is used for \textsc{3sg}, \textsc{1pl} and \textsc{3pl} alike, and most closely resembles the \textsc{1pl} of Palula. On the other hand, a~particular verb form built up with the \textsc{1sg} subjunctive, an \textit{n}-element and a~\isi{gender/number agreement} suffix (giving \textit{dumnoo, dumnee, dumni, dumne} `will give') is used as a~future in \iliSauji. This morphological construction is obviously a~\iliSauji-specific innovation; it is very likely that the combination \textit{n} + \isi{gender/number agreement} is somehow related to the \isi{copula} \textit{(hinoo, hinee, hini, hine):} \textit{dumnoo {\textless} dum hinoo}. For the Past \isi{Imperfective} (\sectref{subsec:9-1-6}), the other instance where the \isi{imperfective} stem + \isi{person agreement} forms are used in Palula (with the past \isi{tense} marker \textit{de}), \iliSauji uses a~construction with the present \isi{imperfective} verb form (\textit{-aan}) followed by a~suffigated form of \textit{aalo} `was \textsc{(msg)}' or one of its \isi{gender}/number alternants: \textit{thaan"=aloo} `was doing', etc. The ancient agreement pattern (in person) has therefore been given up almost exclusively in favour of the new \isi{gender}/number pattern in \iliSauji \citep[46--54]{buddruss1967}.


\largerpage[-1]
The near absence of any forms related to the Palula future in \iliKalkoti tends to point in the same direction. Elicitation of future propositions tends to produce the same \isi{imperfective} verb forms as for most present \isi{tense} propositions. I only have a~few examples where some of the forms with \isi{person agreement} are used: \textsc{1sg} \textit{-am/-um} (as in \textit{ma ɡuwaa th-am} `What should I do?'), \textsc{2/3sg} \textit{-ä}, and \textsc{3pl} \textit{-ään}, rather closely corresponding to the Palula \textsc{1sg}, \textsc{3sg} and \textsc{3pl} forms, respectively. Although the form sharing between the \textsc{2sg} and \textsc{3sg} forms of the aorist is also found in Kohistani \iliShina \citep[39]{schmidt2004}, I would rather assume a~loss of contrast, probably as the result of sound changes in a~more distant past in \iliKalkoti (and Kohistani \iliShina), than to suggest a~split into \textsc{2sg} and \textsc{3sg} in Palula, although the actual form of the modern and somewhat ``mysterious'' \textsc{2sg} in Palula is an~innovation peculiar to this \iliShina variety. See \sectref{subsec:9-1-2} for a description of the \isi{Future} as a~TMA"=category, and \sectref{subsec:9-1-6} for a description of the Past \isi{Imperfective}.


\spitzmarke{\isi{Present}.} The frequently used present \isi{tense} is regularly formed with a~suffix
\textit{~--áan} (see \tabref{tab:8-18}), which invariably carries the \isi{accent}.


\largerpage[-1]
The origin of this element is not an~entirely uncontroversial issue. Both \citet[22]{morgenstierne1941} and \citet[48]{buddruss1967} clearly state that the \textit{-áan} of Palula as well as the virtually identical element in closely related \iliSauji goes back to the \iliOIA present active participle \textit{-ant} or \textit{-antaka}, even this with numerous parallels (involving various degrees of reduction) in other \iliNIA languages \citep[270--271]{masica1991}. What complicates the picture is the small class of verbs in Palula, already mentioned, that form their present \isi{tense} not with \textit{-áan}-\textsc{agr} but with \textit{-áand}-\textsc{agr}. Of course, we may decide that these are a~few residual forms, occurring only with a~group of high"=frequency motion verbs, which for some reason or another have resisted a~change affecting all other verbs (further supported by the fact that the form is missing altogether in \iliSauji and \iliKalkoti), but a~question of separate origins is equally justified. We have in any case no difficulty in showing a~development \textit{-ant {\textgreater} -áand}, as it is entirely parallel to other examples of \textit{a}-lengthening and voicing in the language (compare with \iliOIA \textit{vansantá {\textgreater} *basant {\textgreater}} Palula \textit{basaánd} `spring'). However, if we want to propose that the form \textit{-áan} is merely a~reduced form of \textit{-áand}, we would want to explain why neither \textit{dáanda} `teeth' nor \textit{páanda} `path' (both going back to forms containing an~\iliOIA \textit{ant}-segment) have been similarly reduced to \textit{**daana} and \textit{**paana} in the modern languages. We also have a~particular problem posed by the \isi{imperfective}/present verb forms of \iliKalkoti with three different \isi{gender}/number allomorphs: \textit{-uun, -iin, -aan}. I see it as very unlikely that the different vowel qualities would be straightforward examples of umlauts triggered by a~now lost final vowel of an~agreement suffix. Although \isi{umlaut} is a~feature of the historical development of \iliKalkoti, we do not have any parallel cases where \textit{‌a‌} or \textit{‌aa‌} has developed into \textit{‌uu‌} or \textit{‌ii‌} elsewhere in this variety. Another possible origin is in a~\isi{gender}/number agreeing \isi{auxiliary} \textit{hino} or \textit{hano}, etc. `is' being attached to the aspectually unmarked verb stem (as suggested by Ruth Laila Schmidt, pc). 



This historical \isi{participial} form, regardless of its exact origin, has probably entered the TMA system as an~aspectual marker with a~rather limited \isi{imperfective}~-- probably progressive~-- use, but has steadily gained ground within the \isi{imperfective} realm, marginalising the former present \isi{tense} to the non"=present \isi{imperfective}, thus establishing itself as the sole marker of present"=\isi{tense} reference. An argument for regarding this as primarily a~\isi{tense} category and the suffix as a~present"=\isi{tense} marker and not an~aspect marker is that the unmarked \isi{imperfective} as well as the \isi{perfective} verb form can be further specified with \isi{tense} auxiliaries (see \sectref{subsec:9-1-5}), whereas this is normally not the case with the present.\footnote{There are a~few exceptions in my data where the present"=\isi{tense} \isi{auxiliary} is used with the present verb form, but it is very unusual and perhaps only used when the verb form is interpreted as a~participle rather than as a~\isi{finite} verb. The past \isi{auxiliary} \textit{de}, however, is considered entirely ungrammatical following this form.} 


The suffix \textit{-áan} never appears as the final segment of a~verb, but to it is always added a~number/\isi{gender} suffix, agreeing with the \isi{subject} argument (see \sectref{subsec:8-4-1}). The present \isi{tense} \isi{inflection} undergoes an~\isi{umlaut}"=process when one of the feminine agreement suffixes (with a~high"=front vowel) is attached, with the form \textit{-éen} as a~result. As already pointed out above, a~handful of verbs use the present \isi{tense} \isi{inflection} \textit{-áand} instead, and even undergo \isi{umlaut} formation with the feminine suffixes resulting in the forms \textit{wh-éend-i} `she is coming down', \textit{wh-éend"=im} `they (\textsc{pl}) are coming down', etc. In some speech varieties (particularly within the A dialect), a~third variant is presenting itself in one of the verb classes, as the result of interaction with the final vowel of the \textit{a}-ending verbs, potentially creating a~new \isi{causative} present"=\isi{tense} marker, as shown in \tabref{tab:8-25}. The temporal versus aspectual character the TMA"=category \isi{Present}, using this particular verb form, is further discussed in \sectref{subsec:9-1-3}.


\begin{table}[t]
\caption{{Present} formation with a-ending L-verbs}
\begin{tabularx}{\textwidth}{ l P{25mm} Q P{25mm} Q }
\lsptoprule
&
Singular &
&
Plural &
\\\midrule
\textsc{m}
&
\textit{(pil-áin-u)}
\textit{{\textless}pila(y)-áan-u} &
`(\textsc{msg}) am/are/is making someone drink' &
\textit{(pil-áin-a)}
\textit{{\textless}pila(y)-áan-a} &
`(\textsc{mpl}) are making someone drink'\\
\textsc{f}
&
\textit{(pil-éin-i)}
\textit{{\textless}pila(y)-éen-i} &
`(\textsc{fsg}) am/are/is making someone drink' &
\textit{(pil-éin"=im)}
\textit{{\textless}pila(y)-éen-im} &
`(\textsc{fpl}) are making someone drink'\\\lspbottomrule
\end{tabularx}
\label{tab:8-25}
\end{table}


\spitzmarke{\isi{Obligative}.} This chiefly modal category has an~invariant suffix"=accented form
\textit{-aiṇḍeéu} (in the A dialect mostly pronounced \textit{-eeṇḍeéu}\textsf{)}, added to
basically any verb stem (\tabref{tab:8-27}). It seems that it in most cases simply substitutes for the vowels in vowel"=ending stems. This exemplifies a~verb form that seems to ``fall between two stools'' as far as finiteness is concerned (see \sectref{subsec:9-2-3} for a~further discussion). \citeauthor{schmidt2003}'s (\citeyear[139]{schmidt2003}) ``injunctive'' in Kohistani \iliShina, with its invariant form \textit{-óoṇṭha} and its similar semantics, is probably related to the Palula obligative.


\begin{table}[t]
\caption{{Obligative} formation}

\begin{tabular}{llll}
\lsptoprule
Stem &
\isi{Obligative} &
\\\midrule
\textit{har-} &
\textit{har"=aiṇḍeéu} &
`has/is to be taken'\\
\textit{laɡayé-} &
\textit{laɡay"=aiṇḍeéu} &
`has/is to be attached'\\
\textit{čooṇṭá-} &
\textit{čooṇṭ-aiṇḍeéu} &
`has/is to be written'\\
\textit{nikhé-} &
\textit{nikh"=aiṇḍeéu} &
`has to appear'\\
\textit{the-} &
\textit{th"=aiṇḍeéu} &
`has/is to be done'\\
\textit{su-} &
\textit{su-(w)aiṇḍeéu} &
`has to sleep'\\\lspbottomrule
\end{tabular}
\label{tab:8-27}
\end{table}




\spitzmarke{\isi{Infinitive}.} The infinitive is formed with \textit{-áai} (in
B \textit{-ái})\footnote{It could be argued that it is altogether deaccented, either
  functioning as a~clitic or together with the next morpheme constituting a~phonological word.},
leaves the stem unaccented, and except for in a~few cases (stems with final \textit{u} and in Biori
\textit{bayái} `to go' and \textit{khayái} `to eat') replaces the final stem"=vowel (see
\tabref{tab:8-24}). It appears to be in free variation with the form \textit{-áa}.

\begin{table}[ht]
\caption{{Infinitive} formation}
\begin{tabular}{lll}
\lsptoprule
Stem &
\isi{Infinitive} &
\\\midrule
\textit{utráp-} &
\textit{utrap-áa(i)} &
`run'\\
\textit{samá-} &
\textit{sam-áa(i)} &
`build, make'\\
\textit{kha-} &
\textit{kh-áa(i)} (B \textit{khay-ái}) &
`eat'\\
\textit{su-} &
\textit{su-áa(i)} /swaːj/ &
`sleep'\\\lspbottomrule
\end{tabular}
\label{tab:8-24}
\end{table}

It is, however, still somewhat doubtful whether what I have labelled here an~infinitive really is to
be considered an~independent verb form and a~free morpheme rather than a~secondary stem formation
(see \sectref{subsec:9-3-6}).

\spitzmarke{\isi{Converb}.} Undoubtedly the converb (or the ``\isi{conjunctive participle}'' as it often is
called when referring to IA languages) is the most frequent and important non"=\isi{finite} form, not only
in Palula but in IA languages in general \citep[323]{masica1991}. There are two frequently occurring
and regular forms of the converb, an~accented suffix \textit{-í} added to consonant"=stems,
\textit{e}-ending stems and (replacing the final vowel of) \textit{i}-ending stems, and
an~ending \textit{aá} replacing the final vowel of \textit{a}-ending verb stems. Verb stems
with other vowel endings (\textit{su-} `sleep' and \textit{ru-} `cry') have gone through
assimilation to \textit{-eé}. However, a~few verbs (the \textit{áand}-verbs, see above)
keep a~vowel"=segment of the stem and add \textit{-í} (without assimilation or forming
a~\isi{diphthong}). In addition, a~small group of verbs (\textit{the-}\footnote{The converb of this verb, \textit{the}, is under certain circumstances lengthened to \textit{thée} (especially as lexicalised with a meaning approximately corresponding to `then', often beginning a new utterance, referring back to the immediately preceding context, often along with a postposed clitic \textit{ba}: \textit{thée=ba} `And then...').} `do', \textit{de-} `give', \textit{ǰe-} `hit',
\textit{be-} `go', \textit{bhe-} `become') have a~converb identical to the stem, i.e. with a~short
\textit{-e}. Examples are provided in \tabref{tab:8-23}. For further examples and a description of \isi{Converb} uses, see \sectref{subsec:9-3-1}, \sectref{subsec:13-3-1}, and \sectref{subsec:13-4-1}.


\begin{table}[ht]
\caption{{Converb} formation}
\begin{tabularx}{.75\textwidth}{ l@{\hspace{40pt}} l@{\hspace{40pt}} Q }
\lsptoprule
Stem &
\isi{Converb} &
\\\midrule
\textit{čar-} &
\textit{čar-í} &
`having grazed' (\textit{itr})\\
\textit{lamá-} &
\textit{lamaá} &
`having hung' (\textit{tr})\\
\textit{uthí-} &
\textit{uthí} &
`having stood up'\\
\textit{ukhé-} &
\textit{ukha-(y)í} &
`having gone up'\\
\textit{de-} &
\textit{de} &
`having given'\\
\textit{su-} &
\textit{seé} &
`having slept'\\\lspbottomrule
\end{tabularx}
\label{tab:8-23}
\end{table}


\spitzmarke{Copredicative participle.\footnote{See \citet[17-20]{haspelmath1995} for a~more precise definition.}} This participle seems to be invariably formed from the verb stem with an~\isi{accent}"=bearing suffix \textit{-íim} (\tabref{tab:8-26}). It is formally reminiscent of the instrumental \isi{noun} suffix, with which it has obvious semantic parallels. 


\begin{table}[ht]
\caption{Copredicative participle formation}

\begin{tabularx}{.75\textwidth}{ Q Q l@{\hspace{20pt}} }
\lsptoprule
Stem &
Copredicative participle &
\\\midrule
\textit{buláḍ-} &
\textit{bulaḍ-íim} &
`calling, searching'\\
\textit{ḍuḍúr-} &
\textit{ḍuḍur-íim} &
`rolling, tumbling'\\
\textit{khaṣaalé-} &
\textit{khaṣeel-íim} &
`dragging'\\
\textit{čulá-} &
\textit{čula-íim} &
`waving, shaking'\\
\textit{dac̣hé-} &
\textit{dac̣h-\'{ị}im} &
`looking'\\
\textit{de-} &
\textit{da(y)-íim} &
`giving'\\
\textit{ǰe-} &
\textit{ǰa(y)-íim} &
`hitting'\\\lspbottomrule
\end{tabularx}
\label{tab:8-26}
\end{table}


Due to its adverbial character (see \sectref{subsec:9-3-5} and \sectref{subsec:13-4-1} for details), there are rather severe semantic restrictions on what verbs can occur as copredicative participles. For further examples of the use of the \isi{Copredicative Participle}, see \sectref{subsec:9-3-5} and \sectref{subsec:13-4-1}.  

\largerpage

\spitzmarke{Verbal \isi{noun}.} The verbal \isi{noun} is probably in some sense a~secondary verb form, based (at
least in the A dialect) on the infinitive to which the invariable suffix \textit{-nií} is added
(\tabref{tab:8-28}). This nominalising suffix always carries the \isi{accent}, while the infinitival part
is deaccented (and its vowel subsequently shortened).


\begin{table}[ht]
\caption{Verbal {noun} formation}
\begin{tabular}{llll}
\lsptoprule
Stem &
\isi{Infinitive} &
\isi{Verbal Noun} &
\\\midrule
\textit{utráp-} &
\textit{utrap-áai} &
\textit{utrap"=ai"=nií} &
`run'\\
\textit{samá-} &
\textit{sam-áai} &
\textit{sam"=ai"=nií} &
`build, make'\\
\textit{kha-} &
\textit{kh-áai} &
\textit{kh"=ai"=nií} &
`eat'\\
\textit{su-} &
\textit{su-áai} &
\textit{su"=ai"=nií} /swajˈniː/ &
`sleep'\\\lspbottomrule
\end{tabular}
\label{tab:8-28}
\end{table}


As a~non"=\isi{finite} verb form with \isi{noun}"=like qualities, it may inflect for case and perhaps also for number (although there is no strong evidence for that in my data). For further examples and a description of the various uses of the \isi{Verbal Noun}, see \sectref{subsec:9-3-3}, \sectref{sec:13-4}-\sectref{sec:13-5}, and \sectref{subsec:13-6-6}.


\spitzmarke{Agentive verbal \isi{noun}.} The agentive verbal \isi{noun} is, like the verbal \isi{noun}, a nominalised form, but with a more restricted application than the latter (see \sectref{subsec:9-3-4}). It is formed with a suffix \textit{-áaṭ} (or its \isi{umlaut} variant \textit{-éeṭ}) followed by a \isi{gender}/number suffix. 


\subsection{Verb forms derived from the {perfective} stem}
\label{subsec:8-4-3}

\spitzmarke{Perfective.} This is (when used alone) the single most frequently used verb form in
narrative types of \isi{discourse}, mainly occurring in its \isi{Simple Past} function (see \sectref{subsec:9-1-3}). With an added \isi{auxiliary}, \textit{hín-} `is' it forms the periphrastic category \isi{Perfect} (\sectref{subsec:9-1-7}), and with the \isi{auxiliary} \textit{de} `\textsc{pst}' it forms the \isi{Pluperfect} (\sectref{subsec:9-1-8}).


The most common \isi{perfective} stems end with an \textit{l}-element,
also being the defining feature of the class referred to as L-verbs. The final \isi{syllable} of the stem
(which alternatively could be analysed as a~\isi{perfective} suffix) is always accented and thus carries
the \isi{accent} of the whole word. Like the present"=\isi{tense} inflected stem, the \isi{perfective} stem must be
followed by a~number/\isi{gender} suffix (see \sectref{subsec:8-4-1} and \tabref{tab:8-19}), agreeing with
the \isi{intransitive} \isi{subject} or the \isi{transitive} \isi{direct object}.



The ending elements of the L- and T-based \isi{perfective} verb forms are obviously quite old. The \isi{perfective} forms belonging to the class of T-verbs with a~discernable \textit{t}-element, sometimes realised as \textit{d} or \textit{ṭ}, most certainly derive from an~\iliOIA past (\isi{passive}) participle \textit{-ta} \citep[952]{whitney1960}, representing an~early development of a~perfectivity category, contrasting initially with an~aspectually unmarked plain verb stem. This element has a~number of parallels in other \iliNIA languages \citep[269, 272]{masica1991}. The other perfectivity marking element, with \textit{l}, is a~more recent development and can be traced back to the \iliPrakrit \textit{-illa} (Ruth Laila Schmidt, pc); outside \iliShina, it mostly occurs in \iliNIA languages in the eastern and southern parts of the subcontinent \citep[270]{masica1991}. The T-forming class is clearly a~kind of residual and closed category in Palula as well as in its closest"=related varieties (particularly in \iliSauji and \iliKalkoti), and it is limited in number and productivity, whereas the paradigm of the L-forming class (and its subclasses) has become the normative or system"=defining structural property \citep[104]{mcmahon1994} for verbs, evidenced by the inclusion of rather recent (although few and far between) loans into this class, such as B \textit{newešíl-} `wrote' (from \iliKhowar \textit{niweš-}). The class as such has been \isi{subject} to much more levelling and innovation as compared to the older T-class (Ruth Laila Schmidt, pc).


The non"=\isi{finite} \isi{Perfective Participle} (see \sectref{subsec:9-3-2}) is formally identical to the \isi{finite} \isi{Simple Past} (\tabref{tab:8-19}) to which it is diachronically related. Perfective Participles are essentially adjectival but may also function as nominal forms and therefore be additionally inflected for case. These are distinguished from the \isi{perfective} used for \isi{finite} \isi{Simple Past} only by their distribution, and when they happen to be inflected like nouns.


\section{Valency"=changing morphology}
\label{sec:8-5}

Although mechanisms involved in stem"=\isi{derivation} are \isi{derivational} rather than \isi{inflectional}, the area of \isi{valency} is a~morphologically important and complex issue in Palula like in many other IA languages (as pointed out by \citealt[315]{masica1991}). The two primarily morphological processes of \isi{valency} reduction and \isi{valency addition}, each with its unique suffix, interact with transitivity, causativity, and in a~more limited sense with voice. Palula verb stems come with, or are normally reserved for, a~certain degree of \isi{valency}~-- they are either \isi{intransitive} or \isi{transitive}~-- but this basic \isi{valency} can be changed morphologically, resulting in derived stems with either an~increased \isi{valency} (\sectref{subsec:8-5-1}) or a~reduced \isi{valency} (\sectref{subsec:8-5-2}). Of the two, the former process seems synchronically slightly more productive and more frequently occurring in the language.


\subsection{Valency addition}
\label{subsec:8-5-1}

The productive (and regular) way of increasing or adding \isi{valency} is by adding an~accented suffix \textit{~--á} or \textit{~--awá}\footnote{Which really can be understood as a~secondary or
  doubled \isi{causative} \textit{-á-á}.} to another (\isi{imperfective}) verb stem, as can be seen in
\tabref{tab:8-29}. The verbs derived in this way all belong to the \textit{a}"=ending L"=class. The \isi{valency}"=increasing suffix added to an~\isi{intransitive} \isi{imperfective} stem \mbox{\textit{(núuṭ-)}} thus derives a~\isi{transitive} \isi{imperfective} verb stem \textit{(nuuṭá-)}, and when added to an~already \isi{transitive} stem \textit{(the-)}, a~\isi{causative} verb is derived \textit{(thawá-)}. As this process is a~matter of stem formation, any \isi{inflectional} morphology, as described above, occurs after the \isi{derivational} suffix: \textit{pašáan-a} `are seeing' {\textgreater} \textit{pašawa-áan-a} /paɕaˈwajna/ `are showing', as is also apparent from the examples below.


\begin{table}[ht]
\caption{Regular {valency} addition}
\begin{tabularx}{\textwidth}{ l P{20mm} l Q Q }
\lsptoprule
\multicolumn{2}{ l}{Non"=derived stem} &
&
\multicolumn{2}{ l}{Stem derived by \isi{valency} addition}\\\midrule
\textit{buǰ-} &
`understand' &
\centering {\textgreater} &
\textit{buǰ-á-} &
`make someone understand'\\
\textit{núuṭ-} &
`return, turn around' &
\centering {\textgreater} &
\textit{nuuṭ-á-} &
`turn someone or something around' \\
\textit{paš-} &
`see' &
\centering {\textgreater} &
\textit{paš-awá-} (B \textit{paš-á-}) &
`show'\\
\textit{the-} &
`do' &
\centering {\textgreater} &
\textit{th"=awá-} &
`have someone do'\\\lspbottomrule
\end{tabularx}
\label{tab:8-29}
\end{table}

\largerpage
For the \isi{perfective} stems, which were analysed as including, e.g., an \textit{l}-element, \isi{valency addition} takes place before that element (thus within the stem itself), compare with the stems in examples (\ref{ex:8-5}) and (\ref{ex:8-6}).

\begin{exe}
\ex
\label{ex:8-5}
%modified
\gll xu bhíilam ma \textbf{nuuṭíl-i} hín-i\\
but of.fear \textsc{1sg.nom} return.\textsc{pfv-f} be.\textsc{prs-f} \\
\glt `But because of fear I turned back.' (A:CAV025)
\end{exe}
\begin{exe}
\ex
\label{ex:8-6}
\gll ɡhaš-í ba ɡhúuṛu \textbf{nuuṭóol-u} {\upshape (<} nuuṭ-á-íl-u\upshape{)}\\
catch-\textsc{cv} \textsc{top} horse turn.around.\textsc{pfv"=msg} \\
\glt `Holding it he turned the horse around.' (A:MAB044)
\end{exe}

There are also some non"=productive (or irregular) derivations (\tabref{tab:8-irradd}), reflecting historical or alternative patterns, with root vowel alternation, \isi{aspiration} alternation, and a~derivative segment \textit{-aal}. The alternation \textit{már-/mhaaré-} is related to the \iliOIA \isi{causative} \textit{-áya}, which was accompanied by a~strengthened grade of the root \citep[316--321]{masica1991}. While the \isi{causative} suffix itself was reduced to \textit{-ē} already in \iliMIA (which the underlying \textit{-e} in the \textit{e}-ending L-verbs may be descended from), the root vowel alternation survived, at least in a~few verbs. The more productive \textit{-á} and \textit{-awá}, on the other hand, most probably go back to the \iliOIA \isi{causative} pseudo"=allomorph (to borrow \citeauthor{masica1991}'s term) \textit{-āpaya}, which in Sanskrit became a~productive form of the \isi{causative}. It has since eroded substantially, phonetically speaking. The history of the segment \textit{-aal} (occurring in a~few verbs) is much less certain. It may have a~connection with the regular \isi{causative} marker \textit{-ar} in other \iliShina varieties (\citeauthor{radloffshakil1998} w. Shakil \citeyear[26]{radloffshakil1998}) and/or the irregular \isi{causative} formation of `bring up' from `come up' with \textit{-l} (\textit{ukāɡ {\textgreater} ukālũɡ}) found also in Kohistani \iliGawri \citep[88]{baart1999a}; there are also a~few causatives in \iliUrduHindi \citep[87]{schmidt1999} and in \iliSiraiki \citep[74]{shackle1976} that are formed with a~suffix containing an \textit{l}-element. 



\begin{table}[H]
\caption{Irregular {valency} addition}
\begin{tabularx}{\textwidth}{ l@{\hspace{30pt}} Q l@{\hspace{30pt}} l@{\hspace{30pt}} Q }
\lsptoprule
\textit{mar-} &
`die' &
\centering {\textgreater} &
\textit{mhaaré-} &
`kill'\\
\textit{čar-} &
`graze' (\textit{itr}) &
\centering {\textgreater} &
\textit{čaaré-} &
`graze' (\textit{tr})\\
\textit{whe-} &
`come down' &
\centering {\textgreater} &
\textit{whaalé} &
`take down'\\
\textit{ukhé-} &
`come up' &
\centering {\textgreater} &
\textit{uk(u)aalé} &
`bring up'{\protect\footnotemark}\\\lspbottomrule
\end{tabularx}
\label{tab:8-irradd}
\end{table}

\footnotetext{The form without \textit{-u-} has been noted for Biori Palula.}


In the examples above, the semantics is not radically altered in the process. That, however, is not always the case, and the respective stems have sometimes, since being derived from others lived their own lives, so to speak, shifting semantic focus or developing secondary senses. Some have obviously drifted apart more than others, as can been seen in \tabref{tab:8-addsem}. 


\begin{table} 
\caption{Valency addition and semantic shifts}
\begin{tabularx}{\textwidth}{ l@{\hspace{30pt}} Q l@{\hspace{30pt}} l@{\hspace{30pt}} Q }
\lsptoprule
\textit{muč-} &
`rain' &
\centering {\textgreater} &
\textit{muč-á} &
`open, release'\\
\textit{phed-} &
`arrive, reach' &
\centering {\textgreater} &
\textit{phed-á} &
`send, take'\\
\textit{de-} &
`give' &
\centering {\textgreater} &
\textit{d-awá} &
`ask for'\\
\textit{ǰháan-} &
`know, recognise' &
\centering {\textgreater} &
\textit{ǰhaan-á} &
`wake up'\\
\textit{pal-} &
`hide' (\textit{itr}) &
\centering {\textgreater} &
\textit{pal-á} &
`steal, hide' (\textit{tr}) \\\lspbottomrule
\end{tabularx}
\label{tab:8-addsem}
\end{table}


As a~great number of verbs in the \textit{a}-ending class are either inherently \isi{transitive} (without any apparent \isi{intransitive} counterpart that it has been derived from) or are derived transitives, the final \textit{-á} has in itself become a~transitivity marker of some sort. The \isi{causative} or \isi{valency}"=increasing formation, as described above, can apply to \isi{intransitive} and \isi{transitive}, as well as stems already derived by the valancy"=increasing formation. 


\subsubsection*{Transitives derived from \isi{intransitive} verbs}

First, the formation can be applied to \isi{intransitive} verbs. Although this, strictly speaking, is not part of our focus here, we can see that the same construction is used regardless of the degree of voluntary control that the causee exercises: compare (\ref{ex:8-7}) with (\ref{ex:8-8}), (\ref{ex:8-9}) with (\ref{ex:8-10}), and (\ref{ex:8-11}) with (\ref{ex:8-12}).

\begin{exe}
\ex
\label{ex:8-7}
%modified
\gll hanú \textbf{su-áan-u} \\
\textsc{3msg.prox.nom} sleep-\textsc{prs"=msg} \\
\glt `He's sleeping.' (B:DHE1523)
\end{exe}
\begin{exe}
\ex
\label{ex:8-8}
%modified
\gll ma teeṇíi putr \textbf{su(w)a-(y)áan-u}  \\
\textsc{1sg.nom} \textsc{refl} son make.sleep-\textsc{prs"=msg} \\
\glt `I make my son sleep [I put my son to bed].' (B:DHE6695)
\end{exe}
\begin{exe}
\ex
\label{ex:8-9}
%modified
\gll ma \textbf{hansíl-u} \\
\textsc{1sg.nom} laugh.\textsc{pfv"=msg} \\
\glt `I laughed.' (A:PHN6002)
\end{exe}
\begin{exe}
\ex
\label{ex:8-10}
%modified
\gll míi tas \textbf{hanséel-i} \\
\textsc{1sg.gen} \textsc{3sg.acc} make.laugh.\textsc{pfv-f} \\
\glt `I made her laugh.' (A:HLE2546)
\end{exe}
\begin{exe}
\ex
\label{ex:8-11}
%modified
\gll se éeḍ-im bhíilam \textbf{khonḍíl-im} \\
\textsc{def} half-\textsc{fpl} fearfully speak.\textsc{pfv"=fpl} \\
\glt `The rest [the other half] of them [the women] spoke fearfully.' (A:BEZ022)
\end{exe}
\begin{exe}
\ex
\label{ex:8-12}
%modified
\gll eeṛé kúṛi teeṇíi dhi-á \textbf{khoṇḍéel"=im} \\
\textsc{rem} woman \textsc{refl} daughter-\textsc{pl} make.speak.\textsc{pfv"=fpl} \\
\glt `That woman made her daughters speak.' (A:HLE2553)
\end{exe}

While all the verb stems thus regularly derived causatively end up in one and the same morphological class (\tabref{tab:8-addcl}), namely the \textit{a}-ending L-class, the corresponding \isi{intransitive} verb could come from virtually any \isi{verb class}, although the most common source is the consonant"=ending class of L-verbs.


\begin{table} 
\caption{Valency addition and {verb class} membership}
\begin{tabularx}{\textwidth}{ l  P{10mm}  l  l  l  Q  l }
\lsptoprule
\textit{utráp-} &
`run' &
C-ending L &
\centering {\textgreater} &
\textit{utrapá-} &
`make someone run' &
\textit{a}-ending L\\
\textit{bheš-} &
`sit down' &
C-ending T &
\centering {\textgreater} &
\textit{bhešá-} &
`make someone sit, seat' &
\textit{a}-ending L\\
\textit{buuḍé-} &
`grow old' &
\textit{e}-ending L &
\centering {\textgreater}\par
&
\textit{buuḍá-} &
`make someone grow old (grieve)' &
\textit{a}-ending L\\
\textit{urbhí-} &
`fly'
&
{\textit{i}-ending T}
&
\centering {\textgreater}\par
&
\textit{urbhá-}
&
`fly (something), make something fly' &
{\textit{a}-ending L}
\\\lspbottomrule
\end{tabularx}
\label{tab:8-addcl}
\end{table}

\subsubsection*{Causatives derived from \isi{transitive} verbs}

The same process of \isi{valency addition} can also be applied to \isi{transitive} verbs. Here we start out with a~verbal event involving two participants (as in (\ref{ex:8-13})), a~\isi{subject} and a~\isi{direct object}. Through the process of causation we end up with a~verbal event involving three participants (as in (\ref{ex:8-14})): a) a~causer, b) a~manipulee (the person actually carrying out the verb act on behalf of the ultimate causer), and c) a~\isi{direct object}. The manipulee is marked as such with a~special marker, a~grammaticalised converb of \textit{ṣaawá-} `dress, turn on'. 


\ea
\label{ex:8-13}
%modified
\gll ǰaanɡul-á ma \textbf{bhanǰóol-u} \\
Jangul-\textsc{obl} \textsc{1sg.nom} beat.\textsc{pfv"=msg} \\
\glt `Jangul beat me.' (A:HUA120)
\end{exe}
\begin{exe}
\ex
\label{ex:8-14}
%modified
\gll tíi asáam ṣaawaá kučúru \textbf{bhanǰa-wóol-u}\\
\textsc{3sg.obl} \textsc{1pl.acc} \textsc{manip} dog beat-\textsc{caus.pfv"=msg}\\
\glt `She/He made us beat the dog.' (A:HLE2527)
\z

\largerpage
By the same process some \isi{transitive} verbs can also be made into the equivalent of \isi{ditransitive} verbs, like \textit{pilá-} `feed, give to drink (especially used with children and domestic animals as indirect objects)' in example (\ref{ex:8-15}), from \textit{pil-} `drink'. 

\ea
\label{ex:8-15}
%modified
\gll aníi wée wíi hín-u činaaróom aniniaám wíi keéna \textbf{pil-a-áan-u} thaní  \\
\textsc{3sg.prox.obl} in water be.\textsc{prs"=msg} Chinarom \textsc{3pl.prox.acc} water why.not drink-\textsc{caus"=prs"=msg} \textsc{quot} \\
\glt ` ``Here is water, why not have them [the goats] drink in Chinarom'', he said.' (A:PAS070)
\z

\subsubsection*{Second causatives}

It is also possible to form so"=called second causatives, with a~doubled \isi{causative} suffix \textit{-awá (-á + -á} with an~epenthetic \textit{-w-} in between), as in (\ref{ex:8-16}). Primarily its function is to derive causatives from derived transitives. Here the person uttering the sentence caused or manipulated his son to make the guests drink tea, thus the initiator of the action is one step further removed from the action as compared to (\ref{ex:8-15}), making the verb in (\ref{ex:8-16}) a~four"=argument verb. 


\ea
\label{ex:8-16}
\gll míi teeṇíi putr-á ṣaawaá teeṇíi práač-am čái \textbf{pil-a-wéel-i} {\upshape(<} pil-á-á-íl-i\upshape{)}\\
\textsc{1sg.gen} \textsc{refl} son-\textsc{obl} \textsc{manip} \textsc{refl} guest-\textsc{pl.obl} tea drink-\textsc{caus"=caus.pfv-f}\\
\glt `I had my son give the guests tea to drink.' (A:HLE2506)
\z


\subsection{Valency reduction}
\label{subsec:8-5-2}

Although seldom as elaborate as \isi{valency addition}, the corresponding \isi{valency}"=decreasing process is very similar. That, too, is regularly carried out by a~suffix, in this case an~accented \textit{-íǰ} added to the verb stem (or replacing the final vowel of an \textit{a}-ending verb stem), as can be seen in \tabref{tab:8-30}. The verb stems so derived are without exception part of the consonant"=ending L-class, and a~majority of the corresponding non"=derived verbs originate in the a-ending L-class.


\begin{table}[ht]
\caption{Regular {valency} reduction}
\begin{tabularx}{\textwidth}{ Q Q l Q Q }
\lsptoprule
\multicolumn{2}{ l}{Non"=derived stem} &
&
\multicolumn{2}{ l}{Stem derived by \isi{valency} reduction}\\\midrule
\textit{bilá-} &
`melt' (\textit{tr}) &
\centering {\textgreater} &
\textit{bil-íǰ-} &
`melt' (\textit{itr})\\
\textit{lamá-} &
`hang' (\textit{tr}) &
\centering {\textgreater} &
\textit{lam-íǰ} &
`hang' (\textit{itr})\\
\textit{ḍhanɡá-} &
`bury' &
\centering {\textgreater} &
\textit{ḍhanɡ-íǰ-} &
`be buried'\\
\textit{paš-} &
`see' &
\centering {\textgreater} &
\textit{paš-íǰ-} &
`be seen' \\
\textit{de-} &
`give' &
\centering {\textgreater} &
\textit{da-íǰ-} &
`be given'\\\lspbottomrule
\end{tabularx}
\label{tab:8-30}
\end{table}


The \isi{valency}"=decreasing suffix added to a~\isi{transitive} stem \textit{(bilá-)} thus derives an~\isi{intransitive} (or \isi{passive}) verb \textit{(bilíǰ-)}. Again, like with \isi{valency addition}, any \isi{inflectional} morphology occurs after the \isi{derivational} suffix: \textit{paš-áan-a} `are seeing' {\textgreater} \textit{pašiǰ-áan-a} `are seen', etc. (Compare the \isi{transitive} (\isi{perfective}) verb stem \textit{ḍhanɡóol-} `bury' in (\ref{ex:8-17}) with its corresponding \isi{valency}"=descreased stem \textit{ḍhanɡiǰíl-} `be buried' in (\ref{ex:8-18})).

\ea
\label{ex:8-17}
%modified
\gll qadím ṭópa dít-ii pahúrta tas \textbf{ḍhanɡóol-u} \\
Qadim down give.\textsc{pptc"=gen} after \textsc{3sg.acc} bury.\textsc{pfv"=msg}\\
\glt `After he had knocked down Qadim, he buried him.' (A:SHA037) 
\ex
\label{ex:8-18}
%modified
\gll aaṣṭ=bhiš-á kaal-á pahúrta míi háat-a ba se  \textbf{ḍhanɡiǰíl-i}\\
eight=twenty-\textsc{pl} year-\textsc{pl} after \textsc{1sg.gen} hand-\textsc{obl} \textsc{top} \textsc{3fsg.nom} be.buried.\textsc{pfv-f}\\
\glt `160 years later she was buried by my hand.' (A:PAS133)
\z


Historically \citep[316--317]{masica1991}, the \textit{íǰ}"=suffix undoubtedly goes back to an~\iliOIA \isi{passive} in \textit{-yá}, which in some \iliMIA dialects was phonetically strengthened to \textit{-iǰǰa}. A similar form is found in a~number of modern IA languages, e.g., \textit{-íiǰ} in Gilgiti \iliShina (\citeauthor{radloffshakil1998} w. Shakil \citeyear[116]{radloffshakil1998}). There are also a~number of \isi{intransitive} verbs in Palula with a \textit{ǰ}-element which may originally have been derived from \isi{transitive} verb stems, but for which the \isi{valency}"=reduced member is now either the sole survivor, or has due to a~semantic shift lost its productive connection with a~\isi{transitive} counterpart. There is no \isi{valency}"=reducing process comparable to the doubled \isi{causative} suffix.


\section{Complex predicates}
\label{sec:8-6}


Two other constructions (which were alluded to in the beginning of this chapter) that in some sense also may be described as \isi{derivational}, although not morphological in a~formal sense, are the ``\isi{conjunct verb}'' and the ``\isi{compound verb}''. Although two separate phenomena, they are similar in that they are both typically (but not exclusively) \iliIndoAryan, and both are complex constructions in the sense that they use ``building material'' from more than one lexeme and still in many ways function lexically like a~single verb stem. The two terms are in themselves confusing and prone to be mixed up, and possibly it would make more sense if they were swapped for one another (as already pointed out by \citealt[326]{masica1991}), but in order to avoid adding even more confusion to the discussion, I will primarily use what has become~-- more or less~-- standard terminology in South Asian linguistics, but also under each heading mention and discuss some alternative terms that are also widely used, some of them for the same or closely related phenomena elsewhere in the world.


\subsection{Conjunct verbs}
\label{subsec:8-6-1}

\textit{Conjunct verbs} are complex constructions that function as lexical units, usually consisting of a~verb preceded by a~\isi{noun} or an~\isi{adjective} (but also words from other parts of speech are possible). There are also some cases where it is not obvious to what part of speech the non"=verb component (which will be referred to as a \textit{host}) belongs, sometimes because it does not occur in the language outside of this particular \isi{conjunct verb} construction. 



The verb in such a~construction comes from a~small set of verb stems that I will refer to as \textit{verbalisers}. This \isi{verbaliser} does not contribute to the construction with much more than the general ``verbness'' (including the ablility to be inflected), whereas it is the host that tends to contribute the main semantic content to the complex. The verb is here merely a ``dummy'' without any semantic weight of its own. Some conjunct verbs formed through such host"=\isi{verbaliser} combinations are exemplified in \tabref{tab:8-31}. Usually the \isi{host element} occurs immediately preceding the \isi{verbaliser}, as can be seen in examples (\ref{ex:8-19})--(\ref{ex:8-20}).


\begin{table}[ht]
\caption{Derivations of conjunct verbs}
\begin{tabularx}{\textwidth}{ l l l l l l Q }
\lsptoprule
Host &
&
\multicolumn{2}{l}{Verbaliser} &
&
\multicolumn{2}{l}{Conjunct verb} \\\midrule
\textit{káaṇ} &
`ear' (\textsc{nn}) &
\textit{the-} &
`do' &
{\textgreater} &
\textit{káaṇ thíilu} &
`listened'\\
\textit{tanɡ} &
`narrow' (\textsc{adj}) &
\textit{the-} &
`do' &
{\textgreater} &
\textit{tanɡ thíila} &
`troubled'\\
\textit{široó} &
`start' (\textsc{nn}) &
\textit{the-} &
`do' &
{\textgreater} &
\textit{široó thíilu} &
`started'\\
\textit{rhoó} &
`song' (\textsc{nn}) &
\textit{de-} &
`give' &
{\textgreater} &
\textit{rhoó dítu} &
`sang'\\
\textit{dhreéɡ} &
-- &
\textit{de-} &
`give' &
{\textgreater} &
\textit{dhreéɡ dítu} &
`stretched out'\\
\textit{ašáq} &
`love' (\textsc{nn}) &
\textit{bhe-} &
`become' &
{\textgreater} &
\textit{ašáq bhíli} &
`fell in love'\\
\textit{teér} &
`lapsed' (\textsc{adj}) &
\textit{bhe-} &
`become' &
{\textgreater} &
\textit{teér bhíla} &
`went by'\\\lspbottomrule
\end{tabularx}
\label{tab:8-31}
\end{table}


\begin{exe}
\ex
\label{ex:8-19}
%modified
\gll tíi /{\ldots}/ tanaám bíiḍ-a \textbf{tanɡ} \textbf{thíil-a} \textbf{hín-a} \\
\textsc{3sg.ob} {} \textsc{3pl.acc} much-\textsc{pl} narrow do.\textsc{pfv"=mpl} be.\textsc{prs"=mpl} \\
\glt `He has troubled them a~lot.' (A:KIN003)
\end{exe}
\begin{exe}
\ex
\label{ex:8-20}
%modified
\gll se phaí se mac̣hook-á the \textbf{ašáq} \textbf{bhíl-i} \\
\textsc{def} girl \textsc{def} Machoke-\textsc{obl} to  love become.\textsc{pfv-f} \\
\glt `The girl fell in love with this Machoke.' (A:MAA005)
\end{exe}

The three verbalisers exemplified in the table, `do', `give' and `become', are the only ones that seem to be used productively to construct conjunct verbs (although there are a~few infrequent combinations with other verbs that come close). It is, however, no mere coincidence that we find these particular verbs in the ``dummy'' role; as we shall see, they are functionally precisely what it takes. 


So, what is the function? First, it is a~\isi{derivational} process, deriving verbs from other parts of speech, primarily~-- but not exclusively~-- from nouns. A wide range of phenomena that are lexicalised as nouns, adjectives or something else can thus easily be depicted as events in Palula, as illustrated in examples (\ref{ex:8-21}) and (\ref{ex:8-22}). 

\begin{exe}
\ex
\label{ex:8-21}
%modified
\gll be teeṇíi maǰí \textbf{ǰarɡá} \textbf{th-íia} \\
\textsc{1}\textsc{pl.nom} \textsc{refl} inside council do-\textsc{1pl} \\
\glt `We will discuss it among ourselves.' (A:MAR015)
\end{exe}
\begin{exe}
\ex
\label{ex:8-22}
%modified
\gll kuṛíina ta támbul"=am"=ii ǰ-íin \textbf{rhoo-á} \textbf{d-íin} \\
woman.\textsc{pl} \textsc{cntr} drum-\textsc{pl.obl"=gen} beat-\textsc{3pl} song-\textsc{pl} give-\textsc{3pl}\\
\glt `The women were beating the drums and singing.' (A:JAN034)
\end{exe}

That is further achieved in an~economical way, limited strictly to only a~few \isi{inflectional} paradigms. An extended advantage of this ``verbal flexibility'' is the relative ease with which entirely new verbs can be made or brought into the language, sometimes as totally novel constructions, sometimes as calques of \isi{conjunct verb} constructions in languages of wider communication. This does not necessarily mean that a~new verb will be made just to fill a~gap or to replace one that is already there; instead these constructions could elaborate on or refine the meaning of other verbs. We also find, perhaps not surprisingly, a~substantial number of loan words (primarily from \iliUrdu or \iliPashto) in the host slot of conjunct verbs, such as \textit{daawát} and \textit{muqaabilá} in (\ref{ex:8-23}).

\ea
\label{ex:8-23}
%modified
\gll deeúl-ii xálak"=am \textbf{daawát} \textbf{dít-i} atsharíit"=am the ki \textbf{muqaabilá} \textbf{th-íia}\\
Dir-\textsc{gen} people-\textsc{pl.obl} invitation give.\textsc{pfv-f} Ashreti-\textsc{pl.obl} to that contest do-\textsc{1pl}\\
\glt `The people from Dir invited the the Ashreti people to have a competition with them.' (A:CHA001)
\z

But also lexical material from a~structurally and culturally very different language like \iliEnglish can thus be made into ``new'' verbs (usually via \iliUrdu, but people with direct access to \iliEnglish have no problem applying it on the spot), as exemplified in (\ref{ex:8-24}) and (\ref{ex:8-25}).

\begin{exe}
\ex
\label{ex:8-24}
%modified
\gll c̣hatróol-a the bi \textbf{fuúṇ} \textbf{thíil-i} de\\
Chitral-\textsc{obl} to also phone do.\textsc{pfv"=f} \textsc{pst}\\
\glt `I also phoned Chitral.' (A:CHN070403)
\end{exe}
\begin{exe}
\ex
\label{ex:8-25}
\gll čúur reet-í ǰheez-íi fláiṭ na bhíl-i  hín-i aáǰ bi \textbf{kansál} \textbf{bhíl-i}\\
four night-\textsc{pl} airplane-\textsc{gen} flight \textsc{neg} become.\textsc{pfv-f}  be.\textsc{prs-f} today also cancelled become.\textsc{pfv-f}\\
\glt `There have been no flights for four days, and also today it was cancelled.' (A:CHN070110)
\end{exe}

Having this type of mechanism, we need to answer the question why Palula needs three different ``dummy'' verbs when one would be the most economical. The answer lies in what the verb contributes in addition to its ``verbness'', namely \isi{valency}: \textit{bhe-} `become' provides an~\isi{intransitive} frame, \textit{the-} `do' a~\isi{transitive}, and \textit{de-} `give' a~\isi{transitive} frame with place for an~\isi{oblique} object. 


There are, however, some complexities connected with \isi{argument structure}, and to what extent the \isi{host element} takes part in it; this will be discussed further on (see \sectref{subsec:12-2-8}). This, and related issues, is also described at greater length in a separate article \citep{liljegren2010}.


\subsection{Compound verbs}
\label{subsec:8-6-2}

The other strategy for expanding the verb lexicon, commonly found in IA languages, uses two verb stems, a~construction sometimes referred to in South Asian linguistics \citep[326]{masica1991} as the \textit{compound verb} construction. The first verb stem (\textsc{verb1}), occurring in a~non"=\isi{finite} form, carries the main semantic content, while the second (\textsc{verb2}), drawn from a~small set of verbs, carries the inflections as a~\isi{finite} verb. The second verb is usually \isi{subject} to semantic bleaching but without being fully grammaticalised. The second component (\textsc{verb2}) has variously been referred to as a \textit{vector}, an \textit{intensifier}, an \textit{operator} and an \textit{explicator}. I will use \textit{vector}. This is certainly not a~verb \isi{derivation} in the same sense as the \textit{conjunct verb} construction (although for instance \citeauthor{butt1993} (\citeyear[31]{butt1993}; \citeyear[49]{butt2010}) considers the \textsc{verb2} of compound verbs as well as the \isi{verbaliser} of conjunct verbs as ``light verbs'', a~particular class of verbs being used in forming complex predicates, whether the other~-- in IA, preceding~-- component in the complex is a~verb, a~\isi{noun} or an~\isi{adjective}). \citet[326--330]{masica1991} argues that the construction primarily functions as a~specification of Aktionsart as far as IA languages are concerned, whereas in other descriptions the relexicalisation taking place is being emphasised, i.e., the meaning of the complex is not really predictable from its components but must be learnt \citep[143]{schmidt1999}. Although a~true innovation of IA languages (\citealt[326]{masica1991}; \citealt{hook1977}) in more recent times (in \iliNIA, or possibly \iliMIA), it has several parallels in other parts of the world \citep[348--349]{hook1977}, where they have sometimes been labelled \textit{serial verb constructions} \citep{ansaldo2006}, as well as in other parts of Asia \citep[559]{ebert2006}. 



\citet[20]{schmidt2004b} points out that the \isi{compound verb} construction is a~far more marginal feature in Kohistani \iliShina than in \iliUrduHindi and \iliPunjabi, and my present assumption is that the same is true of Palula, but this issue will need further research. The inventory of vectors commonly include: 1. directionals (`go', `come'), 2. disposals (`throw', `send', `put aside'), 3. verbs expressing suddenness or unexpectedness (`fall', `rise'), and 4. auto- and other benefactives (`give', `take'). Only categories 1 and 2 seem to appear in what can be termed as possible examples of compound verbs in Palula, as shown by the examples (\ref{ex:8-26}) and (\ref{ex:8-27}), but even then relatively infrequently. 

\begin{exe}
\ex
\label{ex:8-26}
%modified
\gll bhun whaí ba \textbf{uḍheew-í} \textbf{wháat-u} \\
down get.down.\textsc{cv} \textsc{top} flee-\textsc{cv} get.down.\textsc{pfv"=msg} \\
\glt `He got down [from the tree] and escaped [downhill].' (B:CLE377)
\end{exe}
\begin{exe}
\ex
\label{ex:8-27}
%modified
\gll muṭ-á sanɡí so amzarái seéb /{\ldots}/ \textbf{ɡhaṇḍ-í} \textbf{ɡaíl-u} \\
tree-\textsc{obl} with \textsc{def} lion sir {} tie-\textsc{cv} throw.\textsc{pfv"=msg} \\
\glt `[He] tied up Mr. Lion to a~tree.' (A:KIN023)
\end{exe}

The first verb occurs as a~non"=\isi{finite} converb, and carries the main semantic content, while the second is a~regularly inflected \isi{finite} verb but is semantically light when considering the meaning of the sentence as a~whole. It is, however, not always easy to make a~clear differentiation between possible instances of compound verbs and the regular use of a~\isi{Converb}, the latter in which it is the head of a~dependent \isi{clause}. 



I am in no position to say whether this is a~construction on the rise, due to contact with lowland languages where this is a~conspicuous feature, or rather is one dwindling in importance and productivity. It should be noted that the construction also exists in neighbouring \iliKalasha, where it reinforces already morphologically expressed inferentiality contrasts \citep[1--4]{bashir1993}.