\chapter{Grammatical relations}
\label{chap:11}

Palula, like many related IA languages, displays an~intricate and quite complex relationship between the grammatical cases expressed, the particular case forms and agreement patterns available and the various functions a~\isi{noun} \isi{phrase} can have in a~given utterance \citep[230--231]{masica1991}. A complicating factor is the type of split ergativity displayed, related to aspect on the one hand and the nature of the NP on the other. The former is a~rather straightforward matter, with ergativity being a~feature only of \isi{perfective} clauses, whereas the relationship between properties of the NP and ergativity is less transparent, with several different cutoff points, some of them less expected from a~typological standpoint. Word order is unmarkedly \isi{intransitive} \isi{subject}~-- verb and \isi{transitive} \isi{subject}~-- object~-- verb, but it allows for quite a~deal of pragmatic flexibility. As the two other factors weigh heaviest as far as \isi{grammatical relations} and \isi{alignment} are concerned, word order will not enter into the present discussion. This also goes along with the general observation that the presence of the other mechanisms in a~language correlates to a~relative flexibility in the basic word order \citep[14--15]{blake2001}.



Following \citet[6--8]{dixon1994}/\citet[76--77]{dixon2010} and \citet[402]{bickel2011}, I will be using the following abbreviations for the three \isi{grammatical relations} (also syntactic primitive relations): S~-- \isi{intransitive} \isi{subject}, A~-- \isi{transitive} \isi{subject}, and O~-- \isi{transitive} object.\footnote{O corresponds to P (for patient) in some of the typological literature \citep{comrie1989,croft2003}.\par } This means that a~verb with one core \isi{noun} \isi{phrase} is \isi{intransitive} and the sole argument relation we refer to as S, a~verb with two core \isi{noun} phrases is \isi{transitive} and the two argument relations we call A and O. In Palula, the unmarked order of the arguments is S-V\textit{itr} and A-O-V\textit{tr}, respectively.


\section{Verb agreement}
\label{sec:11-1}

As already described in \sectref{subsec:8-4-1}, \isi{grammatical relations} can be reflected in the marking of the \isi{predicate} itself, i.e., by \isi{verb agreement}. In Palula, the verb can only display agreement with a~single argument in one and the same \isi{clause}. While agreement in person is limited to the non"=present \isi{imperfective} (\isi{Future} and Past \isi{Imperfective}), the main type of agreement is one in \isi{gender} and number, found in the \isi{perfective} categories and in the \isi{Present} (see \sectref{sec:9-1}). However, as far as grammatical \isi{alignment} is concerned, the dividing line goes between the \isi{perfective} and all non"=\isi{perfective} TMA categories: In the \isi{perfective} there is \isi{ergative} \isi{verb agreement}, and in the non"=\isi{perfective} there is \isi{accusative} \isi{verb agreement}. 


\subsection{Accusative alignment}
\label{subsec:11-1-1}


In the non"=present \isi{imperfective} -- Past \isi{Imperfective} in (\ref{ex:11-1}) and \isi{Future} in (\ref{ex:11-2}) --, the verb always agrees in person in accordance with an~\isi{accusative} \isi{alignment}, i.e., with the \isi{subject}, whether S, as in the \isi{intransitive} \isi{clause} in (\ref{ex:11-1}), or A, as in the \isi{transitive} \isi{clause} in (\ref{ex:11-2}). 

\begin{exe}
\ex
\label{ex:11-1}
%modified
\glll ak praší phará \textbf{se} b-\textbf{éen} de \\
\textsc{idef} slope along \textsc{3pl.nom} go-\textsc{3pl} \textsc{pst} \\
{} {} {}  \textbf{S} \\
\glt `They were moving along a slope.' (B:AVA211)

\ex
\label{ex:11-2}
%modified
\glll \textbf{se} \textbf{zinaawur-aán} xu ma kh-\textbf{óon} \\
\textsc{def} beast-\textsc{pl} but \textsc{1sg.nom} eat-\textsc{3pl} \\
 \textbf{A} {} {}  \textbf{O} \\
\glt `The beasts will eat me.' (A:KAT059)
\end{exe}

An \isi{accusative agreement} pattern is also applied in the \isi{Present}, as can be seen in examples (\ref{ex:11-3})--(\ref{ex:11-4}), although here in the form of number and \isi{gender} agreement.

\begin{exe}
\ex
\label{ex:11-3}
%modified
\glll \textbf{ma} rhoošíia sóon-a the bi-áan-\textbf{u}\\
\textsc{1sg.nom} tomorrow pasture-\textsc{obl} to go-\textsc{prs-}\textsc{msg}\\
\textbf{S}\\
\glt `Tomorrow I will go to the high pasture.' (A:SHY028)

\ex
\label{ex:11-4}
%modified
\glll \textbf{tu} aniaám keé-na kha-áan-\textbf{u}\\
\textsc{2sg.nom} \textsc{3pl.prox.acc} why-\textsc{neg} eat-\textsc{prs-}\textsc{msg} \\
\textbf{A} \textbf{O}\\
\glt `Why don't you eat these?' (A:KAT067)
\end{exe}

\subsection{Ergative alignment}
\label{subsec:11-1-2}


When, on the other hand, any of the TMA categories based on the \isi{perfective} are used, the verb always agrees~-- in accordance with an~\isi{ergative alignment}~-- with S, as in (\ref{ex:11-5}), or O as in (\ref{ex:11-6}) and (\ref{ex:11-7}), whereas it never agrees in \isi{gender} and number with A.

\begin{exe}
\ex
\label{ex:11-5}
%modified
\glll \textbf{čhéeli} eetáa the \textbf{ɡíi} de ta\\
she.goat[\textsc{fsg}] there to go.\textsc{pfv.}\textsc{f} \textsc{pst} \textsc{sub}\\
 \textbf{S} \\
\glt `The goat had gone there.' (A:CAV026)

\ex
\label{ex:11-6}
%modified
\glll ínc̣-a \textbf{čhéeli} khéel-\textbf{i}\\
bear[\textsc{msg}]-\textsc{obl} she.goat[\textsc{fsg}] eat.\textsc{pfv-}\textsc{f}\\
\textbf{A} \textbf{O} \\
\glt `The bear ate the goat.' (A:PAS056)

\ex
\label{ex:11-7}
%modified
\glll kúṛi teeṇíi \textbf{ɡhúuṛu} deec̣hinéeti ḍáḍi nuuṭóol-\textbf{u}\\
woman[\textsc{fsg}] \textsc{refl} horse[\textsc{msg}] right side turn.\textsc{pfv"=msg}\\
\textbf{A} {} \textbf{O} {} {}   \\
\glt `The woman turned her horse to the right.' (A:UXW028)
\end{exe}


Even when O occurs in a non"=\isi{nominative} form (described in detail in \sectref{subsec:11-2-2}), as in (\ref{ex:11-7b}), the verb still agrees in \isi{gender} and number with that particular \isi{noun} \isi{phrase} argument.

\begin{exe}
\ex
\label{ex:11-7b}
%modified
\glll \textbf{tíi} /{\ldots}/ \textbf{tanaám} bíiḍ-a tanɡ thíil-\textbf{a} hín-\textbf{a} \\
\textsc{3sg.obl} {} \textsc{3pl.acc} much-\textsc{mpl} narrow do.\textsc{pfv}-\textsc{mpl} be.\textsc{prs}-\textsc{mpl} \\
\textbf{A} {} \textbf{O} \\
\glt `He has troubled them a lot.' (A:KIN003)
\end{exe}


\section{NP case differentiation}
\label{sec:11-2}

Three sub"=instances of case differentiation will be exemplified below, as they relate to \isi{grammatical relations} and \isi{alignment}: \isi{inflectional} case marking of nouns, \isi{pronominal} case differentiation and NP"=internal marking.


\subsection{Inflectional case marking}
\label{subsec:11-2-1}


As we saw in \sectref{sec:4-5}, the central case distinction made inflectionally is that between the \isi{nominative} and the \isi{oblique} cases. As for the relations we are interested in, the \isi{nominative} is used for S, A and O alike in the non"=\isi{perfective} categories, whereas in the \isi{perfective}, A, as in (\ref{ex:11-10}), is singled out as coded in the \isi{oblique} case (\textit{kaṭamuš-á}) versus the nominatively\footnote{In the description of a~system with a~consistently \isi{ergative} case \isi{alignment}, this would be referred to as absolutive, a~term I prefer not to use when describing the split \isi{ergative} system of Palula.} coded (\textit{kaṭamúš}) S in (\ref{ex:11-8}) and O in (\ref{ex:11-9}).

\begin{exe}
\ex
\label{ex:11-8}
%modified
\glll \textbf{kaṭamúš-\textsc{ø}} /{\ldots}/ sóon-a dúši ɡúum  hín-u \\
Katamosh {} pasture-\textsc{obl} toward go.\textsc{pfv.msg}  be.\textsc{prs"=msg} \\
\textbf{S} \\
\glt `Katamosh set out to the high pasture.' (A:KAT010)

\ex
\label{ex:11-9}
%modified
\glll íṇc̣-a \textbf{kaṭamúš-\textsc{ø}} aamúuṣṭ-u hín-u \\
bear-\textsc{obl} Katamosh forget.\textsc{pfv"=msg} be.\textsc{prs"=msg} \\
 \textbf{A} \textbf{O} \\
\glt `The bear forgot about Katamosh.' (A:KAT140)

\ex
\label{ex:11-10}
%modified
\glll \textbf{kaṭamuš-á} ɡábina khóol-u hín-u \\
Katamosh-\textsc{obl} nothing eat.\textsc{pfv"=msg} be.\textsc{prs"=msg} \\
 \textbf{A} \textbf{O} \\
\glt `Katamosh didn't eat anything.' (A:KAT065)
\end{exe}

This, however, is a~somewhat simplified picture. In actual fact, not all nouns make the distinction between the \isi{nominative} and the \isi{oblique}, and some make it in the plural and not in the singular. The forms themselves occurring as morphological markers of ergativity also differ between nouns belonging to different declensions (see \sectref{sec:4-6}). 



\begin{table}[p]
\caption{Case and number differentiation in the \textit{a}-\isi{declension} (\isi{perfective}): \textit{putr} `son' \textsc{(m)}} 
\begin{tabularx}{\textwidth}{ Q Q Q }
\lsptoprule
&
\textsc{sg} &
\textsc{pl}\\\midrule
S/O &
\textit{putr} &
\textit{putrá} \\
A &
\textit{putrá} &
\textit{putróom}\\\lspbottomrule
\end{tabularx}
\label{tab:11-adecl}
\end{table}


\begin{table}[p]
\caption{Case and number differentiation in the \textit{i}-\isi{declension} (\isi{perfective}): \textit{preṣ} `mother-in-law' \textsc{(f)}}
\begin{tabularx}{\textwidth}{ Q Q Q }
\lsptoprule
&
\textsc{sg} &
\textsc{pl}\\\midrule
S/O &
\textit{preṣ} &
\textit{preṣí} \\
A &
\textit{preṣí} &
\textit{preṣíim} \\\lspbottomrule
\end{tabularx}
\label{tab:11-idecl}
\end{table}


\begin{table}[p]
\caption{Case and number differentiation in the \textit{m}-\isi{declension} (\isi{perfective}): \textit{méemi} `grandmother' \textsc{(f)}}
\begin{tabularx}{\textwidth}{ Q Q Q }
\lsptoprule
&
\textsc{sg} &
\textsc{pl}\\\midrule
S/O/A &
\textit{méemi} &
\textit{méemim} \\\lspbottomrule
\end{tabularx}
\label{tab:11-mdecl}
\end{table}


\begin{table}[p]
\caption{Case and number differentiation in the \textit{aan}-\isi{declension}, V-ending (\isi{perfective}): \textit{baačaá} `king' \textsc{(m)}}
\begin{tabularx}{\textwidth}{ Q Q Q }
\lsptoprule
&
\textsc{sg} &
\textsc{pl}\\\midrule
S/O &
\textit{baačaá} &
\textit{baačaán} \\
A &
\textit{baačaá} &
\textit{baačaanóom} \\\lspbottomrule
\end{tabularx}
\label{tab:11-aanvdecl}
\end{table}

\begin{table}[p]
\caption{Case and number differentiation in the \textit{aan}-\isi{declension}, C-ending (\isi{perfective}): \textit{anɡreéz} `Brit' \textsc{(m)}}
\begin{tabularx}{\textwidth}{ Q Q Q }
\lsptoprule
&
\textsc{sg} &
\textsc{pl}\\\midrule
S/O &
\textit{anɡreéz} &
\textit{anɡreezaán} \\
A &
\textit{anɡreezá} &
\textit{anɡreezaanóom} \\\lspbottomrule
\end{tabularx}
\label{tab:11-aancdecl}
\end{table}


Although there is a~form syncretism between the \isi{oblique} singular and the \isi{nominative} plural in the large \textit{a}- and \textit{i}-declensions (\tabref{tab:11-adecl} and \tabref{tab:11-idecl}), that does not distort the \isi{nominative}"=\isi{oblique} contrast \textit{per se}. Here, a~suffix \textit{-a} or \textit{-i} is the morphological reflex of ergativity in the singular and a~suffix -\textit{óom} or -\textit{íim} in the plural. In the \textit{m}-\isi{declension} (\tabref{tab:11-mdecl}), on the other hand, there is no differentiating \isi{ergative} case marking available at all. For some \textit{aan}-\isi{declension} nouns (\tabref{tab:11-aanvdecl}), the case differentiation is neutralised in the singular but maintained in the plural, while for a~few others (\tabref{tab:11-aancdecl}) there are four distinct forms available: \isi{nominative} singular, \isi{nominative} plural, \isi{oblique} singular and \isi{oblique} plural. The \textit{ee}-\isi{declension} nouns (not displayed here) largely make the same distinctions as \textit{a}- and \textit{i}-\isi{declension} nouns. With respect to frequency, the ``full contrast pattern'' represents the large majority of all Palula nouns (the \textit{a}-\isi{declension} and the \textit{i}-\isi{declension} together making up 70 per cent of all nouns), masculine as well as feminine, while the ``no contrast pattern'' is relatively small (about sixteen per cent), comprising exclusively feminine nouns.



\subsection{Pronominal case differentiation}
\label{subsec:11-2-2}


The \isi{noun} \isi{phrase} slot could of course also be filled with a~\isi{pronoun}, and here too we have different forms bearing a~relation to case. Apart from the singling out of A in the \isi{perfective}, \textit{asím} in (\ref{ex:11-12}), we also have \isi{pronominal} forms particular to O, \textit{asaám} in (\ref{ex:11-13}), vis-à-vis S, \textit{be} in (\ref{ex:11-11}) and A (in the \isi{perfective} and the non"=\isi{perfective} alike).

\begin{exe}
\ex
\label{ex:11-11}
%modified
\glll rhootašíia \textbf{be} ɡíia  \\
morning \textsc{1pl.nom} go.\textsc{pfv.pl}  \\
 {} \textbf{S} \\
\glt `In the morning we left.' (A:GHA006)

\ex
\label{ex:11-12}
%modified
\glll \textbf{asím} ǰinaazá khaṣeel-í wheelíl-u de \\
\textsc{1pl.erg} corpse drag-\textsc{cv} take.down.\textsc{pfv"=msg} \textsc{pst}  \\
 \textbf{A} \textbf{O} \\
\glt `We dragged the corpse down.' (A:GHA044)

\ex
\label{ex:11-13}
%modified
\glll karáaṛu \textbf{asaám} kh-úu \\
leopard \textsc{1pl.acc} eat-\textsc{3sg}\\
\textbf{A} \textbf{O} \\
\glt `The leopard will eat us.' (B:FOY025)
\end{exe}

This, again, is only part of the whole picture. Starting with the personal pronouns proper, these do not uniformly have the same number of forms or make the same distinctions formally, as can be seen in \tabref{tab:11-1}. The first- and second"=plural personal pronouns make a~three"=way distinction, with unique \isi{ergative} forms, \textit{asím} and \textit{tusím}, whereas the first- and second"=singular only have two forms each (with \isi{nominative}"=\isi{accusative} neutralisation as well as an~\isi{ergative}"=\isi{genitive} neutralisation). The demonstratives, which are used as third"=person pronouns, see \tabref{tab:11-2}, show differentiation to the same extent as the plural personal pronouns, i.e., with a~three"=way case contrast. 


\begin{table}[ht]
\caption{Personal pronouns and case differentiation in the perfective}
\begin{tabularx}{.5\textwidth}{ l@{\hspace{15pt}} Q l@{\hspace{15pt}} Q }
\lsptoprule
& O &
\multicolumn{1}{l}{S} &
 A\\\midrule
\textsc{1sg} &
 \textit{ma} &
 \textit{ma} &
 \textit{míi} \\
\textsc{2sg} &
 \textit{tu} &
 \textit{tu} &
 \textit{thíi} \\
\textsc{1pl} &
 \textit{asaám} &
 \textit{be} &
 \textit{asím} \\
\textsc{2pl} &
 \textit{tusaám} &
 \textit{tus} &
 \textit{tusím} \\\lspbottomrule
\end{tabularx}
\label{tab:11-1}
\end{table}


\begin{table}[ht]
\caption{Demonstrative case differentiation in the \isi{perfective} (only the \isi{remote} set represented)}

\begin{tabularx}{.5\textwidth}{ l@{\hspace{15pt}} Q l@{\hspace{15pt}} Q }
\lsptoprule
&
 O &
\multicolumn{1}{ l}{ S} &
 A\\\midrule
\textsc{sg} &
 \textit{tas} &
 \textit{so} \textsc{(m)} / \textit{se} \textsc{(f)} &
 \textit{tíi} \\
\textsc{pl} &
 \textit{tanaám} &
 \textit{se} &
 \textit{taním} \\\lspbottomrule
\end{tabularx}
\label{tab:11-2}
\end{table}


\subsection{NP"=internal marking}
\label{subsec:11-2-3}


Case marking is also relevant for dependents in the \isi{noun} \isi{phrase}, although it has a~much more limited scope (for details, see \sectref{sec:10-3}). There are two kinds of NP"=internal agreement: a) \isi{determiner agreement} and b) \isi{adjectival agreement}. Determiners occur in a~maximum of two forms, one of them occurring only with a~singular masculine \isi{noun} head in the \isi{nominative}, the other occurring elsewhere, i.e., with non"=\isi{nominative} singular heads, feminine and plural heads. Adjectives of the inflecting category display three different forms to reflect properties of the \isi{noun} head: one for a~singular masculine head in the \isi{nominative}, another for non"=\isi{nominative} singular and plural masculine head, and a~third for feminine heads, regardless of case or number. The agreement displayed by dependents within the \isi{noun} \isi{phrase} is therefore not adding anything to the differentiation already made explicit by the case"=inflected head as far as case is concerned. 


\section{The split system summarised}
\label{sec:11-3}


Summarising the findings in \sectref{sec:11-1}--\sectref{sec:11-1}, we have two dimensions on which \isi{ergative} vs. \isi{accusative} \isi{alignment} and their expressions depend in Palula. First and foremost, the presence of \isi{ergative alignment} is aspectually determined. While \isi{accusative} properties can be present regardless of the TMA category realised, it is in the \isi{perfective} only that an (additional) \isi{ergative} pattern is found. A consistent correlation (see \figref{fig:11-1}) exists between \isi{perfective} aspect and \isi{ergative} \isi{verb agreement}, and an~\isi{accusative} \isi{verb agreement} and non"=\isi{perfective} categories. 

\begin{figure}[ht]
\begin{tabularx}{\textwidth}{ l@{\hspace{25pt}} C C C @{\hspace{25pt}}l }
\midrule
Aspect &
 A &
 S &
 O &
Alignment\\\midrule
Non"=\isi{perfective} &
\ligrcell{~}
& \ligrcell{~}
&
&
Accusative\\
Perfective &
& \ligrcell{~}
& \ligrcell{~}
&
Ergative\\\midrule
\end{tabularx}
% \edcomm[inline]{Wäre das nicht besser als Tabelle?}
% \authcomm[inline]{Ja, vielleicht. Irgendwelche Vorschläge?}
\caption{Correlations between aspect and \isi{alignment} in \isi{verb agreement} (shading represents \isi{verb agreement})}
\label{fig:11-1}
\end{figure}


Much less straightforward is the relationship between the nature of the NP and case marking. Even within the same aspectual category (the \isi{perfective}) we have examples of non"=differentiation (ASO all the same as far as case marking is concerned), a~two"=way differentiation (A marked differently from S and O) as well as a~tripartite differentiation (A, S and O all distinguished by case marking). \tabref{tab:11-3} illustrates how case differentiation is displayed for four different categories of NPs in Palula: 

\begin{enumerate}
\item \textsc{Pronoun1} are the first- and second"=person plural as well as all the third"=person pronouns; they display a~tripartite subsystem. 
\item \textsc{Noun1} are all the nouns that make a~\isi{nominative}/\isi{oblique} distinction, and \textsc{Pronoun2} are the first- and second"=singular pronouns; they display an \isi{ergative} subsystem. 
\item \textsc{Noun2} are the nouns that do not make a~\isi{nominative}/\isi{oblique} distinction; they display a~neutral subsystem.
\end{enumerate}
 
\begin{table}[ht]
\caption{Morphologically realised case distinctions related to \isi{grammatical relations} (The case marking below the dotted line applies in the \isi{perfective} only. In the non"=\isi{perfective}, A is treated like S)}

\begin{tabularx}{\textwidth}{ l l@{\hspace{14pt}}  Q  l@{\hspace{14pt}}  Q }
\lsptoprule
&
\multicolumn{1}{ l}{\textsc{Pronoun1}} &
\multicolumn{1}{ l}{\textsc{Noun1}} &
\multicolumn{1}{ l}{\textsc{Pronoun2}} &
\textsc{Noun2}\\\hline
S &
Nominative &
Nominative &
Nominative &
Nominative\\
O &
Accusative &
Nominative &
Nominative &
Nominative\\\cdashline{2-5}
A &
Ergative/\isi{oblique} &
Oblique &
Ergative/\isi{genitive} &
Nominative\\\lspbottomrule
\end{tabularx}
\label{tab:11-3}
\end{table}


\section{Alignment and split features in the region and beyond}
\label{sec:11-4}
 
How do the features summarised above relate to those found in the surrounding region and in related languages? As far as the presence of (morphological) ergativity is concerned, the situation in Palula is far from unique, neither among \iliNIA languages in the region \citep{edelman1983,skalmowski1974,liljegren2014} nor beyond \citep{deosharma2006,klaiman1987,stronski2009,verbeke2011}, but its manifestations and more precise characteristics are manifold and quite diverse in what \citet[250]{masica2001} describes as an ``\isi{ergative} belt'' stretching from the north"=eastern part of the subcontinent all the way to Caucasus, with modern \iliPersian as one major exception in the middle of it. This belt includes \iliIndoAryan, Iranian and Tibeto"=Burman, as well as the isolate \iliBurushaski and some of the language families represented in the Caucasus. 



While \isi{ergative}  case marking is conditional in Palula, it is applied across the board in \iliBurushaski and in the \iliShina varieties spoken adjacent to it (as far as case marking is concerned\footnote{As pointed out to me by Carla Radloff (pc), there is in Gilgit \iliShina a~clearcut split between accusatively aligned \isi{verb agreement} and ergatively aligned case marking (invariably with \textit{-se} or \textit{-s}).}). Although the latter is due to substratum effects from \iliBurushaski according to \citet[248]{masica2001}, another phenomenon, termed ``dual ergativity'' by \citet[213]{hookkoul2004}, is observed in certain Eastern (including Kohistani) \iliShina varieties, where there is a~TMA"=related (\isi{imperfective} vs. \isi{perfective} according to \citealt[51--53]{schmidtkohistani2008}) alternation between ergativity markers of IA origin and an~ergativity marker supposedly of \iliTibetan origin (\citealt[214]{hookkoul2004}; \citealt[211]{bailey1924}). 


\largerpage
However, far more common in \iliNIA languages as well as in Tibeto"=Burman, is some sort of TMA split between \isi{ergative} patterns and \isi{accusative} patterns \citep[248]{masica2001}, usually between \isi{perfective} and non"=\isi{perfective} tenses \citep[342--343]{masica1991}. This may be manifested, as in Palula, in the agreement of \isi{transitive} verbs with O in the \isi{perfective} along with a~distinctive case marking of A. That is the case in \iliUrduHindi \citep[124]{schmidt1999} as well as in many of the other major \iliNIA languages of the subcontinent \citep[248]{masica2001}. Geographically closer to Palula, this is also observed for the Kohistani languages (\citealt[136]{baart1999a}; \citealt[34]{hallberghallberg1999}; \citealt[93--95]{lunsford2001}\footnote{In the case of \iliTorwali, ergativity is also manifested in the future \isi{tense}.}). While ergativity is seen in the case marking of A in other \iliShina varieties, \isi{verb agreement} with O is not a~feature of Gilgiti or Kohistani \iliShina. Instead, as in neighbouring \iliDameli and \iliGawarbati (personal observations), the verb agrees (accusatively) with S or A and never with O, whether or not there are other manifestations of ergativity or accusativity.



A number of Iranian languages in the region also display split ergativity \citep{payne1980}, although with certain peculiarities. For one, \iliPashto exemplifies a~\isi{tense} split rather than an~aspectual split, with \isi{verb agreement} with O and \isi{ergative} case marking of A in past tenses, regardless of aspect (\citealt[4--5]{tegey1977}; \citealt[71--72]{lorenz1979}). In addition, a~class of \isi{intransitive} verbs (expressing involuntary activity) also require an~ergatively marked A \citep[112]{babrakzai1999}, a~phenomenon also described by \citet[217]{hookkoul2004} for \iliIndoAryan \iliKashmiri.\footnote{This is also found in \iliUrdu to a~limited extent \citep[168]{schmidt1999}.} 



While most languages within this so"=called \isi{ergative} belt show some \isi{ergative} features, there are nevertheless some where they are entirely absent. In the immediate vicinity of Palula, the most notable examples are \iliKalasha and \iliKhowar \citep[41]{bashir1988}. While this absence is a~retention feature of \iliKalasha and \iliKhowar, in some languages in other parts of the subcontinent, such as Bengali, Oriya and \iliSinhalese, a~former \isi{ergative} construction has probably been replaced only later by a~consistent \isi{accusative} \isi{alignment} \citep[343--344]{masica1991}.



A number of different patterns are observed in the region as far as case marking, case syncretism and various types of NP splits being realised. Those languages manifesting \isi{verb agreement} with O in some TMA categories, also tend to case mark A distinctively vis-à-vis S and O, but there are also those languages that maintain a~tripartite S vs. O vs. A differentiation, if not for nouns, at least for the pronouns or a~subset of them. In \iliPunjabi, there is a~shared \textsc{1sg} \isi{nominative}/\isi{oblique} form, whereas \textsc{2sg}, \textsc{1pl}, as well as \textsc{2pl} and the third"=person pronouns differentiate between these two cases \citep[229]{bhatia1993}. In \iliGawri, the \textsc{1sg} and \textsc{2sg} make a~\isi{subject} vs. object/\isi{oblique}/agent distinction, the \textsc{1pl} and \textsc{2pl} a~\isi{subject}/agent/object vs. \isi{oblique} distinction, a~\isi{subject} vs. agent vs. object/\isi{oblique} distinction in the \textsc{3sg}, and a~\isi{subject}/object/\isi{oblique} vs. agent distinction in the \textsc{3pl} \citep[39]{baart1999a}. 



In neighbouring \iliDameli (\citealt{morgenstierne1942} and own observations), there is a~\isi{nominative} vs. \isi{accusative}/\isi{ergative} differentiation\footnote{This rather unexpected grouping of A with O vs. S is sometimes referred to as ``double"=\isi{oblique}'' \citep{payne1980}.} in first- and second"=singular as well as in plural, but where \textsc{1sg} and \textsc{1pl} \isi{nominative} somewhat surprisingly have merged. For the demonstratives functioning as third"=person pronouns, the situation is further complicated by animacy distinctions. In \iliGawarbati (\citealt{morgenstierne1950} and personal observations), there seems to be an~almost complete \isi{nominative} vs. \isi{accusative} vs. \isi{ergative} differentiation upheld in all persons and in singular and plural (with nouns and pronouns alike), but only in so far as the NPs are de\isi{finite} and occur in the \isi{perfective}. However, due to the lack of a~more comprehensive study of the language sufficient to base any conclusions on, the analysis remains tentative. 
 

For a more comprehensive treatment of \isi{alignment} patterns and areality in the region, see \citet{liljegren2014}. 