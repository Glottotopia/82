\chapter{Typological overview}
\label{chap:2}

This section is a~brief overview covering the most central features of Palula. For more in"=depth coverage of each topic, and for information on those not explicitly covered here, the later chapters will need to be consulted.


\section{Phonology}
\label{sec:2-1}


With its 32--37 members, the Palula consonant inventory is moderately large to large \citep[10--13]{maddieson2005a}. There are five basic places of articulation (labial, dental, retroflex, palatal and velar), with a~voicing contrast in the plosive and fricative sets, and an~aspiration contrast in the plosive and affricate sets: 


\begin{table}[H]
\begin{tabular}{ l@{\hspace{20pt}} l@{\hspace{20pt}} l@{\hspace{20pt}} l@{\hspace{20pt}} l@{\hspace{20pt}} l@{\hspace{20pt}} l@{\hspace{20pt}} }
/p &
t &
ʈ &
&
k &
(q) &
\\
pʰ &
tʰ &
ʈʰ &
&
kʰ &
&
\\
b &
d &
ɖ &
&
ɡ &
&
\\
&
ʦ &
ʈʂ &
ʨ &
&
&
\\
&
(ʦʰ) &
(ʈʂʰ) &
ʨʰ &
&
&
\\
(f) &
s &
ʂ &
ɕ &
x &
&
h\\
&
z &
(ʐ) &
ʑ &
ɣ &
&
\\
m &
n &
ɳ &
&
&
&
\\
&
r &
ɽ &
&
&
&
\\
&
l &
&
&
&
&
\\
w &
&
&
j/ &
&
&
\\
\end{tabular}
\end{table}


Palula has ten phonemic vowels, comprising five basic qualities, each having a~long and a~short counterpart. This inventory forms a~symmetrical and typologically common system: 

\begin{center}
\begin{tabular}{ll@{\hspace{20pt}}ll@{\hspace{20pt}}ll@{\hspace{20pt}}ll@{\hspace{20pt}}ll}
/iː &
i &
&
&
&
&
&
&
uː &
u\\
&
&
eː &
e &
&
&
oː &
o &
&
\\
&
&
&
&
aː &
a/ &
&
&
&
\\
\end{tabular}
\end{center}



Vowel nasalisation is a~marginal, possibly emerging, feature in the language, but not so far fully contrastive.


The language has a~complex syllable structure \citep[54--57]{maddieson2005b}, permitting three consonants in the onset position and two in the coda position (although a limited number of consonant combinations are permitted before or after the vowel nucleus). There is a~tendency to drop the final consonant in word final clusters:


\begin{table}[H]
\begin{tabularx}{\textwidth}{ l l l l}
CCC before V: &
vd plosive + /r/ + /h/&
/grheːɳɖ/ &
`knot'\\
&
vd plosive/nasal + approximant + /h/&
/ˈdjhuːɽi/ &
`granddaughter'\\
CC before V: &
plosive/nasal + /r/ &
/kraːm/ &
`work'\\
&
C + approximant &
/swaːnu/ &
`is sleeping'\\
CCC after V: &
nasal + /d/ + /r/ &
/jaːndr/ &
`mill'\\
CC after V: &
nasal + C &
/ɡʰreːɳɖ/ &
`knot'\\
&
fricative + plosive (/s + t/, /ʂ + ʈ/) &
/ɡʰoːʂʈ/ &
`house'\\
&
C+ /r/ (/t + r/) &
/putr/ &
`son'\\
\end{tabularx}
\end{table}


Main stress falls on the final or the penultimate syllable of the lexical root. The stressed
syllable receives pitch accent, phonetically realised as: a) high level or high falling on a~short vowel
[\'{}], represented in this work as \textit{á} (in polysyllabic words, elsewhere no marking); b)
low rising on a~long vowel [\v{}], represented \textit{aá}; or c) high falling on a~long vowel [\^{}],
represented as \textit{áa}. Pitch accent is contrastive:


\begin{table}[H]
\begin{tabularx}{\textwidth}{ l l Q l l Q l l Q }
/seːtí/ &
\textit{seetí} &
`looked after' &
/sêːti/ &
\textit{séeti} &
`thigh' &
&
&
\\
/děːdi/ &
\textit{deédi} &
`burnt \textsc{f}' &
/dêːdi/ &
\textit{déedi} &
`grand\-mother' & & &
\\
/hár/ &
\textit{har} &
`every' &
/hǎːr/ &
\textit{haár} &
`defeat' &
/hâːr/ &
\textit{háar} &
`take away!' \\
\end{tabularx}
\end{table}


\section{Morphology}
\label{sec:2-2}

Palula morphology is suffixing, and formatives are almost exclusively concatenative \citep[86--9]{bickelnichols2005a}, with a~moderately high degreee of synthesis \citep[94--97]{bickelnichols2005c}. 


Nouns are inflected for number (singular, plural) and case (nominative, oblique, genitive). In most of the declensional classes nominative plural and oblique singular are cumulated into a~single formative \citep[90--93]{bickelnichols2005b}. The genitive (at least in the plural) can be analysed as suffixed to the oblique rather than to the nominative stem. The noun in the example is \textit{ṣinɡ} `horn':


\begin{table}[H]
\begin{tabular}{ l l l }
&
\textbf{Singular} &
\textbf{Plural} \\
Nominative &
\textit{ṣinɡ} &
\textit{ṣínɡ-a} \\
Oblique &
\textit{ṣínɡ-a} &
\textit{ṣínɡ-am} \\
Genitive &
\textit{ṣínɡ-ii} &
\textit{ṣínɡ-am"=ii} \\
\end{tabular}
\end{table}



There are three main functions of the oblique case of nouns: a) as the transitive subject in the Perfective; b) as the form to which postpostions are added; and c) as a~locative. A number of other case"=like functions (such as recipients) and more peripheral arguments appear as postpositional phrases.


Palula displays core"=case asymmetry \citep[206--209]{iggesen2005}, within the category of nouns as well as for NPs at large (more on pronouns below). While most nouns make a~nominative"=oblique distinction, one declensional class in particular does not make this distinction at all, whereas some of the pronouns make an~even more fine"=tuned nominative"=accusative"=ergative distinction:

\begin{table}[H]
\begin{tabular}{ l l | l | l }
&
\multicolumn{1}{l}{\textbf{`man'}} &
\multicolumn{1}{l}{\textbf{`woman'}} &
\textbf{\textbf{\textsc{3sg}}}\\
Nominative &
\textit{míiš} &
\textit{kúṛi} &
\textit{so} \\\cline{4-4}
Accusative &
\textit{míiš} &
\textit{kúṛi} &
\textit{tas} \\\cline{2-2}\cline{4-4}
Ergative &
\textit{míiša} &
\textit{kúṛi} &
\textit{tíi} \\
\end{tabular}
\end{table}


Palula has a~fairly typical Indo"=European two"=gender system, which is primarily sex"=based \citep[130--133]{corbett2005a}. A noun is either masculine or feminine, a~property established through morphological agreement. The basis for gender assignment is semantic as well as formal \citep[134--137]{corbett2005b}.


Adjectives are inflected for agreement in gender (masculine, feminine), number (singular, plural) and case (nominative, non"=nominative). The adjective in the example is \textit{paṇáaru} `white':


\begin{table}[H]
\begin{tabular}{ l l l l }
&
\textbf{Masculine} &
&
\textbf{Feminine} \\
&
Singular &
Plural &
\\
Nominative &
\multicolumn{1}{l|}{\textit{paṇáaru}} &
\multicolumn{1}{l|}{\textit{paṇáara}} &
\textit{paṇéeri} \\\cline{2-2}
Non"=nominative &
\textit{paṇáara} &
\multicolumn{1}{ l|}{\textit{paṇáara}} &
\textit{paṇéeri} \\
\end{tabular}
\end{table}


Finite verbs are inflected for tense"=aspect, mood (in a~limited sense) and agreement in a)
gender/number, \textit{or} b) person (the type of agreement expressed depending on tense, see \sectref{sec:2-3}
below). There are also some nonfinite forms. For the sake of a~more economical presentation that
takes verbs of different inflectional classes into account, all verbs are analysed as having
a~perfective and a~non"=perfective stem. The verb in the example is \textit{til-} `walk':


\begin{table}[H]
\begin{tabular}{ l l l l }
&
&
\textbf{Singular} &
\textbf{Plural} \\
\textbf{Non"=perfective stem} &
&
&
\\
Present &
\textsc{m} &
\textit{til-áan-u} &
\textit{til-áan-a} \\
&
\textsc{f} &
\textit{til-éen-i} &
\textit{til-éen"=im} \\
Future &
1 &
\textit{tíl-um} &
\textit{til-íia} \\
&
2 &
\textit{tíl-aṛ} &
\textit{tíl-at} \\
&
3 &
\textit{tíl-a} &
\textit{tíl-an} \\
Imperative &
&
\textit{tíl} &
\textit{tíl-ooi} \\
Infinitive &
&
\textit{til-áai} &
\\
Converb &
&
\textit{til-í} &
\\
Obligative &
&
\textit{til"=eeṇḍeéu} &
\\
Copredicative Participle &
&
\textit{til-íim} &
\\
Verbal Noun &
&
\textit{til"=ainií} &
\\
Agentive Verbal Noun &
\textsc{m} &
\textit{til-áaṭ-u} &
\textit{til-áaṭ-a} \\
&
\textsc{f} &
\textit{til-éeṭ-i} &
\textit{til-éeṭ-im} \\
\textbf{Perfective stem} &
&
&
\\
Perfective/Perfective Participle &
\textsc{m} &
\textit{tilíl-u} &
\textit{tilíl-a} \\
&
\textsc{f} &
\textit{tilíl-i} &
\textit{tilíl-im} \\
\end{tabular}
\end{table}

The pronoun system proper (i.e. 1st and 2nd person) is interesting in that it makes more distinctions in the plural than in the singular:

\begin{table}[H]
\begin{tabular}{ l l l | l l }
&
\textbf{Nominative} &
\multicolumn{1}{l}{\textbf{Accusative}} &
\textbf{Genitive} &
\textbf{Ergative} \\
\textsc{1sg} &
\textit{ma} &
\textit{ma} &
\textit{míi} &
\textit{míi} \\\cline{2-5}
\textsc{2sg} &
\textit{tu} &
\textit{tu} &
\textit{thíi} &
\textit{thíi} \\\cline{2-5}
\textsc{1pl} &
\multicolumn{1}{ l|}{\textit{be}} &
\textit{asaám} &
\multicolumn{1}{ l|}{\textit{asíi}} &
\textit{asím} \\\cline{2-5}
\textsc{2pl} &
\multicolumn{1}{ l|}{\textit{tus}} &
\textit{tusaám} &
\multicolumn{1}{ l|}{\textit{tusíi}} &
\textit{tusím} \\
\end{tabular}
\end{table}


The forms \textit{ma} and \textit{tu} are glossed throughout this work as nominative and \textit{míi} and \textit{thíi} as genitive, regardless of their functions in the clause.


The demonstratives, which are used as third"=person pronouns, essentially make the same case distinctions as the plural personal pronouns. Additionally they display gender distinctions (in the nominative singular) as well as a~three"=way deictic contrast:


\begin{table}[H]
\begin{tabular}{ l l l l | l | l | l }
&
&
&
\textbf{Nominative} &
\textbf{Accusative} &
\textbf{Genitive} &
\textbf{Oblique} \\
Proximate &
\textsc{sg} &
\textsc{m} &
\textit{nu} &
\textit{nis} &
\textit{nisíi} &
\textit{níi} \\\cline{4-4}
&
&
\textsc{f} &
\textit{ni} &
\textit{nis} &
\textit{nisíi} &
\textit{níi} \\\cline{5-7}
&
\textsc{pl} &
&
\textit{ni} &
\textit{ninaám} &
\textit{niníi} &
\textit{niním} \\\cline{4-7}
Distal &
\textsc{sg} &
\textsc{m} &
\textit{lo} &
\textit{las} &
\textit{lasíi} &
\textit{líi} \\\cline{4-4}
&
&
\textsc{f} &
\textit{le} &
\textit{las} &
\textit{lasíi} &
\textit{líi} \\\cline{5-7}
&
\textsc{pl} &
&
\textit{le} &
\textit{lanaám} &
\textit{laníi} &
\textit{laním} \\\cline{4-7}
Remote &
\textsc{sg} &
\textsc{m} &
\textit{so} &
\textit{tas} &
\textit{tasíi} &
\textit{tíi} \\\cline{4-4}
&
&
\textsc{f} &
\textit{se} &
\textit{tas} &
\textit{tasíi} &
\textit{tíi} \\\cline{5-7}
&
\textsc{pl} &
&
\textit{se} &
\textit{tanaám} &
\textit{taníi} &
\textit{taním} \\
\end{tabular}
\end{table}


There is relatively little synchronically productive derivational morphology in the language, but a~productive process for deriving verbs from other categories from within the language as well as entirely from novel or non"=native elements is the use of verbalisers such as \textit{the-} `do' and \textit{bhe-} `become':


\begin{table}[H]
\begin{tabularx}{\textwidth}{ l l l l l l l}
\textit{madád} &
`help' &
+ &
\textit{the-} &
{\textgreater} &
\textit{madád thíili} &
`helped'\\
\textit{tanɡ} &
`narrow' &
+ &
\textit{the-} &
{\textgreater} &
\textit{tanɡ thíilu} &
`troubled'\\
\textit{milaáu} &
`joined' &
+ &
\textit{bhe-} &
{\textgreater} &
\textit{milaáu bhílu} &
`met'\\
\textit{ašáq} &
`love' &
+ &
\textit{bhe-} &
{\textgreater} &
\textit{ašáq bhílu} &
`fell in love with'\\
\end{tabularx}
\end{table}



\section{Syntax}
\label{sec:2-3}
Word order in Palula is almost exclusively head"=final. This is seen in the word order in noun phrases (determiner"=noun, adjective"=noun, numeral"=noun, genitive"=noun), adjective phrases (adjunct"=adjective), and postpositional phrases (NP"=postpostion), as well as in clauses (SV and AOV), the latter of which will be obvious from most of the examples given below.


As far as alignment is concerned, Palula displays an~intricate split system. In the Perfective, the pattern is essentially ergative, seen in example (\ref{ex:2-1}) as a~non"=nominatively marked agent"=subject and verbal agreement with the feminine direct object, whereas in the non"=perfective categories, it is essentially accusative, which can be observed from the nominatively marked agent"=subject in (\ref{ex:2-2}), which is also the NP that the transitive verb agrees with in gender and number.

\begin{exe}
\ex
\label{ex:2-1}
\glll [íṇc̣-a] [čhéeli] khéel-i \\
bear[\textsc{msg}]-\textsc{ob} she.goat[\textsc{fsg}] eat.\textsc{pfv-}\textsc{f}\\
\textbf{A} \textbf{O} \\
\glt `The bear ate the goat.' (A:PAS056)
\end{exe}

\begin{exe}
\ex
\label{ex:2-2}
\glll [iṇc̣] áaṇc̣-a khiyáan-u\\
bear[\textsc{msg}] raspberry[\textsc{m}]-\textsc{pl} eat.\textsc{pres-m.sg} \\
\textbf{A} \textbf{O} \\
\glt `The bear is eating raspberries.' (A:KAT145)
\end{exe} 
Several NP splits further complicate the picture.


Agreement is part of all finite verb forms, but the particular agreement features realised are related to tense/aspect. In Future and Past Imperfective, the verb agrees with its target in person, as in (\ref{ex:2-3}), whereas in Present and the categories based on the Perfective, the verb agrees in gender and number, as can be seen in (\ref{ex:2-4}).

\begin{exe}
\ex
\label{ex:2-3}
\gll míi se preṣ-íi se bhraawú ma [ɡhaš-íin] de \\
\textsc{1sg.gn} \textsc{def} mother.in.law-\textsc{gn} \textsc{def} brother.\textsc{pl} \textsc{1sg.nom} catch-\textsc{3pl} \textsc{pst} \\
\glt `The brothers of that mother"=in"=law of mine captured me.' (A:HUA122)
\end{exe}

\begin{exe}
\ex
\label{ex:2-4}
\gll táapaṛ-a túuri íṇc̣-a čhéeli [ɡhašíl-i] hín-i \\ 
hill below bear-\textsc{ob} she.goat[\textsc{f}] catch.\textsc{pfv.fsg} be.\textsc{prs-f} \\
\glt `Below the hill, the bear had captured the goat.' (PAS054)
\end{exe}

Sentences lacking an~overt copula are allowed, and for predicate nominals in the Present tense, as the one shown in (\ref{ex:2-5}), they are normative.

\begin{exe}
\ex
\label{ex:2-5}
\gll míi baábu áak zamindaár míiš \\ 
\textsc{1sg.gn} father \textsc{idef} farmer man \\
\glt `My father is a~farmer.' (OUR002)
\end{exe}
Although it is possible to conjoin clauses with a~conjunctive suffix (also used for conjoining noun phrases), other strategies are preferred, such as juxtaposition for symmetrical clauses or the overwhelmingly favoured Converb construction, exemplified in (\ref{ex:2-6}), which is used for a~great variety of same"=subject clause combinations.

\begin{exe}
\ex
\label{ex:2-6}
\gll ti ba [bhun whay-í ba] [so mhaás muṭ-í bhun wheel-í ba] [teeṇíi ɡhooṣṭ-á the ɡhin-í] ɡáu \\
\textsc{3sg.ob} \textsc{prt} down come.down-\textsc{cv} \textsc{prt} \textsc{def.msg.nom} meat tree-\textsc{gn} down take.down-\textsc{cv} \textsc{prt} \textsc{refl} house-\textsc{ob} to take-\textsc{cv} go.\textsc{pfv.msg} \\
\glt `He came down [having come down], took down the meat from the tree [having taken down the meat from the tree], and brought it to his house.' (B:SHB762)
\end{exe}

In complex constructions, the unmarked order is a~complement clause followed by (or embedded in) the main clause (\ref{ex:2-7}), and similarly an~adverbial clause followed by (or, again, embedded in) the main clause (\ref{ex:2-8}). 

\begin{exe}
\ex
\label{ex:2-7}
\gll [neečíir theníi-a] díiš-a xalk-íim xwaaíš thíil-i \\
	hunt do.\textsc{vn"=pl} village\textsc{-ob} people\textsc{-pl.ob} desire do.\textsc{pfv-f} \\
\glt `People in the village wanted to go hunting.' (B:AVA200)
\end{exe}

\begin{exe}
\ex
\label{ex:2-8}
\gll[raaǰaá múṛ-u ta] putr-óom tasíi hukumát bulooṣṭéel-i \\
	king die.\textsc{pfv"=msg} \textsc{prt} son\textsc{-pl.ob} \textsc{3sg.gn} government snatch.\textsc{pfv-f} \\
\glt `When the king died, the sons seized the power.' (A:MAB003)
\end{exe}

However, a~post"=posed construction with the complementiser \textit{ki} is also commonly used (\ref{ex:2-9}), especially for utterance complements.

\begin{exe}
\ex
\label{ex:2-9}
\gll ɡhueeṇíi-am maníit-u ki [ni bíiḍ-a zinaawúr xálaka hín-a] \\
	Pashtun-\textsc{pl.ob} say.\textsc{pfv-m.sg} \textsc{Comp} \textsc{3pl.prox.nom} much-\textsc{mpl} wild people be.pres-\textsc{mpl} \\
\glt `The Pashtuns said: ``These are very wild people.''' (CHA008)
\end{exe}

Polar interrogatives are formed with a~clitical question particle \textit{-ee} (B. \textit{-aa}), as in (\ref{ex:2-10}), whereas an~indefinite"=interrogative pronoun (or other proform), such as \textit{kasée} (B.) `whose' in (\ref{ex:2-11}), is used in content interrogatives. 

\begin{exe}
\ex
\label{ex:2-10}
\gll ux-á díi hooǰóol-u ki tu insaán-ee \\
	camel-\textsc{ob} from ask.\textsc{pfv"=msg} \textsc{prt} \textsc{2sg.nom} human.being-\textsc{q} \\
\glt `He asked the camel: ``Are you a~man?''' (A:KIN007)
\end{exe}


\begin{exe}
\ex
\label{ex:2-11}
\gll aní [kasée] ziaarat-í thaní \\
	\textsc{prox.nom.pl} whose shrine-\textsc{pl} \textsc{qt} \\
\glt `Whose shrines are these?' (B:FOR026)
\end{exe}

Negation is formed with a~separate and invarable negative particle \textit{na}, preceding the predicate (\ref{ex:2-12}).

\begin{exe}
\ex
\label{ex:2-12}
\gll muṣṭúk-a xálak-a dhii-á díi [na] khooǰ-óon de \\
	of.past-\textsc{mpl} people-\textsc{pl} daughter-\textsc{ob} from \textsc{neg} ask-\textsc{3pl} \textsc{pst} \\
\glt `People in the old days were not asking their daughter [who she wanted to marry].' (A:MAR018)
\end{exe}