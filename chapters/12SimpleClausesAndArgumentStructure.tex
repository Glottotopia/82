\chapter{Simple clauses and argument structure}
\label{chap:12}

In this chapter, I present the different types of simple clauses found in Palula. In the first section (\sectref{sec:12-1}), nonverbal or nominal clauses are introduced, and the presence vs. absence of an~overt copula is discussed. In the second section (\sectref{sec:12-2}), verbal clauses are introduced, with a~focus on subclassification of verbs based on argument structure and transitivity. At the end of that section (in \sectref{subsec:12-2-8}), a~tentative analysis of conjunct verbs and the syntactic role of the host element in these constructions is offered. 


\section{Nonverbal predicates}
\label{sec:12-1}

The type of construction and the particular copular form a~nominal predication typically occurs with is primarily a~question of phrasal identity (whether it is another NP, an~adjective phrase, or a~locative expression), but it is also related to the semantic distinction between equation/identification, property and location, as well as to tense.


Although \tabref{tab:12-1} is a~generalisation that does not take every single instance into account, it shows a~very clear tendency for the different phrasal categories~-- and the semantic relationships coded by them~-- to be mapped systematically to certain copular forms and expressions. 


\begin{table}[ht]
\caption{Distribution of standard copular forms and expressions}
\begin{tabularx}{\textwidth}{ l Q Q Q }
\lsptoprule
&
NP~-- NP
(Equation/{\allowbreak}Identification) &
NP~-- AP
(Property) &
NP~-- Loc
(Location)\\\hline
Present &
ZERO &
\textit{hin-} &
\textit{hin-}\\
Past &
\textit{de} &
\textit{de} &
\textit{heensíl- de}\\\lspbottomrule
\end{tabularx}
\label{tab:12-1}
\end{table}

% \begin{table}[ht]
% \caption{Distribution of standard copular forms and expressions}
% \begin{tabularx}{\textwidth}{ l Q Q Q }
% \lsptoprule
% &
% NP~-- NP
% (Equation/{\allowbreak}Identification) &
% NP~-- AP
% (Property) &
% NP~-- Loc
% (Location)\\\hline
% Present &
% \multicolumn{1}{l:}{ ZERO} &
% \multicolumn{2}{l}{ \textit{hin-}}\\\cdashline{2-4}
% Past &
% \multicolumn{2}{l:}{ \textit{de}} &
%  \textit{heensíl- de} \\\lspbottomrule
% \end{tabularx}
% \label{tab:12-1}
% \end{table}


There is also a~special affinity between the copular clauses with a~locational expression and two other constructions: existentials and possessives. Besides the standard copular forms, some alternative copulas that occur are presented under each section.


\subsection{Copular clauses with nominal predicates}
\label{subsec:12-1-1}


As seen in examples (\ref{ex:12-1})--(\ref{ex:12-3}), predicate noun phrases typically occur without an~overt copula in the present tense. \citet[337]{masica1991} mentions this feature in Eastern NIA languages and in Sinhalese, where NP + NP as well as NP + AP clauses without a~copula are normative, but he also remarks that it is not found in the more central Urdu-Hindi. \citet[118--122]{baart1999a}, likewise, describes left"=out copulas as possible in some Gawri clauses expressing identity, which also seems to be permitted, or even typical, in Khowar (own observations). The absence of a~copula form is in Palula regardless of the phrase being used for classification (ascriptively) or identification. 

\begin{exe}
\ex
\label{ex:12-1}
\gll [míi nóo]\textsubscript{\textsc{sbj}} [laalzamaán]\textsubscript{\textsc{prd}} \\
\textsc{1sg.gen} name Lal.Zaman  \\
\glt `My name is Lal Zaman.' (A:KEE001)
\end{exe}
\begin{exe}
\ex
\label{ex:12-2}
\gll [ma]\textsubscript{\textsc{sbj}} [tarkaáṇ]\textsubscript{\textsc{prd}} bi
     [misrí]\textsubscript{\textsc{prd}} bi \\
\textsc{1sg.nom} carpenter also mason also  \\
\glt `I'm a~carpenter as well as a~mason.' (A:HOW009)

\ex
\label{ex:12-3}
\gll [aní]\textsubscript{\textsc{sbj}} [mheerabaán thaní ak kúṛi-e ziaarát]\textsubscript{\textsc{prd}}, [aṛé]\textsubscript{\textsc{sbj}} ba [tesée dhii-yí ziaarát]\textsubscript{\textsc{prd}} \\
\textsc{3fsg.prox.nom} Meherban \textsc{quot} \textsc{idef} woman-\textsc{gen}  shrine \textsc{3fsg.dist.nom} \textsc{prt} \textsc{3sg.gen} daughter-\textsc{gen}  shrine \\
\glt `This (one) is the shrine of a~woman called Meherban, and that (one) is her daughter's shrine.' (B:FOR034-5)
\end{exe}

The unmarked word order in such clauses is \textsc{sbj - prd}, as in the examples, but a~reverse word order, for focus and various other discourse purposes is also possible with this type of clause ((\ref{ex:12-4}) and (\ref{ex:12-5})), as well as clauses with a~discontinuous predicate (\ref{ex:12-6}).

\begin{exe}
\ex
\label{ex:12-4}
\gll muṣṭuí niiɡiraá [mal-íim"=ii kasubɡár]\textsubscript{\textsc{prd}} [ma]\textsubscript{\textsc{sbj}} \\
of.past since property-\textsc{pl.obl"=gen} professional \textsc{1sg.nom}  \\
\glt `I have been a~shepherd for a~long time.' (A:KEE003)
\end{exe}
\begin{exe}
\ex
\label{ex:12-5}
\gll se nóo-wa maǰí [aakatí nóo-wa]\textsubscript{\textsc{prd}} [aní]\textsubscript{\textsc{sbj}} \\
\textsc{def} name-\textsc{pl} among some name-\textsc{pl} 3\textsc{mpl.prox.nom} \\
\glt `Some names (among these names) are these{\ldots}' (A:SEA002)
\end{exe}
\begin{exe}
\ex
\label{ex:12-6}
\gll [beezaadxaan-íi]\textsubscript{\textsc{prd{\ldots}}} [ma]\textsubscript{\textsc{sbj}} [putr]\textsubscript{\textsc{{\ldots}prd}} \\
Bezad.Khan-\textsc{gen} \textsc{1sg.nom} son  \\
\glt `I am the son of Bezad Khan' (A:GHA001)
\end{exe}

The absence of an~overt copula is, as noted above, normally confined to the present tense, whereas clauses of the above type with past tense reference, as in (\ref{ex:12-7}) and (\ref{ex:12-8}), appear with a~Past tense copula \textit{de}, reflecting a~widespread pattern for languages that allow so"=called copula dropping (\citealt[34]{pustet2003}; \citealt[120]{givon2001a}) as well as other South Asian or neighbouring languages where copula"=less sentences are allowed or normative (\citealt[339]{masica1991}; \citealt[121]{baart1999a}). However, it should be pointed out that an equally plausible analysis (although not adopted in this work) is to regard the form \textit{de}, which bears no formal resemblance to present tense \textit{hin-}, as a past tense marker without any actual copular meaning attached to it, regardless of its occurrence with nonverbal predicates (as described in this section) or its auxiliar use in periphrastic tense-aspect formations (see \sectref{subsec:9-1-5}). 

\begin{exe}
\ex
\label{ex:12-7}
\gll [miirǰamadaár]\textsubscript{\textsc{prd}} ba [tasíi nóo]\textsubscript{\textsc{sbj}} de \\
Mir.Jamadar \textsc{prt} \textsc{3sg.gen} name be.\textsc{pst} \\
\glt `And Mir Jamadar was his name.' (A:GHA051)
\end{exe}
\begin{exe}
\ex
\label{ex:12-8}
\gll [áak]\textsubscript{\textsc{sbj}} ba [habibulaaxaán thaní míiš]\textsubscript{\textsc{prd}} de \\
one \textsc{prt} Habibullah.Khan \textsc{quot} man be.\textsc{pst} \\
\glt `And one of them was a~man named Habibullah Khan.' (A:ACR023)
\end{exe}

However, the presence of a~copula in the past tense is not an~absolute, as past"=tense copula"=less sentences do occur, especially in ``list"=like'' discourses such as in (\ref{ex:12-9}).

\begin{exe}
\ex
\label{ex:12-9}
\gll [mhamadíin-e putr]\textsubscript{\textsc{sbj}} ba [xaeerudíin]\textsubscript{\textsc{prd}}, \textsc{[}xaeerudíin-e putr]\textsubscript{\textsc{sbj}} ba [ɣeyratxaán]\textsubscript{\textsc{prd}} \\
Mahmuddin-\textsc{gen} son \textsc{prt} Khairuddin Khairuddin-\textsc{gen} son \textsc{prt} Ghairat.Khan \\
\glt `And Mahmuddin's son was Khairuddin, and Khariuddin's son Ghairat Khan.' (B:ATI016-7)
\end{exe}

Copular clauses involving a~change of state can be formed with \textit{bhe-} `become' (not to be confused with the superficially similar \textit{bhe}-conjuncts, see \sectref{subsec:12-2-8}), as in (\ref{ex:12-10}).

\begin{exe}
\ex
\label{ex:12-10}
\gll [ma]\textsubscript{\textsc{sbj}} [míiš]\textsubscript{\textsc{prd1}} na de, bálki maǰburí ki [míiš]\textsubscript{\textsc{prd2}} bhíl-i de \\
\textsc{1sg.nom} man \textsc{neg} be.\textsc{pst}  however necessity by man become.\textsc{pfv-f} \textsc{pst} \\
\glt `I was not a~man, but out of necessity I had become a~man.' [Uttered by a~woman in a~story who had on a~previous occasion dressed up like a~man.] (A:UXW061)
\end{exe}

Another (normally intransitive) verb \textit{ɡir-} `turn' may rarely be used in a~similar way, carrying the approximate meaning `X turns into Y'.



The above mentioned \textit{bhe-} `become' and \textit{ɡir-} `turn', along with a~few other verbs, such as \textit{dhar-} `remain' and \textit{yhe-} `come', as they are used in some clauses, seem to occupy an~intermediate position between copular and full verbs, and can thus be described as semi"=copulas \citep[5--6]{pustet2003}. 



A special type of subjectless copular clause is illustrated in (\ref{ex:12-11})--(\ref{ex:12-12}), thus referring to a~temporal setting introduced earlier. Often such expressions occur at the beginning of a~story. The standard Past copula \textit{de} is used.

\begin{exe}
\ex
\label{ex:12-11}
\gll [lhoók-u díiš]\textsubscript{\textsc{prd}} de \\
small-\textsc{msg} village be.\textsc{pst} \\
\glt `It was a~small village.' (A:JAN003)
\end{exe}
\begin{exe}
\ex
\label{ex:12-12}
\gll [c̣hiṇ zamaan-á]\textsubscript{\textsc{prd}} de \\
dark time-\textsc{pl} be.\textsc{pst} \\
\glt `These were dark times.' (A:JAN010)
\end{exe}

\subsection{Copular clauses with adjectival predicates}
\label{subsec:12-1-2}


While clauses with a~predicate adjective phrase do occur without an~overt copula, they seem to do so with less regularity than those with a~predicate noun phrase. While it is left out in some fixed expressions, such as in greetings, a~present"=tense form \textit{hin-} (agreeing in gender and number with the NP) of the standard copula is normally present in running discourse, as is evident from examples (\ref{ex:12-13}) and (\ref{ex:12-14}). 

\begin{exe}
\ex
\label{ex:12-13}
\gll aṛó ǰinaazá asaám díi ɡhašá, [zhaáy]\textsubscript{\textsc{sbj}} [naawás]\textsubscript{\textsc{prd}} hín-i \\
\textsc{dist.msg.nom} corpse \textsc{1pl.acc} from take.out.\textsc{imp.sg}  place dangerous be.\textsc{prs-f} \\
\glt `Hold this corpse away from us, the place is dangerous.' (A:GHA039)
\end{exe}
\begin{exe}
\ex
\label{ex:12-14}
\gll [moosúm]\textsubscript{\textsc{sbj}} típa [šuy]\textsubscript{\textsc{prd}} hín-u \\
weather now good be.\textsc{prs"=msg} \\
\glt `The weather is good now.' (B:VIS245)
\end{exe}

The nearly obligatory copula dropping with nominal predicates compared to the optional or variable pattern with adjectival predicates may imply that the present"=tense copula \textit{hin-} is not entirely devoid of meaning \citep[8, 31, 66]{pustet2003}, which is also supported by the existential use of \textit{hin-} (\sectref{subsec:12-1-3}), but that is a~matter for further research.



Examples (\ref{ex:12-15}) and (\ref{ex:12-16}) show that with past"=tense reference, \textit{de} is used, just as with the nominal predicates.

\begin{exe}
\ex
\label{ex:12-15}
\gll [so báaṭ]\textsubscript{\textsc{sbj}} [ɣoṛ]\textsubscript{\textsc{prd}} de \\
\textsc{def.msg.nom} stone greasy be.\textsc{pst} \\
\glt `The stone was greasy.' (A:BRE012)
\end{exe}
\begin{exe}
\ex
\label{ex:12-16}
\gll tasíi nóo kaṭamúš de, [so]\textsubscript{\textsc{sbj}} [bíiḍ-u trók-u]\textsubscript{\textsc{prd}} de \\
\textsc{3sg.gen} name Katamosh be.\textsc{pst} \textsc{3sg.nom} very-\textsc{msg}  thin-\textsc{msg} be.\textsc{pst} \\
\glt `His name was Katamosh and he was very thin.' (A:KAT002-3)
\end{exe}

When simple"=present (\textit{hin-}) or past (\textit{de}) time reference is not sufficient, the relevant forms of the existential (see \sectref{subsec:12-1-4}) verb \textit{háans-} `stay, remain, find oneself, be present' is used in place of the standard copula. In (\ref{ex:12-17}), the longer duration of the happiness needs to be expressed with the Past Imperfective of \textit{háans-}.

\begin{exe}
\ex
\label{ex:12-17}
\gll [se] heewand-á [bíiḍ-a xušaán] hóons"=an de \\
\textsc{3pl.nom} winter-\textsc{obl} very-\textsc{pl} happy stay-\textsc{3pl} \textsc{pst} \\
\glt `They were/remained very happy during the winter.' (A:SHY008)
\end{exe}

As with noun phrase predicates, adjective phrase predicates, as in example (\ref{ex:12-18}), also occur with \textit{bhe-} `become' as a~copula.

\begin{exe}
\ex
\label{ex:12-18}
\gll aní čhoot-á kha ta, [tu]\textsubscript{\textsc{sbj}} [čaáx]\textsubscript{\textsc{prd}} bh-íiṛ \\
\textsc{prox} cheese-\textsc{pl} eat.\textsc{imp.sg} \textsc{prt} \textsc{2sg.nom} fat become-\textsc{2sg} \\
\glt `Eat this cheese, and you will become fat.' (A:KAT075)
\end{exe}

Some predicative adverbial phrases can function just like predicative adjective phrases. However, in such clauses the adverbial phrase, `alone' in (\ref{ex:12-19}), in actual fact denotes a~property of the subject.

\begin{exe}
\ex
\label{ex:12-19}
\gll [be]\textsubscript{\textsc{sbj}} [khilaí]\textsubscript{\textsc{prd}} ba na de \\
\textsc{1pl.nom} alone \textsc{prt} \textsc{neg} be.\textsc{pst} \\
\glt `We were not alone.' (A:ACR017)
\end{exe}


\subsection{Copular clauses with locative expressions}
\label{subsec:12-1-3}

In clauses with a~predicate locative phrase, the copula is always overt. With present"=time reference the standard copula is used in its present"=tense form \textit{hín-} (agreeing in gender and number with the NP). The semantics of such clauses, examples (\ref{ex:12-20})--(\ref{ex:12-21}), is to declare something or someone referred to by a~definite NP as present (or absent) in a~certain location.

\begin{exe}
\ex
\label{ex:12-20}
\gll [míi ɡhoóṣṭ]\textsubscript{\textsc{sbj}} [lookúṛi]\textsubscript{\textsc{prd}} hín-u \\
\textsc{1sg.gen} house Lokuri be.\textsc{prs"=msg} \\
\glt `My house is in Lokuri.' (A:OUR001)
\end{exe}
\begin{exe}
\ex
\label{ex:12-21}
\gll [so iškaarí méeš]\textsubscript{\textsc{sbj}} [muṭ-á wée]\textsubscript{\textsc{prd}} hín-u \\
\textsc{def.msg.nom} hunting man tree-\textsc{obl} in be.\textsc{prs"=msg} \\
\glt `The hunter is in the tree.' (B:CLE368)
\end{exe}

With past tense reference the Pluperfect of the verb \textit{háans-} `stay, remain, find oneself, be present', as in (\ref{ex:12-22}), is used.

\begin{exe}
\ex
\label{ex:12-22}
\gll dac̣h-íi ta [kúṛi]\textsubscript{\textsc{sbj}} ba [ɡhooṣṭ-á šíiṭi]\textsubscript{\textsc{prd}} na heensíl-i de \\
look-3\textsc{sg} \textsc{prt} woman \textsc{prt} house-\textsc{obl} inside \textsc{neg} stay.\textsc{pfv-f} \textsc{pst} \\
\glt `Looking (around), he saw that the woman was not in the house.' (A:WOM656)
\end{exe}

Normally the subject precedes the locative predicate phrase, but this can be reversed, as in (\ref{ex:12-23}), when the focus is shifted, although the order locative"=subject is more typical of the closely related existential construction (see \sectref{subsec:12-1-4}).

\begin{exe}
\ex
\label{ex:12-23}
\gll [se baṭ-á ǰhulí]\textsubscript{\textsc{prd}} [se kuṇaak-íi paaṇṭí]\textsubscript{\textsc{sbj}} bi heensíl-i de, [teewiz-í]\textsubscript{\textsc{sbj}} bi heensíl-i de \\
\textsc{def} stone-\textsc{obl} on \textsc{def} child-\textsc{gen} clothes  also stay.\textsc{pfv-f}
\textsc{pst} amulet-\textsc{pl} also stay.\textsc{pfv-f} \textsc{pst} \\
\glt `On the stone were the child's clothes and also his amulets.' (A:BRE012)
\end{exe}

\subsection{Other copular or copula"=like expressions}
\label{subsec:12-1-4}

\spitzmarke{Existentials.} Although similar to copular clauses with a~locative expression, the function of existentials is not to specify the location of a~known entity but to assert the existence of some previously unintroduced entity. 


Therefore, while the subjects of local copular clauses are definite, the subjects of existential clauses are indefinite. Normally the word order is also the reverse of most local copular clauses, i.e., the locative expression precedes the subject noun phrase, and can probably not be considered a~predicate in the same sense as in the copular locative expressions. The verb used is either \textit{háans-} `stay, remain, find oneself, be present', especially with past time reference (\ref{ex:12-25}), or the present form (\ref{ex:12-24})~-- and occasionally the Past form~-- of the standard copula.

\begin{exe}
\ex
\label{ex:12-24}
\gll aní dees-óom atshareet-á wée [qariibán čúur zára kušúni]\textsubscript{\textsc{sbj}} hín-a \\
\textsc{prox} day-\textsc{pl.obl} Ashret-\textsc{obl} in about  four thousand inhabitants be.\textsc{prs"=mpl}\\
\glt `In these days there are about 4,000 inhabitants in Ashret.' (A:PAS007)

\ex
\label{ex:12-25}
\gll bhuná [áak ɡíri]\textsubscript{\textsc{sbj}} heensíl-i \\
below \textsc{idef} rock stay.\textsc{pfv-f} \\
\glt `Down below there was a~big rock.' (A:GHA043)
\end{exe}

In many existential expressions, (\ref{ex:12-26})--(\ref{ex:12-28}), there is no overt location at all.

\begin{exe}
\ex
\label{ex:12-26}
\gll koó hín=ee, yh-óoi thaní \\
someone be.\textsc{prs.mpl}=\textsc{q} come-\textsc{imp.pl} \textsc{quot} \\
\glt `If there is anyone, come!' (A:JAN038)
\end{exe}
\begin{exe}
\ex
\label{ex:12-27}
\gll miír thaní áak míiš heensíl-u de \\
Mir \textsc{quot} \textsc{idef} man stay.\textsc{pfv"=msg} \textsc{pst} \\
\glt `There was a~man called Mir.' (A:GHA051)
\end{exe}
\begin{exe}
\ex
\label{ex:12-28}
\gll eesé dášum maǰí dúu bhraawú de \\
\textsc{rem} ten.\textsc{obl} among two brother.\textsc{pl} be.\textsc{pst} \\
\glt `Among them there were two brothers.' (A:PAS011)
\end{exe}

The (normally) intransitive verb \textit{yhe-} `come' could in some sentences be analysed as having a~function similar to \textit{háans-} with the approximate meaning `X comes into existence', such as in (\ref{ex:12-29}).

\begin{exe}
\ex
\label{ex:12-29}
\gll tasíi watan-í qahtí yhéel-i \\
\textsc{3sg.gen} country-\textsc{obl} famine come.\textsc{pfv-f} \\
\glt `There was a~famine in his country.' (A:ABO019)
\end{exe}

Another similarly functioning verb is \textit{lhaíǰ-} `to be found', the passive or valency"=reduced form of \textit{lhay-} `find'.


\spitzmarke{Possessives.} There are two types of possessive constructions in which both are similar in structure to the existential expressions, where the possessor is expressed either as a~genitive NP or as a~postpositional phrase with `from'. There is an~approximate but not absolute correspondence between the first construction, in (\ref{ex:12-30})--(\ref{ex:12-32}), and inalienable possession and the second construction, in (\ref{ex:12-33})--(\ref{ex:12-34}), and alienable possession.

\begin{exe}
\ex
\label{ex:12-30}
\gll [tasíi] ba ɡa wása na heensíl-u \\
\textsc{3sg.gen} \textsc{prt} any strength \textsc{neg} stay.\textsc{pfv"=msg} \\
\glt `And he had no strength at all [lit: And his strength was not present].' (A:GHA017)
\end{exe}
\begin{exe}
\ex
\label{ex:12-31}
\gll [har qóom"=ii] [har qabilá-ii] teṇteeṇíi ǰhaníi dasturá haans-áan-u \\
every tribe-\textsc{gen} every clan-\textsc{gen} \textsc{refl} marriage.\textsc{gen}  customs stay-\textsc{prs"=msg}\\
\glt `Each tribe and clan has its own marriage customs [lit: Every tribe's and every clan's custom of marriage is present].' (A:MAR001)

\ex
\label{ex:12-32}
\gll muṣṭóoi zamaná-ii [áak míiš-ii] áak lhéṇḍ-u putr de maní \\
of.past time-\textsc{gen} \textsc{idef} man-\textsc{gen} \textsc{idef} bald-\textsc{msg} son be.\textsc{pst} \textsc{hsay} \\
\glt `Once upon a~time a~man had a~bald son [lit: In the past of a~man a~bald son was].' (A:KAT001)
\end{exe}
\begin{exe}
\ex
\label{ex:12-33}
\gll [ma díi] paiseé náhin-a \\
\textsc{1sg.nom} from money.\textsc{pl} \textsc{neg}.be.\textsc{prs"=mpl} \\
\glt `I don't have any money [lit: From me money is not.].' (B:ANG008)
\end{exe}
\begin{exe}
\ex
\label{ex:12-34}
\gll misrí yhóol-u seentá [misrí díi] tsaṭák hóons-a \\
mason come.\textsc{pfv"=msg} \textsc{condh} mason from hammer stay-\textsc{3sg} \\
\glt `When the mason comes he will have a~hammer [lit: When the mason has come, from the mason a~hammer will be present].' (A:HOW010)
\end{exe}

\spitzmarke{Transitive copular clauses.} The verb \textit{the-} `do' can be used as the transitive equivalent of the adjectival copular \textit{bhe-} `become', as shown in example (\ref{ex:12-35}).

\begin{exe}
\ex
\label{ex:12-35}
\gll ɡhaḍeerá tas peerišaán thíil-u de \\
elder.\textsc{obl} \textsc{3sg.acc} worried do.\textsc{pfv"=msg} \textsc{pst} \\
\glt `The elder had made him worried.' (A:Q9.0088)
\end{exe}

\section{Verbal predicates}
\label{sec:12-2}

\subsection{Argument structure and transitivity}
\label{subsec:12-2-1}


Like most IA languages, there is a~strict distinction between intransitive and transitive verbs in Palula. Almost without exception, a~verb stem is either intransitive or transitive and cannot be ambivalent or polyvalent as far as transitivity is concerned. There is, on the other hand, fairly productive valency"=changing morphology (as described in \sectref{sec:8-5}) by which a~stem can increase or decrease its valency. A verb's transitivity is primarily diagnosed on basis of absence/presence of ergative morphology in the perfective and the aspectual shift between accusative alignment and ergative alignment as far as verb agreement is concerned (see \sectref{sec:11-1}). 


Within each of these two main categories, intransitive verbs and transitive verbs, there is another distinction made between simple intransitive/transitive verbs and intransitive/transitive verbs with an~indirect object. I am here using a~very broad definition of an~indirect object, as a~non"=nominative/non"=ergative argument, usually coded by a~postposition. 


The four resulting argument structures (in \tabref{tab:12-2}) cover a~large majority of all verbs. Also a~few verbs displaying a~non"=standard pattern are discussed, as well as complement"=taking verbs and the somewhat analytically challenging conjunct verbs.


\begin{table}[ht]
\caption{Valency patterns summarised}

\begin{tabularx}{\textwidth}{ Q Q Q }
\lsptoprule
Basic pattern &
Intransitive &
Transitive\\\hline
Simple &
NP\textsc{sbj} V &
NP\textsc{sbj} NP\textsc{do} V\\
With indirect object &
NP\textsc{sbj} PP/NP\textsc{io} V &
NP\textsc{sbj} NP\textsc{do} PP/NP\textsc{io} V\\\lspbottomrule
\end{tabularx}
\label{tab:12-2}
\end{table}


\subsection{Simple intransitive verbs}
\label{subsec:12-2-2}


The typical pattern for intransitive verbs is to take a~single argument in the form of a~subject noun phrase coded in the nominative: \textbf{NP\textsc{sbj}} \textbf{V}. This pattern is exemplified in (\ref{ex:12-36}) and (\ref{ex:12-37}) with the verbs `die' and `break', respectively.

\begin{exe}
\ex
\label{ex:12-36}
\gll aakatí reet-í baád [so]\textsubscript{\textsc{sbj}} múṛ-u \\
some night"=\textsc{pl} after \textsc{3msg.rem.nom} die.\textsc{pfv"=msg} \\
\glt `A few days later, he died.' (A:ABO024)
\end{exe}
\begin{exe}
\ex
\label{ex:12-37}
\gll andáa bhíl-u ta [šíin"=ii čoreé šeenbóo-a]\textsubscript{\textsc{ sbj}} phooṭíl-a \\
like.that become.\textsc{pfv"=msg} \textsc{prt} bed-\textsc{gen} all.four leg-\textsc{pl}  break.\textsc{pfv"=mpl}\\
\glt `When that happened, all four legs of the bed broke.' (A:GHU024)
\end{exe}

Many such verbs (\tabref{tab:12-poc}) are process verbs and the subject noun phrase has a~semantic role that (following \citealt[125]{givon2001a}) could be described as a~patient"=of"=change, whether human, animate or inanimate.


\begin{table}[H]
\caption{Examples of simple intransitive verbs taking a patient"=of"=change subject}
\begin{tabularx}{\textwidth}{ l@{\hspace{25pt}} l@{\hspace{25pt}} l@{\hspace{25pt}} l@{\hspace{25pt}} }
\lsptoprule
\textit{mar-} &
`die' &
\textit{bhakulé-} &
`fatten'\\
\textit{ǰandé-} &
`become alive, regain strength' &
\textit{čiiré-} &
`be delayed'\\
\textit{phooṭé-} &
`break' &
\textit{čhinǰ-} &
`fall'\\
\textit{buuḍé-} &
`grow old' &
\textit{dhraǰ-} &
`stretch'\\
\textit{buc̣haalé-} &
`become hungry' &
\textit{baḍ-} &
`grow'\\\lspbottomrule
\end{tabularx}
\label{tab:12-poc}
\end{table}


Other verbs (\tabref{tab:12-pos}) can take as a~single argument an~agent subject or a~patient"=of"=state subject.


\begin{table}[H]
\caption{Examples of simple intransitive verbs taking an agent"=subject or a patient"=of"=state subject}
\begin{tabularx}{\textwidth}{ l@{\hspace{25pt}} l@{\hspace{25pt}} l@{\hspace{25pt}}
    l@{\hspace{25pt}} }
\lsptoprule
\textit{uthí-} &
`stand up, get up' &
\textit{muutré-} &
`urinate'\\
\textit{bheš-} &
`sit down' &
\textit{su-} &
`sleep, fall asleep'\\\lspbottomrule
\end{tabularx}
\label{tab:12-pos}
\end{table}


Example (\ref{ex:12-38}) illustrates the use of \textit{uthí-} `stand up, get up'.

\begin{exe}
\ex
\label{ex:12-38}
\gll [ma]\textsubscript{\textsc{sbj}} roošnaám raɣáṣṭi uth-áan-u  \\
1\textsc{sg.nom} morning early get.up-\textsc{prs"=msg} \\
\glt `I get up early in the morning.' (B:MOR001)
\end{exe}

It should be noted that these verbs primarily are process verbs with a~punctual interpretation, closely corresponding to English `stand up, sit down' etc. The corresponding stative meaning `stand, sit' is derived through a~resultative construction (see \sectref{subsec:9-1-7}). Compare examples (\ref{ex:12-39}) and (\ref{ex:12-40}), where the verb \textit{bheš-} `sit (down)' in the latter is expressed resultatively, and hence receives a~stative (durative) interpretation.

\begin{exe}
\ex
\label{ex:12-39}
\gll so [bhéṭ-u seentá] so bi [bhéš-a de] maní \\
\textsc{3msg.nom} sit.down.\textsc{pfv"=msg} \textsc{condh} \textsc{3msg.rem.nom} also  sit.down-\textsc{3sg} \textsc{pst } \textsc{hsay} \\
\glt `When he had sat down, the other one obviously also sat down.' (A:UXB017)

\ex
\label{ex:12-40}
\gll aḍaphaár whayí dac̣h-íi ta amzarái [bheš-í hín-u] \\
halfways come.down.\textsc{cv} look-\textsc{3sg } \textsc{prt} lion sit.down-\textsc{cv} be.\textsc{prs"=msg} \\
\glt `He comes down halfway and sees the lion sitting there.' (A:KAT086)
\end{exe}

\subsection{Simple transitive verbs}
\label{subsec:12-2-3}


The typical pattern for transitive verbs is to take as arguments: a) one subject noun phrase, always coded in the nominative in the imperfective, while in the perfective some NPs are non"=optionally coded in a~non"=nominative case (oblique or ergative), and b) one direct object noun phrase coded in the nominative or the accusative (again depending on the nature of the NP, but regardless of aspect, see \sectref{sec:11-2}). The unmarked word order is subject preceding direct object: \textbf{NP\textsc{sbj}} \textbf{NP\textsc{do}} \textbf{V}.


This pattern is exemplified in (\ref{ex:12-41}) and (\ref{ex:12-42}) with the verbs `eat' and `kill', respectively.

\begin{exe}
\ex
\label{ex:12-41}
\gll [karáaṛu]\textsubscript{\textsc{sbj}} [asaám]\textsubscript{\textsc{do}} khúu \\
leopard \textsc{1pl.acc} eat.3\textsc{sg} \\
\glt `The leopard will eat us.' (B:FOY)
\end{exe}
\begin{exe}
\ex
\label{ex:12-42}
\gll [tíi]\textsubscript{\textsc{sbj}} [áa ḍáaɡ]\textsubscript{\textsc{do}} mheeríl-u \\
\textsc{3sg.obl} \textsc{idef} deer kill.\textsc{pfv"=msg} \\
\glt `He killed a~deer.' (A:THA002)
\end{exe}

The direct object in many transitive verbs of this type (\tabref{tab:12-poco}) has a~patient"=of"=change role corresponding rather closely with the role of the subject of the corresponding intransitive verb (for example \textit{mar-/mhaaré-} `die/kill', \textit{phooṭé-/phooṭá-} `break\textit{(itr)}/break\textit{(tr)}'}).


\begin{table}[H]
\caption{Examples of simple transtive verbs taking a patient"=of"=change object}
\begin{tabularx}{\textwidth}{ l@{\hspace{25pt}} l@{\hspace{25pt}} l@{\hspace{25pt}}
    l@{\hspace{25pt}} }
\lsptoprule        
\textit{mhaaré} &
`kill' &
\textit{pil} &
`drink'\\
\textit{ǰandá} &
`make alive' &
\textit{kha} &
`eat'\\
\textit{phooṭá} &
`break' &
\textit{pičhá} &
`sweep, wipe'\\
\textit{samá} &
`build, put together' &
\textit{taapé} &
`heat up'\\
\textit{čooṇṭá} &
`write, embroider' &
\textit{ɡhuaṛá} &
`boil'\\\lspbottomrule
\end{tabularx}
\label{tab:12-poco}
\end{table}


Following \citet[127]{givon2001a}, such verbs could be further classified according to several types of change: a) creation (e.g., \textit{samá-}), b) destruction (e.g., \textit{kha-}), c) change in physical condition (e.g., \textit{phooṭá-}), d) change in surface conditions (e.g., \textit{pičhá-}), e) change in internal qualities (e.g., \textit{taapé-}). It is especially in the categories c. and e. that we find most parallels between the direct object and the subject in typical intransitive verbs.


Some less prototypically transitive verbs (\tabref{tab:12-exps}) also conform to this pattern, e.g., those with experiencer"=subjects rather than agent"=subjects as in the verbs exemplified above.


\begin{table}[H]
\caption{Examples of simple transitive verbs taking an experiencer"=subject}
\begin{tabularx}{\textwidth}{ l@{\hspace{25pt}} l@{\hspace{25pt}} l@{\hspace{25pt}}
    l@{\hspace{25pt}} }
\lsptoprule
\textit{paš-} &
`see' &
\textit{ṣuṇ-} &
`hear'\\\lspbottomrule
\end{tabularx}
\label{tab:12-exps}
\end{table}


Example (\ref{ex:12-43}) illustrates the use of \textit{paš-} `see'.

\begin{exe}
\ex
\label{ex:12-43}
\gll eetíi maǰí [luumée]\textsubscript{\textsc{sbj}} [kaṭamuš-íi lhéṇḍ-i kakaríi]\textsubscript{\textsc{do}} dhríṣṭ-i hín-i \\
\textsc{3sg.rem.obl} in fox.\textsc{obl} Katamosh-\textsc{gen} bald-\textsc{f} scalp see.\textsc{pfv-f} be.\textsc{prs-f} \\
\glt `There the fox saw Katamosh's bald scalp.' (A:KAT152)
\end{exe}

\subsection{Intransitive verbs with an~indirect object}
\label{subsec:12-2-4}

The next major type of intransitive verbs take, in addition to a~subject, an~indirect object (the term here used in a~broad sense, which will become even more obvious when discussing the parallel situation with transitive verbs taking an~indirect object). The indirect object occurs mostly as a~postpositional phrase, but occasionally as a~non"=nominative noun phrase. In the typical cases the subject is an~agent and the indirect object a~locative: \textbf{NP\textsc{sbj}} \textbf{PP/NP\textsc{io}} \textbf{V}.


This pattern is exemplified in (\ref{ex:12-44}) and (\ref{ex:12-45}) with the verbs `enter' and `reach' respectively.

\begin{exe}
\ex
\label{ex:12-44}
\gll [ak čoór]\textsubscript{\textsc{sbj}} [tesée ɡhooṣṭ-á]\textsubscript{\textsc{io}} ačíit-u \\
\textsc{idef} thief \textsc{3sg.gen} house-\textsc{obl} enter.\textsc{pfv"=msg} \\
\glt `A thief entered his house.' (B:THI)
\end{exe}
\begin{exe}
\ex
\label{ex:12-45}
\gll [tusaám the]\textsubscript{\textsc{io}} rhootašíia [páanǰ toobak-í]\textsubscript{\textsc{sbj}} phéd-an  \\
2\textsc{pl.acc} to tomorrow five gun-\textsc{pl} reach-\textsc{3pl} \\
\glt `Tomorrow five guns will reach you [i.e., will be sent to you].' (A:GHA085)
\end{exe}

Many verbs of this type (\tabref{tab:12-mot}) code events of motion.


\begin{table}[H]
\caption{Examples of intranstive verbs with an indirect object}
\begin{tabularx}{\textwidth}{ l@{\hspace{25pt}} l@{\hspace{25pt}} l@{\hspace{25pt}}
    l@{\hspace{25pt}} }
\lsptoprule
\textit{yhe-} &
`come' &
\textit{ač-} &
`enter, go in'\\
\textit{whe-} &
`come/go down' &
\textit{lanɡ-} &
`cross'\\
\textit{ukhé-} &
`come/go up' &
\textit{phed-} &
`arrive, reach'\\
\textit{be-} &
`go' &
\textit{nam-} &
`get down'\\
\textit{nikhé-} &
`appear, come out' &
\textit{uḍhéew-} &
`flee'\\\lspbottomrule
\end{tabularx}
\label{tab:12-mot}
\end{table}


For some of these verbs, especially those which are already spatially defined (e.g., \textit{whe-} and \textit{ukhé-}), the argument status of the indirect object is somewhat doubtful, and they may alternatively be classified as simple intransitive verbs with possible (but optionally occurring) locative complements or elaborators \citep[304--305]{allerton2006}.



Whether verbs whose meanings include vertical directional senses `up', `down', `level', is a~feature of an~area comprising several languages, particularly a~mountainous region such as the Hindukush, is a~matter for further research. It is in any case a~feature pointed out by \citet[9]{noonan2003} and summarised as the presence of ``vertical case and vertical verbs'' in some languages of the Himalayas. In Palula, spatial and vertical differentiation among pronouns and adverbs, can certainly be seen as part of the same phenomenon. 


\subsection{Transitive verbs with an~indirect object}
\label{subsec:12-2-5}


The contrast between simple intransitive verbs and intransitive verbs with an~indirect object is more or less parallel to a~contrast between simple transitive verbs and transitive verbs with an~indirect object. We recognise here quite a~few verbs as transitive (or causative) counterparts of the verbs presented in the section above and sometimes they also happen to be transparent morphological derivations of those (compare with \textit{lanɡ-/lanɡá-} `cross/take across'). 


To the pattern of simple transitive verbs is added an~indirect object, occurring mostly as a~postpositional phrase but alternatively as a~non"=nominative noun phrase. Again, the indirect object is a~very broadly defined argument type that includes a~whole range of postpositional phrases and noun phrases coded as non"=core participants. In the typical case the subject is an~agent, the direct object a~patient whose physical location is being changed, and the indirect object a~locative: \textbf{NP\textsc{sbj}} \textbf{PP/NP\textsc{io}} \textbf{NP\textsc{do}} \textbf{V}. 


This pattern is exemplified in (\ref{ex:12-46}) with the verb `throw'.

\begin{exe}
\ex
\label{ex:12-46}
\gll [tíi]\textsubscript{\textsc{sbj}} bi [teeṇíi zaán]\textsubscript{\textsc{do}} [wíi-a]\textsubscript{\textsc{io}} ɡaíl-i hín-i \\
\textsc{3sg.obl} also own self water-\textsc{obl} throw.\textsc{pfv-f} be.\textsc{prs-f} \\
\glt `He threw himself into the water.' (A:SHY062)
\end{exe}

The movement can also be extended into the abstract realm, as with the verb \textit{ɡaḍé-} `take off, extract' in (\ref{ex:12-47}).

\begin{exe}
\ex
\label{ex:12-47}
\gll [ḍaaktar-á]\textsubscript{\textsc{sbj}} [bidráaɡu]\textsubscript{\textsc{do}} [xatrá díi]\textsubscript{\textsc{io}} ɡaḍíl-u \\
doctor-\textsc{obl} ill.person danger from take.off.\textsc{pfv"=msg} \\
\glt `The doctor brought the patient out of danger.' (A:Q9.0594)
\end{exe}

Many verbs of this type (\tabref{tab:12-trmot}) code events of movement (concrete as well as abstract) caused by a~human agent, some of them inclusive of a~vertical specification (as some of the intransitive verbs with an~indirect object).


\begin{table}[H]
\caption{Examples of transitive motion verbs with an indirect locational object}
\begin{tabularx}{\textwidth}{ l@{\hspace{25pt}} l@{\hspace{25pt}} l@{\hspace{25pt} }
    l@{\hspace{25pt}} }
\lsptoprule
\textit{čhooré} &
`put' &
\textit{ɡaḍé} &
`take off, extract\\
\textit{bhešá} &
`seat' &
\textit{lanɡá} &
`take across'\\
\textit{ɡalé} &
`throw' &
\textit{whaalé} &
`carry down, take down'\\
\textit{lamá} &
`hang' &
\textit{ukuaalé} &
`carry up, take up'\\\lspbottomrule
\end{tabularx}
\label{tab:12-trmot}
\end{table}


Other events (\tabref{tab:12-ben}), coded with verbs traditionally labelled bi- or ditransitive, can easily be seen as extensions of these, where the indirect object instead of being a~locational goal (often) is a~human ``goal'' with a~benefactive role. There is, not surprisingly, some overlap between these two, so that some of the verbs may be used with a~locative non"=human goal as well as with a~benefactive human goal.


\begin{table}[H]
\caption{Examples of transitive verbs with an indirect benefactive object}
\begin{tabularx}{\textwidth}{ l@{\hspace{25pt}} Q l@{\hspace{25pt}} Q }
\lsptoprule
\textit{de-} &
`give' &
\textit{amzayé-} &
`send (something)'\\
\textit{aṭé-} &
`bring (something to someone)' &
\textit{phedá-} &
`take (something to someone)'\\
\textit{phrayé-} &
`send (someone)' &
&
\\\lspbottomrule
\end{tabularx}
\label{tab:12-ben}
\end{table}


Examples (\ref{ex:12-48}) and (\ref{ex:12-49}) illustrate the use of \textit{phedá-} `take, make reach' and \textit{phrayé-} `send', respectively.

\begin{exe}
\ex
\label{ex:12-48}
\gll [se ɣar-í the]\textsubscript{\textsc{io}} [asím]\textsubscript{\textsc{sbj}} [tas]\textsubscript{\textsc{do}} phedóol-u \\
\textsc{def} peak-\textsc{obl} to \textsc{1pl.erg} \textsc{3sg.acc} take.\textsc{pfv"=msg} \\
\glt `We took him to the peak.' (A:GHA029)
\end{exe}
\begin{exe}
\ex
\label{ex:12-49}
\gll [šišíi-e hakim-á]\textsubscript{\textsc{sbj}} [thíi so duṣmán]\textsubscript{\textsc{do}} [nawaab-á ḍáḍi]\textsubscript{\textsc{io}} phreyíl-u thaní \\
Shishi-\textsc{gen} ruler-\textsc{obl} \textsc{2sg.gen} \textsc{def.msg.nom} enemy  prince-\textsc{obl} toward send.\textsc{pfv"=msg} \textsc{quot} \\
\glt `He said: The ruler of Shishi has sent that enemy of yours in the direction of the prince [of Dir].' (B:ATI031)
\end{exe}

In fact, not even \textit{de-} `give', the verb most typically fitting the description ditransitive in a~comparative perspective, is essentially or inherently ditransitive, but merely a~transitive verb with an~indirect object which has a~certain preference for the goal"=role, but which shows quite some variability in this respect (compare with \sectref{subsec:12-2-6}, \sectref{subsec:12-2-8}). \citet[43]{baart1999a} notes a~similar range of usage and variability in the valency"=pattern of GIVE in Gawri.



The pattern is extended to other verbs with ``movements'' (\tabref{tab:12-trind}) that only allow for an~abstract interpretation.


\begin{table}[H]
\caption{Other transitive verbs with an indirect object}
\begin{tabularx}{\textwidth}{ l@{\hspace{25pt}} l@{\hspace{25pt}} l@{\hspace{25pt}}
    l@{\hspace{25pt}} }
\lsptoprule
\textit{pašawá-} &
`show' &
\textit{mané-} &
`tell, read'\\\lspbottomrule
\end{tabularx}
\label{tab:12-trind}
\end{table}


Example (\ref{ex:12-50}) illustrates the use of \textit{mané-} `tell, read'.

\begin{exe}
\ex
\label{ex:12-50}
\gll [so]\textsubscript{\textsc{sbj}} [nis]\textsubscript{\textsc{do}} [har áak-a the]\textsubscript{\textsc{io}} man-áan-u \\
\textsc{3sg.nom} \textsc{3sg.prox.acc} every one-\textsc{obl} to say-\textsc{prs"=msg} \\
\glt `He's telling everybody about it.' (A:Q9.575)
\end{exe}

With some verbs of this kind, an~alternation in coding between the two objects is possible; each corresponding to the particular perspective taken describing what is virtually one and the same event. 

\begin{exe}
\ex
\label{ex:12-51}
\gll aní phaí manɡée wée wíi puuríl-u \\
\textsc{prox} girl pot.\textsc{obl} into water fill.\textsc{pfv"=msg} \\
\glt `This girl filled the pot with water [lit: filled water into the pot].' (A:CHE070918)
\end{exe}
\begin{exe}
\ex
\label{ex:12-52}
\gll aní phaí wíi-yii manɡái puuríl-i \\
\textsc{prox} girl water-\textsc{gen} pot fill.\textsc{pfv-f} \\
\glt `This girl filled the pot with water [lit: filled the pot of water].' (A:CHE070918)
\end{exe}

In (\ref{ex:12-51}), the content, i.e., the `water', is coded as a~direct object (agreeing with the finite verb) and the vessel as a~locative indirect object, whereas in (\ref{ex:12-52}), the vessel is coded as a~direct object (agreeing with the finite verb) and the content in the genitive.\footnote{It is equally possible to interpret the genitive NP `of water' in this sentence as a~dependent of `pot', rather than an ``indirect object'', thus making the sentence a~simple transitive one; however, that is a~less likely interpretation in the sentence \textit{tíi baalṭí wíiyii puuríli} `She filled the bucket with water', where the direct object instead precedes the genitive NP.} 


Some verbs coding transactions, such as `ask for' in (\ref{ex:12-53}) and (\ref{ex:12-54}), seem to take a~direct object and two potential indirect objects, although in actual usage only one of these indirect objects seem to appear as independent sentence arguments at a~time. 

\begin{exe}
\ex
\label{ex:12-53}
\gll phaíi báabu ǰhaamatreé díi xarčá bi dawa-áan-u \\
girl.\textsc{gen} father son.in.law.\textsc{obl} from compensation also ask.for-\textsc{prs"=msg} \\
\glt `The girl's father demands compensation from the son"=in"=law.' (A:MAR032)
\end{exe}
\begin{exe}
\ex
\label{ex:12-54}
\gll taním teeṇíi bharíiw-a the baačaa-íi=ee wazíir"=ii dhii-á dawéel"=im \\
3\textsc{pl.erg} \textsc{refl} husband-\textsc{obl} to king-\textsc{gen=cnj}  minister-\textsc{gen} daughter-\textsc{pl} ask.for.\textsc{pfv"=fpl}\\
\glt `They demanded the king's and the minister's daughters [in marriage] for their husband.' (A:UXW059)
\end{exe}

\subsection{Non"=standard valency patterns}
\label{subsec:12-2-6}

\spitzmarke{Dummy"=subject or subjectless verbs.} Although Palula generally seems to avoid subjectless clauses, there are a~few instances, particularly in weather expressions (\tabref{12-opt}), where that is possible.


\begin{table}[H]
\caption{Examples of verbs with an optional subject}
\begin{tabularx}{\textwidth}{ l@{\hspace{25pt}} l@{\hspace{25pt}} l@{\hspace{25pt}}
    l@{\hspace{25pt}} }
\lsptoprule
\textit{beedhré-} &
`clear up' &
\textit{muč-} &
`rain'\\\lspbottomrule
\end{tabularx}
\label{tab:12-opt}
\end{table}


These two verbs are exemplified in (\ref{ex:12-55}) and (\ref{ex:12-56}). 

\begin{exe}
\ex
\label{ex:12-55}
\gll (aaɡhaá) beedhríil-u \\
sky clear.up.\textsc{pfv"=msg} \\
\glt `The sky cleared/It cleared up.' (A:HLE3047)
\end{exe}
\begin{exe}
\ex
\label{ex:12-56}
\gll ǰulaí yúun-a daš taarexée biooṛkúi$\sim$ sax mút-u \\
July month-\textsc{obl} \textsc{10} date.\textsc{gen} Biori.valley heavy  rain.\textsc{pfv"=msg} \\
\glt `The tenth of July it rained heavily in Biori Valley.' (B:FLO166)
\end{exe}

In both of these cases, the verb is strictly limited to certain forms (those with masculine singular agreement) and each is, generally speaking, only used with one particular noun, \textit{beedhré-} with \textit{aaɡhaá} and \textit{muč-} with \textit{báaṣ (}B \textit{baṣ)} `rain'. In the latter case, the noun needs to be explicitly included in the A variety, whereas it is often left implicit in B. In any case, it is clear that even when the subject is present, it does not add anything semantically to the clause that is not already implicit in the meaning of the verb.


Some other weather expressions make use of the verb \textit{de-} `give' (\tabref{tab:12-wea}), which normally is a~transitive verb with an~indirect object, but appears here as formally intransitive (rendering `fall' a~better gloss for \textit{de-}). However, it would also be possible to analyse these expressions as subject"=deprived, resulting in the object formally filling the subject position, thus agreeing with the verb in the imperfective as well as in the perfective. 


\begin{table}[H]
\caption{Examples of weather expressions with the verb \textit{de-}}
\begin{tabularx}{\textwidth}{ l@{\hspace{25pt}} l@{\hspace{25pt}} l@{\hspace{25pt}} l@{\hspace{25pt}} }
\lsptoprule
\textit{kir de-} &
`snow' &
\textit{húuši de-} &
`blow'\\
\textit{áašuṇḍ de-} &
`hail' &
&
\\\lspbottomrule
\end{tabularx}
\label{tab:12-wea}
\end{table}


The expression \textit{kir de-} is exemplified in (\ref{ex:12-57}).

\begin{exe}
\ex
\label{ex:12-57}
\gll panǰ phuṭ-í kir dít-u \\
five foot-\textsc{pl} snow give.\textsc{pfv"=msg} \\
\glt `[We] had five feet of snow.' (B:AVA200)
\end{exe}

\spitzmarke{Intransitive verbs with a~non"=nominative experiencer.} A special case of intransitive verbs with an~indirect object (\sectref{subsec:12-2-4}) can be said to be used for a~number of constructions where a~sensation is coded as the nominative subject~-- agreeing with the verb~-- whereas the human or animate experiencer appears as a~non"=nominative (often coded like or perceived as a~locative) NP or PP. Some examples can be seen in \tabref{tab:12-nnom}. Formally the pattern is equal to \sectref{subsec:12-2-4}, but due to the special status of the non"=nominative argument, this particular construction deserves special treatment here. Note that the verbs used in these constructions have more generic meaning and scope when used in other, non"=experiential, clauses: \textit{de-} `give', \textit{ṣač-} `adhere to, climb', \textit{yhe-} `come', \textit{ḍhak-} `touch'.


\begin{table}[H]
\caption{Examples of verbs with a non"=nominative experiencer}
\begin{tabularx}{\textwidth}{ Q Q Q Q }
\lsptoprule
\textit{bhíili/šid de-} &
`feel fear/cold' &
\textit{níindra yhe-} &
`feel sleepy'\\
\textit{bhíili/šid ṣač-} &
`feel fear/cold' &
\textit{ǰáar ḍhak-} &
`be affected by fever'\\\lspbottomrule
\end{tabularx}
\label{tab:12-nnom}
\end{table}


Some of these are illustrated in examples (\ref{ex:12-58})--(\ref{ex:12-60}).

\begin{exe}
\ex
\label{ex:12-58}
\gll [tusaám]\textsubscript{\textsc{io(exp)}} [níindra]\textsubscript{\textsc{sbj}} yh-éend"=im \\
\textsc{2pl.acc} sleep come-\textsc{prs"=fpl} \\
\glt `You are feeling sleepy.' (A:CHE071001)

\ex
\label{ex:12-59}
\gll [míiš-a]\textsubscript{\textsc{io(exp)}} [bhíili]\textsubscript{\textsc{sbj}}
ṣéet-i \\
man-\textsc{obl} fear adhere.\textsc{pfv-f} \\
\glt `The man was overcome by fear.' (A:CHE070927)

\ex
\label{ex:12-60}
\gll [tas the]\textsubscript{\textsc{ io(exp)}} [marɡ-íi ǰáar]\textsubscript{\textsc{sbj}} ḍhakíl-u \\
3\textsc{sg.acc} to death-\textsc{gen} fever touch.\textsc{pfv"=msg} \\
\glt `He got a~severe fever.' (A:ABO023)
\end{exe}

Similar expressions that code an~experiencer non"=nominatively and the sensation nominatively, often referred to as ``dative subjects'', are found in various South Asian languages (both in IA and non"=IA languages\footnote{Outside IA, however, non"=nominative experiencers are primarily found in Dravidian \citep[260--263]{abbi1990}, a~reason for Dravidian to be suggested as a~possible source of the construction \citep[136]{hock1990}. It is not a~typical feature of Tibeto"=Burman (\citealt[260]{abbi1990}; \citealt[82]{bickel2004}), and they are obviously only found in those Tibetan languages in close contact with IA (\citealt[83, 88]{bickel2004}; \citealt[8--9]{noonan2003}).}), as shown by \citet[326--330]{hook1990b} and \citet[256--263]{abbi1990}. However, a~construction involving (formal) causatives for expressing involuntary experience, found in Kalasha \citep[310]{bashir1990} as well as in Gilgiti Shina \citep{hookzia2005}, is not at all evidenced in Palula. 



In some sense, the conjunct verb \textit{ḍhoó/ḍhoowá de-} `see, notice/appear' (see \sectref{subsec:12-2-8}) can also be described along the same lines, where the experiencer (if explicit in the clause) occurs as a~non"=nominative NP or a~PP, while what appears or is seen is a~nominative NP agreeing with the verb, as can be seen in (\ref{ex:12-61}).

\begin{exe}
\ex
\label{ex:12-61}
\gll akáaš kaal-á padúši [nis the] [dúu ziaarat-í] ḍhoowá dít-im \\
eleven year-\textsc{pl} after \textsc{3sg.prox.acc} to two shrine-\textsc{pl} \textsc{host} give.\textsc{pfv"=fpl} \\
\glt `Eleven years later two shrines appeared to him/Eleven years later he saw two shrines.' (B:FOR021)
\end{exe}

Another construction in which the experiencer also is ``demoted'' to non"=nominative coding is found with e.g., some bodily sensations, as in (\ref{ex:12-62}), where the body part occurs, in this case `teeth', as the head of the subject NP, and the experiencer appears as the genitive modifier `my'.

\begin{exe}
\ex
\label{ex:12-62}
\gll míi dáand-a šila-yáan-a \\
\textsc{1sg.gen} tooth-\textsc{pl} hurt-\textsc{prs"=mpl} \\
\glt `I've got a~toothache [lit: My teeth are hurting].' (B:DHE5112)
\end{exe}

\spitzmarke{Intransitive verbs with a~postpositional object.} What can be viewed as a~metaphoric extension of the pattern of intransitive verbs with an~indirect object (see \sectref{subsec:12-2-4}) is the pattern of intransitive verbs with a~dative or patient indirect object (\tabref{tab:12-ipost}). Such indirect objects are coded with oblique case and a~postposition although they in a~semantic sense can be said to be primary objects.


\begin{table}[H]
\caption{Examples of intrantive verbs with a postpositional object}
\begin{tabularx}{\textwidth}{ l@{\hspace{25pt}} l@{\hspace{25pt}} l@{\hspace{25pt}}
    l@{\hspace{25pt}} }
\lsptoprule
\textit{khóṇḍ-} &
`talk (with)' &
\textit{uṭík-} &
`be angry (with) [lit: jump at]'\\\lspbottomrule
\end{tabularx}
\label{tab:12-ipost}
\end{table}


Example (\ref{ex:12-63}) illustrates the use of \textit{khóṇḍ-} `talk with'.

\begin{exe}
\ex
\label{ex:12-63}
\gll [ma]\textsubscript{\textsc{sbj}} [tas sanɡí]\textsubscript{\textsc{io(primary object)}} khóṇḍ-um de ta bi so kráam th-íi de \\
\textsc{1sg.nom} \textsc{3sg.acc} with talk-\textsc{1sg} \textsc{pst}  \textsc{prt} also \textsc{3sg.nom} work do-\textsc{3sg} \textsc{pst} \\
\glt `He continued working while I talked to him.' (A:Q9.1105)
\end{exe}

Some conjunct verbs with \textit{bhe-} (\sectref{subsec:12-2-8}) display the same pattern: \textit{ašáq bhe-} `fall in love (with)', \textit{milaáu bhe-} `meet'.


\spitzmarke{Transitive verb with a~genitive object.} One verb, the transitive \textit{ǰe-} `hit, beat', is exceptional in that it codes the direct object~-- or at least what seems to be the only explicit object~-- in the genitive (instead of the usual nominative/accusative), as seen in (\ref{ex:12-64}).

\begin{exe}
\ex
\label{ex:12-64}
\gll [kuṛíina]\textsubscript{\textsc{sbj}} ta [támbul"=am-ii]\textsubscript{\textsc{gen}} ǰ-íin  \\
woman.\textsc{pl} \textsc{prt} drum-\textsc{pl.obl"=gen} beat-\textsc{3pl}  \\
\glt `And the women were beating the drums.' (A:JAN034)
\end{exe}

That we are indeed dealing with a~formally (and not only ``logically'') transitive verb is confirmed by the ergative pattern in the perfective (i.e., non"=nominative coding of the subject and non"=subject verbal agreement). Here, however, another anomaly shows up, namely that the verb agrees, seemingly ``by default'', in the feminine singular, even when the explicit object is masculine, as in (\ref{ex:12-65}), or plural.

\begin{exe}
\ex
\label{ex:12-65}
\gll [aṛé míiš-a]\textsubscript{\textsc{sbj}} [aní dúu lhoók-a kučúr-am"=ii]\textsubscript{\textsc{gen}} ǰít-i \\
\textsc{dist} man-\textsc{obl} \textsc{prox} two small-\textsc{obl} dog-\textsc{pl.obl"=gen} beat.\textsc{pfv-f} \\
\glt `That man hit these two small dogs.' (A:ADJ80)
\end{exe}

\citet[43]{baart1999a}, who observes the same anomaly in Gawri with a~verb meaning `hit, beat' (where it happens to be one sense of the verb `give'), suggests by pointing to a~synonymous construction that this particular pattern is historically elliptic, leaving out a~feminine singular noun meaning `stroke', which is itself the direct object being modified by the genitive noun phrase. 


The genitive coding does not show up when \textit{ǰe-} is being used as a~verbaliser in a~conjunct (\sectref{subsec:12-2-8}).


\subsection{Verbs with clausal complements}
\label{subsec:12-2-7}

The syntax of verbs that take a~clause as their complements will be treated in detail elsewhere (\sectref{sec:13-5}). As a~part of this chapter, only some major subtypes, as they relate to argument structure, are pointed out and exemplified.


\spitzmarke{Modality verbs.} With modality verbs (\tabref{tab:12-mod}), the subject of the main clause~-- which is headed by the finite modality verb~-- is coreferent with the subject of the complement clause, the latter being left unexpressed. The complement"=clause verb appears either as an~Infinitive or a~Verbal Noun. Quite a~few modality verbs are conjunct verbs (the ones in the right column).


\begin{table}[H]
\caption{Examples of modality verbs}
\begin{tabularx}{\textwidth}{ l@{\hspace{25pt}} l@{\hspace{25pt}} l@{\hspace{25pt}}
    l@{\hspace{25pt}} }
\lsptoprule
\textit{bha-} &
`can/could, be able' &
\textit{inkaár the-} &
`refuse'\\
\textit{ṣáat-} &
`begin'{\protect\footnotemark} &
\textit{iraadá the-} &
`plan'\\
\textit{dawá-} &
`want' &
\textit{koošíš the-} &
`plan' \\\lspbottomrule
\end{tabularx}
\label{tab:12-mod}
\end{table}

\footnotetext{This modality verb, which as a~non"=modal verb corresponds to the imperfective stem
  \textit{ṣač-} of a~very polysemous verb with meanings such as `climb, argue, be lit', also
  implies aspectuality and occurs in my data only in the perfective, hence the perfective stem
  given.}

Examples (\ref{ex:12-66}) and (\ref{ex:12-67}) illustrate the use of \textit{bha-} `can/could, be able' and \textit{ṣáat-} `begin', respectively.
\begin{exe}
\ex
\label{ex:12-66}
\gll tanaám maǰí [áak míiš]\textsubscript{\textsc{sbj}} muṭ-á ǰe ukh-áai bhóo de \\
\textsc{3pl.acc} among one man tree-\textsc{obl} up ascend-\textsc{inf}  be.able.\textsc{3sg} \textsc{pst} \\
\glt `Of them only one man was able to climb the tree.' (A:UNF007)
\end{exe}
\begin{exe}
\ex
\label{ex:12-67}
\gll [čhéeli]\textsubscript{\textsc{sbj}} líiwee čaapeerá ɡiráa ṣéet-i \\
goat in.there around turn.\textsc{inf} begin.\textsc{pfv-f} \\
\glt `The goat started to turn from side to side in there.' (B:FOX)
\end{exe}

The scope of these verbs, the grammatical categories available to them, and the degree of their grammaticalisation varies extensively. For example, some of them are modality verbs only, as is the case of \textit{bha-}, whereas some other verbs in this group, such as \textit{ṣáat-}, display multiple membership. Although occurring sometimes in otherwise perfectly parallel constructions (as far as surface valency is concerned), their inherent transitivity/intransitivity and argument coding is kept intact. That is, with the transitive modality verb \textit{bha-}, the direct object of the complement clause is coreferent with the direct object of the main clause (\ref{ex:12-68}), whereas with the intransitive modality verb \textit{ṣáat-,} the direct object of the complement clause stands outside the argument structure of the modality verb (\ref{ex:12-69}).

\begin{exe}
\ex
\label{ex:12-68}
\gll [asím]\textsubscript{\textsc{sbj}} [pileeṭ-íi buṭheé mhaás]\textsubscript{\textsc{do}} kháai bhóol-u \\
\textsc{1pl.erg} plate-\textsc{gen} all meat eat.\textsc{inf} be.able.\textsc{pfv"=msg} \\
\glt `We were able to eat all the meat on the plates.' (A:CHE070920)
\end{exe}
\begin{exe}
\ex
\label{ex:12-69}
\gll [be]\textsubscript{\textsc{sbj}} mhaás kháai/khainií ṣáat-a \\
\textsc{1pl.nom} meat eat.\textsc{inf}/eat.\textsc{vn} begin.\textsc{pfv"=mpl} \\
\glt `We started to eat meat.' (A:CHE070920)
\end{exe}

Complement"=taking conjunct verbs (such as \textit{iraadá the-}) are a~rather fluid and open class, with the potential of easily introducing new modality senses through direct calques from e.g. Urdu. 


\spitzmarke{Manipulation verbs.} The (human) agent"=subject of the main clause~-- which is headed by a~manipulation verb~-- manipulates the behaviour of a~manipulee, the latter being coreferential with the agent of the complement clause. The complement"=clause verb often appears as a~Verbal Noun coded as an~indirect object, \textit{bainií the} in (\ref{ex:12-70}), of the main clause. 

\begin{exe}
\ex
\label{ex:12-70}
\gll ɡhaḍeer-á xálak-a bainií the uṛíit-a \\
elder-\textsc{obl} people-\textsc{pl} go.\textsc{vn} to let.\textsc{pfv"=mpl} \\
\glt `The elder allowed people to leave.' (A:Q6.13.03)
\end{exe}

There is also a~special causative construction, using \textit{ṣaawaá}, the converb of \textit{ṣaawá-} `dress (someone), turn on (e.g., a~light or a~fire)' preceded by the manipulee coded as an~indirect object. The inflected finite verb in this construction could be any verb derived causatively (see \sectref{subsec:8-5-1}). The causative verb in (\ref{ex:12-71}) is derived from \textit{aṭé-} `bring'.

\begin{exe}
\ex
\label{ex:12-71}
\gll míi tas teeṇíi putr-á ṣaawaá aṭawóol-u \\
\textsc{1sg.gen} \textsc{3sg.acc} \textsc{refl} son-\textsc{obl} \textsc{manip} cause.to.bring.\textsc{pfv"=msg} \\
\glt `I made my son bring him.' (A:HLE2589)
\end{exe}

\spitzmarke{Perception, cognition and utterance (PCU) verbs.} The subject of a~PCU verb (\tabref{tab:12-pcu}), perceives or cognises a~state or event, or utters a~proposition concerning a~state or event. The complement clause corresponds to what is perceived, cognised or uttered. Some such verbs are only or primarily used with a~clausal complement, whereas a~few of them display multiple membership.


\begin{table}[H]
\caption{Examples of PCU verbs}
\begin{tabularx}{\textwidth}{ l@{\hspace{25pt}} l@{\hspace{25pt}} l@{\hspace{25pt}}
    l@{\hspace{25pt}} }
\lsptoprule
\textit{buǰ-} &
`understand' &
\textit{mané-} &
`say'\\
\textit{dun-} &
`think' &
\textit{ṣuṇ-} &
`hear'\\
\textit{khooǰá-} &
`ask' &
\textit{ṭaaké-} &
`call out to'\\\lspbottomrule
\end{tabularx}
\label{tab:12-pcu}
\end{table}


There are reasons for not regarding the complement clause as a~direct object, one being the non"=verb"=final position of the complement clause, the other being that the structure stays the same regardless of the ``inherent'' argument structure of the PCU verb being used. In the examples (\ref{ex:12-72}) and (\ref{ex:12-74}), \textit{mané-} `say' and \textit{khooǰá-} `ask' are both transitive verbs with indirect objects (although explicit only in (\ref{ex:12-74})), whereas \textit{buǰ-} `understand' in (\ref{ex:12-73}) is intransitive. Transitive PCU verbs with a~complement display masculine"=singular agreement in the perfective, by default. This is similar to the ``mismatch'' between argument coding and the valency patterns with the intransitive vs. transitive modality verbs above.

\begin{exe}
\ex
\label{ex:12-72}
\gll [ɡhueeṇíi-am]\textsubscript{\textsc{sbj}} maníit-u ki [ni bíiḍ-a zinaawúr xálak-a hín-a]\textsubscript{\textsc{cpl}} \\
Pashtun-\textsc{pl.obl} say.\textsc{pfv"=msg} \textsc{c}\textsc{om}\textsc{p} \textsc{3pl.prox.nom} very-\textsc{mpl}  wild people-\textsc{pl} be.\textsc{prs"=mpl} \\
\glt `The Pashtuns said: These are very wild people.' (A:CHA008)
\end{exe}
\begin{exe}
\ex
\label{ex:12-73}
\gll haṛé waxtíe [ṣiáal=ee lhooméi]\textsubscript{\textsc{sbj}} bi búd-a ki \textsc{[}karáaṛu asaám khúu]\textsubscript{\textsc{cpl}} \\
\textsc{dist} time.\textsc{gen} jackal=\textsc{cnj} fox also  understand.\textsc{pfv"=mpl} \textsc{comp} leopard \textsc{1pl.acc} eat.\textsc{3sg} \\
\glt `At that time also the jackal and the fox understood: The leopard will eat us.' (B:FOY)
\end{exe}
\begin{exe}
\ex
\label{ex:12-74}
\gll [ṣiúul-a]\textsubscript{\textsc{sbj}} [lhooméea díi]\textsubscript{\textsc{io}} khooǰúul-u ki [ée lhooméi{\ldots}]\textsubscript{\textsc{cpl}} \\
jackal-\textsc{obl} fox.\textsc{obl} from ask.\textsc{pfv"=msg} \textsc{comp} oh fox \\
\glt `The jackal asked the fox: Oh, fox{\ldots}' (B:FOY)
\end{exe}

Normally the complement clause is preceded by the complementiser \textit{ki}, as in the examples above, but there are also other (and additional) strategies available, some of them with relevance only for individual PCU verbs (see \sectref{subsec:13-5-1}). 


\subsection{Valency patterns of conjunct verb constructions}
\label{subsec:12-2-8}

Conjunct verbs (see \sectref{subsec:8-6-1}) are special in that the host (i.e., the element that combines with a~verbaliser to form a~conjunct verb) in some of them stands in no grammatical relation whatsoever to other parts of the sentence, whereas with others it functions as a~direct object of the verbaliser and as such controls agreement in the perfective. Using various diagnostics, some scholars (\citealt[201]{verma1993}; \citealt[165]{mohanan1993}) have suggested that only some of these combinations are ``true'' conjunct verbs (particularly those of the former kind), while the rest are normal syntactic combinations of nouns and verbs. 



This, however, seems somewhat oversimplified and would possibly exclude (as pointed out by \citealt[160]{masica1993}) the more productive in the language as well as obscure the mechanisms by which this pattern has come about in the first place. Therefore I prefer to regard all of them, at least preliminarily, as conjunct"=verb constructions, but on different levels of lexicalisation or grammaticalisation, and with a~possible subclassification into: a) conjuncts with a~host contributing to the argument structure (what I refer to as ``non-incorporating'' below) and b) conjuncts with hosts playing no role in argument structure and agreement patterns (referred to below as ``incorporating''). The latter is supported by a~distinction suggested by \citet[69--74]{jaeger2006} between light"=verb constructions (by him defined more narrowly than \citealt{butt2010}) and periphrastic constructions (discussed by \citeauthor{jaeger2006} particularly with reference to so"=called `do'-periphrasis), also acknowledging that the two constructions in fact form a~cross"=linguistic continuum and may share functional properties in individual languages. 



\subsubsection*{Conjunct verbs with \textit{the-} `do'}

The most productive means of forming conjunct verbs is with \textit{the-} `do', and we accordingly notice a~wide variety of constructions and possible argument structures for conjuncts formed this way. For a~large part of them, but far from all, the host is a~relatively recent loan from another language. All of these constructions are transitive, which is obvious from the ergative alignment in the perfective. While \citet[201]{verma1993} suggests a~subclassification of conjuncts into three types~-- lexically complex, syntactically complex and a~purely analytical sequence (the latter only superficially looking like a~conjunct verb)~-- I see no reason to make a~distinction between the two latter categories. Instead I will distinguish between \textit{incorporating conjunct verbs} and \textit{non-incorporating conjunct verbs} only, basically following Haig (\citeyear{haig2002}) in his work on complex predicates in Kurdish.


\spitzmarke{Incorporating \textit{the}-conjuncts (no host"=verbaliser agreement).} These conjuncts (examples listed in \tabref{tab:12-3}) are internally complex but externally unitary. While being phonologically two words (each carrying its own accent), the complex is syntactically equal to simple transitive verbs.


\begin{table}[ht]
\caption{Incorporating \textit{the}-conjuncts}
\begin{tabularx}{.75\textwidth}{ l@{\hspace{45pt}} Q }
\lsptoprule
Host + Verbaliser &
\\\hline
\textit{tanɡ the-} &
`trouble'\\
\textit{band the-} &
`close, block, stop'\\
\textit{teér the-} &
`spend'\\
\textit{čhin the-} &
`wean, abandon'\\
\textit{široó the-} &
`start'\\
\textit{bayaán the-} &
`tell'\\
\textit{hawaalá the-} &
`hand over'\\
\textit{raál the-} &
`raise, lift'\\
\textit{káaṇ the-} &
`listen to'\\
\textit{ḍóo the-} &
`carry (a person)'\\
\textit{baáṭ the-} &
`sharpen (a tool)'\\
\textit{ǰamá the-} &
`gather'\\
\textit{muqarár the-} &
`fix'\\
\textit{rekaáḍ the-} &
`record'\\\lspbottomrule
\end{tabularx}
\label{tab:12-3}
\end{table}


This also means that the direct object in such constructions is an~argument of the complex as a~whole, and the complex can be analysed as a~single predicate. It is still possible, however, for the host to be separated from the verbaliser by, for instance, a~negation. All conjunct \textit{the}-verbs with an~adjective host, such as \textit{tanɡ the-} exemplified in (\ref{ex:12-75})~-- or at least what is an~adjective in other uses or in a~donor language~-- belong to this particular type. What is more interesting, however, is that we also find complexes of this type formed with a~noun as a~host (at least in a~historical"=etymological sense) that in no way participates as an~NP in the argument structure. The conjunct \textit{káaṇ the-} in (\ref{ex:12-76}) is an~example. 

\begin{exe}
\ex
\label{ex:12-75}
\gll [dunyáii zinaawur-aán]\textsubscript{\textsc{do}} [thíi]\textsubscript{\textsc{sbj}} aní zanɡal-í buṭheé tanɡ thíil-a hín-a \\
world.\textsc{gen} beast-\textsc{pl} \textsc{2sg.gen} \textsc{prox} forest-\textsc{obl} all  narrow do.\textsc{pfv"=mpl} be.\textsc{prs"=mpl} \\
\glt `You have troubled all the beasts in this forest.' (A:KIN013)

\ex
\label{ex:12-76}
\gll [eesé waxt-íi peeɣambár hazrát iliaás aleehisalaam-íi beetí]\textsubscript{\textsc{do}} káaṇ na thíi de \\
\textsc{rem} time-\textsc{gen} prophet lord Elijah peace.upon.him-\textsc{gen} word.\textsc{pl} \textsc{host} \textsc{neg} do.\textsc{3sg} \textsc{pst} \\
\glt `He was not listening to the words of the prophet of that time, Elijah (PBUH).' (A:ABO011)
\end{exe}

I suggest that a~host element, which may be homonymous with another noun in the language~-- such as \textit{káaṇ} `ear' in \textit{káaṇ the-} `listen'~-- with which it is historically related, has in fact ceased to constitute an~NP in this construction, or, in the case of a~loan from another language~-- such as the Perso"=Arabic noun \textit{široó} in \textit{široó the-} `start'~-- it may never even have been interpreted as an~NP from the very start (which is probably the case with the English origin element \textit{rekaáḍ} in \textit{rekaáḍ the-} `record' as well), no matter how much of a~noun it is in the donor language. This process of lexical incorporation \citep[203]{verma1993} is simply a~convenient way of deriving new verbs, combining the semantics (or some aspects of it) of the host element with the ``verbness'' of the verbaliser with the resulting verb behaving, as far as argument structure is concerned, in no way different from any other simple transitive verb (compare with \citealt[286]{haspelmath1993} for Lezgian). In some cases it is not entirely unlikely that the complex as a~whole is a~calque from another language, without even going through a~process of lexical incorporation, as there are several close parallels in Urdu as well as in Pashto to Palula conjuncts. The conjunct verb of this type as a~whole is lexical in nature \citep[199]{verma1993}.



\spitzmarke{Non"=incorporating \textit{the}-conjuncts (host"=verbaliser agreement).} In the next type of conjunct verb, the verbaliser does agree with the nominal host (in the perfective), thus, if holding on to the conjunct idea, meaning that the predicate agrees with an~element internal to itself \citep[168]{mohanan1993}. Examples of conjuncts of this kind are given in \tabref{tab:12-4}.


\begin{table}[ht]
\caption{Non"=incorporating \textit{the}-conjuncts}
\begin{tabularx}{.75\textwidth}{ l@{\hspace{45pt}} Q} 
\lsptoprule
Host + verbaliser &
\\\hline
\textit{madád the-} &
`help'\\
\textit{šuweelí the-} &
`show goodness'\\
\textit{ǰarɡá the-} &
`consult'\\
\textit{muaáf the-} &
`forgive'\\
\textit{salaám the-} &
`greet'\\
\textit{ǰhaní the-} &
`marry'\\
\textit{bhootíi the-} &
`plough'\\
\textit{phóom the-} &
`take care of'\\
\textit{ṭoóp the-} &
`charge'\\
\textit{kráam the-} &
`work'\\
\textit{nimóos the-} &
`pray (ritual prayers)'\\
\textit{damá the-} &
`rest, take a break'\\
\textit{kaš the-} &
`smoke, take a draught'\\
\textit{amál the-} &
`obey, give heed, follow'\\
\textit{hamlá the-} &
`attack'\\\lspbottomrule
\end{tabularx}
\label{tab:12-4}
\end{table}


As with the former type, many of the host elements are transparent loans, such as \textit{madád} in (\ref{ex:12-77}) and (\ref{ex:12-78}), but there are also a~good number of hosts with a~longer history in the language. Quite a number of those, however, are not actually used, or have any clear homonyms, outside their conjunct use.

\begin{exe}
\ex
\label{ex:12-77}
\gll [ma]\textsubscript{\textsc{sbj}} šíiṭi be [tes sanɡí]\textsubscript{\textsc{io}} [madád]\textsubscript{\textsc{do}} th-áam  \\
\textsc{1sg.nom} inside go.\textsc{cv} \textsc{3sg.acc} with help do-\textsc{1sg} \\
\glt `When I have gone inside, I will help him.' (B:FOY)
\end{exe}
\begin{exe}
\ex
\label{ex:12-78}
\gll [eetíi]\textsubscript{\textsc{sbj}} [ma sanɡí]\textsubscript{\textsc{io}} [madád]\textsubscript{\textsc{do}} thíil-i  \\
\textsc{3sg.rem.obl} \textsc{1sg.nom} with help[\textsc{fsg]} do.\textsc{pfv-f}  \\
\glt `He helped me.' (A:GHA069)
\end{exe}

One solution to the somewhat puzzling situation of the host being part of the argument structure is to analyse the noun in the complex as itself assuming the role of a~predicate (\citealt[204--212]{verma1993}; \citealt[164--170]{mohanan1993}). This predicate chooses a~suitable verbaliser to team up with and contributes itself to the total number of arguments of the clause, but at the same time it assigns roles and case marking to its ``own'' arguments. I hold, however, that it is sufficient and far easier to see the valency pattern displayed (e.g., the host coded as a~direct object and the occurrence of indirect objects, i.e., certain postpositional phrases) as a~property (or lexical specification) of each individual construction in its entirety. 

\begin{exe}
\ex
\label{ex:12-79}
\gll [so]\textsubscript{\textsc{sbj}} [ɣariib"=aan-óom the]\textsubscript{\textsc{io}} bíiḍ-i [phóom]\textsubscript{\textsc{do}} th-íi de  \\
\textsc{3sg.nom} poor-\textsc{pl"=obl} to much-\textsc{f} care do-\textsc{3sg}  \textsc{pst} \\
\glt `He used to take good care of the poor.' (A:ABO004)
\end{exe}
\begin{exe}
\ex
\label{ex:12-80}
\gll [ṣuú]\textsubscript{\textsc{sbj}} [tusaám ǰhulí]\textsubscript{\textsc{io}} [hamlá]\textsubscript{\textsc{do}} th-áan-u  \\
king \textsc{1pl.acc} on attack do-\textsc{prs"=msg}  \\
\glt `The king is going to attack you.' (B:ATI040)
\end{exe}

As far as perfective agreement, the verbaliser agrees in the same manner with the host as any transitive verb does with its direct object, while the subject is ergatively coded. A remaining argument (which in that case could be considered a ``logical object'' vis-à-vis the host) has to be coded as an~indirect (postpositional) object, e.g., \textit{ɣariibaanóom the} `to the poor' in (\ref{ex:12-79}) and \textit{tusaám ǰhulí} `on you' in (\ref{ex:12-80}). The postposition used for that argument is (as noted above) specified by each conjunct. Some of these are displayed in \tabref{tab:12-5}.


\begin{table}[ht]
\caption{Postpositions in the valency pattern of some \textit{the}-conjuncts}

\begin{tabularx}{\textwidth}{ l@{\hspace{25pt}} l@{\hspace{25pt}} Q }
\lsptoprule
 \textbf{IO Postposition} &
&
Conjunct verb\\\hline
\textit{sanɡí} &
`with' &
\textit{madád the-, šuweelí the-, ǰarɡá the-}\\
\textit{the} &
`to' &
\textit{phóom the-, muaáf the-, salaám the-} \\
\textit{ǰhulí} &
`on' &
\textit{amál the-, hamlá the-} \\
\textit{dúši} &
`toward' &
\textit{ṭoóp the-} \\\lspbottomrule
\end{tabularx}
\label{tab:12-5}
\end{table}


The complexes, which \citet[201]{verma1993} considers ``purely analytical'' sequences and not ``true'' conjuncts, only differ from the ones already discussed in that they do not include a~postpositional phrase in their valency pattern. To treat them as different phenomena on the basis that they are logically intransitive, i.e., not having a ``logical object'' beside the direct object"=coded ``formal'' host, is not a~very convincing argument; they are in fact only ``logically intransitive'' in the trivial sense of translation equivalence (as pointed out by \citealt[157]{masica1993}). These would in Palula be conjuncts such as \textit{kráam the-} `work', \textit{nimóos (}B \textit{nimáas) the-} `pray', exemplified in (\ref{ex:12-81}), and \textit{hiǰrát the-} `migrate'. 

\begin{exe}
\ex
\label{ex:12-81}
\gll [ak buzrúɡ]\textsubscript{\textsc{sbj}} teeṇíi ɡhooṣṭ-á maṇḍayí [nimáas]\textsubscript{\textsc{do}} th-íi de \\
\textsc{idef} wise.man \textsc{refl} house-\textsc{obl} porch.\textsc{obl}  prayer do-\textsc{3sg} \textsc{pst} \\
\glt `A wise man was praying on the porch of his own house.' (B:THI)
\end{exe}

Just as with the other non-incorporating conjuncts, the verb agrees with the nominal host in gender/number (in the perfective), the latter thus being part of its argument structure in the form of a~direct object. 


Looking at the conjunct \textit{hiǰrát the-} `migrate' in (\ref{ex:12-82}), we also notice that this indeed specifies a~particular valency pattern, including a~source (realised by \textit{taayúu} `from there' in the example) beside the subject and the direct object. 

\begin{exe}
\ex
\label{ex:12-82}
\gll [tíi]\textsubscript{\textsc{sbj}} taayúu [hiǰrát]\textsubscript{\textsc{do}} thíil-i  \\
\textsc{3sg.obl} from.there migration do.\textsc{pfv-f} \\
\glt `He migrated from there.' (B:ATI002)
\end{exe}

The distinction suggested by Verma becomes even more superficial when considering a~conjunct like \textit{bhootíi the-} `plough', which can code the non"=nominative argument denoting `field' alternatively as a~locative NP (oblique case), as in (\ref{ex:12-84}), and a~postpositional phrase, as in (\ref{ex:12-83}).

\begin{exe}
\ex
\label{ex:12-83}
\gll [tíi]\textsubscript{\textsc{sbj}} [teeṇíi c̣híitr-a wée]\textsubscript{\textsc{io}} [bhootíi]\textsubscript{\textsc{do}} thíil-i  \\
\textsc{3sg.obl} \textsc{refl} field-\textsc{obl} in ploughing do.\textsc{pfv-f} \\
\glt `He ploughed his own field.' (A:CHE070925)
\end{exe}
\begin{exe}
\ex
\label{ex:12-84}
\gll [taním]\textsubscript{\textsc{sbj}} [teeṇíi c̣híitr"=am]\textsubscript{\textsc{loc}} [bhootíi-a]\textsubscript{\textsc{do}} thíil"=im  \\
\textsc{3pl.erg} \textsc{refl} field-\textsc{pl.obl} ploughing-\textsc{pl} do.\textsc{pfv"=fpl} \\
\glt `They ploughed their own fields.' (A:CHE070925)
\end{exe}

Another argument for treating all these instances of \textit{the}-conjuncts as a~single phenomenon is the means of passivisation/valency reduction. While other transitive nouns are amenable to morphological passivisation with the affix \textit{-íǰ} (see \sectref{subsec:8-5-2}), \textit{the}-conjuncts~-- regardless of subtype -- are passivised or valency"=reduced by substituting \textit{the-} for \textit{bhe-} `become', as shown in (\ref{ex:12-85})--(\ref{ex:12-87}), the other main verbaliser (see below).

\begin{exe}
\ex
\label{ex:12-85}
\gll théeba [ǰhaníi deés]\textsubscript{\textsc{sbj}} muqarár bh-áan-u  \\
then wedding.\textsc{gen} day fixing become.\textsc{prs"=msg} \\
\glt `Then the day of the wedding is being fixed.' (A:MAR073)
\end{exe}
\begin{exe}
\ex
\label{ex:12-86}
\gll [tasíi mansubá ǰhulí]\textsubscript{\textsc{io}}
     [amál]\textsubscript{\textsc{sbj}} bhíl-i  \\
\textsc{3sg.gen} planning on obedience become.\textsc{pfv-f} \\
\glt `His plan was followed.' (A:Q9.0739)
\end{exe}
\begin{exe}
\ex
\label{ex:12-87}
\gll muloó díi ɣeér [kráam]\textsubscript{\textsc{sbj}} na bh-áan-u  \\
mullah from without work \textsc{neg} become.\textsc{prs"=msg} \\
\glt `Without a~mullah the work is not being done.' (A:MAR043)
\end{exe}

\subsubsection*{Conjunct verbs with \textit{bhe-} `become'}

The other productive means of forming conjunct verbs is with \textit{bhe-}, some examples given in \tabref{tab:12-6}, but here we do not observe the same variety of construction types and argument structures as with the \textit{the}-conjuncts, the most obvious reason being that all complexes with \textit{bhe-} are intransitive and thus bound to have fewer potential arguments. For the most part, but again not exclusively, the host element is often an~element copied from another language. 


While verbaliser"=host agreement seemed to be relatively common with \textit{the}-conjuncts, this is, as far as I have been able to observe, not found with \textit{bhe}-conjuncts outside of passive constructions derived from \textit{the}-conjuncts with host agreement exemplified above in (\ref{ex:12-85})--(\ref{ex:12-87}). 


\begin{table}[ht]
\caption{\textit{bhe}-conjuncts}
\begin{tabularx}{.75\textwidth}{ l@{\hspace{45pt}} Q }
\lsptoprule
Host + verbaliser &
\\\hline
\textit{teér bhe-} &
`pass, go by'\\
\textit{široó bhe-} &
`begin'\\
\textit{darák bhe-} &
`appear, turn up'\\
\textit{rawaán bhe-} &
`move, get going'\\
\textit{raál bhe-} &
`rise, climb'\\
\textit{ǰamá bhe-} &
`gather, assemble'\\
\textit{ašáq bhe-} &
`fall in love (with)'\\
\textit{ṭinɡ bhe-} &
`challenge, face'\\
\textit{ḍup bhe-} &
`drown'\\
\textit{óol bhe-} &
`(stand) guard'\\
\textit{čhub bhe-} &
`ride (on horse, etc.)'\\
\textit{mušqúl bhe-} &
`get on well (with)'\\
\textit{milaáu bhe-} &
`meet'\\\lspbottomrule
\end{tabularx}
\label{tab:12-6}
\end{table}


Although a~number of these conjuncts are made up of the same host element as some of the \textit{the}-conjuncts, such as \textit{teér bhe-} in (\ref{ex:12-88}) and \textit{ǰamá bhe-} in (\ref{ex:12-89}) they are not simply derived from these, as the meaning of the corresponding \textit{bhe}-conjunct is not necessarily passive, but rather they stand in the same kind of relationship to one another as any other transitive/intransitive pairs with or without a~common base segment.

\begin{exe}
\ex
\label{ex:12-88}
\gll [dees-á]\textsubscript{\textsc{sbj}} teér bhíl-a \\
day-\textsc{pl} passed become.\textsc{pfv"=mpl} \\
\glt `The days went by.' (A:SHY037)
\end{exe}
\begin{exe}
\ex
\label{ex:12-89}
\gll [buṭheé]\textsubscript{\textsc{sbj}} be ǰamá bhíl-a hín-a  \\
all go.\textsc{cv} collection become.\textsc{pfv"=mpl} be.\textsc{prs"=mpl} \\
\glt `They all went and gathered.' (A:KAT120)
\end{exe}

The historical"=etymological identity of the host element as far as part of speech is concerned is not unitary: a) \textit{teér} originates in Pashto where it is an~adjective `passed, gone by, done, accomplished, spent (as time), over, lapsed, that has been', and even there is used in a~conjunct, in some tenses morphologically incorporated into the verb (\textit{teredal}), with a~meaning very similar to the Palula conjunct; b) \textit{ǰamá} is of Perso"=Arabic origin and is listed as a~noun or verbal noun in Urdu, Pashto as well as in Persian (the three most likely donor candidates) with an~approximate meaning `collection, assembly, congregation, whole'; c) \textit{raál} is used in Palula outside of this construction as an~adverb meaning `high'. The crucial point is that it is not interpreted as an~NP, even when the origin happens to be nominal. 


As with many of the \textit{the}-conjuncts, some \textit{bhe}-conjuncts are also specified for an~argument structure involving postpositionally coded participants, such as `Machoke' in (\ref{ex:12-90}), and `he' in (\ref{ex:12-91}). Some of these indirect objects would fit the description ``human dative''. 

\begin{exe}
\ex
\label{ex:12-90}
\gll [se phaí]\textsubscript{\textsc{sbj}} [se mac̣ook-á the]\textsubscript{\textsc{io}} ašáq bhíl-i \\
\textsc{def} girl \textsc{def} Machoke-\textsc{obl} to love  become.\textsc{pfv-f} \\
\glt `The girl fell in love with Machoke.' (A:MAA005)
\end{exe}
\begin{exe}
\ex
\label{ex:12-91}
\gll óo, [ma]\textsubscript{\textsc{sbj}} [tas sanɡí]\textsubscript{\textsc{io}} milaáu bhíl-u hin-u \\
yes \textsc{1sg.nom} \textsc{3sg.acc} with meeting become.\textsc{pfv"=msg}  be.\textsc{prs"=msg} \\
\glt `Yes, I have met him.' (A:TAQ037)
\end{exe}

\subsubsection*{Conjunt verbs with \textit{de-} `give' and other verbalisers}

The situation is in many ways much more complex when it comes to conjunct verbs formed with
verbalisers other than the above mentioned \textit{the-} and \textit{bhe-}, as they tend to fade out
into the area of idiomatic expressions and it is difficult to specify any grammatical conditions for
choosing any of them instead of the transitively triggered \textit{the}-conjuncts or the
intransitively triggered \textit{bhe}-conjuncts. While DO and BE/BECOME verbalisers in IA
languages do not to any larger extent contribute to the semantics of the complexes, that is less
true when it comes to other verbalisers \citep[78--79]{gambhir1993}, ``shading off'', as
\citet[157]{masica1993} phrases it, through GIVE and TAKE into individual combinations of certain
verbs and nonverbal elements of a~mostly idiomatic character.



Nevertheless, some of the observations and categorisations outlined above for \textit{the-} and \textit{bhe-} are in some ways relevant even for conjuncts with \textit{de-} `give'. As already noted above, \textit{de-} is already as a~simple verb rather polysemous and as such is associated with a~few different valance patterns, something that further blurs the picture.


\spitzmarke{Incorporating \textit{de}-conjuncts (no host"=verbaliser agreement).} These conjuncts (examples given in \tabref{tab:12-7}) are internally complex but externally unitary, as we saw with the incorporating \textit{the}-conjuncts above. 


\begin{table}[ht]
\caption{Incorporating \textit{de}-conjuncts}
\begin{tabularx}{.75\textwidth}{ l@{\hspace{45pt}} Q }
\lsptoprule
Host + verbaliser &
\\\hline
\textit{dhreéɡ de-} &
`lie down, stretch out'\\
\textit{baás de-} &
`spend the night, stay (over night)'\\
\textit{ṭópa de-} &
`defeat, put down'\\
\textit{ḍhoó/ḍhoowá de-} &
`appear/notice'\\\lspbottomrule
\end{tabularx}
\label{tab:12-7}
\end{table}


Often no exact meaning can be assigned to the host element, and the basic meaning of \textit{de-} has been considerably bleached, the reason why the complex meaning seldom can be deduced or guessed from its individual components. The two conjuncts in \tabref{tab:12-1} are exemplified in (\ref{ex:12-92}) and (\ref{ex:12-93}).

\begin{exe}
\ex
\label{ex:12-92}
\gll tuúš muṣṭú b-íi ta, [áak luumái]\textsubscript{\textsc{sbj}} ḍhoó dít-i hín-i \\
a.little before go-\textsc{3sg} \textsc{prt} \textsc{idef} fox.[\textsc{fsg]} \textsc{host} give.\textsc{pfv-f} be.\textsc{prs-f} \\
\glt `Walking a~little forward, a~fox appeared/Walking a~little forward, (he) noticed a~fox.' (A:KAT011)
\end{exe}
\begin{exe}
\ex
\label{ex:12-93}
\gll [so]\textsubscript{\textsc{sbj}} ba [dharaṇ-í pharé]\textsubscript{\textsc{io}} dhreéɡ dít-u hín-u \\
\textsc{3msg.nom} \textsc{prt} ground-\textsc{obl} along \textsc{host}  give.\textsc{pfv"=msg} be.\textsc{prs"=msg} \\
\glt `And he lay down on the ground.' (A:UNF010)
\end{exe}

These complexes are intransitive, but other non"=nominative arguments may appear, which also opens up the possibility for analysing e.g., \textit{ḍhoó/ḍhoowá de-} as taking a~non"=nominative experiencer subject (either explicitly or implicitly), see \sectref{subsec:12-2-6}. 


\spitzmarke{Non"=incorporating \textit{de}-conjuncts (host"=verbaliser agreement).} In other \textit{de}-conjuncts (\tabref{tab:12-8}), the verbaliser does agree with the nominal host (in the perfective), and is therefore an~argument of the clause. 


\begin{table}[ht]
\caption{Non"=incorporating \textit{de}-conjuncts}
\begin{tabularx}{.75\textwidth}{ l@{\hspace{45pt}} Q }
\lspbottomrule
\textbf{Host + verbaliser} &
\\\hline
\textit{haát de-} &
`touch'\\
\textit{póo de-} &
`step on'\\
\textit{azaáb de-} &
`punish'\\
\textit{ḍaaká de-} &
`rob'\\
\textit{čóoḍ de-} &
`applaud, clap hands'\\\lspbottomrule
\end{tabularx}
\label{tab:12-8}
\end{table}


It is obvious from (\ref{ex:12-94}) and (\ref{ex:12-95}) that the host element in this case is
perceived as an~NP, even when it does not appear synchronically as a~noun in the language outside of
the particular conjunct. (Although \textit{póo} is related to the word for foot in the protolanguage, \textit{p\'{\={a}}da-} \citep[8056]{turner1966},\footnote{Compare with Palula \textit{naalpoó} `barefoot'} the word in current use with the meaning `foot' is
\textit{khur}).

\begin{exe}
\ex
\label{ex:12-94}
\gll [taním]\textsubscript{\textsc{sbj}} [šóo wée]\textsubscript{\textsc{io}} [haát]\textsubscript{\textsc{do}} na dít-i \\
\textsc{3pl.erg} vegetables in hand \textsc{neg} put.\textsc{pfv-f} \\
\glt `They didn't touch the vegetables.' (A:UXW035)
\end{exe}
\begin{exe}
\ex
\label{ex:12-95}
\gll [míi]\textsubscript{\textsc{sbj}} [áa ǰhandra-í ǰhulí]\textsubscript{\textsc{io}} [póo]\textsubscript{\textsc{do}} dít-i  \\
\textsc{1sg.gen} \textsc{idef} snake-\textsc{obl} on host put.\textsc{pfv-f} \\
\glt `I stepped on a~snake.' (A:TAQ167)
\end{exe}

These complexes are transitive with an~indirect object, although what is coded as the indirect object is semantically really the (most affected) patient. 


\spitzmarke{Extended uses of GIVE or non"=incorporating \textit{de}-conjuncts?} For one group of \textit{de}-conjuncts, which in no formal sense behaves differently than the conjuncts above, it can be argued that the basic meaning of \textit{de-} as `give' is preserved to such an~extent that it would be trivial to consider them conjuncts. But we can also argue that these constructions (where the receiver is coded with the postposition \textit{the}) happen to coincide with the basic semantics of sentences where a~simple verb \textit{de-} is used to describe how concrete objects are being handed over to someone. Examples of such conjuncts (or standard combinations of an~abstract object and GIVE) are \textit{húkum de-} `order' (\ref{ex:12-96}), \textit{daawát de-} `invite', \textit{xabaár de-} `inform' and \textit{baát de-} `tell' (\ref{ex:12-97}).

\begin{exe}
\ex
\label{ex:12-96}
\gll [alaahtaalaá]\textsubscript{\textsc{sbj}} [farišteém the]\textsubscript{\textsc{io}} [húkum]\textsubscript{\textsc{do}} dít-u \\
God angel.\textsc{pl.obl} to order  give.\textsc{pfv"=msg} \\
\glt `God ordered the angels.' (A:ABO031)
\end{exe}
\begin{exe}
\ex
\label{ex:12-97}
\gll [baát]\textsubscript{\textsc{do}} [máa=the]\textsubscript{\textsc{io}} da  \\
word \textsc{1sg.nom=}to give.\textsc{imp.sg} \\
\glt `Tell me!' (A:MAA008)
\end{exe}

Most conjuncts of this type, such as \textit{daawát de-} `invite' in (\ref{ex:12-98}), can also take a~clausal complement, thus function as PCU"=verbs (see \sectref{subsec:12-2-7}).

\begin{exe}
\ex
\label{ex:12-98}
\gll [deeúl-ii xálak"=am]\textsubscript{\textsc{sbj}} [daawát]\textsubscript{\textsc{do}} [dít-i] [atsharíit"=am the]\textsubscript{\textsc{io}} ki [muqaabilá th-íia]\textsubscript{\textsc{cpl}}  \\
Dir-\textsc{gen} people-\textsc{pl.obl} invitation give.\textsc{pfv-f}  Ashreti-\textsc{pl.obl} to \textsc{comp} contest do-\textsc{1pl} \\
\glt `The people of Dir invited the Ashretis to compete with them' (A:CHA001)
\end{exe}

\spitzmarke{Conjuncts with other verbs.} There are a~few other combinations of a~verb and a~host element that may be considered conjuncts (\tabref{tab:12-9}), some of them already pointed out as participating in non"=standard valency patterns (\sectref{subsec:12-2-6}). In particular the transitive \textit{ǰe-} `hit' used as a~verbaliser should be mentioned.


\begin{table}[ht]
\caption{Conjuncts with other verbs}
\begin{tabularx}{.75\textwidth}{ l@{\hspace{45pt}} Q }
\lspbottomrule
\textbf{Host + verbaliser} &
\\\hline
\textit{laambú ǰe-} &
`take a bath'\\
\textit{axsí ǰe-} &
`play ``akhsi'''\\
\textit{bhéenɡi ǰe-} &
`swim'\\
\textit{muṣteekíim ǰe-} &
`punch (somebody)'\\
\textit{šóo ǰháan-} &
`like, be fond of'\\
\textit{breéx nikhé-} &
`have pain (in ribs)'\\
\textit{ǰáar dhak-} &
`have a fever'\\
\textit{audás phooṭá-} &
`urinate'\\
\textit{qalahúr ɡhaṇḍé-} &
`target'\\
\textit{níindram be-} &
`fall aspleep'\\
\textit{níindra yhe-} &
`feel sleepy'\\
\textit{bhíili ṣač-} &
`feel fear'\\\lspbottomrule
\end{tabularx}
\label{tab:12-9}
\end{table}


Most, or maybe all, such conjuncts, are of the non"=incorporating (host"=agreeing) type, as can be seen in (\ref{ex:12-99}) and (\ref{ex:12-100}). 

\begin{exe}
\ex
\label{ex:12-99}
\gll haṛíiwee [peereṇíi-a]\textsubscript{\textsc{sbj}} har deés [laambú]\textsubscript{\textsc{do}} ǰ-éen de \\
in.there fairy-\textsc{pl} every day bath hit-\textsc{3pl} \textsc{pst} \\
\glt `Every day the fairies used to take a~bath in it.' (B:FOR029)
\end{exe}
\begin{exe}
\ex
\label{ex:12-100}
\gll [míiš-a]\textsubscript{\textsc{sbj}} ba [axsí]\textsubscript{\textsc{do}} ǰ-íin  \\
man-\textsc{pl} \textsc{prt} ``akhsi'' hit-\textsc{3pl} \\
\glt `The men are playing ``akhsi''.' (A:JAN035)
\end{exe}

However, this again borders on the idiomatic, and these possible conjuncts are in any case less central and productive than the ones formed with \textit{the-} and \textit{bhe-}.