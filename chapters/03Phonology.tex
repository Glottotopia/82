\chapter{Phonology}
\label{chap:3}

\section{Consonants}
\label{sec:3-1}

\subsection{Consonant inventory}

\authcomm[inline]{I added the above subheading as agreed with Martin Haspelmath (for reasons of
  consistency, cf. \sectref{sec:3-1} on consonants). Are you ok with this?}

\begin{table}[ht]
\caption{Inventory of consonants (IPA). Marginal phonemes and combinations with aspiration within parentheses; vl.=voiceless, vd.=voiced}
\begin{tabularx}{\textwidth}{ P{20mm} Q Q Q Q Q Q Q Q }
\lsptoprule
&
&
La\-bi\-al &
Den\-tal &
Retro\-flex &
Pa\-la\-tal &
Ve\-lar &
Post\-velar &
Glot\-tal \\\hline
Plo\-sive &
vl. &
\textbf{p} &
\textbf{t} &
\textbf{ʈ} &
&
\textbf{k} &
\textbf{(q)} &
\\
&
&
(pʰ) &
(tʰ) &
(ʈʰ) &
&
(kʰ) &
&
\\
&
vd. &
\textbf{b} &
\textbf{d} &
\textbf{ɖ} &
&
\textbf{ɡ} &
&
\\
&
&
(bʰ) &
(dʰ) &
(ɖʰ) &
&
(ɡʰ) &
&
\\
Affricate &
vl. &
&
\textbf{ʦ} &
\textbf{ʈʂ} &
\textbf{ʨ} &
&
&
\\
&
&
&
(ʦʰ) &
(ʈʂʰ) &
(ʨʰ) &
&
&
\\
Fricative &
vl. &
\textbf{(f)} &
\textbf{s} &
\textbf{ʂ} &
\textbf{ɕ} &
\textbf{x} &
&
\textbf{h}\\
&
vd. &
&
\textbf{z} &
\textbf{(ʐ)} &
\textbf{ʑ} &
\textbf{ɣ} &
&
\\
&
&
&
(zʰ) &
(ʐʰ) &
(ʑʰ) &
&
&
\\
Nasal &
&
\textbf{m} &
\textbf{n} &
\textbf{ɳ} &
&
&
&
\\
&
&
(mʰ) &
(nʰ) &
&
&
&
&
\\
Flap &
&
&
\textbf{r} &
\textbf{ɽ} &
&
&
&
\\
&
&
&
(rʰ) &
&
&
&
&
\\
Lateral approximant &
&
&
\textbf{l} &
&
&
&
&
\\
&
&
&
(lʰ) &
&
&
&
&
\\
Approximant &
&
\textbf{w} &
&
&
\textbf{j} &
&
&
\\
&
&
(wʰ) &
&
&
(jʰ) &
&
&
\\\lspbottomrule
\end{tabularx}
\label{tab:3-1}
\end{table}

The consonant inventory is rather symmetrical, with the dental and retroflex places of articulation displaying the most well"=developed system of manner contrasts. The ancient (OIA) contrast between the three sibilants /s ʂ ɕ/ is preserved \citep[442]{cardona1987}, with the present voicing contrast probably not evolving until quite recently, partly through lenition of voiced affricates, partly through foreign loans. 

While the plosive and fricative sets show a~contrast in voicing nearly throughout, voiced counterparts are missing in the small affricate set. Mirroring that is a~general allophonic variation (see below) between voiced fricative and affricate pronunciations. The voiced palatal fricative could equally well be treated as an~affricate, as that is the more common allophone (especially in the A. dialect), but to provide more symmetry to the system, I have chosen to include it among the voiced fricatives,\footnote{As pointed out by \citet[34]{zoller2005}, this particular asymmetry within the affricate/fricative sets is a~feature shared by a~number of languages of northern Pakistan, ``due to an [sic] lenition process which is more advanced in case of the voiced phonemes than in case of the voiceless phonemes''.} while in the common transcription it is represented as \textit{ǰ}.


The post"=velar (or uvular) place of articulation is represented by a~voiceless post"=velar or uvular plosive /q/ alone. This marginal phoneme is only pronounced distinctly post"=velar by some educated speakers~-- and even then rather inconsistently~-- when occurring in loanwords of mainly Perso"=Arabic origin. In many speakers' pronunciation, however, it normally tends to approximate a~velar fricative pronunciation [x], thus not contrasting with the phoneme /x/. The fricatives /z, x, ɣ/ are rather frequent in present"=day Palula, and many of the words probably have a~long history in the language, although they to a~large extent are found in vocabulary borrowed from languages in the immediate region, and to a~much lesser extent are found in inherited vocabulary affected by phonological processes. A labio"=dental [f] is sometimes heard in more recent loans, primarily from Urdu and English, but with many speakers it alternates freely with or is entirely replaced by the native voiceless plosive [pʰ], especially when occurring with aspiration, hence /f/ is considered a~marginal phoneme. 


The voiced retroflex fricative /ʐ/ is also a~marginal phoneme, but it is included for comparative reasons; an~even more rarely occurring voiced retroflex affricate sound [ɖʐ] is tentatively analysed as an~allophone of the same phoneme. 


There is insufficient proof to regard a~velar nasal [ŋ] as a~phoneme independent from /n/, as it only occurs before /k/ and /ɡ/, or as a~variant pronunciation of /nɡ/: [ŋɡ]$\sim$[ŋ]. 


Although contrastive, aspiration is not considered a~feature of individual consonant segments in this description (as indicated by the parenthetical treatment in \tabref{tab:3-1}) and will be discussed in \sectref{subsec:3-4-1}.

\subsection{Distribution and variation}
\label{subsec:3-1-1}

Examples of the distribution of consonants are shown in \tabref{tab:3-2}. 


The retroflex consonants are in some descriptions called retracted \citep[16]{schmidtkohistani2008} or cerebrals \citep{morgenstierne1941}; it has been questioned whether these consonants in HKIA languages are retroflex in the same sense or to the same extent as in the main NIA languages or in Dravidian languages. 


\begin{sidewaystable}[p!]
\caption{The distribution of consonants: word"=initial, medial, and final.{\protect\footnotemark}
  The occurrences within parentheses are matters of interpretation (see \sectref{subsec:3-2-3},
  \sectref{subsec:3-4-1})}
\begin{tabularx}{\textwidth}{ l@{\hspace{20pt}} l@{\hspace{20pt}} Q l@{\hspace{20pt}} Q Q Q Q }
\lsptoprule
/p/ &
&
/piːli/ &
`drank (\textsc{f)}' &
/ɕopu/ &
`navel' &
/ʈip/ &
`drop'\\
/b/ &
&
/biːɖi/ &
`many (\textsc{f)}' &
/ʑabal/ &
`iron bar' &
/ɖaːb/ &
`plain'\\
/f/ &
&
/fasil/ &
`crop' &
/xafa/ &
`upset' &
/muaːf/ &
`excuse'\\
/m/ &
&
/miːɕa/ &
`men' &
/hiːmaːl/ &
`glacier' &
/braːm/ &
`joint'\\
/w/ &
&
/wiːwaj/ &
`wife's brother' &
/heːwaːnd/ &
`winter' &
(/ɡʰaːw/ &
`cow')\\
/r/ &
&
/reːti/ &
`nights' &
/beːriʂ/ &
`summer' &
/anɡoːr/ &
`fire'\\
/l/ &
&
/leːdi/ &
`found (\textsc{f)}' &
/baliː/ &
`roof end' &
/ʨʰaːl/ &
`goat kid'\\
/t/ &
&
/teːti/ &
`hot (\textsc{f)}' &
/pʰutu/ &
`fly' &
/baːt/ &
`word'\\
/d/ &
&
/deːdi/ &
`father's mother' &
/leːdi/ &
`found (\textsc{f)}' &
/ɕid/ &
`coldness'\\
/n/ &
&
/neːɽi/ &
`stream bed' &
/ʑaːnu/ &
`person' &
/soːn/ &
`pasture'\\
/s/ &
&
/seːti/ &
`looked after (\textsc{cv)}' &
/buːsi/ &
`kiss' &
/deːs/ &
`day'\\
/z/ &
&
/zeːri/ &
`supplication' &
/baːzoːr/ &
`bazaar' &
/anɡreːz/ &
`Brit'\\
/ʦ/ &
\textit{(ts)} &
/ʦiːpi/ &
`squeezed (\textsc{cv)}' &
/buʦu/ &
`stick' &
/uʦ/ &
`spring'\\
/ɕ/ &
\textit{(š)} &
/ɕeːmi/ &
`spleen' &
/huːɕi/ &
`wind' &
/diːɕ/ &
`village'\\
/ʨ/ &
\textit{(č)} &
/ʨeːri/ &
`spouted jug' &
/kuʨuru/ &
`dog' &
/baːlbaʨ/ &
`child'\\\lspbottomrule
\end{tabularx}
\end{sidewaystable}

\addtocounter{table}{-1}
\begin{sidewaystable}[p!]
\caption{The distribution of consonants: word"=initial, medial, and final. The
  occurrences within parentheses are matters of interpretation (see \sectref{subsec:3-2-3}, \sectref{subsec:3-4-1}). (continued)}
\begin{tabularx}{\textwidth}{ l@{\hspace{20pt}} l@{\hspace{20pt}} Q l@{\hspace{20pt}} Q Q Q Q }
\lsptoprule
/ʑ/ &
\textit{(ǰ)} &
/ʑeːli/ &
`bore (\textsc{f)}' &
/beːʑi/ &
`heifer' &
/raːʑ/ &
`rope'\\
/j/ &
\textit{(y)} &
/jiːɽi/ &
`sheep' &
/lʰaːja/ &
`will find' &
(/babaːj/ &
`apple')\\
/ʈ/ &
\textit{(ṭ)} &
/ʈaːka/ &
`call!' &
/beːʈi/ &
`lamb' &
/baːʈ/ &
`stone'\\
/ɖ/ &
\textit{(ḍ)} &
/ɖaːka/ &
`robbery' &
/ɡeːɖi/ &
`big (\textsc{f)}' &
/haːɖ/ &
`bone'\\
/ɳ/ &
\textit{(ṇ)} &
-- &
&
/deːɳi/ &
`calf (of leg)' &
/bʰeːɳ/ &
`sister'\\
/ɽ/ &
\textit{(ṛ)} &
-- &
&
/deːɽi/ &
`beard' &
/kiroːɽ/ &
`chest'\\
/ʂ/ &
\textit{(ṣ)} &
/ʂeːti/ &
`disputed (\textsc{f)}' &
/kʰaʂiː/ &
`hoe' &
/baːʂ/ &
`rain'\\
/ʐ/ &
\textit{(ẓ)} &
/ʐami/ &
`sister's husband' &
/ʈʂaɳʐa/ &
`torch' &
/riːʐ/ &
`track'\\
/ʈʂ / &
\textit{(c̣)} &
/ʈʂiːnki/ &
`twittered (\textsc{cv)}' &
/teːʈʂi/ &
`wood chisel' &
/dʰraːʈʂ/ &
`grape'\\
/k/ &
&
/kati/ &
`how many?' &
/bakaːra/ &
`flock' &
/ɖoːk/ &
`back'\\
/ɡ/ &
&
/ɡaɖi/ &
`taken out (\textsc{cv)}' &
/siɡal/ &
`sand' &
/pʰaːɡ/ &
`fig'\\
/x/ &
&
/xati/ &
`letters' &
/maːxaːm/ &
`evening' &
/mux/ &
`face'\\
/q/ &
&
/qisa/ &
`story' &
/alaːqa/ &
`area' &
/aɕaq/ &
`love'\\
/ɣ/ &
&
/ɣeːri/ &
`caves' &
/kaːɣaːz/ &
`paper' &
/baːɣ/ &
`garden'\\
/h/ &
&
/hari/ &
`removed \textsc{(cv)}' &
(/kuhiː/ &
`well') &
-- &
\\\lspbottomrule
\end{tabularx}
\label{tab:3-2}
\end{sidewaystable}

\footnotetext{In Palula common transcription: \textit{píili, šópu, ṭip, bíiḍi, ǰabál,
    ḍáab, fásil, xafá, muaáf, míiša, hiimaál, bráam, wíiwai,
    heewaánd, ɡhaáu, reetí, béeriṣ, anɡóor, léedi, balíi, čhaál,
    téeti, phútu, baát, déedi, léedi, šid, néeṛi, ǰáanu, sóon,
    seetí, búusi, deés, zeerí, baazóor, anɡreéz, tsiipí, bútsu, uts,
    šéemi, húuši, díiš, čéeri, kučúru, baalbáč, ǰéeli,
    béeǰi, ráaǰ, yíiṛi, lháaya, babaái, ṭaaká, beeṭí, báaṭ,
    ḍaaká, ɡéeḍi, haáḍ, déeṇi, bheéṇ, déeṛi, kiroóṛ, ṣéeti,
    khaṣíi, báaṣ, ẓamí, ẓaṇẓá, ríiẓ, c̣iinkí, téec̣i, dhraác̣,
    katí, bakáara, ḍóok, ɡaḍí, síɡal, phaáɡ, xatí, maaxaám, mux,
    qisá, alaaqá, ašáq, ɣeerí, kaaɣaáz, baáɣ, harí, kuhíi}.}



I am presently in no position to determine the exact nature of retroflexion in Palula, but I prefer, nevertheless, to retain the term, as the most prominent feature in the pronunciation of the ``retroflex'' consonants is the articulation with the tip of the tongue against a~place at the rear end of the alveolar ridge and usually with the tongue slightly curled back. The dental consonants on the other hand are indeed dental, often articulated against the lower as well as the upper teeth. Generally the area of contact between the tongue and the place of articulation is larger than in the case of the retroflex consonants.


The palatal consonants can also be described as alveolo"=palatal, with the blade of the tongue against the area covering the rear part of the alveolar and the front part of the palate, and with the tip of the tongue behind the lower teeth. 

\subsubsection*{Plosives}
/p/: [p], /b/: [b], /t/: [t̪], /d/: [d̪], /ʈ/: [ʈ]$\sim$[ṯ], /ɖ/: [ɖ]$\sim$[ḏ], /k/: [k], /ɡ/: [ɡ], (/q/: [q]$\sim$[x]).

With respect to frequency, the voiceless plosives can be considered the unmarked subset, occurring almost twice as often as their voiced counterparts. The voiced plosives do not commonly occur word"=finally, and when they do they tend to be devoiced. 


Intervocalically, the voiced plosives are often fricativised, and occur with aspiration (see \sectref{subsec:3-4-1}). Also, the voiceless plosives tend to show lenition, e.g. /pʰ/ with an~alternating pronunciation [f]$\sim$[ɸ]$\sim$[pʰ], as in /pʰeːrimaː/ \textit{(pheerimaá)} `Ferima (\textsc{place name)}'. 


The phonemic status of [q] was already commented on above.


\subsubsection*{Affricates}

/ʦ/: [ts]$\sim$[s], /ʈʂ/: [ʈʂ]$\sim$[ʂ], /ʨ/: [tɕ].


Affricates occur at three places of articulation, but with respect to frequency the dentals are quite limited compared to the other two. The explanation of the missing voicing contrast is partly explainable (as already commented on above) by the overlap or neutralisation between the affricate and fricative sets. 


There is also a~less consistent neutralisation of the contrast between aspirated dental (\ref{ex:3-1}) and retroflex voiceless (\ref{ex:3-2}) affricates and their fricative counterparts (but as far as I have been able to observe, never between aspirated voiceless palatal affricates and fricatives), apparently limited to certain lexical items. 

\begin{exe}
\ex
\label{ex:3-1}
\gll /aʦʰareːt/: [aʦʰaɾěːt]$\sim$[asaɾěːt] \\
(atshareét) \\
\glt `Ashret'

\ex
\label{ex:3-2}
\gll /aːʂaːṛ/: [aːʂǎːɽ]$\sim$[aːʈʂʰǎːɽ] \\
(aaṣaáṛ) \\
\glt `apricot'
\end{exe}

\subsubsection*{Fricatives}

(/f/: [f]$\sim$[pʰ]), /s/: [s], /z/: [z], /ʂ/: [ʂ], /ʐ/: [ʐ]$\sim$[ɖʐ], /ɕ/: [ɕ], /ʑ/: [dʑ]$\sim$[ʑ], /x/: [x], /ɣ/: [ɣ], /h/: [ɦ]$\sim$[h].


As already pointed out in connection with the affricates, there is a~close link between the affricate set and the fricative set, with some overlaps and neutralisations taking place between them. The voiced palatal fricative is alternatively realised as [ʑ] and [dʑ] (more often with an~affricate pronunciation in A.) and the voiced retroflex as [ʐ] and [ɖʐ], whereas /z/ seems to occur consistently as [z] and never with an~affricate pronunciation.


The marginal phoneme /f/ is often realised as [pʰ], thus neutralised with /pʰ/, as in /faːjda/ \textit{(faaidá)} `benefit': [pʰaːjdá]$\sim$[faːjdá].


The voiced retroflex fricative is extremely rare, occurring only in a~few words, with [ɖʐ] most likely an~allophone of it in /ʐʰaɳʐiːr/ (ẓ\textit{haṇẓíir)} `chain': [ɖʐʰaɳɖʐ îːɾ]. 


There is a~strong affinity between /h/, almost exclusively occurring word initially, and aspiration (\sectref{subsec:3-4-1}), /ʰ/ (hence the same representation in the Palula common transcription), especially that which is often described as voiced aspiration or breathiness. Historical occurrences of word medial /h/ through movement to syllable onsets have often been reinterpreted as voiced aspiration. In the present language, it only rarely occurs intervocalically, and even then often with an~interpretational ambivalence: [ɾʱaiː] \textit{(rhayíi)} `footprints': /rʰajiː/ or /rahiː/.


\subsubsection*{Nasals}

/m/: [m], /n/: [n̪]$\sim${}[ɲ]$\sim$[ŋ], /ɳ/: [ɳ].


Phonetically there are at least five places of articulation attested for nasals: labial, dental, retroflex, palatal and velar. The palatal nasal, however, is analysed as deriving from a~sequence of /n/ + a~palatal consonant, as it never occurs in any other environment. The same analysis may be applied to the velar nasal, where the sequence /n/ + a~velar stop usually is the likely source.


The case is a~little more complicated with the retroflex nasal, /ɳ/. Although it is clear in some cases, that retroflexion is the result of assimilation with an~adjacent retroflex consonant, this cannot always be concluded. Whereas a~retroflex nasal normally does not occur word initially (although the word /ɳiɳeː/ \textit{(ṇiṇeé)} `popcorn' can be cited as an~isolated example), it contrasts intervocalically, cf. \textit{(déeni)} `is giving' and \textit{(déeṇi)} `calf (of the leg)' in (\ref{ex:3-3}) (and word"=finally \textit{(kan)} `shoulder' vs. \textit{(kaṇ)} `ear' in B. (\ref{ex:3-4})), with dental /n/. 

\begin{exe}
\ex
\label{ex:3-3}
/deːni/~-- /deːɳi/

\ex
\label{ex:3-4}
/kan/~-- /kaɳ/
\end{exe}

The labial and the dental nasals are very frequent in the language, as these two segments are part of some of the most productive inflectional forms in the language.

\subsubsection*{Flaps}

/r/: [ɾ], /ɽ/: [ɽ].


While /r/ commonly occurs word"=initially, intervocalically, and word"=finally, the occurrence of /ɽ/
is more restricted. In B. it does not occur word"=initially at all, whereas in A. it occurs in free
variation with /l/ in ``weak'' forms of a~series of demonstratives (\ref{ex:3-5}), \textit{(lo)} or
\textit{(ṛo)} `he, that', etc., related to ``strong'' forms of the same series with an~intervocalic
/ɽ/, \textit{(eeṛó)} `he, that', etc., but otherwise not.

\begin{exe}
\ex
\label{ex:3-5}
/lo/$\sim${}/ɽo/ \\
`he, that' 
\end{exe}

\subsubsection*{Lateral approximant}

/l/: [l]($\sim$[l] B.).


Preceded by a~back vowel /a aː o oː u uː/, /l/ is being velarised, but only markedly so in the B. variety: cf. non"=velarised \textit{(khéeli)} `ate \textsc{f}' and velarised \textit{(khúulu)} `ate \textsc{msg}'.

\begin{exe}
\ex
\label{ex:3-6}
/kʰeːli/ [kʰêːli] -- /kʰuːlu/ [kʰûːlu] 
\end{exe}

\subsubsection*{Approximants}

/w/: [β̞]$\sim$[ʋ], /y/: [j].


In the speech of my main A. consultant, the front"=most approximant /w/ is usually pronounced bilabially [β̞], but with many speakers this phoneme seems to alternate between a~bilabial and something close to a~labiodental [ʋ] pronunciation.


The two approximants are sometimes challenging in terms of interpretation, and are in various ways susceptible to articulatory fluctuation or variation, especially when occurring intervocalically, an~issue that will be further discussed in connection with vowels.


\section{Vowels}
\label{sec:3-2}

\subsection{Vowel inventory}

\authcomm{I added the above subheading as agreed with Martin Haspelmath (for reasons of
  consistency, cf. \sectref{sec:3-1} on consonants). Are you ok with this?}\authcomm{That should be fine, yes}For the vowels, there are five contrasting places of articulation, as can be seen in \tabref{tab:3-3}: a) close front, b) close back, c) open front, d) open back rounded, and e) open back unrounded. Together with phonemic length contrasts there is a~ten"=vowel system. A convincing and consistent contrast (as the one shown for Gilgiti Shina, \citealt[19]{radloff1999} between oral and nasalised vowels has not been found. Instead, nasalisation seems to be a~marginal suprasegmental feature of a~limited number of lexemes. Apart from those, nasalisation is a~non"=contrastive phonetic property of vowels occurring adjacent to a~nasal consonant. 



\begin{table}[ht]
\caption{Inventory of vowels, with IPA symbols}
\begin{tabularx}{\textwidth}{ Q Q Q Q Q }
\lsptoprule
&
&
Front
&
Back unrounded &
Back rounded \\\hline
Close &
short &
i &
&
u\\
&
long &
iː &
&
uː\\
Open &
short &
e &
a &
o\\
&
long &
eː &
aː &
oː\\\lspbottomrule
\end{tabularx}
\label{tab:3-3}
\end{table}

\subsection{Distribution and variation}
\label{subsec:3-2-1}

\tabref{tab:3-4} exemplifies target articulations of the vowels, all of which take on more centralised qualities in natural and connected speech. Generally, the short vowels /i/, /a/, and /u/ tend to be pronounced as less peripheral than their long counterparts. The short /i/ is not necessarily more open than the long /iː/, but it has a~rather more central pronunciation; the short /u/, on the other hand, is both more open and slightly more central than the long /uː/; the short /a/ is also slightly less open and more fronted than the long /aː/. 


\footnotetext{In Palula common transcription, the words are: \textit{ɡir, ɡiír, ṭíki, tíiṇi, preṣ, keéṇ, ṭéka, teeká, šak, káaṇ, ṭáka, ṭaaká, sum, kúuṇ, thúki, thúuṇi, khoṇḍ, kóoṇ, tróki, ṭooká}..}


\begin{table}[ht]
\caption{Vowel contrasts exemplified (see \sectref{subsec:3-4-3} for details on pitch accent){\protect\footnotemark}}
\begin{tabularx}{\textwidth}{ l l l@{\hspace{40pt}} l l l Q }
 \lsptoprule
/i/ &
/ɡir/ &
`turn around!' &
/iː/ &
\textit{(ii)} &
/ɡǐːr/ &
`saw'\\
&
/ʈíki/ &
`bread' &
&
&
/tî:ɳi/ &
`sharp'\\
/e/ &
/preʂ/ &
`mother"=in"=law' &
/eː/ &
\textit{(ee)} &
/kěːɳ/ &
`cave'\\
&
/ʈéka/ &
`peaks' &
&
&
/ʈeːká/ &
`labour'\\
/a/ &
/ɕak/ &
`doubt' &
/aː/ &
\textit{(aa)} &
/kâːɳ/ &
`ear'\\
&
/ʈáka/ &
`insult' &
&
&
/ʈaaká/ &
`call!' \\
/u/ &
/sum/ &
`dry mud' &
/uː/ &
\textit{(uu)} &
/kûːɳ/ &
`corner'\\
&
/tʰúki/ &
`spittle' &
&
&
/tʰûːɳi/ &
`pillar'\\
/o/ &
/kʰoṇḍ/ &
`speak!' &
/oː/ &
\textit{(oo)} &
/kôːɳ/ &
`arrow'\\
&
/tróki/ &
`worn, thin' &
&
&
/ʈoːká/ &
`push!' \\\lspbottomrule
\end{tabularx}
\label{tab:3-4}
\end{table}


Phonetically, there is a~significant difference between short and long vowels. The duration of a~long vowel like /aː/ as compared to its short counterpart /a/ is not just slightly longer but usually at least twice the duration.


Environment as well as accent (see \sectref{subsec:3-4-3}) further influences the exact pronunciation of each of the ten vowels. Under certain conditions, some neutralisations take place (see next section). 


As pointed out already by \citet[58]{morgenstierne1932}, the most important~-- if not all~-- phonological dialect differences between A. and B. concern the vowels rather than the consonants. 

\subsection{Vowel neutralisation}
\label{subsec:3-2-2}

While there is a~consistent contrast between all the five vowel qualities as well as a~contrast in length, when the vowels are accented (see Section \sectref{subsec:3-4-3} for details on accent), these contrasts are fewer and less convincing than when the vowels are unaccented. The two main dialects also show some differences in this regard. 


Whereas B. maintains the /a/ vs. /e/ contrast~-- as is clearly evidenced in the morphological contrast between the general oblique inflection \textit{-a} and the genitive singular \textit{-e} of many masculine nouns~-- this is probably not the case with unaccented short vowels in A., where the unaccented genitive ending instead is /i/, and the contrast between /e/ and /a/ seems to be neutralised. Further, the variable (but grammatically identical) masculine endings [o] and [u] in A. seem to correspond to a~more non"=variable [u] in B. 


Generally speaking, the qualitative contrasts are maintained between the long vowels, whereas evidence for contrast between some of the short unaccented vowels is lacking. Therefore, as far as A. is concerned, all instances of unaccented [u] and [o] are consistently transcribed as \textit{u} in the Palula common transcription, and all instances of unaccented [a] and [e] are rendered as \textit{a}.


\subsection{The status of diphthongs}
\label{subsec:3-2-3}

A complex issue still needing more careful study to be finally resolved concerns the interpretation and representation of ambiguous vowel sequences. However, for the time being there is no strong evidence for stipulating any phonemic diphthongs with a~status comparable to that of the ten vowels already being introduced.

The sequences of vowels we find in lexical stems all consist of at least one close vowel (also interpretable as an~approximant), such as [ai], [aːi], [ui], [oːi], [oi], [eːi], [ueː], [uaː], [iaː], [ioː], [iːu], [aːu], [au]. Probably most, if not all, combinations of a~short close vowel and another long or short vowel are possible. Some examples are given in (\ref{ex:3-7}).


\begin{exe}
\extab
\label{ex:3-7}
\begin{tabularx}{\textwidth}{ l l l l }
&
[bʰɾaːdʑai] &
\textit{(bhraaǰái)} &
`sister"=in"=law'\\
&
[baba:i] &
\textit{(babaái)} &
`apple'\\
&
[dʑabui] &
\textit{(ǰabúi)} &
`velum'\\
&
[bʰoːi] &
\textit{(bhoói)} &
`daughter"=in"=law'\\
&
[lʰoilo] &
\textit{(lhóilu)} &
`red'\\
&
[jeːi] &
\textit{(yéei)} &
`mother'\\
&
[nʰiaːɽa] &
\textit{(nhiáaṛa)} &
`near'\\
&
[pʰioːɽ] &
\textit{(phióoṛ)} &
`side (of an~animal)' \\
&
[bʰiuːɽi] &
\textit{(bhiúuṛi)} &
`Biori'\\
&
[ɡʰa:u] &
\textit{(ɡhaáu)} &
`cow'\\
&
[maɳɖau] &
\textit{(maṇḍáu)} &
`veranda'\\
\end{tabularx}
\end{exe}


Taking a~number of factors into account, such as mother"=tongue speakers' counting (``knocking'') syllables, the apparent absence of sequences not including any of the two close vowels [i] and [u], and the evidence for approximants occurring word"=initially as well as intervocalically (and therefore if not being interpreted consonantally leaving a~gap word"=finally), would favour an~approximant interpretation, which would render the following phonemic output: /bʰraːʑáj/, /babǎːj/, /ʑabúj/, /bʰ\v{o}ːj/, /lʰójlu/, /jêːj/, /nʰjâːɽa/, /pʰjôːɽ/, /bʰjûːɽi/, /ɡʰǎːw/, /maɳɖáw/.


However, in a~number of words with a~vowel + [i] sequence, the final [i] can be considered a~feminine gender suffix (in some cases derived by that suffix, at least diachronically, from a~masculine stem), and in the morphological behaviour of monomorphemic stems, such as those exemplified above, it is an~advantage to show that there is an~underlying vowel /i/ or /u/ (rather than a~consonant) involved. Therefore I have chosen to represent them as \textit{bhraaǰái}, \textit{yéei}, etc., in Palula common transcription, to signal precisely the connection between a~stem and its derivations or inflected forms.\footnote{Although not attempted here, an~alternative analysis of [ái] would be to consider it an~allophone of first"=mora accented /ée/.} 


The latter representation makes even more sense for sequences in polymorphemic words, such as \textit{dhióomii} `of the daughters' from \textit{dhií} + \textit{-óom} (\textsc{ob.pl)} + \textit{-ii} (\textsc{gn)}, although the ``surface phonemic'' representation would be /dʰjoːmiː/, the latter taking \textit{de facto} syllabification into account at the expense of morphemic transparency. This holds for inflected forms of verbs as well: A purely phonemic representation such as /swâːnu/ `is sleeping \textsc{msg}' obscures the fact that we have the verbal stem \textit{só-} `sleep' inflected for Present tense \textit{-áan}, and therefore a~Palula common transcription \textit{suáanu} has been chosen for it.


When, on the other hand, there is a~need to show that there indeed is a~syllable break between two successive vowels, whether the word is mono- or polymorphemic, an~approximant, \textit{y} or \textit{w}, is shown as inserted: \textit{khilayí} /kʰilají/ `alone', \textit{bhooyóomii} /bʰoːjôːmiː/ `of the daughters"=in"=law', and \textit{bharíiwa} /bʰariːwa/ `husbands'.


\section{Phonotactics}
\label{sec:3-3}

\subsection{Syllable structure}
\label{subsec:3-3-1}


A typical syllable in Palula is an~open syllable consisting of a~consonant and a~vowel. This is the most common type when the syllable is unaccented. Long, as well as short vowels (\ref{ex:3-8}), could constitute the nucleus of such a~syllable: CV, CVV. There are monosyllabic words (/be/, /wiː/) which conform to this basic CV pattern, but most words are polysyllabic, consisting of two or more CV (or CVV) syllables (such as /ɡuː.li/ and /ku.ɳaː.koːmiː/).


\begin{exe}
\extab
\label{ex:3-8}
\begin{tabular}{ l l l }
/be/ &
\textit{(be)} &
`we'\\
/wiː/ &
\textit{(wíi)} &
`water'\\
/ɡuː.li/ &
\textit{(ɡúuli)} &
`bread'\\
/ku.ɳaː.koːmiː/ &
\textit{(kuṇaakóomii)} &
`of the children'\\
\end{tabular}
\end{exe}



The closed"=syllable pattern, CV(V)C, is also a~very common syllable, see examples in (\ref{ex:3-9}), and the most common one in accented syllables. This type occurs in monosyllabic as well as in polysyllabic words. Commonly, however, a~word is made up of a~combination of open and closed syllables. 


\begin{exe}
\extab
\label{ex:3-9}
\begin{tabular}{ l l l }
/pil/ &
\textit{(pil)} &
`drink!' \\
/ɕiːn/ &
\textit{(\v{s}íin)} &
`bed'\\
/ʈiːn.ʨuk/ &
\textit{(ṭíinčuk)} &
`scorpion'\\
/bʰan.ʑuːm/ &
\textit{(bhanǰúum)} &
`I will beat'\\
/ʈom.bu/ &
\textit{(ṭómbu)} &
`stem'\\
/piɳ.ɖuː.ru/ &
\textit{(piṇḍúuru)} &
`round'\\
/heː.wan.da/ &
\textit{(heewandá)} &
`winter (\textsc{ob)}'\\
\end{tabular}
\end{exe}


Even onsetless syllables, V(C) or VV(C), occur in Palula (\ref{ex:3-10}), though less frequently. That means that both the onset and the coda is optional, i.e. a~vowel nucleus can occur alone or at least word"=initially. Whether this is also possible word"=medially or word"=finally is an~interpretational issue, but in any case, there are no single phonological words consisting of only a~vowel nucleus.\footnote{An alternative analysis not attempted in this work would be to regard a~glottal stop as a~consonant phoneme preceding all vowels that are here considered word"=initial, thus doing away with onsetless syllables altogether.}


\begin{exe}
\extab
\label{ex:3-10}
\begin{tabular}{ l l l }
/u.ɽi/ &
\textit{(uṛí)} &
`pour!' \\
/ux/ &
\textit{(ux)} &
`camel'\\
/oː.ɖʰoːl/\ \ &
\textit{(ooḍhóol)} &
`flood'\\
\end{tabular}
\end{exe}


The minimal word can therefore be defined as consisting of a~vowel nucleus plus either an~onset or a~coda consonant. There seems also to be further constraints on words belonging to the major open classes as opposed to words from closed classes when it comes to minimal words. Nouns, adjectives and verbs (except for Imperative forms and a~few participle forms) must consist of at least a~short vowel plus a~coda, or an~onset plus a~long vowel. Pronouns, on the other hand, may very well consist of only a~short vowel with an~onset: /ma/ `I', /be/ `we', etc.


\subsection{Consonant clusters}
\label{subsec:3-3-2}


The preservation of a~number of clusters, especially some that occur word finally, sets Palula off as more conservative than most other Shina varieties. 


If characterising aspiration as a~feature of units larger than segments (as we have done so far and will continue to do), there is a~maximum onset of two consonants in Palula words, as can be seen in (\ref{ex:3-11}). The second member of such a~cluster is either /r/ or an~approximant (if we go with the analysis presented above, \sectref{subsec:3-2-3}), usually, but not exclusively, preceded by a~plosive. 


An interesting reflection is that the clusters with voiced plosives and + /r/ always co"=occur with aspiration, whereas voiceless aspirated plosives in such clusters are rare indeed, the verb /pʰrajaːnu/ \textit{(phrayé-)} `send' being the only example found so far in my data. 


Some clusters with approximants have evolved through metathesis, especially in the A. dialect: /ɡʰweːɳiː/ \textit{(ɡhueeṇíi)} `Pashtoon' {\textless} /uɡheːɳiː/ (which is still the form heard in the B. dialect). Others seem to be in the process of evolving, evidenced by parallel forms: /ukʰaːndu/$\sim$/kʰwaːndu/ \textit{(ukháandu)} `is coming/going up \textsc{msg}', perhaps pointing to a~preference for Cw and Cj clusters vis-à-vis word initial V-syllables.


\begin{exe}
\extab
\label{ex:3-11}
\begin{tabular}{ l l l l }
/pr/ &
/praːʨu/ &
\textit{(práaču)} &
`guest'\\
/pʰr/ &
/pʰrajaːnu/ &
\textit{(phrayáanu)} &
`is sending (\textsc{msg)}'\\
/bʰr/ &
/bʰroː/ &
\textit{(bhroó)} &
`brother'\\
/tr/ &
/troki/ &
\textit{(tróki)} &
`thin, weak (\textsc{f)}'\\
/dʰr/ &
/dʰruːk/ &
\textit{(dhrúuk)} &
`gorge, stream' \\
/kr/ &
/kraːm/ &
\textit{(kráam)} &
`work' \\
/ɡʰr/ &
/ɡʰreːɳɖ/ &
\textit{(ɡhreéṇḍ)} &
`knot' \\
/mr/ &
/mrinɡa/ &
\textit{(mrínɡa)} &
`deer'\\
/nj/ &
/njaːʈa/ &
\textit{(niaaṭá)} &
`shave!, shear!' \\
/nʰj/ &
/nʰjaːɽa/ &
\textit{(nhiáaṛa)} &
`near'\\
/pj/ &
/pjaːla/ &
\textit{(piaalá)} &
`cup'\\
/pʰj/ &
/pʰjoːɽ/ &
\textit{(phióoṛ)} &
`side (of animal)' \\
/dʰj/ &
/dʰjuːɽi/ &
\textit{(dhiúuṛi)} &
`granddaughter'\\
/sw/ &
/sweːni/ &
\textit{(suéeni) } &
`is sleeping (\textsc{f)}'\\
/ɡʰw/ &
/ɡʰweːɳiː/ &
\textit{(ɡhueeṇíi)} &
`Pashtun'\\
/dʰw/ &
/dʰweːli/ &
\textit{(dhuéeli)} &
`washed (\textsc{f)}'\\
\end{tabular}
\end{exe}


Consonant clusters in final position, see (\ref{ex:3-12}), seem to be subject to a~much higher degree of variability, although the position also seems to be slightly more permissive. The more frequently occurring type consists of nasal + plosive/affricate/fricative, the other ones being more marginal in occurrence. 


\begin{exe}
\extab
\label{ex:3-12}
\begin{tabular}{ l l l l }
/nd/ &
/daːnd/ &
\textit{(dáand)} &
`tooth'\\
/ɳɖ/ &
/ɡʰreːɳɖ/ &
\textit{(ɡhreéṇḍ)} &
`knot'\\
/nk/ &
/ɡʰroːnk/ &
\textit{(ɡhroónk)} &
`worm'\\
/nɡ/ &
/ɕoːnɡ/ &
\textit{(šóonɡ) } &
`branch'\\
/nʑ/ &
/paːnʑ/ &
\textit{(páanǰ) } &
`five' \\
/nɕ/ &
/bʰeːnɕ/ &
\textit{(bheénš) } &
`beam (of wood)' \\
/ɳʈʂ/ &
/iɳʈʂ/ &
\textit{(iṇc̣) } &
`bear' \\
/tr/ &
/suːtr/ &
\textit{(súutr)} &
`thread' \\
/st/ &
/ɡʰraːst/ &
\textit{(ɡhraást)} &
`wolf' \\
/ʂʈ/ &
/ɡʰoːʂʈ/ &
\textit{(ɡhoóṣṭ) } &
`house'\\
\end{tabular}
\end{exe}


The final affricate or fricative is always articulated, even if sometimes only weakly, whereas the
nasal (homorganic with the affricate or fricative) is sometimes~-- more with some speakers than
others and depending on word emphasis~-- phonetically absent but leaving a~trace of nasalisation on
the preceding vowel. Even the nasal + plosive sequences are subject to much variability. With some
speakers and dialects, one of the phonemes in the sequence is altogether absent, sometimes the nasal
(then leaving the preceding vowel nasalised, ex. /naːnɡ/
(náanɡ) `finger or toe"=nail' B.: [n̪\^{ã}ːg]), sometimes the stop (ex. /ɕaːnɡ/ (\v{s}áanɡ)
`branch' B.: [ɕâːŋ]), whereas in the corresponding inflected forms the stop would never be
omitted: [n̪ûːŋɡa] (núunɡa), [ɕûːŋɡa]
(\v{s}úunɡa). This especially pertains to the /n +
d/ sequences in singular nouns, where it seems to be more rule than exception that the final stop is
dropped, especially in B., whereas these are clearly articulated when occurring non"=finally in the
inflected forms: /dan/ (dán) `tooth' B., but
/daːnda/ (dáanda) teeth'; pan/(pán) path' B., but /paːnda/
(páanda) paths'.


A special case is the final cluster /ndr/ in /jaːndr/ (yáandr) `mill'. This is the only three"=consonant cluster at a~word boundary discovered so far, but its exact phonetic realisation is not entirely easy to define in terms of segments. With some speakers, the /n/ is clearly articulated, whereas the /dr/ part is only faintly present, and in other pronunciations the final /r/ gets a~schwa"=like sound attached to it, in practice making /dr/ the onset of an~additional syllable. As with the above mentioned clusters occurring at the end of singular nouns, the same cluster stretching over a~syllable boundary in an~inflected form of the same noun is clearly and unambiguously articulated: /jaːn.dra/ `mills'.


In the final /tr/-cluster, the /r/ is present as a~segment in the speech of all my informants, but its articulation is not exactly identical to its non"=final allophones (as in the inflected forms of the same lexical items). There is a~strong tendency for it to be pronounced with very little energy, almost always being devoiced and sometimes also followed by an~optional very short schwa"=like sound: [putɾ̥(ə)] in /putr/ \textit{(pútr)} `son'. 


As for the realisation of the final /st/ and /ʂʈ/-clusters, there are differences between different speakers, and possibly between different dialects as well. My B. informants tend to articulate both members of the cluster, even in final position, though the plosive is somewhat softened, whereas my A. informants seem to prefer to omit the plosive altogether in final position, e.g [ɡʱ\v{o}ːʂ] `house', [n̪aːs] \textit{(náas)} `nose'. However, in all speech varieties both the fricative and the plosive are clearly present when occurring medially, i.e. in the corresponding inflected forms: [ɡʱoːʂʈá], [nastí].


Clusters occurring intervocalically are all of the same type as those occurring word finally; the nasal + plosive/affricate/fricative type is especially common: /oːmbaːr/ (oombaár) `canal inlet', /ʈombu/ (ṭómbu) `stem', /kaːntiːru/ (kaantíiru) `insane person', /ʈiːnʨuk/ \textit{(ṭíinčuk)} `scorpion', /bʰanʑa/ \textit{(bhanǰá)} `beat!', /ʐʰaɳʐiːr/ \textit{(ẓhaṇẓíir)} `chain', /heːnsili/ \textit{(heensíli)} `stayed, existed \textsc{f}'. 


\section{Suprasegmentals}
\label{sec:3-4}

\subsection{Aspiration and breathiness}
\label{subsec:3-4-1}

In the present description, aspiration has been analysed as a~property or feature of a~word (or more correctly the lexical stem), rather than as a~segment or a~secondary articulation of any one segment. This feature, [ʰ] with a~voiceless consonant and [ʱ] with a~voiced consonant, occurs only once in a~word\footnote{\label{fnt:ftn32} The process of dissimilation of aspirates in two successive syllables is known as \textit{Grassman's Law} within Indo"=European historical linguistics (\citealt[19, 56]{szemerenyi1996}; \citealt[92--93]{lehmann1962}) and has been applied to the development of OIA as well as Greek. A synchronic process or rule restricting the occurrence of aspirated sounds to one per word has also been stipulated for other NIA languages (see for example \citealt[32]{losey2002} for Gojri, and \citealt[34--35]{shackle1976} for Siraiki). This process, however, is not confined to Indo"=European languages (cf. Tibeto"=Burman Manipuri, \citealt[13--14]{bhatningomba1997}).} (in a~majority of the cases word"=intially) and is represented by the letter h in the common transcription.\footnote{It is also possible that a~more careful analysis will result in all instances of /h/ (here considered a~consonant in its own right when occurring alone in syllable onset) being treated as aspiration, i.e. a~feature of the word rather than a~segment.} Some minimal pairs in (\ref{ex:3-13}) illustrate the contrastiveness of this feature.
\begin{exe}
\extab
\label{ex:3-13}
\begin{tabularx}{116mm}{ l l Q l l l Q }
/bʰoːla/ &
\textit{(bhóola)} &
`were able to' &
vs. &
/boːla/ &
\textit{(bóola)} &
`hair'\\
/kʰareːɽi/ &
\textit{(kharéeṛi)} &
`bolt' &
&
/kareːɽi/ &
\textit{(karéeṛi)} &
`leop\-ardess'\\
/wʰiː/
&
\textit{(whíi)}
&
`will come down (\textsc{3sg})' &
&
/wiː/
&
\textit{(wíi)}
&
`water'\\
\end{tabularx}
\end{exe}


The voiced aspirated sounds are normally phonetically realised with breathy voice during their release and/or the immediately following vowel is pronounced with (at least partial) breathy voice: [bʱo̤ːla].


Most Palula consonant phonemes can be accompanied (at syllable onset) by aspiration, as indicated within parentheses in \tabref{tab:3-1} and with examples in (\ref{ex:3-14}), from plosives to approximants. 


\begin{exe}
\extab
\label{ex:3-14}
\begin{tabularx}{\textwidth}{ l l l }
/pʰeːpi/ &
\textit{(phéepi)} &
`paternal aunt'\\
/dʰut/ &
\textit{(dhut)} &
`mouth'\\
/ʈʰonɡi/ &
\textit{(ṭhónɡi)} &
`axe'\\
/ʨʰeːli/ &
\textit{(čhéeli)} &
`she"=goat'\\
/ʑʰaːʈ/ &
\textit{(ǰhaáṭ)} &
`goat's hair'\\
/lʰoːɳ/ &
\textit{(lhoóṇ) } &
`salt'\\
/mʰaːs/ &
\textit{(mhaás)} &
`meat'\\
/jʰuɳɖi/ &
\textit{(yhúṇḍi)} &
`stick'\\
\end{tabularx}
\end{exe}


Only the voiceless fricatives /s ʂ ɕ x/, the distributionally limited /ɳ ɽ/, and the ``new'' phonemes /ɣ x/ do not co"=occur (at syllable onset) with aspiration. This ``generous'', obligatory occurrence of aspiration is not a~feature of most other languages in the immediate region, possibly with the exception of Indus Kohistani, where OIA aspiration, like in Palula, has been preserved and where aspiration is concomitant with most of its consonants \citep[19--25]{hallberghallberg1999}. 


The contrastiveness of aspiration when co"=occurring with the affricates /ts/ and /ʈʂ/ is somewhat doubtful and would deserve further, more detailed investigation. At least for the latter, it seems, it is almost by ``default'' more or less clearly aspirated. A similar hesitation has been expressed on the /ʈʂ/-/ʈʂʰ/ contrast in Khowar \citep[239]{endresenkristiansen1981}.


Whereas the phonetic realisation of the aspiration with the voiceless consonants is more or less equal to a~secondary pronunciation, the voiced aspiration or ``breathiness'' affects the pronunciation of the following vowel and is also somewhat mobile within the syllable, and for some words even beyond the realm of that syllable. Especially in B., there is a~fluctuation in some words, as seen in (\ref{ex:3-15}) and (\ref{ex:3-16}), between an~intervocalic /h/ and the aspiration or ``breathiness'' as described above, the intervocalic /h/ probably representing an~older pattern.\footnote{Possibly this is preserved to a~larger extent in the conservative variety spoken in Puri (mainly agreeing with Biori Palula), where the following nominative"=oblique alternation was recorded: /brʰuː/ \textit{(bhruú)} `brother'~-- /brahu/ `brothers'. The unstable character of the phoneme /h/ and voiced aspiration in Kalasha is also commented on by \citet[50]{morchheegaard1997}.}

\begin{exe}
\extab
\label{ex:3-15}
/lʰóylo/$\sim$/lohílo/ \\
\textit{(lhóilu)} \\
`red' (B.)

\extab
\label{ex:3-16}
/bʰjûːɽi/$\sim$/bihûːɽi/ \\
\textit{(bhiúuṛi)} \\
`Biori' (B.)
\end{exe}

But also in A. there are words, as in example (\ref{ex:3-17}), for which the location of aspriation is alternating (between speakers and possibly even with one and the same speaker).

\begin{exe}
\extab
\label{ex:3-17}
/ɡʰaɖeːró/$\sim$/ɡaɖʰeːró/ \\
(ɡhaḍeeró) \\
`elder'
\end{exe}

The aspiration feature certainly has multiple diachronic sources: One is the OIA aspiration, preserved in Palula to an~extent not evidenced in the major Shina varieties,\footnote{On the contrary \citet[30]{schmidtkohistani2008} state that the voiced aspirates in modern Kohistani Shina have come into the language through borrowing, while they have been lost in OIA voiced aspirate cognates.} such as in /ɡʰuːɽu/ \textit{(ɡhúuṛu)} `horse' {\textless} OIA \textit{ɡhōṭa} \citep[4516]{turner1966}. Another is the above mentioned intervocalic /h/ advanced to a~more word"=initial position. Finally, an~old point of aspiration can be advanced, such as in /ɡʰoːʂʈ/ \textit{(ɡhoóṣṭ)} `house' {\textless} OIA \textit{ɡōṣṭhá} \citep[4336]{turner1966}. Other words may have followed other routes possibly further reinforced by the rising pitch of a~second mora accent (see \sectref{subsec:3-4-3} below). In any case, not all occurrences of aspiration, even when concomitant with plosives, are justified or explained solely by etymology (as pointed out by \citealt[57]{morgenstierne1932}).


Further study is needed to determine to what extent aspiration is preserved in two aspirated words that are compounded. There is an~indication that the primary stressed part of the compound keeps its aspiration, while the other point of aspiration is entirely or partly deaspirated (cf. comment in footnote \ref{fnt:ftn32} on \textit{Grassman's Law}): /dʰut/ \textit{(dhut)} `mouth' + /ɡʰaːnu/ \textit{(ɡháanu)} `large' {\textgreater} /dutaɡʰaːnu/ \textit{(dutaɡháanu)} `talkative'.


The interaction between accent and aspiration is another topic for further research. As voiced aspiration/breathiness quite often precedes a~second"=mora accented long vowel, the voiced aspiration may have influenced or reinforced the rising pitch of that accent. However, it should be pointed out that breathiness also occurs before unaccented, /ʑʰamatroː/ \textit{(ǰhamatroó)} `son"=in"=law', as well as first"=mora accented vowels, /oːɖʰoːl/ \textit{(ooḍhóol)} `flood'; hence the two suprasegmental features are, in essence, independent.\footnote{This contrasts with the situation in Kalam Kohistani and Torwali (both languages belonging to the Kohistani group of HKIA languages), where aspiration is only contrastive with voiceless consonants, whereas breathiness is a~feature only optionally concomitant with low tone (\citealt[92]{baart1999b}; \citealt[36--37]{lunsford2001}). } 


\subsection{Nasalisation}
\label{subsec:3-4-2}


As mentioned earlier, nasalisation seems to be a~marginal suprasegmental feature associated with a~limited number of lexemes, of which some can be seen in (\ref{ex:3-18}). It is unclear whether in all those lexemes there is a~historical loss of a~nasal segment.


\begin{exe}
\extab
\label{ex:3-18}
\begin{tabularx}{\textwidth}{ l l l }
/ʑʰ\~{i}ː/ &
\textit{(ǰhií$\sim$) } &
`louse'\\
/kũj/ &
\textit{(kúi$\sim$)} &
`valley'\\
/hẽːta/ &
\textit{(hée$\sim$ta)} &
`would (\textsc{condl)}'\\
/sẽːta/ &
\textit{(sée$\sim$ta)} &
`should (\textsc{condh)}'\\
/dʰrũːʂ/ &
\textit{(dhruú$\sim$ṣ)} &
`Drosh'\\
\end{tabularx}
\end{exe}

\subsection{Pitch accent}
\label{subsec:3-4-3}

A phonological word in Palula may carry one, and only one, accent. Phonetically the accent is primarily realised as relatively higher pitch, accompanied to some extent by higher amplitude. Generally speaking, in a~single word, accent is associated with high pitch, and the corresponding lack of accent is associated with low (or default) pitch. 

The accent"=bearing unit is the mora, which means that accent can be associated with a~short vowel (as in (\ref{ex:3-19})), \textit{or} the first mora of a~long vowel (as in (\ref{ex:3-20})), \textit{or} the second mora of a~long vowel (as in (\ref{ex:3-21})).


\begin{exe}
\extab
\label{ex:3-19}
\begin{tabularx}{\textwidth}{ l l l }
\multicolumn{3}{l}{Accent on short vowel:}\\
/ʂíʂ/ &
\textit{(ṣiṣ)} &
`head'\\
/híɽu/ &
\textit{(híṛu)} &
`heart'\\
/kilí/ &
\textit{(kilí)} &
`key'\\
\end{tabularx}
\end{exe}

\begin{exe}
\extab
\label{ex:3-20}
\begin{tabularx}{\textwidth}{ l l l }
\multicolumn{3}{ l}{First"=mora accent on long vowel:}\\
/ɖôːk/ &
\textit{(ḍóok)} &
`back'\\
/pûːtri/ &
(\textit{púutri)} &
`granddaughter'\\
/aʈʂîː/ &
\textit{(ac̣híi)} &
`eye'\\
\end{tabularx}
\end{exe}


\begin{exe}
\extab
\label{ex:3-21}
\begin{tabularx}{\textwidth}{ l l l }
\multicolumn{3}{ l}{Second"=mora accent on long vowel:}\\
/bǎːt/ &
\textit{(baát)} &
`talk, word, issue'\\
/kuɳǎːk/ &
\textit{(kuṇaák)} &
`child'\\
/bǎːbu/ &
\textit{(baábu)} &
`father'\\
\end{tabularx}
\end{exe}

This means that the pitch accent\footnote{Pitch accent here is of the kind also observed in e.g. Lithuanian \citep[73--82]{szemerenyi1996}, a~mora accent which is ``free within limits''.} (henceforth only \textit{accent}) has one of the following phonetic manifestations:
\begin{enumerate}
\item[a)] high level or falling on a~short vowel [\'{}], represented in this work with an~acute accent mark: \textit{á} (only in polysyllabic words, elsewhere no marking), 
\item[b)] rising on a~long vowel [\v{}], represented with an~acute accent mark on the second vowel symbol, \textit{aá}, or 
\item[c)] falling on a~long vowel [\^{}], represented with an~acute accent mark on the first vowel symbol, \textit{áa}.
\end{enumerate}
A word, as referred to here, is either a~bare stem, such as /děːs/ \textit{(deés)} `day', /paːɽú/ \textit{(paaṛú)} `magician', or a~stem with one or more suffixes added to it, such as /deːs-ôːm-iː/ \textit{(dees-óom"=ii)} `of the days', /ʑôːn-um/ \textit{(ǰhóon"=um)} `I will know'. Even though some combinations of syllables and accents are more common than others, and there are restrictions on accent placement (see below), the location of the accent within a~given word is not entirely predictable. Therefore, accent in Palula must be defined as lexical. 


Difference in accent placement is in a~few cases, as in (\ref{ex:3-22}), the only phonemic contrast between two lexical items.


\begin{exe}
\extab
\label{ex:3-22}
\begin{tabularx}{116mm}{ l l Q l l l Q }
/ʨûːr/ &
\textit{(čúur)} &
`four' &
\textbf{vs.} &
/ʨǔːr/ &
\textit{(čuúr)} &
`hot fire'\\
/káti/ &
\textit{(káti)} &
`saddle' &
&
/katí/ &
\textit{(katí)} &
`how many?'\\
/dêːdi/ &
\textit{(déedi)} &
`grand\-mother' &
&
/děːdi/ &
\textit{(deédi)} &
`burnt'\\
/dʰûːra/ &
\textit{(dhúura)} &
`distant' &
&
/dʰuːrá/ &
\textit{(dhuurá)} &
`separate'\\
\end{tabularx}
\end{exe}

Although this kind of minimal pair is not particularly common in the language, the system definitely allows them to occur.

\subsubsection*{The position of accent on stems}

Accent falls either in the final or the penultimate syllable in a~non"=verbal lexical stem, as seen
in (\ref{ex:3-23}) and (\ref{ex:3-24}), whereas in verbal stems, as in (\ref{ex:3-25}), the accent
is always on the final syllable.


\begin{exe}
\extab
\label{ex:3-23}
\begin{tabularx}{\textwidth}{ l l l }
\multicolumn{3}{l}{Lexical accent on noun stems:}\\
/tôːruɳ/ &
\textit{(tóoruṇ)} &
`forehead'\\
/paːlǎː/ &
\textit{(paalaá)} &
`leaf'\\
/kuɳôːku/ &
\textit{(kuṇóoku)} &
`puppy'\\
/ɡʰaɖeːró/ &
\textit{(ɡhaḍeeró)} &
`elder'\\
\end{tabularx}

\extab
\label{ex:3-24}
\begin{tabularx}{\textwidth}{ l l l }
\multicolumn{3}{ l }{Lexical accent on some other non"=verbal stems:}\\
/típa/ &
\textit{(típa)} &
`now'\\
/pʰaré/ &
\textit{(pharé)} &
`along, toward'\\
/eːtríli/ &
\textit{(eetríli)} &
`the day before yesterday'\\
/taːqatwár/ &
\textit{(taaqatwár)} &
`powerful'\\
\end{tabularx}

\extab
\label{ex:3-25}
\begin{tabularx}{\textwidth}{ l l l }
\multicolumn{3}{ l}{Lexical accent on verb stems:}\\
/krín/ &
\textit{(krín)} &
`sell'\\
/piʨʰíl/ &
\textit{(pičhíl)} &
`slip'\\
/karoːɽé/ &
\textit{(karooṛé)} &
`dig, scratch'\\
\end{tabularx}
\end{exe}

\subsubsection*{Accent properties of suffixes}

There are two types of suffixes: those that carry their own accent, which we will refer to as
accent"=bearing suffixes, and those that are accent"=neutral. When a~suffix of the first type is added
to a~stem, the accent of the stem is eliminated, and the word accent falls on the suffix. When
a~suffix of the second type is added to a~stem, the stem accent may be retained. However, under
certain conditions, even in the latter case, the accent (defined by the lexical stem) shifts from
the stem to the suffix, a~matter we will return to in \sectref{subsec:3-5-1}.

\begin{table}[ht]
\caption{Accent"=bearing suffixes}
\begin{tabularx}{\textwidth}{ l l l Q }
\lsptoprule
Suffix &
Function &
Example &
\\\hline
\textit{-í} &
plural &
\textit{kuḍ-í} &
`walls'\\
\textit{-í} &
oblique, locative &
\textit{dukeen-í} &
`in the shop'\\
\textit{-íim} /îːm/ &
plural oblique &
\textit{dukeen-íim} &
`in the shops'\\
\textit{-í} &
converb &
\textit{ɡhin-í} &
`having taken'\\
\textit{-áan} /âːn/ &
present &
\textit{ɡhin-áan-u} &
`(he) is taking'\\
\textit{-íia} /îjːa/ &
\textsc{1pl} &
\textit{ɡhin-íia} &
`we will take'\\
\textit{-íl} &
perfective (stem) &
\textit{čhin-íl-i} &
`(was) cut'\\
\textit{-eeṇḍeéu} /eːɳɖěːw/ &
obligative &
\textit{ɡhin"=eeṇḍeéu} &
`has to be taken'\\
\textit{-áaṭ} /âːʈ/ &
agentive &
\textit{čhin-áaṭ-u} &
`the person cutting'\\
\textit{-áai} /âːj/ &
infinitive &
\textit{ɡhin-áai} &
`to take'\\
\textit{-ainií} /ajnǐː/ &
verbal noun &
\textit{ɡhin"=ainií} &
`taking'\\
\textit{-íim} /îːm/ &
copredicative &
\textit{khaṣeel-íim} &
`dragging'\\
\textit{-íǰ} /íʑ/ &
passive (stem) &
\textit{paš-íǰ-aṛ} &
`it will be seen\newline (by you)'\\
\textit{-á} &
causative (stem) &
\textit{pal-á} &
`hide (it)!'\\\lspbottomrule
\end{tabularx}
\label{tab:3-5}
\end{table}


The most productive accent"=bearing suffixes (mainly verbal) are presented in \tabref{tab:3-5}. Some of them also have other allomorphs, such as \textit{-éen}, \textit{-áand}, and \textit{-éend} of \textit{-áan}.


Some of these suffixes may be added cumulatively, whereby the accent is carried by the last accent"=bearing suffix.


As with the accent"=bearing suffixes, only the most productive accent"=neutral suffixes are presented in \tabref{tab:3-6}, excluding possible allomorphs.



\begin{table}[ht]
\caption{Accent"=neutral suffixes}
\begin{tabularx}{\textwidth}{ l@{\hspace{20pt}} l@{\hspace{20pt}} l@{\hspace{20pt}} Q }
\lsptoprule
Suffix &
Function &
Example &
\\\hline
\textit{-um} &
\textsc{1sg} &
\textit{ɡhín-um} &
`I will take'\\
\textit{-aṛ} /aɽ/ &
\textsc{2sg} &
\textit{ɡhín-aṛ} &
`you (\textsc{sg}) will take'\\
\textit{-a} &
\textsc{3sg} &
\textit{ɡhín-a} &
`he/she/it will take'\\
\textit{-at} &
\textsc{2pl} &
\textit{ɡhín-at} &
`you (\textsc{pl}) will take'\\
\textit{-an} &
\textsc{3pl} &
\textit{ɡhín-an} &
`they will take'\\
\textit{-u} &
msg &
\textit{ɡhináan-u} &
`(he, etc.) is taking'\\
\textit{-a} &
mpl &
\textit{ɡhináan-a} &
`(they) are taking'\\
\textit{-i} &
f &
\textit{ɡhinéen-i} &
`(she, etc.) is taking'\\
\textit{-a} &
plural &
\textit{díiš-a} &
`villages'\\
\textit{-a} &
oblique &
\textit{díiš-a} &
`in the village'\\
\textit{-am} &
plural oblique &
\textit{díiš-am} &
`in the villages'\\
\textit{-ii} /iː/ \textit{(B.-e)} &
genitive &
\textit{díiš-ii (B. díiš-e)} &
`of the village'\\\lspbottomrule
\end{tabularx}
\label{tab:3-6}
\end{table}


Some quantitative and qualitative morphophonemic alternations relating to the position of the accent in the word will be dealt with in \sectref{subsec:3-5-1}.


\section[Morphophonology]{Morphophonology{\protect\footnotemark}}
\label{sec:3-5}

\footnotetext{In this section, only the Palula common transcription is being used (as in the rest of this work) without any accompanying IPA transcription.}

\subsection{Morphophonemic alternations relating to accent}
\label{subsec:3-5-1}

A number of segmental modifications (primarily in the nominal paradigm) are related to accent, or more precisely to the position of the accent within the word; whether they are described as synchronically productive processes or the result of a~diachronic process, the latter also offering some explanations to the more regular dialectal variation observed. 


\begin{table}[ht]
\caption{ Accent"=related alternations in the paradigm}
\begin{tabularx}{\textwidth}{ l@{\hspace{30pt}} Q Q Q }
\lsptoprule
Stem &
&
Inflected form &
\\\hline
\textit{ḍheér} &
`stomach' &
\textit{ḍheer-í} &
`stomachs'\\
\textit{kuṇaák} &
`child' &
\textit{kuṇaak-á} &
`children'\\
\textit{ṣiṣ} &
`head' &
\textit{ṣiṣ-óom} &
`heads (\textsc{ob)}'\\
\textit{haál} &
`plough' &
\textit{hal-á} &
`ploughs'\\\lspbottomrule
\end{tabularx}
\label{tab:3-7}
\end{table}


\tabref{tab:3-7} illustrates the main types of alternations: \textit{ḍheér-ḍheerí} has an~accent alternating between the stem and an~accent"=bearing suffix without any concomitant segmental modification (see \sectref{subsec:3-4-3} and \tabref{tab:3-5}), \textit{kuṇaák-kuṇaaká} has an~accent shifting from the stem to an~accent"=neutral suffix without any concomitant segmental modification (see \sectref{subsec:3-4-3} and \tabref{tab:3-6}), \textit{ṣiṣ-ṣiṣóom} has an~accent shifting from the stem to an~affix"=neutral suffix with accompanying suffix modification (the unaccented allomorph being \textit{-am}), and \textit{haál-halá} has an~accent shifting from the stem to an~affix"=neutral suffix with accompanying stem vowel modification. These four and some of their more salient subtypes will be further discussed and illustrated in the rest of this subsection.


\subsubsection*{Accent alternation (accent"=bearing suffix) without modification}

A stem inflected with one of the accent"=bearing suffixes presented in \tabref{tab:3-5} will without
exception carry its accent on the accent"=bearing unit of the suffix, and not on the lexically
defined position in the stem. Some further examples are given in \tabref{tab:3-8}.


\begin{table}[ht]
\caption{ Accent alternating between stem and accent"=bearing suffix}
\begin{tabularx}{\textwidth}{ l@{\hspace{25pt}} Q l@{\hspace{25pt}} Q }
\lsptoprule
Stem &
&
Inflected form &
\\\hline
\textit{preṣ} &
`mother"=in"=law' &
\textit{preṣ-í} &
`mothers"=in"=law'\\
\textit{keéṇ} &
`cave' &
\textit{keeṇ-í} &
`in the cave'\\
\textit{khoṇḍ} &
`speak!' &
\textit{khoṇḍ-íia} &
`we will speak'\\
\textit{til} &
`walk!' &
\textit{til-áan-a} &
`(they, etc.) are walking'\\\lspbottomrule
\end{tabularx}
\label{tab:3-8}
\end{table}

\subsubsection*{Accent alternation (accent"=neutral suffix) without modification}

In some cases, the accent shifts to the suffix in spite of it being an~accent"=neutral suffix. While
the previously mentioned process is fully predictable from the type of suffix itself, the reason for
the current process happening is instead to be found in the lexical stem itself.


For the most part (as far as the nominal paradigm is concerned), this shift takes place when the
lexical stem accent is on the last mora, as in \tabref{tab:3-9}.



\begin{table}[ht]
\caption{ Accent shift from final"=moraic accented stems to accent"=neutral suffix}
\begin{tabularx}{\textwidth}{ l@{\hspace{25pt}} Q l@{\hspace{25pt}} Q }
\lsptoprule
Stem &
&
Inflected form &
\\\hline
\textit{putr} &
`son' &
\textit{putr-á} &
`sons'\\
\textit{oóṛ} &
`chicken' &
\textit{ooṛ-á} &
`chickens'\\
\textit{dheeṛúm} &
`pomegranate' &
\textit{dheeṛum-á} &
`pomegranates'\\
\textit{atshareét} &
`Ashret' &
\textit{atshareet-á} &
`in Ashret'\\\lspbottomrule
\end{tabularx}
\label{tab:3-9}
\end{table}


While this is true for all polysyllabic stems, there are quite many monosyllabic stems for which accent shift remains non"=predictable from a~purely synchronic perspective. On the one hand, there are those final"=mora accented stems that do not produce an~accent shift (\tabref{tab:3-10}), and on the other hand, there are those non"=final mora accented stems that contrary to expectation do undergo an~accent shift (the latter will be discussed below in the discussion of stem modification).



\begin{table}[ht]
\caption{ Stems with final"=mora accent not displaying accent shift}
\begin{tabularx}{.75\textwidth}{ l@{\hspace{20pt}} Q l@{\hspace{20pt}} Q }
\lsptoprule
Stem &
&
Inflected form &
\\\hline
\textit{dhut} &
`mouth' &
\textit{dhút-a} &
`mouths'\\
\textit{iṇc̣} &
`bear' &
\textit{íṇc̣-a} &
`bears'\\
\textit{bhruk} &
`kidney' &
\textit{bhrúk-a} &
`kidneys'\\
\textit{haát} &
`hand' &
\textit{háat-a} &
`hands'\\\lspbottomrule
\end{tabularx}
\label{tab:3-10}
\end{table}

\subsubsection*{Accent alternation with suffix modification}

In some cases, an~accented suffix vowel is (qualitatively or quantitatively) modified as compared
with an~unaccented allophone. This can be seen in \tabref{tab:3-11}. This particularly concerns the
accent"=neutral plural oblique suffix \textit{-am} and the genitive suffix \textit{-e} (in B. only).

The alternation in the paradigms to some extent reflects general vowel shifts in the language
\textit{(a {\textgreater} aa, aa {\textgreater} oo/uu, ee {\textgreater} ii)} conditioned by accent,
and in the B. dialect also by syllable structure, to which we will have reason to return to in the
discussion below on stem modification.

In the A. dialect, \textit{-am} regularly has the form \textit{-óom} when accented.

\begin{table}[ht]
\caption{Accent shift with suffix modification (A. dialect)}
\begin{tabularx}{\textwidth}{ l@{\hspace{20pt}} l@{\hspace{20pt}} Q Q Q }
\lsptoprule
Stem &
&
Plural nom &
Plural ob &
Plural gn \\\hline
\textit{deés} &
`day' &
\textit{dees-á} &
\textit{dees-óom} &
\textit{dees-óom"=ii} \\
\textit{ɡhoóṣṭ} &
`house' &
\textit{ɡhooṣṭ-á} &
\textit{ɡhooṣṭ-óom} &
\textit{ɡhooṣṭ-óom"=ii} \\
\textit{kuṇaák} &
`child' &
\textit{kuṇaak-á} &
\textit{kuṇaak-óom} &
\textit{kuṇaak-óom"=ii} \\\lspbottomrule
\end{tabularx}
\label{tab:3-11}
\end{table}


In the B. dialect, \textit{\--am} regularly takes the form \textit{-áam} in closed syllables, and \textit{-úum} in open syllables (i.e. when followed by a~genitive suffix), as illustrated in \tabref{tab:3-12}. Also the unaccented genitive suffix \textit{-e} corresponds regularly to an~accented form \textit{-í}.



\begin{table}[ht]
\caption{Accent shift with suffix modification (B. dialect)}
\begin{tabularx}{\textwidth}{ Q l l l l l }
\lsptoprule
Stem &
&
Singular gn &
Plural nom &
Plural ob &
Plural gn \\\hline
\textit{deés} &
`day' &
\textit{dees-í} &
\textit{dees-á} &
\textit{dees-áam} &
\textit{dees}-\textit{úum-e}\\
\textit{ɡhoóṣṭ} &
`house' &
\textit{ɡhooṣṭ-í} &
\textit{ɡhooṣṭ-á} &
\textit{ɡhooṣṭ-áam} &
\textit{ɡhooṣṭ-úum-e} \\
\textit{kuṇaák} &
`child' &
\textit{kuṇaak-í} &
\textit{kuṇaak-á} &
\textit{kuṇaak-áam} &
\textit{kuṇaak-úum-e} \\\lspbottomrule
\end{tabularx}
\label{tab:3-12}
\end{table}

\subsubsection*{Accent alternation with stem modification}

In the nominal paradigm of the A. dialect, some vowel modifications affecting the nominative have been blocked by accent shift in the inflected forms, resulting in alternations between \textit{aa} and \textit{a}, which is obvious with the nouns in \tabref{tab:3-13}. The lengthening of the accented vowels has produced a~second"=mora accent in polysyllabic stems and vowels preceded by aspiration. 

\begin{table}[ht]
\caption{Alternations between a~and aa (A. dialect)}
\begin{tabularx}{.75\textwidth}{ Q Q Q }
\lsptoprule
Stem &
&
Inflected form\\\hline
\textit{báaṭ} &
`stone' &
\textit{baṭ-á} \\
\textit{heewaán(d)} &
`winter' &
\textit{heewand-á} \\
\textit{dhaán} &
`goat' &
\textit{dhan-á}\\
\textit{sáar} &
`lake' &
\textit{sar-í}\\
\textit{aaṣaáṛ} &
`apricot' &
\textit{aaṣaṛ-í} \\
\textit{c̣haár} &
`waterfall' &
\textit{c̣har-í} \\\lspbottomrule
\end{tabularx}
\label{tab:3-13}
\end{table}


Along the same lines, there are regular alternations (\tabref{tab:3-14}) between accented stems with \textit{ii, oo} (B. \textit{uu}) and \textit{aa} vs. unaccented stems with \textit{ee, aa} and \textit{a}.



\begin{table}[ht]
\caption{Alternations in the verbal paradigm between: a$\sim$áa, aa$\sim$óo and ee$\sim$íi}
\begin{tabularx}{\textwidth}{ Q l Q l }
\lsptoprule
Form with\newline stem accent &
&
Form with\newline suffix accent &
\\\hline
\textit{páaš-um} &
`I will see' &
\textit{paš-áan-u} &
`(he) is seeing'\\
\textit{ǰhóon"=um (B.~ǰhúun"=um)} &
`I will know' &
\textit{ǰhaan-áan-u} &
`(he) is knowing'\\
\textit{uḍhíiw"=um} &
`I will escape' &
\textit{uḍheew-áan-u} &
`(he) is escaping'\\\lspbottomrule
\end{tabularx}
\label{tab:3-14}
\end{table}

\subsubsection*{Other alternations}

Some other alternations having to do with the interaction between stem vowels and suffix vowels will
be discussed at length in the chapters on noun and verb morphology, \sectref{chap:4} and
\sectref{chap:8}, respectively.

\subsection{Morphophonemic alternations relating to syllable structure}
\label{subsec:3-5-2}


Morphophonemic alternations relating to syllable structure, already briefly touched upon above (see \tabref{tab:3-12}), is exclusively a~feature of the B. dialect. The alternations are between closed syllables with \textit{ee, aa}, and \textit{a} vs. open syllables with \textit{ii, uu}, and \textit{aa}. It is most clearly displayed in the nominal paradigm, as shown in \tabref{tab:3-15}.



\begin{table}[ht]
\caption{Alternations between: a$\sim$áa, aãúu and ee\~{}íi (B. dialect)}
\begin{tabularx}{\textwidth}{ Q Q Q }
\lsptoprule
Stem &
&
Inflected form\\\hline
\textit{kaṇ} &
`ear' &
\textit{káaṇa} \\
\textit{kram} &
`work' &
\textit{kráama} \\
\textit{dan} &
`tooth' &
\textit{dáanda} \\
\textit{ooḍháal} &
`flood' &
\textit{ooḍhúula} \\
\textit{sáan} &
`pasture' &
\textit{súuna} \\
\textit{baazáar} &
`bazaar' &
\textit{baazúura} \\
\textit{méeš} &
`man' &
\textit{míiša} \\
\textit{šéen} &
`bed' &
\textit{šíina} \\\lspbottomrule
\end{tabularx}
\label{tab:3-15}
\end{table}

\subsection{Umlaut}
\label{subsec:3-5-3}

There are numerous examples, in the nominal (\tabref{tab:3-16}), adjectival (\tabref{tab:3-19}), and verbal paradigms (\tabref{tab:3-17}) of anticipatory fronting (``umlaut'') of aa to ee when preceding an~i in the following syllable. The anticipation normally does not occur if the a~is short. 


\begin{table}[ht]
\caption{Alternations in the nominal paradigm between aa and umlaut"=ee}
\begin{tabularx}{\textwidth}{ Q Q Q Q }
\lsptoprule
Form without umlaut &
&
Form with umlaut &
\\\hline
\textit{baát} &
`word, issue' &
\textit{beetí} &
`words, issues'\\
\textit{ɡilaás} &
`glass' &
\textit{ɡileesí} &
`glasses'\\
\textit{hiimaál} &
`glacier' &
\textit{hiimeelí} &
`glaciers'\\
\textit{kitaáb} &
`book' &
\textit{kiteebí} &
`books'\\\lspbottomrule
\end{tabularx}
\label{tab:3-16}
\end{table}


The nouns `glass' and `book' in \tabref{tab:3-16} show that this process has been productively extended even to relatively recent loans.



\begin{table}[ht]
\caption{Alternations in the verbal paradigm between aa and umlaut"=ee}
\begin{tabularx}{\textwidth}{ l Q l Q l l }
\lsptoprule
\multicolumn{2}{l}{Form without umlaut} &
\multicolumn{2}{l}{Form with umlaut} &
\multicolumn{2}{l}{Form with umlaut}\\\hline
\textit{mhaaráanu} &
`(he) is killing' &
\textit{mheerí} &
`having killed' &
\textit{mheerílu} &
`killed'\\
\textit{phaaláanu} &
`(he) is splitting' &
\textit{pheelí} &
`having splitted' &
\textit{pheelílu} &
`split'\\
\textit{ǰhaanáanu} &
`(he) is knowing' &
\textit{ǰheení} &
`having known' &
\textit{ǰheenílu} &
`knew'\\\lspbottomrule
\end{tabularx}
\label{tab:3-17}
\end{table}


Umlaut is also, as can be seen in \tabref{tab:3-18}, applied to verbal suffixes (or the final part of the perfective stem) in anticipation of a~following adjectival feminine agreement suffix -\textit{i}.



\begin{table}[ht]
\caption{Umlaut in verbal suffixes anticipating feminine agreement suffixes}
\begin{tabularx}{\textwidth}{ l@{\hspace{30pt}} Q l@{\hspace{30pt}} Q }
\lsptoprule
\multicolumn{2}{l}{Form without umlaut} &
\multicolumn{2}{l}{Form with umlaut}\\\hline
\textit{mhaaráanu} &
`(he) is killing' &
\textit{mhaaréeni} &
`having killed'\\
\textit{phaaláanu} &
`(he) is splitting' &
\textit{phaaléeni} &
`having splitted'\\
\textit{phooṭóolu} &
`broke (\textsc{msg)}' &
\textit{phooṭéeli} &
`broke (\textsc{fsg)}'\\
\textit{mučóolu} &
`opened (\textsc{msg)}' &
\textit{mučéeli} &
`opened (\textsc{fsg)}'\\
\textit{láadu} &
`found (\textsc{msg)}' &
\textit{léedi} &
`found (\textsc{fsg)}'\\
\textit{nikháatu} &
`(he) appeared' &
\textit{nikhéeti} &
`(she) appeared'\\\lspbottomrule
\end{tabularx}
\label{tab:3-18}
\end{table}


Also in the adjectival stem (\tabref{tab:3-19}), we find umlaut in anticipation of a~feminine agreement suffix. 


\clearpage


\begin{table}[ht]
\caption{Umlaut in adjectival stems anticipating feminine agreement suffixes}
\begin{tabularx}{\textwidth}{ l@{\hspace{20pt}} Q l@{\hspace{20pt}} Q }
\lsptoprule
\multicolumn{2}{l}{Form without umlaut} &
\multicolumn{2}{l}{Form with umlaut}\\\hline
\textit{paṇáaru} &
`white \textsc{(msg)}' &
\textit{paṇéeri} &
`white \textsc{(fsg)}'\\
\textit{táatu} &
`hot \textsc{(msg)}' &
\textit{téeti} &
`hot \textsc{(fsg)}'\\
\textit{sóoru} &
`fine, whole (\textsc{msg)}' &
\textit{séeri} &
`fine, whole (\textsc{fsg)}'\\\lspbottomrule
\end{tabularx}
\label{tab:3-19}
\end{table}


Umlaut is also applied to derivations of various kinds (\tabref{tab:3-20}) in which the derivational suffix contains \textit{i}.



\begin{table}[ht]
\caption{Umlaut in derivations}
\begin{tabularx}{\textwidth}{ l@{\hspace{25pt}} Q l@{\hspace{25pt}} Q }
\lsptoprule
\multicolumn{2}{l}{Corresponding form without umlaut} &
\multicolumn{2}{l}{Derived form with umlaut}\\\hline
\textit{káaku} &
`older brother' &
\textit{kéeki} &
`older sister'\\
\textit{kuṇóoku} &
`puppy' &
\textit{kuṇéeki} &
`female dog'\\
\textit{ɡhwaaṇaá} &
`Pashto (language)' &
\textit{ɡhweeṇíi} &
`Pashtun (person)'\\
\textit{bakaraál} &
`shepherd' &
\textit{bakareelí} &
`shepherding'\\\lspbottomrule
\end{tabularx}
\label{tab:3-20}
\end{table}