\chapter{Nouns}
\label{chap:4}

\section{The noun and its properties}
\label{sec:4-1}


Distributionally (i.e., syntactically) the noun functions as the head of a~noun phrase argument. Within the noun phrase, the head is placed finally, as in `man' in example (\ref{ex:4-1}), and preceding modifiers agree with the head in gender, number and case (for further details, see Chapter \sectref{chap:10}).


\begin{exe}
\ex
\label{ex:4-1}
%modified
\gll aní dúu dhríɡ-a \textbf{míiš-a} \\
	\textsc{prox} two tall-\textsc{mpl} man-\textsc{pl} \\
\glt `these two tall men' (A:ADJ048)
\end{exe}

On the clause level, such a~phrase, where a~noun is the head, can serve as a~subject, as in (\ref{ex:4-2}), or a~direct object, as in (\ref{ex:4-3}).

\begin{exe}
\ex
\label{ex:4-2}
%modified
\gll \textbf{se} \textbf{kúṛi} búd-i ki anú míi bharíiw na\\
	\textsc{def} woman understand.\textsc{pfv-f} \textsc{compl} \textsc{prox.msg.nom} \textsc{1sg.gen} husband \textsc{neg}\\
\glt `The woman understood that this wasn't her husband.' (A:WOM646)

\ex
\label{ex:4-3}
%modified
\gll teeṇíi čúti-m de \textbf{baṭ} uc̣h-áan-u \\
	\textsc{refl} paw-\textsc{pl} give.\textsc{cv} stone pick.up-\textsc{prs"=msg} \\
\glt `It [the leopard] picked up a~stone with his paws.' (B:SHB749)
\end{exe}

It can also be the head of a postpositionally marked noun phrase, as in (\ref{ex:4-4}), a noun phrase expressing location, as in (\ref{ex:4-5}), or a nominal predicate, as in (\ref{ex:4-5b}).

\begin{exe}
\ex
\label{ex:4-4}
%modified
\gll \textbf{áa} \textbf{baṭ-á} ǰhulí dhreéɡ de \\ 
\textsc{idef} stone-\textsc{obl} on stretched.out be.\textsc{pst} \\
\glt `She stretched out on top of a~stone.' (A:BRE009)

\ex
\label{ex:4-5}
%modified
\gll \textbf{hasé} \textbf{díiš-a} hateeṇ-ú ɣam bhíl-u \\
	\textsc{rem} village-\textsc{obl} such-\textsc{msg} grief become.\textsc{pfv"=msg} \\
\glt `There was such grief in the village.' (B:AVA221)

\ex
\label{ex:4-5b}
%modified
\gll \textbf{lhoók-u} \textbf{díiš} de \\
small-\textsc{msg} village be.\textsc{pst} \\
\glt `It was a~small village.' (A:JAN003)
\end{exe}


Phonologically, the most frequent Palula noun stem (i.e., the nominative singular form) in my data consists of two syllables, comprising about half of all nouns in my database, 30 per cent are monosyllabic, and 16 per cent are three syllables. Four"=syllable nouns are rare in my database, and possibly most if not all of these are either derived from three"=syllable words or are compounds of two noun stems. There is no evidence of noun stems exceeding four syllables. The minimal Palula noun stem consists of a~rhyme where the nucleus is built up by a~short vowel and a~coda consonant: \textit{uṭ} `camel' (B), \textit{uts} `well'. Though simple vowel words do exist in Palula, there are no such nouns.\footnote{A potential counter-example is \textit{(óo$\sim$)} `mouth', although the status of the nasalised vowel as a single segment is phonologically questionable.} A noun stem with a~consonant onset must have at least a~long vowel nucleus, or a~short vowel and a~consonant coda: \textit{bíi} `seed', \textit{kuḍ} `wall'. There are no nouns consisting of only consonant onset and a~short vowel, though there are such words belonging to other parts of speech.


Morphologically, the prototypical Palula noun is inflected for number (\sectref{sec:4-4}) and case (\sectref{sec:4-5}), which is intimately related with declensional membership (\sectref{sec:4-6}) as well as inherent grammatical gender (\sectref{sec:4-3}). 


\section{Noun morphology}
\label{sec:4-2}

\subsection{Inflectional morphology}
\label{subsec:4-2-1}

Palula has two grammatical genders: masculine and feminine. Gender is an~inherent, lexically defined, property of the noun and is partly predictable on semantic and formal grounds (see \sectref{sec:4-3}). Nouns are inflected to show number (\sectref{sec:4-4}) and case  (\sectref{sec:4-5}). An example paradigm is displayed in \tabref{tab:4-nouns}, representing one of a handful of Palula noun declensions (\sectref{sec:4-6}). 


\begin{table}[ht]
\caption{Inflection of nouns}
\begin{tabularx}{\textwidth}{ Q Q Q Q Q }
\lsptoprule
&
Nominative &
Oblique &
Genitive &
\\\hline
Singular &
\textit{c̣híitr} &
\textit{c̣híitr-a} &
\textit{c̣híitr-ii} &
`field' (\textsc{m})\\
Plural &
\textit{c̣híitr-a} &
\textit{c̣híitr-am} &
\textit{c̣híitr-am-ii} &
\\\lspbottomrule
\end{tabularx}
\label{tab:4-nouns}
\end{table}


As far as case inflection is concerned, the primary distinction is between nominative (\sectref{subsec:4-5-1}) and oblique (\sectref{subsec:4-5-2}) case, but other inflectionally expressed cases, such as genitive (\sectref{subsec:4-5-3}), will also be discussed. Gender, number and case control agreement within the noun phrase (\sectref{sec:10-3}), and therefore have relevance also for adjectives, demonstratives and numerals. Gender and number have further relevance for clause"=level agreement patterns (\sectref{sec:11-1}--\sectref{sec:11-2}). Definiteness and deixis are central components of the noun phrase but are not part of the inflectional system of the noun and will be discussed in \sectref{sec:5-2}.


As in many other NIA languages, the Palula nominal paradigm is built up from a~combination of inherited synthetic elements, ``new'' agglutinative elements and analytical elements \citep[212]{masica1991}. The declensional classes (\sectref{sec:4-6}) suggested in this chapter are mainly based on the formation of the plural, but case forms as well as gender play a~role. On functional grounds, this chapter includes slightly more than what would typically be included under the heading of morphology proper.


\subsection{Derivational morphology}
\label{subsec:4-2-2}

There are only a few regular processes by which nouns are derived morphologically from other parts of speech. The clearest examples are nominalisations of verbs, although it should be kept in mind that those are usually occurring in specialised grammatical constructions, such as the Verbal Noun (described in \sectref{subsec:9-3-3}), occurring as the non-finite predicate in a number of dependent clauses, and the Agentive Verbal Noun (see \sectref{subsec:9-3-4}), the latter which is for instance used in forming an inceptive construction.


The derivation of deadjectival nouns is idiosyncratic, and semi"=regular at best, with a few pairs pointing to what might earlier have been productive morphological processes: \textit{šidal-aár} `coldness' < \textit{šidáalu} `cold'; \textit{taapi-aál} `warmth' < \textit{táatu} `warm, hot'.     


Certain adjectives, especially dimensional adjectives modifying humans, and numerals can also be used as heads of noun phrases. Apart from the application of case forms normally reserved for nouns, there is no morphology \textit{per se} involved in such derivation of nouns from adjectives or numerals (see \sectref{subsec:6-3-2} and \sectref{subsec:6-4-2}). 


\section{Gender}
\label{sec:4-3}

A feature associated with, or assigned to, each noun is its grammatical gender. The language has two grammatical genders; each noun is either \textit{masculine}, see examples in (\ref{ex:4-m}), or \textit{feminine}, see examples in (\ref{ex:4-f}), following the general tradition in describing IA languages. 


\begin{exe}
\extab
\label{ex:4-m}
\begin{tabularx}{\textwidth}{ l l }
\multicolumn{2}{l}{Masculine nouns:}\\
\textit{phoó} &
`boy' \\
\textit{ɡhúuṛu} &
`horse, stallion'\\
\textit{ɡhoóṣṭ} &
`house'\\
\end{tabularx}

\extab
\label{ex:4-f}
\begin{tabularx}{\textwidth}{ l l }
\multicolumn{2}{l}{Feminine nouns:}\\
\textit{kéeki} &
`older sister'\\
\textit{phúti} &
`mosquito'\\
\textit{hiimaál} &
`glacier'\\
\end{tabularx}
\end{exe}


In my database, which comprises about 1,300 nouns,\footnote{In the A dialect. There is an~additional database with about 400 B dialect nouns, many of them overlapping with the former.} masculine nouns were slightly more numerous than feminine nouns with 58 per cent masculine as compared to 42 per cent feminine.


There are no traces here of the three"=gender system of OIA, still present in some western NIA languages \citep[220--221]{masica1991}. As in many of the two"=gender systems found in NIA, it is mainly the old masculine and neuter that have merged. Although gender has otherwise been fairly stable as far as inherited vocabulary is concerned, there has been a~rather radical restructuring of the old declensional system, partly as a~result of segmental loss, a~phenomenon we will have reason to discuss further in \sectref{sec:4-6}.


While gender is an~inherent and classificatory property of the noun lexeme itself, it is essentially established through morphological agreement with adjectives/demonstratives (\ref{ex:4-6}) and verbs (\ref{ex:4-7}), for which it is a~variable property. 


\begin{exe}
\ex
\label{ex:4-6}
\gll hasó bidráaɡ-u kuṇaák \\
	\textbf{\textsc{rem.msg.nom}} sick-\textbf{\textsc{msg}} child[\textbf{\textsc{m}}] \\
\glt `that sick child' (B:ATI057)
\end{exe}

\begin{exe}
\ex
\label{ex:4-7}
\gll phaí na yhéel-i \\
	girl[\textbf{\textsc{f}}] not come.\textsc{pfv-\textbf{\textsc{f}}} \\
\glt `The girl didn't come.' (A:SHY058)
\end{exe}

\subsection{Gender assignment}
\label{subsec:4-3-1}


There are plenty of semantic and formal clues to gender assignment in Palula. Looking again at the examples in (\ref{ex:4-m}) and (\ref{ex:4-f}), we do not have a~problem making intelligent guesses as to the gender of the words for `boy' and `older sister', respectively, if we assume that there is a~connection between biological gender or sex and this particular grammatical two"=gender differentiation between feminine nouns and masculine nouns \citep[102]{dahl2000}. Those nouns, as is the case for other nouns denoting humans and higher animates, are indeed assigned gender according to meaning~-- masculine gender if male, and feminine if female, a~phenomenon we refer to as semantic gender assignment \citep[7--32]{corbett1991}. For the word `horse, stallion', we can similarly draw the conclusion that it is a~masculine noun (especially if we are told that another word is used to refer to a~mare). The other three words, referring to either lower animates, inanimate objects or phenomena, have also~-- some way or another~-- been assigned to one of the two genders: \textit{hiimaál} and \textit{phúti} are feminine nouns and \textit{ɡhoóṣṭ} is a~masculine noun. We cannot, however, see any obvious semantic reason for this to be so, and we will therefore have to look at other assigning criteria.


Male--female word pairs with a~common lexical ``root'' are frequent in the Palula lexicon, especially as kinship terms, and the derivation of a~feminine counterpart from a~masculine (except in a~few cases when it may be the other way around) could be described as a~rather productive state of affairs, even synchronically. Some examples are given in \tabref{tab:4-1}.


Beside the obvious correlation between biological and grammatical gender, we have a~somewhat related but less regular connection between gender and relative size/power. Alternatively there is some kind of complementarity implied. This is mainly detectable in lexical pairs of the kind displayed in \tabref{tab:4-2}, which although similar to the male--female pairs in \tabref{tab:4-1}, are related to other scales than pure masculinity versus femininity and also with a~more approximate similarity in kind. While this derivative process (in its essence very much like diminutive formation) may have been productive in the past, it is uncertain how productive this extended gender differentiation is among today's Palula speakers.



\begin{table}[ht]
\caption{Male/female pairs vis-à-vis gender}
\begin{tabularx}{\textwidth}{ l Q l Q }
\lsptoprule
Masculine &
&
Feminine &
\\\hline
\textit{práaču} &
`male guest' &
\textit{préeči} &
`female guest'\\
\textit{dóodu} &
`paternal grandfather' &
\textit{déedi} &
`paternal grandmother'\\
\textit{phoó} &
`boy' &
\textit{phaí} &
`girl'\\
\textit{phóopu} &
`father's sister's husband' &
\textit{phéepi} &
`father's sister'\\
\textit{jhambróoṛu} &
`bridegroom' &
\textit{ǰhambréeṛi} &
`bride'\\
\textit{khaár} &
`donkey' &
\textit{khári} &
`female donkey'\\
\textit{púšu} &
`tom"=cat' &
\textit{púši} &
`cat'\\\lspbottomrule
\end{tabularx}
\label{tab:4-1}
\end{table}

For most higher animates, the masculine noun is the ``default'' gender, and the feminine, when used, is a~marked form (compare with \sectref{subsec:4-3-3}). Practically, that means that either only a~masculine form is available or that the masculine member in a~masculine--feminine pair is used when no specification is needed. However, in a~few cases, the feminine is the default, even when there is a~distinct masculine form available, such as \textit{luumái} `fox' vs. \textit{luumóo} `male fox', and the above exemplified (\tabref{tab:4-1}) pair \textit{púši} `cat' vs. \textit{púšu} `tom"=cat' (compare with \citealt[103--104]{dahl2000}). A different relation holds between masculine \textit{čhaál} `goat kid' and feminine \textit{čhéeli} `goat', where the feminine noun is not the female counterpart to the masculine but instead the generic term for goat, whereas the masculine is used to refer to the young kid regardless of sex.


\begin{table}[ht]
\caption{Masculine/feminine lexical pairs}
\begin{tabularx}{\textwidth}{ l Q l Q }
\lsptoprule
Masculine &
&
Feminine &
\\\hline
\textit{phútu} &
`fly' &
\textit{phúti} &
`mosquito'\\
\textit{khaláaṛu}
&
`large leather bag, made from skin of a~he"=goat' &
\textit{khaléeṛi}
&
`smaller leather bag, made from the skin of a~she"=goat'\\
\textit{anɡúṛu} &
`thumb, big toe' &
\textit{anɡúṛi} &
`finger, toe'\\
\textit{ac̣hibáaṛu} &
`eyebrow' &
\textit{ac̣hibéeṛi} &
`eyelashes'\\\lspbottomrule
\end{tabularx}
\label{tab:4-2}
\end{table}

Now we will turn to formal properties that may stand in relation to gender association. First, a~comparison of a~large number of nouns reveals that particular phonological properties really seem to be related to either of the two genders. A word"=final \textit{u} is for instance a~formal feature of a~major group of masculine nouns, as are \textit{á} and \textit{oó} for smaller groups of nouns that are all masculine (\ref{ex:4-m2}).


\begin{exe}
\extab
\label{ex:4-m2}
\begin{tabularx}{\textwidth}{ l l l }
\textbf{final u} &
\textit{áanɡu} &
`sickle' (\textsc{m})\\
\textbf{final á} &
\textit{kuṇḍá} &
`hook, peg' (\textsc{m})\\
\textbf{final oó} &
\textit{muuṣoó} &
`elbow' (\textsc{m})\\
\end{tabularx}
\end{exe}

A word"=final unaccented \textit{i} is in the same way a~feature of a~very large group of feminine nouns, and \textit{ái} for a~smaller group of them (\ref{ex:4-f2}).


\begin{exe}
\extab
\label{ex:4-f2}
\begin{tabularx}{\textwidth}{ l l l }
\textbf{final i} &
\textit{thúuṇi} &
`pillar' (\textsc{f})\\
&
\textit{súuri} &
`sun' (\textsc{f})\\
\textbf{final ái} &
\textit{kooɡái} &
`cheek' (\textsc{f})\\
\end{tabularx}
\end{exe}

This kind of correlation between phonological properties and gender, not uncommon in IA \citep[219]{masica1991} and referred to as ``overt gender''\footnote{Alternative terms offered by \citet[219]{masica1991} are \textit{thematic, strong thematic, extended, augmented} or \textit{enlarged}.} (see also \citealt[44, 62]{corbett1991}), is of course significant, but whether we should talk about ``gender marking'' is not equally clear. We can hypothesise that \textit{-u} is a~nominative case marker, available for some nouns like \textit{ɡhúuṛu}, for which segmental material is added to what we may view as a~root \textit{ɡhúuṛ-} when inflecting for case and number. For a~large number of feminine nouns with a~final \textit{i} (including \textit{ɡhúuṛi} `mare'), we may in addition to that describe them as derived from such a~masculine root with the help of a~derivational affix \textit{-i} denoting something female (or something smaller of approximately the same kind). In a~diachronic perspective both seem in fact to go back to OIA derivational suffixes with~-- at least partly~-- diminutive senses: a~masculine \textit{-aka-} and a~feminine \textit{-ikā-} (\citealt[222]{masica1991}; \citealt[15]{morgenstierne1941}; \citealt[29]{buddruss1967}).


\begin{table}[ht]
\caption{Toponyms and gender assignment}
\begin{tabularx}{\textwidth}{ l Q l Q }
\lsptoprule
Masculine &
&
Feminine &
\\\hline
\textit{atshareét} &
`Ashret' &
\textit{ṭiṭlái} &
`Titley'\newline (in Sharadesh)\\
\textit{c̣hatróol} &
`Chitral' &
\textit{bhiúuṛi} &
`Biori'\\
\textit{buzeeɡhaá} &
`Buzegha'\newline (hamlet in Ashret) &
\textit{lawaṇí} &
`Lawani'\newline (in Sharadesh)\\
\textit{baṭsúm} &
`Batsum'\newline (settlement in Ashret) &
\textit{meeṭhíl} &
`Meethil'\newline (field in Ashret)\\\lspbottomrule
\end{tabularx}
\label{tab:4-3}
\end{table}


Also proper nouns referring to inanimates, such as toponyms (\tabref{tab:4-3}), have gender assigned to them. Gender assignment here seems to be primarily morphologically determined, but since phonology is interrelated with morphological behaviour there is no surprise that those toponyms that end with \textit{i} or \textit{ai} are feminine, whereas most of the rest are masculine. In some cases the etymology is overt enough to connect the last part of a~toponym with a~proper noun with a~particular and known gender (such as \textit{-deéš} in \textit{šaṛadeéš}, which is related to the masculine noun \textit{díiš} `village').


There are also, as we shall see in \sectref{sec:4-6}, obvious connections between noun morphology and gender, or what is referred to as morphological gender assignment \citep[34--50]{corbett1991}, which is really a~matter of co"=variation rather than one feature necessarily being primary and the other secondary. Some declensions and subdeclensions consist entirely or to a~large extent of feminine or masculine nouns, whereas for other declensions the gender distribution is quite even. Gender can, for instance, be predicted to some degree from the plural formation or the case inflections applied to a~particular noun.


We can thus say that an~interplay of semantic, phonological and morphological properties make a~noun masculine or feminine, or at least tells us whether a~particular noun is more likely to be one or the other. 


\subsection{Gender stability and consistency}
\label{subsec:4-3-2}


An issue that deserves some attention is the relative stability and variability of gender in Palula. First, across the two main dialects, there is almost no variation at all,\footnote{In my present database I have only eight nouns that differ in gender assignment between A and B.} at least as far as words with a~longer history in the language are concerned. This almost complete correspondence is on the other hand not surprising, considering that the noun paradigms and the declensions are near"=identical in those two varieties. Comparing with something less closely related, the Shina variety of Gilgit, we observe (see \tabref{tab:4-4}) more divergence between that variety and Palula as far as gender assignment is concerned. If we limit our comparison to inanimate and lower animate cognate nouns, and exclude higher animates, which are all assigned gender according to rather transparent semantic criteria, as well as what we with confidence can regard as recent loans, we come up with a 79 per cent correspondence between gender in Palula and gender in Gilgiti. This is still quite a~high figure, considering that Palula and Gilgiti speakers most likely have not had any shared development or even any non"=trivial contact for at least a~few centuries (see \sectref{subsec:1-2-3}).



\begin{table}[ht]
\caption{Gender in Palula and Gilgiti Shina cognate nouns compared (word lists from \citealt{bailey1924} and \citealt{radloff1999}): items (\%)}
\begin{tabularx}{.5\textwidth}{ Q Q Q }
\lsptoprule
&
\multicolumn{2}{l}{Gilgiti} \\
&
\textsc{m} &
\textsc{f}\\\hline
Palula &
&
\\
\textsc{m} &
63(41) &
17(11)\\
\textsc{f} &
15(10) &
58(38)\\\lspbottomrule
\end{tabularx}
\label{tab:4-4}
\end{table}

\subsection{Gender markedness}
\label{subsec:4-3-3}

Morphologically and otherwise, the feminine tends to be the marked gender in Palula. Many nouns with an~ending \textit{i} are derived from a~masculine noun (see \tabref{tab:4-1}), while the opposite seems to be extremely rare (\textit{phóopu} from \textit{phéepi} is a~possible example). Also, the most common complement"=taking verbs agree, by default, with the masculine singular (see \sectref{subsec:13-5-1}) regardless of the content of the complement clause. There is similarly an~indication that in other cases of agreement without an~accessible controller, the masculine singular form of the target is chosen. Perhaps the obligatory number distinction in agreement with masculine heads vis-à-vis the optionality as far as feminine heads are concerned (see \sectref{sec:10-3}) is related to this as well. These observations suggest that the ``default'' in Palula is the masculine. Interestingly, this stands in direct contrast to the feminine as the default gender in Gilgiti Shina \citep[176]{hookzia2005}. This, however, is a~topic for further studies.

\section{Number}
\label{sec:4-4}

Number is one of the basic, and possibly the most straightforward, of the inflectional categories related to nouns in Palula. There are two number categories, singular and plural, with no traces of the OIA dual category. There are a~number of different plural formations in the language: \textit{-a, -i, -m, -ee} and \textit{-aan}, as seen in \tabref{tab:4-5}, some of them with additional stem changes. As these inflections (along with oblique case inflections) form the basis of the declensional distinctions, the various forms will be discussed in more detail in \sectref{sec:4-6}. 

Some groups of nouns (\sectref{subsec:4-4-1}--\sectref{subsec:4-4-3}) are special in that they normally do not pluralise or display a~regular contrast between a~singular and a~plural form.



\begin{table}[ht]
\caption{Examples of plural formation}
\begin{tabularx}{\textwidth}{ Q Q Q Q }
\lsptoprule
Singular &
&
Plural &
\\\hline
\textit{bhit} &
`plank' &
\textit{bhíta} &
`planks'\\
\textit{áanɡu} &
`sickle' &
\textit{áanɡa} &
`sickles'\\
\textit{kitaáb} &
`book' &
\textit{kiteebí} &
`books'\\
\textit{saaréeṇi} &
`wife's sister' &
\textit{saaréeṇim} &
`wife's sisters'\\
\textit{ǰandoó} &
`male goat' &
\textit{ǰandeé} &
`male goats'\\
\textit{duṣmaán} &
`enemy' &
\textit{duṣmanaán} &
`enemies'\\\lspbottomrule
\end{tabularx}
\label{tab:4-5}
\end{table}

\subsection{Non"=count nouns}
\label{subsec:4-4-1}


Non"=count or mass nouns normally do not have plural forms, nor are they modified by numerals. Many of them refer to what is perceived as a~substance. Some examples are provided in \tabref{tab:4-ncount}.



\begin{table}[H]
\caption{A selection of non"=count nouns}
\begin{tabularx}{\textwidth}{ Q Q Q Q }
\lsptoprule
\textit{wíi} &
`water' &
\textit{ɡhiíṛ} &
`ghee'\\
\textit{c̣hiír} &
`milk' &
\textit{anɡóor} &
`fire'\\
\textit{číčal} &
`mud, sand' &
\textit{kir} &
`snow'\\
\textit{lhoóṇ} &
`salt' &
\textit{mhaás} &
`meat'\\
\textit{póoṣ} &
`dung' &
\textit{rúǰi} &
`rice'\\
\textit{paaṇṭí} &
`clothes' &
\textit{lheéṣ} &
`plaster'\\\lspbottomrule
\end{tabularx}
\label{tab:4-ncount}
\end{table}

Such nouns are usually quantified by adjectives with the meaning `much', `some', `a little', a~measuring unit or a~noun referring to some sort of container or a~quantifiable part of the whole, as in (\ref{ex:4-8}) and (\ref{ex:4-9}). 

\begin{exe}
\ex
\label{ex:4-8}
%modified
\gll máa=the tuúš \textbf{c̣hiír} da \\
	\textsc{1sg.nom}=to some milk give.\textsc{imp.sg} \\
\glt `Give me some milk!' (A:HLE2298)
\end{exe}

\begin{exe}
\ex
\label{ex:4-9}
%modified
\gll máa=the dúu ɡileesí \textbf{c̣hiír} da \\
	\textsc{1sg.nom}=to two glass.\textsc{pl} milk give.\textsc{imp.sg} \\
\glt `Give me two glasses of milk!' (A:HLE2298)
\end{exe}

Only exceptionally, when speaking of more than one variety of a~substance, for instance, may a~mass noun be pluralised. Some mass nouns may also occur in a~plural form, as `blood' (\ref{ex:4-11}), as well as in a~singular form (\ref{ex:4-10}), while still being regarded as a~substance, sometimes with the connotation of large volumes or weights somehow divided up into several separate chunks or volumes. It would, however, still be ungrammatical to quantify such a~noun directly with a~numeral, whether in its singular or its plural form.

\begin{exe}
\ex
\label{ex:4-10}
%modified
\gll áak ṭip \textbf{ráat} \\
	one drop blood \\
\glt `a drop of blood' (A:HLE2334)
\end{exe}

\begin{exe}
\ex
\label{ex:4-11}
%modified
\gll tasíi múṭii \textbf{rat-á} nikháat-a \\
	\textsc{3sg.gen} arm.\textsc{gen} blood-\textsc{pl} appear.\textsc{pfv"=mpl} \\
\glt `Blood came from [many places of] his arm.' (A:HLE2335)
\end{exe}


Also, some abstract nouns (examples provided in \tabref{tab:4-abst}) are treated as non"=count nouns that do not normally pluralise. 


\begin{table}[H]
\caption{Examples of abstract non-count nouns}
\begin{tabularx}{\textwidth}{ l@{\hspace{40pt}} Q l@{\hspace{40pt}} Q }
\lsptoprule
\textit{insaáf} &
`justice' &
\textit{muúl} &
`value'\\
\textit{rahái} &
`desire, appetite' &
\textit{šid} &
`coldness'\\\lspbottomrule
\end{tabularx}
\label{tab:4-abst}
\end{table}

However, since non"=count nouns also occur in oblique forms (such as \textit{inseefí} from \textit{insaáf} `justice'), and the singular oblique in most Palula declensions is formally identical to a~nominative plural, there is usually no need to come up with a~novel plural form, since the step towards pluralisation is never far away, so to speak.


Other nouns have a~unique reference, at least in one or more senses of the word, and therefore normally do not occur pluralised, but must be considered as special cases of non"=count nouns: \textit{súuri} `sun', \textit{ɡhuaaṇaá} `Pashto (language)' and \textit{xudái} `God'.


\subsection{Collective nouns}
\label{subsec:4-4-2}


While the nouns dealt with above as non"=count nouns, often referring to substances or abstract concepts, are largely limited to the singular, a~few nouns (\ref{ex:4-coll}) have a~collective meaning and occur in a~plural form only, or almost exclusively so.


\begin{exe}
\extab
\label{ex:4-coll}
\begin{tabularx}{\textwidth}{ l@{\hspace{40pt}} Q }
\textit{bakáara} &
`(flock of) sheep and goats'\\
\textit{xálak, xálaka} &
`people'\\
\textit{ɡookh(u)rá} &
`cattle'\\
\end{tabularx}
\end{exe}

\subsection{Proper names}
\label{subsec:4-4-3}

A structural property typical of proper names is that they do not normally pluralise, as they are seen as having a~unique reference, tied to one person or to one place. There is, however, no such restriction on case inflection. As proper names can take on any function in the clause, they also inflect like other nouns for case, as can be seen in (\ref{ex:4-12}).

\begin{exe}
\ex
\label{ex:4-12}
%modified
\gll \textbf{fazelnuur-á} díi ba panǰ putr-á bhíl-a \\
	Fazal.Noor-\textbf{\textsc{obl}} from \textsc{top} five son-\textsc{pl} become.\textsc{pfv"=mpl} \\
\glt `Fazal Noor had five sons.' (B:ATI079)
\end{exe}

Complex names consisting of several elements are common and are treated as lexical units and often as single phonological words (see \sectref{subsec:10-1-3}).

\section{Case}
\label{sec:4-5}

Noun phrases are marked for case, either with noun suffixes, or~-- as far as pronouns are concerned~-- through distinct forms and also by cross"=referencing within the noun phrase between the noun head and its dependents. There is also cross"=referencing within the clause between the verb and one of its arguments (see Chapter \sectref{chap:11}).


Even though the following presentation focuses on case inflection rather than grammatical relations, it is, as \citet[230--231]{masica1991} phrases it, ``easily the most problematic nominal category in NIA''. Case in Palula can, as in many other NIA languages, be described as accumulative inflectional layers with case"=like functions, and the comparison between languages is complicated by the fact that a~function in a~given layer in one language is managed in a~different layer in another language. The actual case forms (mainly the oblique case) in Palula differ between the declensions as do the actual occurrences of certain case distinctions. 


The most basic formal distinction is that between the nominative (or direct) case and the oblique case, but even that is not realised in all declensions, but is, for instance, totally missing in the \textit{m}-declension (see \sectref{subsec:4-6-3}). Form syncretism \citep[27]{matthews1991} between nominative plural and oblique singular is found in most declensions, reminiscent of the paradigm of ``overtly'' masculine nouns in Urdu"=Hindi \citep[1]{schmidt1999}, or those of some typically masculine and typically feminine vowel"=ending nouns in Pashto \citep[726-728]{robsontegey2009}.


The only other inflectionally distinct case for nouns is the genitive, although it must be considered more peripheral than the basic distinction between nominative and oblique. The evidence for ergative, instrumental and vocative cases, respectively, will also be discussed below, whereas a~rather large number of case"=like functions expressed with postpositions will be discussed in \sectref{sec:7-2}.


\subsection{Nominative case}
\label{subsec:4-5-1}

The nominative is the form of the noun used as the citation form, the subject of intransitive verbs, as in (\ref{ex:4-13}), and as the direct object, as in (\ref{ex:4-14}), of most transitive verbs. In the non"=perfective categories, it is also the case with which subjects of transitive verbs occur, as in (\ref{ex:4-14b}). 


\begin{exe}
\ex
\label{ex:4-13}
%modified
\gll miír thaní áak \textbf{míiš} heensíl-u de\\
	Mir called one man exist.\textsc{pfv"=msg} \textsc{pst}\\
\glt `There was a~man called Mir.' (A:GHA051)
\end{exe}

\begin{exe}
\ex
\label{ex:4-14}
%modified
\gll haláal the \textbf{púustu} ɡhaḍ-í \\
	slaughter do.\textsc{cv} skin take.off-\textsc{cv} \\
\glt `After slaughtering it, he took off the skin.' (B:SHB732)
\end{exe}

\begin{exe}
\ex
\label{ex:4-14b}
%modified
\gll so \textbf{musaafár} šukhaáu teeṇíi huǰut-í pharé pail-óo de \\
	\textsc{def.msg.nom} traveller coat \textsc{refl} body-\textsc{obl} toward fold-\textsc{3sg} \textsc{pst} \\
\glt `The traveler was folding his cloak around him.' (A:NOR006)
\end{exe}


For all nouns ending in a~consonant, the nominative is identical to the noun stem, whereas for nouns ending in gender"=typical vowels, it could be argued that these vowels are in fact nominative case"=markers. The latter interpretation may be especially relevant for those nouns ending in \textit{u} and \textit{i}, for which there are distinct vocative forms (see \sectref{subsec:4-5-4}) without gender"=typical ``nominative'' endings.

\subsection{Oblique case}
\label{subsec:4-5-2}

For those nouns that take an~oblique case suffix, this is the form the noun occurs in when followed by a~postposition, as in (\ref{ex:4-15}), when serving as the agent of perfective transitive verbs, as in (\ref{ex:4-16}), and when a~noun is used as an~adverbial of time or place, as in (\ref{ex:4-17}). 


\begin{exe}
\ex
\label{ex:4-15}
%modified
\gll teṇ-teeṇíi \textbf{ɡhooṣṭ-áam} the búi\\
	\textsc{red}-\textsc{refl} house-\textsc{pl.obl} to go.\textsc{imp.pl} \\
\glt `Go each to your own houses!' (B:DHE5705)
\end{exe}

\begin{exe}
\ex
\label{ex:4-16}
%modified
\gll \textbf{míḍ-a} maníit-u ki beedawaá na bh-úuy=a \\
	ram-\textsc{obl} say.\textsc{pfv"=msg} \textsc{comp} impatient \textsc{neg} become-\textsc{imp.pl=q} \\
\glt `The ram said: Let's not be impatient!' (B:SHI005)
\end{exe}

\begin{exe}
\ex
\label{ex:4-17}
%modified
\gll muxáak zamanée \textbf{ɡalaṭúuk-a} ak méeš de \\
	before time.\textsc{gen} Kalkatak-\textsc{obl} one man be.\textsc{pst} \\
\glt `Once there was a~man in Kalkatak.' (B:FLW774)
\end{exe}

This is also the form used for the causee in causative constructions, as in the case of `son' in (\ref{ex:4-18}). 


\begin{exe}
\ex
\label{ex:4-18}
%modified
\gll ma teeṇíi \textbf{putr-á} čéi pila-áan-u \\
	\textsc{1sg.nom} \textsc{refl} son-\textsc{obl} tea give.to.drink-\textsc{prs"=msg} \\
\glt `I make my son drink tea.' (A:DHE6693)
\end{exe}

As mentioned above, only in some of the declensions (see \sectref{sec:4-6}) is there a~nominative--oblique differentiation. The singular oblique is formed by a~suffix \textit{-a, -í}, etc. added to the noun stem, alternatively by an~ending \textit{eé, a} etc. replacing the ending vowel of the nominative. In most declensions, the oblique singular is formally identical to the nominative plural. The oblique plural is formed by a~suffix \textit{-am, -óom, -íim}, or \textit{-eém} added to the noun stem, except for nouns forming a~plural with \textit{-aán}, for which the plural oblique suffix \textit{-óom} (or \textit{-úm}) may be attached subsequent to the plural suffix.

\subsection{Genitive case}
\label{subsec:4-5-3}

The genitive is used for the noun that heads a~possessive construction, see (\ref{ex:4-19})--(\ref{ex:4-20}), and for an~ablative function, either along with an~additional postposition, as in (\ref{ex:4-22}), or alone, as in (\ref{ex:4-21}).


\begin{exe}
\ex
\label{ex:4-19}
%modified
\gll \textbf{khanɡar-íi-e} záxum lab saáz bh-áan-u \\
	sword-\textsc{obl"=gen} wound quickly whole become-\textsc{prs"=msg} \\
\glt `The wound of a~sword heals quickly.' (B:PRB018)
\end{exe}

\begin{exe}
\ex
\label{ex:4-20}
%modified
\gll ma \textbf{šaak-úum-e} ɡhoóṣṭ saáz th-áan-u \\
	\textsc{1sg.nom} wooden-\textsc{obl.pl"=gen} house whole do-\textsc{prs"=msg} \\
\glt `I'm building a~wood house.' (B:DHE6733)
\end{exe}

\begin{exe}
\ex
\label{ex:4-21}
%modified
\gll ma \textbf{c̣hetrúul-e} wh-áand-u \\
	\textsc{1sg.nom} Chitral-\textsc{gen} come.down-\textsc{prs"=msg} \\
\glt `I'm coming [down] from Chitral.' (B:DHE4795)
\end{exe}

\begin{exe}
\ex
\label{ex:4-22}
%modified
\gll ma \textbf{kooḍɡháii} thíi yhóol-u \\
	\textsc{1sg.nom} Kotgha.\textsc{gen} from come.\textsc{pfv"=msg} \\
\glt `I came from Kotgha.' (A:HLE2265)
\end{exe}

The genitive is also used for the object in constructions with the verb \textit{ǰe-} `hit', see (\ref{ex:4-23}). The background of this special case may possibly be explained (as \citealt[43]{baart1999a}, has done for Kohistani Gawri) with an~implicit object with the nominal meaning `hit'.

\begin{exe}
\ex
\label{ex:4-23}
%modified
\gll ma paalaá ɡhin-í \textbf{phút-am-e} ǰ-áan-u \\
	\textsc{1sg.nom} leaf take-\textsc{cv} fly-\textsc{obl.pl"=gen} hit-\textsc{prs"=msg} \\
\glt `I'm driving away flies with leaves.' (B:DHN4851)
\end{exe}

The genitive case inflection is less variable than the oblique. In A, the genitive is formed with an~invariable \textit{-ii} (accented or unaccented, depending on declensional membership), whereas in B it has one accented form \textit{-í} and one unaccented \textit{-e} (also depending on declensional membership). It can be argued that the genitive belongs in a~layer outside of or based on the oblique. This makes most sense for the genitive plural, which is formed by attaching \textit{-ii} to the oblique plural form of the noun: \textit{-am + -ii {\textgreater} -amii, -íim + -ii {\textgreater} -íimii, -óom + -ii {\textgreater} -óomii}. The genitive singular on the other hand is usually the noun stem followed by the genitive suffix, save for the nouns of the \textit{i}-declension in B, where the (strengthened) oblique singular suffix mediates between the stem and the genitive suffix: \textit{ḍheer-íi-e} `belly-\textsc{obl"=gen}'. It is, therefore, possible that an~old genitive plural has been replaced by this new ``peripheral'' genitive, formed analogically from the singular genitive.

\subsection{Other cases or case"=like categories}
\label{subsec:4-5-4}

A few other case categories should also be mentioned, although they are rather marginal or seldom used. 


One of the expressions of an \textit{instrumental} function, or possibly case, is (as far as I have evidence) formally identical to the oblique plural and occurs in the form of `gun' in example (\ref{ex:4-24}), regardless of number reference. 


\begin{exe}
\ex
\label{ex:4-24}
%modified
\gll se míiš-a ba huṇḍíi thíi se bhalá-ii \textbf{toobak-íim} ǰít-i \\
	\textsc{def} man-\textsc{obl} \textsc{top} from.above from \textsc{def} spirit-\textsc{gen} gun-\textsc{ins} hit.\textsc{pfv-f}\\
\glt `The man shot from above with his gun at the spirit.' (A:WOM671)
\end{exe}

This example of case syncretism is in itself a~hint that the origin of the ergative marking is in the instrumental. However, instrumentality is also, and more frequently, expressed with a~converb (see \sectref{subsec:13-4-1}), such as \textit{ɡhiní} from \textit{ɡhin-} `take' (as in example (\ref{ex:4-23}) above).


Unique \textit{vocative} forms (whether we regard it as a~case or not) of nouns only occur with some kinship terms, as in (\ref{ex:4-25})--(\ref{ex:4-26}). These forms are shorter than the corresponding nominative forms, appearing without their ``gender signaling'' \textit{u}- or \textit{i}-endings.


\begin{exe}
\ex
\label{ex:4-25}
%modified
\gll \textbf{phéep} séer-i=ee \\
	father's.sister.\textsc{voc} fine-\textsc{f=q} \\
\glt `How are you auntie [politely addressing any middle"=aged woman]?' (A:HLE3088) [\textit{phéepi} in \textsc{nom.sg}]
\end{exe}

\begin{exe}
\ex
\label{ex:4-26}
%modified
 \gll \textbf{dóod} sóor-u=ee \\
	father's.father.\textsc{voc} fine-\textsc{msg=q} \\
\glt `How are you grandpa [politely addressing any aged man]?' (A:HLE3090) [\textit{dóodu} in \textsc{nom.sg}]
\end{exe}

A great number of other case"=like meanings are expressed with the help of postpositions, usually
preceded by the noun in its oblique case form (see \sectref{sec:7-2} for a~further discussion). Postpositions
tend to receive stress like other free morphemes, but in a~few cases, where the postposition is
de"=accented, we may see the early grammaticalisation of a~postposition into a~case suffix. This is
especially true of the postposition \textit{wée} `in, into', as can be seen in (\ref{ex:4-27}),
which in some speech styles is phonologically fused with the singular oblique suffix, possibly
developing into a~more specialised locative case suffix.

\begin{exe}
\ex
\label{ex:4-27}
%modified
 \gll so \textbf{meedóon"=ee} (=meedóon-a wée) nikháat-u \\
	he field\textsc{-``locative''} [=field-\textsc{obl} in] appear.\textsc{pfv"=msg} \\
\glt `He appeared in the field.' (A:ROP003)
\end{exe}

\section{Declensions}
\label{sec:4-6}

  The formal realisations of the categories~-- gender, number and case~-- can be grouped into
  inflectional classes, which I will refer to as noun declensions. There are three main declensions (to which between 80 and 90 per cent of all nouns belong)
  and a~few minor ones. These are primarily
  based on the various plural formations (hence the
  declension labels, ``\textit{a}-declension'', etc., given here) and to a~lesser degree on oblique
  case forms. \citet[219]{masica1991} points out that gender in NIA ``often entails declensional
  differences''. That is true to some extent also with Palula, but there is far from any one"=to"=one
  mapping between gender and declensional affinity. A general overview of the various declensions is displayed in \ref{tab:4-5b}, their characteristic number and case formations, and the gender categories that are represented within each of them.  


\begin{table}[ht]
\caption{Noun declensions, an overview}
\begin{tabularx}{\textwidth}{ Q Q Q Q Q Q}
\lsptoprule
Declension &
Singular &
&
Plural &
&
Gender \\
&
Nominative &
Oblique &
Nominative &
Oblique
\\\hline
\textit{Major:}\\
\textit{a}-decl &
\textit{Ø} &
\textit{-a} &
\textit{-a} &
\textit{-am} &
\textsc{m/f} \\
\textit{i}-decl &
\textit{Ø} &
\textit{-í} &
\textit{-í} &
\textit{-íim} &
\textsc{f/m} \\
\textit{m}-decl &
\textit{Ø} &
\textit{Ø} &
\textit{-m} &
\textit{-m} &
\textsc{f} \\
\textit{Minor:}\\
\textit{ee}-decl &
\textit{Ø} &
\textit{Ø} &
\textit{-eé} &
\textit{-eém} &
\textsc{m}\\
\textit{aan}-decl &
\textit{Ø} &
\textit{-á} &
\textit{-aán} &
\textit{-aanóom} &
\textsc{m} \\\lspbottomrule
\end{tabularx}
\label{tab:4-5b}
\end{table}

  
  Below I present the declensions, one by one;
  there is a~short characterisation of each (including phonological forms and any correlation to
  semantic content), its relative frequency,\footnote{Based on a~database with about 1,700 nouns.} a~number of examples, important subgroups, and some suggestions as to the historical development as far as it is traceable.


\subsection{\textit{a-}declension}
\label{subsec:4-6-1}


Nouns belonging to the \textit{a}-declension form their plural and oblique case with an \textit{a}-suffix, as is evident from \tabref{tab:4-6}. This is the largest declensional class as far as my data goes with 50 per cent of my nouns belonging to this particular declension. A clear majority (79 per cent) of them are masculine. A great number of toponyms are also included.


\begin{table}[ht]
\caption{\textit{a}-declension nouns}
\begin{tabularx}{\textwidth}{ Q : Q Q : Q Q }
\lsptoprule
\multicolumn{2}{l}{Singular} & \multicolumn{2}{l}{Plural}\\
\multicolumn{1}{l}{Nominative} &
Oblique &
\multicolumn{1}{l}{Nominative} &
Oblique &
\\\hline
\textit{kráam} &
\textit{kráam-a} &
\textit{kráam-a} &
\textit{kráam"=am} &
`work' (\textsc{m})\\
\textit{ṣinɡ} &
\textit{ṣínɡ-a} &
\textit{ṣínɡ-a} &
\textit{ṣínɡ-am} &
`horn' (\textsc{m})\\
\textit{c̣hatróol} &
\textit{c̣hatróol-a} &
&
&
`Chitral' (\textsc{m})\\
\textit{bíi} &
\textit{bíi-a} &
\textit{bíi-a} &
\textit{bíi-am} &
`seed' (\textsc{f})\\\lspbottomrule
\end{tabularx}
\label{tab:4-6}
\end{table}


Quite a~number of nouns in this declension can be traced back to the large OIA declension of
masculine and neuter nouns ending in \textit{a}. The OIA neuter nouns have largely fused with the
masculine. In the nominative singular there is a~regular loss of the final OIA segment
\textit{a(s)}, as in a~number of other NIA languages \citep[222]{masica1991}, while the Palula
plural inflection \textit{-a} may reflect one or more of the OIA dual or plural forms that include
a~long \textit{ā: k\'{\={a}}māu, k\'{\={a}}mās, k\'{\={a}}mān,
  k\'{\={a}}mānām} (of \textit{k\'{\={a}}ma} `love', see \citealt[330]{whitney1960}). I
would like to suggest the diachronic developments (partly based on \citealt{turner1966}) in (\ref{ex:4-ety1}).

\begin{exe}
\extab
\label{ex:4-ety1}
\begin{tabular}{ l }
\textit{anɡóor} `fire' (\textsc{m}) {\textless} *\textit{anɡáar} {\textless} *\textit{ánɡaar} {\textless} \textit{ánɡāra-} `glowing charcoal'\\
~~(\textsc{m}, \textsc{n}) \\
\textit{c̣híitr} `field' (\textsc{m}) {\textless} *\textit{c̣héetr} {\textless} \textit{kṣ\'{\={e}}tra-} `land' (\textsc{n}) \\
\textit{bíi} `seed' (\textsc{f}) {\textless} *\textit{bīya} {\textless} \textit{bīǰa-} `seed, semen' (\textsc{n}) \\
\textit{ṣínɡ} `horn' (\textsc{m}) {\textless} *\textit{ṣínɡa} {\textless} \textit{šr̥nɡa-} `horn' (\textsc{n})
\end{tabular}
\end{exe}

Also a~few nouns of the OIA declension with stems ending in a~consonant have ended up here, particularly neuters with a~final \textit{n} in their stems. Some examples are shown in (\ref{ex:4-ety2}). 

\begin{exe}
\extab
\label{ex:4-ety2}
\begin{tabular}{ l }
\textit{bráam} `joint' (\textsc{m}) {\textless} \textit{*bráama} {\textless} \textit{*mrama} {\textless} \textit{*marma} {\textless} \textit{marman-}\\
~~`vulnerable spot; secret; limb, joint' (\textsc{n}) \\
\textit{kráam} `work' (\textsc{m}) {\textless} \textit{*kráama} {\textless} \textit{*kráma} {\textless} \textit{*kárma} {\textless} \textit{kárman-} `act, work'\\
~~(\textsc{n}) \\
\textit{nóo} `name' (\textsc{m})] {\textless} \textit{*nóowa} {\textless} \textit{*náawa} {\textless} \textit{*náama} {\textless} \textit{n\'{\={a}}man-} `name' (\textsc{n})
\end{tabular}
\end{exe}

However, this is far from a~totally uniform declension in Palula. Nouns with an~alternating accent (as explained in \sectref{subsec:3-5-1}) form their plural oblique with \textit{-óom} instead of \textit{-am}, exemplified in \tabref{tab:4-7}, due to an~historical tensing and raising of the accented vowel (\textit{ám {\textgreater} áam {\textgreater} óom}). 


\begin{table}[ht]
\caption{\textit{a}-declension nouns with accent shift}
\begin{tabularx}{\textwidth}{ Q : Q Q : Q Q }
\lsptoprule
\multicolumn{2}{l}{Singular} & \multicolumn{2}{l}{Plural}\\
\multicolumn{1}{l}{Nominative} &
Oblique &
\multicolumn{1}{l}{Nominative} &
Oblique &
\\\hline
\textit{deés} &
\textit{dees-á} &
\textit{dees-á} &
\textit{dees-óom} &
`day' (\textsc{m})\\
\textit{ɡhoóṣṭ} &
\textit{ɡhooṣṭ-á} &
\textit{ɡhooṣṭ-á} &
\textit{ɡhooṣṭ-óom} &
`house' (\textsc{m})\\
\textit{pútr} &
\textit{putr-á} &
\textit{putr-á} &
\textit{putr-óom} &
`son' (\textsc{m})\\
\textit{ṣiṣ} &
\textit{ṣiṣ-á} &
\textit{ṣiṣ-á} &
\textit{ṣiṣ-óom} &
`head' (\textsc{m})\\\lspbottomrule
\end{tabularx}
\label{tab:4-7}
\end{table}

With a~rather high level of confidence we can trace this Palula accent shift to an~originally stressed (or accented) root final \textit{-á} (found among OIA nouns with a~stem ending in \textit{a}), which was deleted through apocope some time between MIA \citep[247--248]{pischel2011} and the emergence of a~Palula proto"=language (in proto"=Shina or at an~even earlier stage), leaving the last remaining vocalic mora of the noun in the nominative singular with the accent, while preserving it on the original segment in the inflected noun forms, as in the OIA \textit{dev\'{\={a}}s, dev\'{\={a}}n, dev\'{\={a}}nām}, the nominative, accusative and genitive plural forms of \textit{devá} `god' respectively \citep[330]{whitney1960}. The proposed diachronic developments are shown in (\ref{ex:4-ety3}). 

\begin{exe}
\extab
\label{ex:4-ety3}
\begin{tabular}{ l }
\textit{deés} `day' (\textsc{m}) {\textless} \textit{*deesá} {\textless} \textit{divasá-} `heaven, day' (\textsc{m})\\
\textit{ɡhoóṣṭ} `house' (\textsc{m}) {\textless} \textit{*ɡhooṣṭá} {\textless} \textit{*ɡooṣṭhá} {\textless} \textit{ɡōṣṭhá-} `cow"=house, \\
~~meeting"=place'~(\textsc{m})\\
\textit{pútr} `son' (\textsc{m}) {\textless} \textit{putrá-} `son' (\textsc{m})\\
\textit{ṣíṣ} `head' (\textsc{m}) {\textless} \textit{*ṣiṣá} {\textless} \textit{*šiiṣá} {\textless} \textit{šīrṣá-} `head, skull' (\textsc{n})
\end{tabular}
\end{exe}

These words all go back to the OIA declension with nouns ending in \textit{a} being mostly masculine, but with a~few Palula masculine nouns in this group originating in OIA neuters. That final"=accented nouns in OIA always end up as accent"=shifting nouns in Palula is, however, not an~infallible rule. There are indeed counterexamples, like \textit{díiš} `village' (\textsc{m}) {\textless} \textit{deeš}\textbf{\textit{á}-} `point, region, part, province, country' (\textsc{m}), where it is likely that a~stress"=shift took place prior to the apocope process (an assumption that is further supported by the regular strengthening of first"=mora accented \textit{ée} into \textit{íi} in the Palula proto"=language, while the quality of second"=mora accented \textit{eé} in most cases has been preserved): \textit{díiš {\textless} déeš} \textit{{\textless} déeša} \textit{{\textless}} \textit{deeš}\textbf{\textit{á}-}. 


Among \textit{a}-inflecting nouns in Palula, I have at least one example of a~word that can be traced back to the rather limited OIA declension of stems with a~final syllabic \textit{r̥: dhií} `daughter' {\textless} \textit{duhitr̥-} `daughter'.


A sub"=irregularity in the \textit{a}-declension is a~group of nouns (\tabref{tab:4-8}) ending with an~accented vowel \textit{ú} or \textit{í} that usually have a~shared singular form (nominative and oblique), while the two plural cases are formally distinct.



\begin{table}[ht]
\caption{\textit{a}-declension nouns with ending ú or í}
\begin{tabularx}{\textwidth}{ Q : Q : Q Q }
\lsptoprule
\multicolumn{1}{l}{Singular}
&
\multicolumn{3}{l}{Plural}\\
\multicolumn{1}{l}{}&
\multicolumn{1}{l}{Nominative} &
\multicolumn{1}{l}{Oblique} &
\\\hline
\textit{kilí} &
\textit{kili(y)-á} &
\textit{kili(y)-óom} &
`key' (\textsc{f})\\
\textit{beeṭí} &
\textit{beeṭi-á} &
\textit{beeṭi-óom} &
`lamb' (\textsc{m})\\
\textit{bhaampú} &
\textit{bhaampu-á} &
\textit{bhaampu-óom} &
`ball' (\textsc{m})\\
\textit{kursí} &
\textit{kursi-á} &
\textit{kursi-óom} &
`chair' (\textsc{f})\\\lspbottomrule
\end{tabularx}
\label{tab:4-8}
\end{table}

Because of the historical development mentioned above, by which accented \textit{á} was strengthened, some nouns (examples in \tabref{tab:4-9}) now show an~alternation (in A) between a~long (accented) vowel \textit{áa} or \textit{aá} in the singular and a~short (unaccented) vowel \textit{a} in the inflected forms. All disyllabic nouns and monosyllabic nouns with aspiration (including a~preceding \textit{h}) have developed a~second"=mora accent \textit{aá} on the lengthened vowel.


\begin{table}[ht]
\caption{\textit{a}-declension nouns with length alternation}
\begin{tabularx}{\textwidth}{ Q : l Q : l Q }
\lsptoprule
\multicolumn{2}{l}{Singular} & \multicolumn{2}{l}{Plural}\\
\multicolumn{1}{l}{Nominative} &
Oblique &
\multicolumn{1}{l}{Nominative} &
Oblique &
\\\hline
\textit{dáar} &
\textit{dar-á} &
\textit{dar-á} &
\textit{dar-óom} &
`door' (\textsc{m})\\
\textit{ḍáaɡ} &
\textit{ḍaɡ-á} &
\textit{ḍaɡ-á} &
\textit{ḍaɡ-óom} &
`markhor' (\textsc{m})\\
\textit{haál} &
\textit{hal-á} &
\textit{hal-á} &
\textit{hal-óom} &
`plough' (\textsc{m})\\
\textit{dhaataár} &
\textit{dhaatar-á} &
\textit{dhaatar-á} &
\textit{dhaatar-óom} &
`fireplace' (\textsc{m})\\\lspbottomrule
\end{tabularx}
\label{tab:4-9}
\end{table}

There are also many highly frequent nouns (examples in \tabref{tab:4-10}) that in the nominative singular end with an~unaccented \textit{u}, as already pointed out in connection with gender assignment, where this ending \textit{u} does not appear in the plural or in any of the case inflected forms. This could in fact qualify as a~very important subdeclension.


\begin{table}[ht]
\caption{\textit{a}-declension nouns with ending unaccented u}
\begin{tabularx}{\textwidth}{ Q : Q Q : Q Q }
\lsptoprule
\multicolumn{2}{l}{Singular} & \multicolumn{2}{l}{Plural}\\
\multicolumn{1}{l}{Nominative} &
Oblique &
\multicolumn{1}{l}{Nominative} &
Oblique &
\\\hline
\textit{ǰáanu} &
\textit{ǰáan-a} &
\textit{ǰáan-a} &
\textit{ǰáan"=am} &
`person' (\textsc{m})\\
\textit{tóoru} &
\textit{tóor-a} &
\textit{tóor-a} &
\textit{tóor"=am} &
`star' (\textsc{m})\\
\textit{phalúuṛu} &
\textit{phalúuṛ-a} &
\textit{phalúuṛ-a} &
\textit{phalúuṛ-am} &
`grain' (\textsc{m})\\
\textit{thaskúuru} &
\textit{thaskúur-a} &
\textit{thaskúur-a} &
\textit{thaskúur"=am} &
`hoe' (\textsc{m})\\\lspbottomrule
\end{tabularx}
\label{tab:4-10}
\end{table}

\citet[15]{morgenstierne1941}, as well as \citet[29]{buddruss1967}, suggests the OIA derivation \textit{-aka-} (also belonging to the large a-ending declension in OIA) as the origin of these typically masculine endings. According to \citet[222]{masica1991}, this \textit{-aka-} (in the nominative singular \textit{-akas}) has been subject to weakening via \textit{-akō {\textgreater} -aɡō {\textgreater} -ahu {\textgreater} -au}, ending up as \textit{-ō} or \textit{-\={a}} in many NIA languages. That may be so, but I also hold it for very likely that this rather major group of masculine nouns has been further expanded through analogy with the \textit{aka}-derived nouns, some of them possibly with an~adjectival origin. In any case, for a~few Palula nouns with an~ending \textit{u} there is indeed evidence of OIA cognates (or established reconstructions) with \textit{-aka-} (or \textit{-uka}) formation, see (\ref{ex:ety4}), although it will be necessary to posit a~stress or accent"=shift (due to reasons I am not able to formulate now) to have taken place for those words that were stressed on the final syllable in OIA. 


\begin{exe}
\extab
\label{ex:4-ety4}
\begin{tabular}{ l }
\textit{bóolu} `hair' (\textsc{m}) {\textless} \textit{*báalo} {\textless} \textit{*báalau} {\textless} \textit{vālaka-} `tail of horse or elephant'\\
~~(\textsc{m})\\
\textit{tóoru} `star' (\textsc{m}) {\textless} \textit{*táaro} {\textless} \textit{*táarau} {\textless} \textit{tāraká-} `belonging to the stars' (\textsc{m})
\end{tabular}
\end{exe}


Earlier, a~derivational suffix forming adjectives from nouns, \textit{-aka-} (and a~group of similar endings) included already in OIA diminutive formations as well as a~number of less easily definable noun"=to-noun derivations. In the older sources, one basic form occurs as well as a~derived form with seemingly identical semantic content \citep[1222]{whitney1960}. The above exemplified \textit{vālaka-} `tail of horse or elephant' is for instance derived from \textit{v\'{\={a}}la-}, glossed very similarly as `hair of tail, tail, hair' by \citet[12056]{turner1966}. \citet[222]{masica1991} describes it as a~diminutive suffix, which later became a~meaningless ``extension''. There is a~corresponding \citep[1181, 1222]{whitney1960} feminine formation made with \textit{-ikā} that will be discussed below. However, as pointed out already, we will certainly find nouns in this declensional subgroup with an~altogether different origin. One such example is \textit{híṛu} `heart' (\textsc{m}) {\textless} \textit{hr̥daya-} `heart', where the final \textit{-aya-} may have gone through a~weakening process, similar to that of \textit{-aka-}, also ending up with a~final \textit{u} vowel.


In a~few nouns (\tabref{tab:4-11}) with an~accented final \textit{ó} in the nominative singular, this \textit{ó} also disappears in the inflected forms, and the inflectional endings are \textit{á} and \textit{óom}, respectively.


\begin{table}[ht]
\caption{\textit{a}-declension nouns with ending accented ó}
\begin{tabularx}{\textwidth}{ l : l l : l Q }
\lsptoprule
\multicolumn{2}{l}{Singular} & \multicolumn{2}{l}{Plural}\\
\multicolumn{1}{l}{Nominative} &
Oblique &
\multicolumn{1}{l}{Nominative} &
Oblique &
\\\hline
\textit{ɡhaḍeeró} &
\textit{ɡhadeer-á} &
\textit{ɡhadeer-á} &
\textit{ɡhaḍeer-óom} &
`elder' (\textsc{m})\\
\textit{c̣haaṇbharó} &
\textit{c̣haaṇbhar-á} &
\textit{c̣haaṇbhar-á} &
\textit{c̣haaṇbhar-óom} &
`load of holly"=oak branches' (\textsc{m})\\\lspbottomrule
\end{tabularx}
\label{tab:4-11}
\end{table}

Another relatively large group of nouns (\tabref{tab:4-12}), ending in \textit{ái}, may be considered a~subdeclension and may eventually develop into a~declension of their own or even fuse with the \textit{ee}-declension (see below), one of the minor declensions described below.\footnote{Note, however, that the ending with the feminines is a~first"=mora accented \textit{ée}, whereas the ending of the masculines is a~second"=mora accented \textit{eé}.} 


\begin{table}[ht]
\caption{\textit{a}-declension nouns with ending ái}
\begin{tabularx}{\textwidth}{ Q : Q Q : Q l }
\lsptoprule
\multicolumn{2}{l}{Singular} & \multicolumn{2}{l}{Plural}\\
\multicolumn{1}{l}{Nominative} &
Oblique &
\multicolumn{1}{l}{Nominative} &
Oblique &
\\\hline
\textit{manɡái} &
\textit{manɡ-ée(-a)} &
\textit{manɡ-ée(-a)} &
\textit{manɡ-éem} &
`water pot' (\textsc{f})\\
\textit{kooɡái} &
\textit{kooɡ-ée} &
\textit{kooɡ-ée} &
\textit{kooɡ-éem} &
`cheek' (\textsc{f})\\
\textit{booǰái} &
\textit{booǰ-ée} &
\textit{booǰ-ée} &
\textit{booǰ-éem} &
`sack' (\textsc{f})\\
\textit{amzarái} &
\textit{amzar-ée} &
\textit{amzar-ée} &
\textit{amzar-éem} &
`lion' (\textsc{m})\\\lspbottomrule
\end{tabularx}
\label{tab:4-12}
\end{table}

The actual pronunciation of the inflected forms seems to be rather variable, with a~preserved ending \textit{a} (in line with the typical \textit{a}-declined nouns) in some of the variant forms. These nouns are almost exclusively feminine.

\subsection{\textit{i}-declension}
\label{subsec:4-6-2}


Nouns belonging to the \textit{i}-declension form their plural and oblique case with an \textit{i}-suffix, as exemplified in \tabref{tab:4-13}. This is the second"=most frequent noun declension with 20 per cent of the nouns in my database belonging to this declension. Most of them, about 70 per cent, are feminine. These nouns regularly form their plural and oblique forms with an~accented suffix. 


\begin{table}[ht]
\caption{\textit{i}-declension nouns}
\begin{tabularx}{\textwidth}{ Q : Q Q : Q Q }
\lsptoprule
\multicolumn{2}{l}{Singular} & \multicolumn{2}{l}{Plural}\\
\multicolumn{1}{l}{Nominative} &
Oblique &
\multicolumn{1}{l}{Nominative} &
Oblique &
\\\hline
\textit{kuḍ} &
\textit{kuḍ-í} &
\textit{kuḍ-í} &
\textit{kuḍ-íim} &
`wall' (\textsc{f})\\
\textit{ḍheer} &
\textit{ḍheer-í} &
\textit{ḍheer-í} &
\textit{ḍheer-íim} &
`belly' (\textsc{f})\\
\textit{préṣ} &
\textit{preṣ-í} &
\textit{preṣ-í} &
\textit{preṣ-íim} &
`mother"=in"=law'~(\textsc{f})\\
\textit{maṇḍáu} &
\textit{maṇḍaw-í} &
\textit{maṇḍaw-í} &
\textit{maṇḍaw-íim} &
`veranda' (\textsc{m})\\\lspbottomrule
\end{tabularx}
\label{tab:4-13}
\end{table}


Like the \textit{a}-declension this is far from a~uniform class, and in some cases it is not entirely clear whether a~noun should be included or rather be considered part of a~separate declension. I have, for instance, chosen to regard a~group of masculine nouns ending in \textit{oó} or \textit{á} as a~separate minor declension (see \sectref{subsec:4-6-4}), although it may possibly be analysed as part of the \textit{i}-declension. Occurring particularly frequently in this declension are plurals with an~additional umlaut (\tabref{tab:4-14}), of which almost all are feminine, a~considerable number of them relatively recent loans from other languages.


\begin{table}[ht]
\caption{\textit{i}-declension nouns with umlaut}
\begin{tabularx}{\textwidth}{ Q : Q Q : Q l }
\lsptoprule
\multicolumn{2}{l}{Singular} & \multicolumn{2}{l}{Plural}\\
\multicolumn{1}{l}{Nominative} &
Oblique &
\multicolumn{1}{l}{Nominative} &
Oblique &
\\\hline
\textit{baát} &
\textit{beet-í} &
\textit{beet-í} &
\textit{beet-íim} &
`talk, issue' (\textsc{f})\\
\textit{dukaán} &
\textit{dukeen-í} &
\textit{dukeen-í} &
\textit{dukeen-íim} &
`shop' (\textsc{f})\\
\textit{ǰum(i)aát} &
\textit{ǰumeet-í} &
\textit{ǰumeet-í} &
\textit{ǰumeet-íim} &
`mosque' (\textsc{f})\\
\textit{himaál} &
\textit{himeel-í} &
\textit{himeel-í} &
\textit{himeel-íim} &
`glacier' (\textsc{f})\\\lspbottomrule
\end{tabularx}
\label{tab:4-14}
\end{table}

As with the \textit{a}-nouns, there are several \textit{i}-nouns that have gone through vowel lengthening in their basic form but have kept a~short vowel in their inflected form, as in the examples in \tabref{tab:4-15}. The origin of this declension is less straightforward as far as OIA is concerned, and the history of the large bulk of this declension may be found either in MIA or in rather recent developments. It is indeed home to a~considerable amount of ``modern'' loans from Urdu.


\begin{table}[ht]
\caption{\textit{i}-declension nouns with length alternation}
\begin{tabularx}{\textwidth}{ Q : Q Q : Q l }
\lsptoprule
\multicolumn{2}{l}{Singular} & \multicolumn{2}{l}{Plural}\\
\multicolumn{1}{l}{Nominative} &
Oblique &
\multicolumn{1}{l}{Nominative} &
Oblique &
\\\hline
\textit{dharaáṇ} &
\textit{dharaṇ-í} &
\textit{dharaṇ-í} &
\textit{dharaṇ-íim} &
`ground, earth' (\textsc{f})\\
\textit{čáar} &
\textit{čar-í} &
\textit{čar-í} &
\textit{čar-íim} &
`grass, fodder' (\textsc{f})\\
\textit{ǰhanɡaár} &
\textit{ǰhanɡar-í} &
\textit{ǰhanɡar-í} &
\textit{ǰhanɡar-íim} &
`liver' (\textsc{f})\\
\textit{c̣haár} &
\textit{c̣har-í} &
\textit{c̣har-í} &
\textit{c̣har-íim} &
`waterfall' (\textsc{f})\\\lspbottomrule
\end{tabularx}
\label{tab:4-15}
\end{table}

The nouns for which I have been able to trace an~OIA cognate and also am able to hypothesise on their development into the Palula form, as shown in (\ref{ex:4-ety5}), are either from the OIA declension with stems ending in a~short vowel \textit{i} or \textit{u}, the declension with stems ending in a~long vowel \textit{ā, ī, ũ}, or are derived nouns in OIA ending in \textit{-ya}.


\begin{exe}
\extab
\label{ex:4-ety5}
\begin{tabular}{ l }
\textit{díṣṭ} `hand"=span' (\textsc{f}) {\textless} \textit{diṣṭi-} `a measure of length' (\textsc{f})\\
\textit{ɡrheéṇḍ} `knot' (\textsc{f}) {\textless} \textit{*ɡrheeṇḍí} {\textless} \textit{*ɡraaṇḍhí} {\textless} \textit{ɡranthí-} `knot, etc.' (\textsc{m})\\
\textit{bheéṇ} `sister' (\textsc{f}) {\textless} \textit{*bhaíṇ} {\textless} \textit{*bhaiṇi} {\textless} \textit{bhaɡinī-} `sister' (\textsc{f})\\
\textit{dharaáṇ} `ground, earth' (\textsc{f}) {\textless} \textit{*dharáṇ} {\textless} \textit{dharáṇī-} `ground' (\textsc{f})\\
\textit{kúḍ} `wall' (\textsc{f}) {\textless} \textit{kuḍya-} `wall' (\textsc{n})\\
\textit{muúl} `price, value' (\textsc{f}) {\textless} \textit{mũlya-} `original value, price' (\textsc{n})
\end{tabular}
\end{exe}


\begin{itemize}[itemsep=0pt, leftmargin=\widthof{\textit{dharaáṇ} `ground, earth' (\textsc{f})}]
\item[\textit{díṣṭ} `hand"=span' (\textsc{f})] {\textless} \textit{diṣṭi-} `a measure of length' (\textsc{f})
\item[\textit{ɡrheéṇḍ} `knot' (\textsc{f})] {\textless} \textit{*ɡrheeṇḍí} {\textless} \textit{*ɡraaṇḍhí} {\textless} \textit{ɡranthí-} `knot, etc.' (\textsc{m})
\item[\textit{bheéṇ} `sister' (\textsc{f})] {\textless} \textit{*bhaíṇ} {\textless} \textit{*bhaiṇi} {\textless} \textit{bhaɡinī-} `sister' (\textsc{f})
\item[\textit{dharaáṇ} `ground, earth' (\textsc{f})] {\textless} \textit{*dharáṇ} {\textless} \textit{dharáṇī-} `ground' (\textsc{f})
\item[\textit{kúḍ} `wall' (\textsc{f})] {\textless} \textit{kuḍya-} `wall' (\textsc{n})
\item[\textit{muúl} `price, value' (\textsc{f})] {\textless} \textit{mũlya-} `original value, price' (\textsc{n})
\end{itemize}

The proposed intermediary forms remain tentative, although most of the processes are evident from parallel comparative material, such as apocope, umlaut, forward shift of aspiration, \textit{a}-lengthening and intervocalic lenition of plosives.

\subsection{\textit{m-}declension}
\label{subsec:4-6-3}

Nouns belonging to the \textit{m}-declension typically form their plural with an~\textit{m}-suffix, but they do not inflect for oblique case. Examples are displayed in \tabref{tab:4-16}. According to my data, this declension is slightly smaller (at 16 per cent) than the \textit{i}-declension. It is also the most homogenous of the main declensions, as all nouns in this class are feminine,\footnote{A possible counter"=example is the Pashto loan \textit{malɡiri} `friend, companion', with the plural form \textit{malɡirim}, which seems to be assigned gender referentially rather than having an~inherent and invariable gender.} and their basic form ends in \textit{i} (unaccented except in a~few cases).


\begin{table}[ht]
\caption{\textit{m}-declension nouns}
\begin{tabularx}{.75\textwidth}{ Q Q l }
\lsptoprule
Singular &
Plural &
\\\hline
\textit{čhéeli } &
\textit{čhéeli-m} &
`she"=goat' (\textsc{f})\\
\textit{déeṛi} &
\textit{déeṛi-m} &
`beard' (\textsc{f})\\
\textit{phéepi} &
\textit{phéepi-m} &
`paternal aunt' (\textsc{f})\\
\textit{phaí} &
\textit{phaíi-m} &
`girl' (\textsc{f})\\\lspbottomrule
\end{tabularx}
\label{tab:4-16}
\end{table}


This pattern is the result of a~rather recent historical process that is only partly clear to me. The basic assumption is that the previous plural marking (as far as there was an~overt marking at all) disappeared in a~process of apocope that affected all final unaccented vowels, and the plural oblique (with an~\textit{m}) was (maybe compensatorily) extended to all nouns with a~plural reference, hence becoming a~plural"=marker rather than a~case"=marker.


The development of this declension is to some extent a~feminine parallel to that of the masculine \textit{u}-ending nouns of the \textit{a}-declension. A feminine derivational suffix \textit{-ikā-} was used in OIA in much the same way as the above"=mentioned \textit{-aka-} \citep[1222]{whitney1960}, while the NIA endings with their origin in \textit{-ikā-} probably retain much more of the Sanskritic diminutive sense \citep[222]{masica1991}. OIA nouns with the ending \textit{-ikā-} are part of the OIA declension 3 together with other nouns ending in a~long vowel. Below are some examples of Palula \textit{m}-declension nouns with probable OIA \textit{ikā}-cognates: 


\begin{itemize}[leftmargin=\widthof{\textit{čhéeli} `goat (she"=goat)' (\textsc{f})x}]
\item[\textit{čhéeli} `goat (she"=goat)' (\textsc{f})] \textit{{\textless} *čhaali {\textless} *čhawali {\textless} čhaɡalikā-} `goat' (\textsc{f})
\item[\textit{déeṛi} `beard' (\textsc{f})] \textit{{\textless} *daaṛi {\textless} dāḍhikā-} `beard' (\textsc{f})
\item[\textit{béeǰi} `heifer' (\textsc{f})] {\textless} \textit{*bíadzi {\textless} dvivatsikā-} `two year old' (\textsc{f})
\end{itemize}

However, there are nouns in the Palula \textit{m}-declension for which we cannot be absolutely
certain that they derive from OIA nouns with an \textit{ikā-}suffix.\footnote{At least there are
  no \textit{ikā}-formations occurring in the older literature, which of course is no proof
  they have never existed.} They may, for instance, derive from nouns ending in a~long vowel
\textit{ī} (\textit{séeti} `thigh' (\textsc{f}) {\textless}? \textit{sakthī-} `thigh, thighbone' (\textsc{f})) or nouns ending in a~short i (\textit{héeṛi} `duck'
(\textsc{f}) {\textless}? \textit{ātí-} `an aquatic bird' (\textsc{f})) and have, for
reasons I am presently unable to explain, come to declensionally converge with the above"=mentioned
\textit{ikā-}derived nouns and not with the Palula \textit{i}-declension nouns. Thus
\citet[222]{masica1991} comments on the development of the NIA feminine marker (such as the ending
\textit{i} of the \textit{m}-declension nouns in Palula): ``Its evolution was no doubt influenced,
however, by the existence of a~Feminine in \textbf{-ī}, at times
restrengthened, in all periods of the language.''


Although regular masculine--feminine pairings with the endings \textit{aka} and \textit{ikā}
respectively seem to have been common already in OIA, I would not suggest that similar pairings so
common in Palula all go back to nouns with these suffixes. It is, I think, much more likely that
many of them were formed at a~much later stage, maybe as feminine derivations of semantically
generic masculine nouns, or are both adjectives that have come to be increasingly used as nouns. My
hypothesis is that the more lexicalised pairs (including those where the feminine counterpart is
primarily a~diminutive (see \tabref{tab:4-2})) belong to an~older layer, whereas simple
male--female pairs (see \tabref{tab:4-1}) are formed to a~large extent at a~later stage, somewhat in
analogy with the former.


Also nouns ending in unaccented \textit{u}, such as the ones in \tabref{tab:4-17}, should probably be considered part of this declension. These, however, show some instability in their inflectional pattern (with an~alternative paradigm: \textit{pháapu, pháapa, pháapam}), and it is possible that in spite of their being feminine are becoming part of the subgroup of the \textit{a}-declension, which drops the final \textit{u} when inflecting. This may or may not be the first step towards a~subsequent gender"=shift. Historically, the few nouns of the \textit{m}-declension that end in a~short unaccented \textit{u} probably derive from OIA feminines with a~suffix \textit{-ukā}, such as \textit{práašu} `rib' (\textsc{f}) {\textless} \textit{paršukā-} `rib' (\textsc{f}).


\begin{table}[ht]
\caption{\textit{m}-declension nouns with ending unaccented u}
\begin{tabularx}{\textwidth}{ Q Q Q }
\lsptoprule
Singular &
Plural &
\\\hline
\textit{máaṭu} &
\textit{máaṭum} &
`neck' (\textsc{f})\\
\textit{pháapu} &
\textit{pháapum} &
`lung' (\textsc{f})\\
\textit{práašu} &
\textit{práašum} &
`rib' (\textsc{f})\\\lspbottomrule
\end{tabularx}
\label{tab:4-17}
\end{table}


\subsection{Smaller declensions and irregular nouns}
\label{subsec:4-6-4}

Apart from the main declensions, there are groups of nouns that, for various reasons, do not conform to any of the previously introduced declensions, although in many respects they seem to share features with one or the other of them.

\subsubsection*{\textit{ee}"=declension}

A group of nouns, presently making up about eight per cent of my database, has a~plural form ending with a~second"=mora accented \textit{eé} (\tabref{tab:4-18}). Characteristically the nominative singular ends in an~accented \textit{á}, and the \textit{eé} may be described as the result of hiatus between the stem \textit{á} and the suffix \textit{-í}, thus basically qualifying to be included in the \textit{i}-declension. (The latter is even more apparent in the B dialect, where the plural form suffix \textit{-í}, has fused to a~much lesser extent with the final stem vowel: \textit{kuṇḍá -} \textit{kuṇḍaí}.) However, contrary to the typical \textit{i}-declension nouns, these nouns do not distinguish between nominative and oblique singular, while that case distinction is upheld in the plural. As it is phonologically rather distinct from other \textit{i}-declension nouns in the present stage of the language (at least in the A dialect), I prefer to treat it as a~declension of its own. This declension includes a~number of nouns with a~long history in the language, but at the same time, it seems a~rather productive one as far as incorporation of more recent loans are concerned.


\begin{table}[ht]
 \label{bkm:Ref193698864}
 \caption{\textit{ee}"=declension nouns}
\begin{tabularx}{\textwidth}{ Q Q Q Q }
\lsptoprule
Singular
&
Plural\\
&
Nominative &
Oblique &
\\\hline
\textit{kuṇḍá} &
\textit{kuṇḍ-eé} &
\textit{kuṇḍ-eém} &
`hook, peg' (\textsc{m})\\
\textit{ǰinaazá} &
\textit{ǰinaaz-eé} &
\textit{ǰinaaz-eém} &
`corpse' (\textsc{m})\\
\textit{qisá} &
\textit{qis-eé} &
\textit{qis-eém} &
`story' (\textsc{m})\\
\textit{paalaá} &
\textit{paal-eé} &
\textit{paal-eém} &
`leaf' (\textsc{m})\\\lspbottomrule
\end{tabularx}
\label{tab:4-18}
\end{table}


All of the nouns from this declension for which gender is known are masculine. This should be compared with the group of nouns, already introduced under the \textit{a}-declension, with a~singular \textit{ái}-ending and a~plural \textit{ée}-ending (i.e., first"=mora accented). Although their plural forms are segmentally very similar to the ones discussed here, I hesitate to include them in this group, as they behave like \textit{a}-declension nouns (and in an~alternative pronunciation receive a~plural ending \textit{-éea} or \textit{-áya}) in most other respects. In addition, almost all of those are feminine.


This group of nouns contains a~large number of non"=inherited words, some rather recent loans from Urdu or Pashto, others with a~longer history in the language, but only a~few that can be regarded as inherited from OIA. The only one I have been able to trace comes from the \textit{a}-ending declension in OIA, and here it is likely that apocope along with stress on the remaining final segment has resulted in a~second"=mora accented long \textit{aá} in Palula: \textit{paalaá} `leaf' (\textsc{m}) {\textless} \textit{*paalaawá-} \textit{{\textless} pallava-} `sprout, twig, blossom' (\textsc{m}, \textsc{n}).


Another, very limited group of masculine nouns, with ending \textit{oó} in the singular (\tabref{tab:4-19}), behave in a~very similar manner, and show similar forms in the plural, but they tend (with a~few exceptions) to use the same form for the singular oblique as for the plural nominative. It makes most sense to consider these nouns as another subcategory of the \textit{ee}-declension.



\begin{table}[ht]
 \label{bkm:Ref193698938}
 \caption{\textit{ee}"=declension nouns with ending oó}
\begin{tabularx}{\textwidth}{ Q : Q Q : Q l }
\lsptoprule
\multicolumn{2}{l}{Singular} & \multicolumn{2}{l}{Plural}\\
\multicolumn{1}{l}{Nominative} &
Oblique &
\multicolumn{1}{l}{Nominative} &
Oblique &
\\\hline
\textit{ǰandoó} &
\textit{ǰand-eé} &
\textit{ǰand-eé} &
\textit{ǰand-eém} &
`he"=goat' (\textsc{m})\\
\textit{ǰhamatroó} &
\textit{ǰhamatr-eé} &
\textit{ǰhamatr-eé} &
\textit{ǰhamatr-eém} &
`son"=in"=law' (\textsc{m})\\
\textit{muuṣoó} &
\textit{muuṣ-eé} &
\textit{muuṣ-eé} &
\textit{muuṣ-eém} &
`elbow' (\textsc{m})\\
\textit{paitsoó} &
\textit{paits-eé} &
\textit{paits-eé} &
\textit{paits-eém} &
`trouser leg' (\textsc{m})\\\lspbottomrule
\end{tabularx}
\label{tab:4-19}
\end{table}

 
This group is closely related to the \textit{u}-ending nouns of the \textit{a}-declension (in the closely related variety spoken in Sau, Afghanistan, these two build up a~declension of their own vis-à-vis the other nouns in the \textit{a}-declension). Like those, the \textit{oo}-ending nouns derive from the OIA declension of \textit{a}-ending nouns. 


\begin{itemize}[leftmargin=\widthof{\textit{ǰamatroó} `son"=in"=law' (\textsc{m})}]
\item[\textit{haṇoó} `egg' (\textsc{m})] {\textless} \textit{*haṇóo-} {\textless} \textit{*aaṇáa-} {\textless} \textit{āṇḍaka-} `egg; testicles' (\textsc{n})
\item[\textit{ǰhamatroó} `son"=in"=law' (\textsc{m})] {\textless} \textit{*ǰamatróo-} {\textless} \textit{*ǰamaatráa-} {\textless} \textit{ǰāmātraka-} `daughter's husband' (\textsc{m})
\item[\textit{muuṣoó} `elbow' (\textsc{m})] {\textless} \textit{*muuṣóo-} {\textless} \textit{*muuṣáa-} {\textless} \textit{*mũṣala-} `muscle'
\end{itemize}

For the first two words we have to assume a~change in accent"=pattern in the development into modern Palula. A second"=mora accented long \textit{aá} would have kept the vowel quality, whereas there is a~regular development of first"=mora accented long \textit{áa} into \textit{óo}, so we assume that after the apocope had taken place we were left with a~final long \textit{áa} with a~first"=mora accent. This could be compared with two other (irregular) \textit{oo}-ending nouns, where the \textit{aa} is preserved in the inflected forms: \textit{phoó} (\textsc{nom.pl} \textit{phaayá}) `boy' {\textless} \textit{*pháa}, and \textit{bhroó} (\textsc{nom.pl} \textit{bhraawú}) `brother' {\textless} \textit{*bhráa}. To that should be added that we find some irregularities in this group that may constitute remnants of an~earlier inflectional pattern, such as with words referring to `descendant of so and so' as with \textit{phaṭakoó} `descendant of Paṭak': \textit{phaṭakeé} \textsc{nom.pl}, \textit{phaṭakeé} \textsc{obl.sg}, \textit{phaṭakúm} \textsc{obl.pl}.

\subsubsection*{\textit{aan}"=declension}

Although nouns forming plural with \textit{-aán} (\tabref{tab:4-20}) in many ways could be described as a~subcategory of the \textit{a}-declension, it is an~interesting group (making up less than five per cent of all nouns) because of its maximum formal number and case differentiation and also because of the semantics of these particular nouns. While in the main declensions there is either a~formal collapse between the nominal plural and the oblique singular or a~total absence of nominal--oblique case distinctions, there is a~four"=way contrast displayed for many of the \textit{aan}-nouns.


\begin{table}[ht]
 \label{bkm:Ref193699042}
 \caption{\textit{aan}"=declension nouns}
\begin{tabularx}{\textwidth}{ Q : Q Q : P{25mm} P{25mm} }
\lsptoprule
\multicolumn{2}{l}{Singular} & \multicolumn{2}{l}{Plural}\\
\multicolumn{1}{l}{Nominative} &
Oblique &
\multicolumn{1}{l}{Nominative} &
Oblique &
\\\hline
\textit{yaár}
&
\textit{yaar-á}
&
\textit{yaar-aán}
\textit{(yaar-á)} &
\textit{yaar"=aan-óom}
\textit{(yaar-óom)} &
{`friend' (\textsc{m})}
\\
\textit{deéw} &
\textit{deew-á} &
\textit{deew-aán} &
\textit{deewaan-óom} &
`giant' (\textsc{m})\\
\textit{ẓamí} &
\textit{ẓamí} &
\textit{ẓami-aán} &
\textit{zami-óom} &
`brother"=in"=law'~(\textsc{m})\\
\textit{anɡreéz} &
\textit{anɡreez-á} &
\textit{anɡreez-aán} &
\textit{anɡreez"=aan-óom} &
`Englishman, foreigner'\\\lspbottomrule
\end{tabularx}
\label{tab:4-20}
\end{table}


Almost all of the nouns in my database with \textit{-aán} as a~plural marker denote male higher animates, often referring to occupations, but there are apparent exceptions such as inanimate \textit{aaluɡaán} `potatoes', low animate \textit{traambuaán} `wasps' and the abstract noun \textit{hamaliyaán} `habits'.


This declension is heterogenous in the sense that many of the nouns have alternative forms. Many of
the nouns are basically accent"=shifting \textit{a}-declension nouns, but have alternative plural
forms with a~plural suffix \textit{-aán} and a~plural oblique in
\textit{-aanóom} (or for some nouns \textit{-aanúm}). To some extent
this may be an~effect of borrowing, where the plural form of some Pashto nouns along with its suffix
has also been copied \textit{saṇḍá} `male buffalo'~-- \textit{saṇḍaɡaán} `male
buffaloes'. It this case, it is not entirely clear whether the word is the one inherited from OIA
(\textit{s\'{\={a}}ṇḍa} `uncastrated (of bull))' that later acquired its morphology through Pashto
influence or as a~whole has been reintroduced via Pashto (\textit{saṇḍa} `a male buffalo') rather
recently. But whatever the origin may be in every single case, it has become a~very common pattern,
especially for nouns with male human reference, even with words that are very clearly inherited from
OIA, such as \textit{saaṇḍú} `wife's sister's husband' ({\textless} \textit{*s\={a}\'{
  }(ṇ)ḍhu-})~-- \textit{saaṇḍuɡaán}. In the latter case, with its epenthetic
\textit{-ɡ-} in the plural, it is tempting to suggest a~form"=analogy with the already
mentioned \textit{saṇḍá~-- saṇḍaɡaán}. \textit{-aan} is a~common plural
formation for nouns denoting human referents not only in Pashto but also in the locally influential
languages Persian \citep[431]{windfuhrperry2009} and Khowar \citep[221--225]{endresenkristiansen1981}.

\subsubsection*{Irregular nouns}

There is a~small group of nouns, exemplified in \tabref{tab:4-21}, that either have unique paradigms or are otherwise highly irregular, and in some cases also display alternative forms for some categories. Most of these are kinship terms. In the case of \textit{maámu/maamaá}, there is probably some sort of paradigm mixing going on, although there may have been a~distinction in the past.


\begin{table}[ht]
 \label{bkm:Ref193699124}
 \caption{Irregular nouns}
\begin{tabularx}{\textwidth}{ Q : Q Q : Q Q }
\lsptoprule
\multicolumn{2}{l}{Singular} & \multicolumn{2}{l}{Plural}\\
\multicolumn{1}{l}{Nominative} &
Oblique &
\multicolumn{1}{l}{Nominative} &
Oblique &
\\\hline
\textit{kúṛi} &
\textit{kúṛi} &
\textit{kuṛíina} &
\textit{kuṛíina} &
`woman, wife' (\textsc{f})\\
\textit{brhoó} &
\textit{bhraa(w)ú} &
\textit{bhraa(w)ú} &
\textit{brhaawóom} &
`brother' (\textsc{m})\\
\textit{phoó} &
\textit{phoó, phoo(w)á} &
\textit{phaayá} &
\textit{phayóom} &
`boy' (\textsc{m})\\
\textit{maámu, }
\textit{maamaá} &
\textit{maáma,}
\textit{maamaá} &
\textit{maamayeé,}
\textit{maameé} &
\textit{maamayúm,}
\textit{maameém} &
`uncle' (\textsc{m})
\\
\textit{muloó} &
\textit{muloó} &
\textit{mulhaán} &
\textit{mulhaanúm} &
`mullah' (\textsc{m})\\\lspbottomrule
\end{tabularx}
\label{tab:4-21}
\end{table}
