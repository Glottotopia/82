\addchap{Acknowledgments}
\begin{refsection}


A large number of people and circumstances have been instrumental in the completion of this work.


My thanks go, first of all, to my main supervisor, Östen Dahl (Stockholm University, SU), who, over
an~extended period of time, read and reread all the different parts and versions of my text,
constantly commenting and challenging my analysis, while also suggesting improvements and pointing
out crucial typological correlations and linguistic references.


I am also indebted to my second supervisor, Joan Baart (Summer Institute of Linguistics, SIL), who,
besides carefully reading and insightfully commenting on my text, served more than anyone else as
an~enormously inspiring and competent mentor in the investigation and analysis of a~largely
undocumented language, especially during our overlapping time periods in Pakistan.


I am very grateful to Ruth Schmidt (University of Oslo), who encouraged my Palula research from the
time we first met, and subsequently read many of my drafts, often offering interesting comparisons
with other Shina varieties and ancient forms of Indo"=Aryan. Without any formal agreement, she
provided qualified mentorship and stirred an~interest in the diachronic aspects of my Palula
studies.


Thanks go to a~number of other people who at different stages took the time to read drafts of the
work, either in its entirety or parts of it, and offered crucial input on structure, style and
content: Marie Crandall (SIL), Andreas Jäger (SU), Masja Koptjevskaja"=Tamm (SU), Eva Lindström (SU),
Stephen Marlett (SIL) and Carla Radloff (SIL). The final copy"=editing of the entire manuscript was
done by Lamont Antieau, who, within a~very short time span, made essential improvements to the
text. Thank you!


Also a~special thanks to Ljuba Veselinova (SU), who helped me produce the maps included in this
work, and to Emil Perder (SU) for the many interesting and rewarding conversations we have had about
an~area and its languages at the centre of our mutual interest.


Moving to Pakistan and settling in Peshawar would have been much harder and not as enjoyable and
fascinating as it turned out to be, if it hadn't been for all the many new friends and people who
gave me and my family a~warm welcome and continued making us feel at home during those years. At the
Frontier Language Institute (FLI) in Peshawar, a~special thanks go to Wayne \& Valerie Lunsford,
without whose unceasing friendship and supportiveness we would never have made it. Thanks also to
Fakhrud Din and Muhammad Zaman Sager, who faithfully included us in their lives and in the lives of
their families and communities. I am also indebted to all the other staff members and students of
FLI, not specifically named here, who have enriched my life and many times made sense of the
seemingly (and actucally) contradictory cultures and sentiments that Pakistan comprises.


For his help in getting to know Chitral, and the Palula community in particular, I need to mention
my friend Fakhrud Din once more, as he alone helped me establish many of those contacts and
connections that turned out to be the longest lasting and most fruitful in my work and continues to
provide me with local knowledge, advice and insights.


Although hospitality is something self"=evident and a~matter of honour in Chitral, I would
nevertheless like to thank those who have provided shelter, sustenance and good company. Especially
(but not exclusively) my thanks go to Atiq Ullah (Biori), Hanif Ullah (Drosh), Hayatud Din
(Kalkatak), Khitabud Din (Ashret), (late) Sahib Shah (Biori), Said Habib (Ashret) and Shaukat Ali
(Kalkatak).


Thanks also to the local leadership in the Palula"=speaking localities and to
Anjuman-e-taraqqi-e-Palula (the Society for the Promotion of Palula) for your cooperativeness and
encouragement.


My main Palula language consultants and coworkers are duly acknowledged in the introductory chapter
of this work, and informants and recorded speakers are listed in the references at the end, the sole
reason why none of these important people are mentioned here.


During the periods spent in Stockholm, the Department of Linguistics, Elisabeth \& Herman Rhodin's
Foundation and P.A.~Siljeström's Foundation have contributed by providing me part"=time employment
and doctoral student scholarships. My field work in Pakistan was conducted while holding a~post as
a~language development consultant (from 2003 under the auspices of FLI) financially supported by
Sida (Swedish International Development Cooperation Agency) through PMU InterLife.


I also want to say thank you to my parents, Evert \& Ulla Liljegren, for bringing forth that
original curiosity, for your continued belief in me (even when it meant letting the grandchildren
spend several years abroad), and for that monthly ``extra'' which made those badly needed family
vacations possible.


Finally, my deepest love and appreciation to Maarit, Johanna and Jonathan, my own family, who can
now take a~deep breath as this piece of work is completed at last.

\printbibliography[heading=subbibliography]
\end{refsection}