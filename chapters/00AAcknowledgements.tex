\addchap{Acknowledgments}
\begin{refsection}

% \authcomm{Right after the title page I would like to have a dedication inserted, written both in Palula orthography and in English: \LARGE\PRL{پالُولہ بھراؤمی نوہ جُلیۡ} \textit{To the Palula people}}
This grammar of Palula is essentially a revised version of my doctoral thesis, successfully defended at Stockholm University in June 2008, at the time titled \textit{Towards a grammatical description of Palula}. Since then, the text has been subject to a series of necessary corrections, structural as well as content"=related changes and a few but substantial additions, and it is my sincere hope that the end result is an improved product as far as user"=friendliness, comprehensiveness and accuracy are concerned.


A large number of people and circumstances have played a vital role in the completion of this work.


First of all, none of this would ever have been possible hadn’t it been for the many open hearts and homes of of the Palula community that I and my family have experienced since the first contacts were established in the summer of 1998. Thank you all for sharing your lives, your knowledge and your stories with us. Although hospitality is something self"=evident and a~matter of honour in Chitral, I would nevertheless like to thank those who so generously have provided shelter, sustenance and good company. Especially (but not exclusively) my thanks go to Muhammad Atiqullah (Biori), Hanifullah (Drosh), Hayatuddin (Kalkatak), Khitabuddin (Ashret), (late) Sahib Shah (Biori), (late) Said Habib (Ashret) and Shaukat Ali (Kalkatak). Thanks also to the local leadership in the Palula"=speaking localities and to Anjuman"=e"=taraqqi"=e"=Palula (the Society for the Promotion of Palula) for your cooperativeness and encouragement. 


For his help in getting to know Chitral, and the Palula community in particular, I must mention my friend Fakhruddin Akhunzada, as he alone helped me establish many of those contacts and connections that turned out to be the longest lasting and most fruitful in my work and continues to provide me with local knowledge, advice and insights. My main Palula language consultants and coworkers are duly acknowledged in the introductory chapter of this work, and informants and recorded speakers are listed in the references at the end, the sole reason why many of these key people are not mentioned here.


Many thanks, to my main PhD supervisor, Östen Dahl, who, over
an~extended period of time, read and reread all the different parts and versions of my text up to the time of my defence, constantly commenting and challenging my analysis, while also suggesting improvements, pointing out crucial typological correlations and linguistic references, and widening my own awareness on a range of linguistic topics. I am also deeply indebted to my second supervisor, Joan Baart, who,
besides carefully reading and insightfully commenting on my text, served more than anyone else as an~enormously inspiring and competent mentor in the investigation and analysis of a~largely undocumented language, especially during our overlapping time periods in Pakistan.


I am very grateful to Ruth Laila Schmidt, who encouraged my Palula research from the time we first met, and subsequently read many of my drafts, often offering interesting comparisons with other Shina varieties and ancient forms of Indo"=Aryan. Without any formal agreement, she
provided qualified mentorship and stirred an~interest in the diachronic aspects of my Palula studies.


Thanks go to a~number of other people who at different (both pre- and post"=dissertation) stages took the time to read drafts of the work, either in its entirety or parts of it, and offered crucial input on structure, style and content: Lamont Antieau, Kimberly Caton, Henk Courtz, Marie Crandall, Peter Hook, Andreas Jäger, Amy Kennemur, Maria Koptjevskaja"=Tamm, Eva Lindström, Stephen Marlett and (late) Carla Radloff. Also a~special thanks to Ljuba Veselinova, who helped me produce the maps included in this work, and to Emil Perder for the many interesting and rewarding conversations we have had about
an~area and its languages at the centre of our mutual interest. Correspondence and conversations at different points of time with a number of other people engaged in linguistic fieldwork in the region, have also been inspirational and contributed to spurring a further interest in areal"=linguistic perspectives on my Palula research; here I would like to mention Elena Bashir, Jan Heegård, Tariq Rahman, Khawaja Rehman, Ronald Trail, Matthias Weinreich and Claus Peter Zoller in particular.


My chief language consultant, Naseem Haider, is acknowledged in the introductory chapter, but I want to mention him here too. Apart from carrying out many interviews and recordings, he made a first transcription of most Palula texts that my analysis is based on, which we later worked on together and gradually refined and discussed at great length and detail. Without his ceaseless patience in providing answers and explanations, and in listening to and helping to thresh out my own hypotheses, the annotion would still have been full of question marks, and my resulting analysis would have been at a much less advanced stage than it is now. Thank you, Naseem, for the privilege of working with you!


Moving to Pakistan and settling in Peshawar would have been much harder and not as enjoyable and fascinating as it turned out to be, if it hadn't been for all the many new friends and people who gave me and my family a~warm welcome and continued making us feel at home during those years. At the Forum for Language Initiatives (formely Frontier Language Institute) in Peshawar, a~special thanks go to Wayne and Valerie Lunsford (and their sons, Sean and Jordan), without whose unceasing friendship and supportiveness we would never have made it. Thanks also to Fakhruddin Akhunzada (again) and Muhammad Zaman Sagar, who faithfully included us in their lives and in the lives of their families and communities. I am also indebted to all the other FLI staff members and all of those language activists and local scholars in FLI’s network, not specifically named here, who have enriched my life and many times made sense of the seemingly (and actucally) contradictory cultures and sentiments that Pakistan comprises.


Furthermore, I am extremely grateful to the staff at Language Science Press. A special thanks to Martin Haspelmath for your encouragement and for offering me the opportunity to publish my work at LSP. I also want to thank Birgit Jänen for your tremendous work with LaTeX conversion and typesetting, and Sebastian Nordhoff for managing the publication process and offering quick answers and relevant solutions to various problems. I also want to thank the four anonymous reviewers who offered many valuable and insightful comments and contributed suggestions that greatly improved the final version of the work. Thanks also to the proofreaders who were engaged at the final stages of the production process.


During my PhD periods spent in Stockholm, the Department of Linguistics, Elisabeth \& Herman Rhodin's Foundation and P.A.~Siljeström's Foundation contributed by providing me part"=time employment and doctoral student scholarships. My field work in Pakistan was conducted while holding a~post as a~language development consultant (from 2003 under the auspices of FLI) financially supported by Sida (Swedish International Development Cooperation Agency), through its frame organisation PMU, 1998—2000, 2003—2006, and 2008—2010. For the period when most of the revisions were implemented, I was supported by the Swedish Research Council as part of the ongoing research project \textit{Language contact and relatedness in the Hindukush region} (421-2014-631).


Finally, my deepest love and appreciation to Maarit, Johanna and Jonathan, my own family, for your love, participation and support throughout all of our adventures.




Henrik Liljegren


\textit{Stockholm, December 2015}


\printbibliography[heading=subbibliography]
\end{refsection}