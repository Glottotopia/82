\chapter{Pronouns}
\label{chap:5}

\section{Introduction and overview}
\label{sec:5-1}


Palula is essentially a `two"=person' language \citep[4--15]{bhat2004}, having distinct first- and second"=person (singular, plural) pronouns, i.e., for the speech act participants, whereas third person is expressed by forms belonging to the category of demonstratives. Third person displays a~number of properties (such as distinction in gender, distance and accessibility) not shared with the first- and second"=person pronouns, but that are shared to a~large extent with a~larger category of pro"=forms, which build a~number of symmetrical sets that cut across several word classes. Personal pronouns proper, other types of pronouns as well as demonstratives (pronominal and adnominal) and with them related forms are treated in this chapter.


\section{Personal pronouns}
\label{sec:5-2}

The personal pronouns, \textsc{1sg}, \textsc{2sg}, \textsc{1pl} and \textsc{2pl}, occur in two to four case forms each, as displayed in \tabref{tab:5-1}.\footnote{Throughout this work, the pronouns are glossed according to formal distinctions only. That means, \textsc{1sg} \textit{ma} and \textsc{2sg} \textit{tu} are glossed as \textsc{nom} regardless of their functions (as subjects, direct objects or postpositional objects, etc.), and \textsc{1sg} \textit{míi} and \textsc{2sg} \textit{thíi} are glossed as \textsc{gen} regardless of their functions (as heads of possessive constructions or subjects of perfective transitive clauses, etc.).} While the singular persons make a~two"=way nominative"=genitive distinction, the plural persons make a~four"=way distinction between nominative, accusative, genitive and ergative.


\begin{table}[ht]
 \label{bkm:Ref193699445}
 \caption{Personal pronouns}
\begin{tabularx}{\textwidth}{ l Q Q Q Q }
\lsptoprule
&
\multicolumn{1}{l}{Nominative} &
\multicolumn{1}{l}{Accusative\footnote{\textit{máa=the} `to me' is special. As far as I know, it is only
 together with this particular postposition that the vowel of the pronoun is lengthened. What this
 shows is that the pronoun forms a~phonological word with the postposition and has gone through
 regular \textit{a}-lengthening.}} &
\multicolumn{1}{l}{Ergative} &
\multicolumn{1}{l}{Genitive}\\\midrule
\textsc{1sg} &
\textit{ma} &
\textit{ma, máa=} &
\textit{míi} &
\textit{míi}\\
\textsc{2sg} &
\textit{tu} &
\textit{tu} &
\textit{thíi} &
\textit{thíi}\\
\textsc{1pl} &
\textit{be} &
\textit{asaám} &
\textit{asím} &
\textit{asíi} (B \textit{asée})\\
\textsc{2pl} &
\textit{tus} &
\textit{tusaám} &
\textit{tusím} &
\textit{tusíi} (B \textit{tusée})\\\lspbottomrule
\end{tabularx}
\label{tab:5-1}
\end{table}




That means that, whereas most nouns only distinguish between nominative and oblique case (beside the genitive forms), and the agent of transitive perfective clauses is expressed in the oblique, the plural personal pronouns have a~unique ergative pronominal form; compare the ergative \textsc{1pl} \textit{asím} in example (\ref{ex:5-2}) with the nominative \textit{be} and accusative \textit{asaám} in example (\ref{ex:5-1}). 

\ea
\label{ex:5-1} 
\gll \textbf{be} eetáa yhóol-a ta hiimaál čhinǰ-í \textbf{asaám} híṛ-a\\
	\textsc{1pl.nom} there.\textsc{rem} come.\textsc{pfv"=mpl} when avalanche strike\textsc{-cv} \textsc{1pl.acc} take.away.\textsc{pfv"=mpl}\\
\glt `When we came there, an~avalanche hit and swept us away.' (A:ACR012) 
\ex
\label{ex:5-2}
%modified
\gll \textbf{asím} tu na bulaḍíl-u hín-u \\
	\textsc{1pl.erg} \textsc{2sg.nom} \textsc{neg} call.\textsc{pfv"=msg} be.\textsc{prs"=msg} \\
\glt `We haven't called you.' (A:GHU030)
\z

The first- and second"=person singular pronouns on the other hand make fewer distinctions than most full nouns, with their two"=way case contrast. As in Kohistani Shina, the oblique case has merged with the nominative \citep[82]{schmidtkohistani2008}. The nominative is thus used for intransitive subjects, as the \textsc{1sg} \textit{ma} in example (\ref{ex:5-3}), all direct objects, as the \textsc{1sg} \textit{ma} in example (\ref{ex:5-5}), as well as for transitive agents in the imperfective, while the genitive covers the possessor of a~possessive construction, as the \textsc{1sg} \textit{míi} in example (\ref{ex:5-3}), as well as the agent of perfective clauses, as exemplified by \textsc{1sg} \textit{míi} in (\ref{ex:5-4}). 


\begin{exe}
\ex
\label{ex:5-3}
%modified
%modified
\gll \textbf{ma} seé hín-u \textbf{míi} kúṛi seé hín-i \\
	\textsc{1sg.nom} sleep.\textsc{cv} be.\textsc{prs"=msg} \textsc{1sg.gen} woman sleep.\textsc{cv} be.\textsc{prs-f} \\
\glt `I was asleep and my wife was also asleep.' (A:HUA015)

\ex
\label{ex:5-4}
%modified
\gll \textbf{míi} lhéṇḍu láad-u \\
	\textsc{1sg.gen} bald.one find.\textsc{pfv"=msg} \\
\glt `I found the bald one.' (A:KAT119)

\ex
\label{ex:5-5}
%modified
\gll lo \textbf{ma} kha-áan-u than-íi de \\
	\textsc{3sg.dist.nom} \textsc{1sg.nom} eat-\textsc{prs"=msg} say-\textsc{3sg} \textsc{pst} \\
\glt `He said, ``It will eat me up''.' (A:GHA018)
\end{exe}

The form used for direct objects is also the form preceding postpositions. It would be possible to analyse some of those combinations of pronoun and postposition (examples (\ref{ex:5-6}) and (\ref{ex:5-7})) as case forms, as the two tend to be phonologically slightly more fused as compared to what is observed with most combinations of full nouns and postpositions. However, for the sake of descriptive economy, I have here chosen to treat them as postpositions. 

\begin{exe}
\ex
\label{ex:5-6}
%modified
\gll \textbf{asaám} \textbf{the} xabaár dít-i \\
	1\textsc{pl}.\textsc{acc} to news give.\textsc{pfv"=fsg} \\
\glt `We heard about it.' (A:GHA006)

\ex
\label{ex:5-7}
%modified
\gll \textbf{tu} \textbf{díi} ma ɡóobina lhéest-i \\
\textsc{2sg.nom} from \textsc{1sg.nom} nowhere escape.\textsc{pfv-f} \\
\glt `Nowhere am I safe from you.' (A:PAS126)
\end{exe}

There is nothing comparable in Palula to the honorific levels in second"=person pronouns found in regionally influential Urdu \citep[17]{schmidt1999}, and to a~lesser degree in Pashto (own personal observations). The use of the second"=person singular is in itself not perceived as derogatory or unsuitable, and neither is the second"=person plural pronoun ever used with singular reference in Palula.

To personal pronouns can also be added a~particle of emphasis or exclusiveness, \textit{eé} (in B \textit{e}), to be compared with exclusiveness as expressed with numerals \sectref{subsec:6-4-1}. It seems to be most frequently used together with the genitive form of the first- and second"=person singular pronouns. In example (\ref{ex:5-8}) it has the approximate meaning `my \textit{own}'.



\begin{exe}
\ex
\label{ex:5-8}
%modified
\gll fazelnuúr ba \textbf{míi} \textbf{e} ɡaaḍubáabu de \\
	Fazel.Noor \textsc{top} \textsc{1sg.gen} \textsc{ampl} grandfather be.\textsc{pst} \\
\glt `And Fazel Noor was my own grandfather.' [Preceded by a~genealogy starting with a~distant forefather.] (B:ATI078)
\end{exe}

However, in uses with nominative pronominal forms, as in (\ref{ex:5-9}), it takes on a~meaning similar to `I myself'. Note that this, in Palula, is a~function distinct from that of the reflexive pronouns (see \sectref{sec:5-4}).

\begin{exe}
\ex
\label{ex:5-9}
%modified
\gll típa aní mauǰudá waqt-íi \textbf{ma} \textbf{eé} bakáara-m-ii kasúb th-áan-u\\
	now \textsc{prox} present time.\textsc{gen} \textsc{1sg.nom} \textsc{ampl} herd-\textsc{obl"=gen} occupation do-\textsc{prs"=msg}\\
\glt `Nowadays I myself am engaged in shepherding.' (A:KEE079)
\end{exe}


A~reduplicated form of a~pronoun and the suffix, as in (\ref{ex:5-10}), is used in approximately the same sense.


\begin{exe}
\ex
\label{ex:5-10}
%modified
\gll \textbf{mam} \textbf{eé} ɡíi hín-i aḍaphaár tií\\
\textsc{1sg.nom.red} \textsc{ampl} go.\textsc{pfv.fsg} be.\textsc{prs-f} halfways until \\
\glt `I myself went halfways.' (A:CAV010)
\end{exe}

The particle \textit{eé} can also be used with  a demonstrative, as in (\ref{ex:5-11}). 


\begin{exe}
\ex
\label{ex:5-11}
%modified
\gll dúu ta \textbf{míi} \textbf{ni} \textbf{eé} ac̣híi-a dúu ba tóoruṇ-a wée \\
	two \textsc{cntr} \textsc{1sg.gen} \textsc{prox} \textsc{ampl} eye-\textsc{pl} two \textsc{top} forehead-\textsc{obl} in \\
\glt `Two were those very eyes of mine, and two were up on my forehead.' (A:HUA114) 
\end{exe}

\section{Demonstratives}
\label{sec:5-3}

\subsection{Relationship to the larger pro"=form system}
\label{subsec:5-2-1}


As mentioned in the introduction to this chapter, the demonstrative pronouns are part of a~larger system of pro"=forms (pro"=NPs, pro"=adverbs, pro"=adjectives, etc.), displaying a certain degree of symmetry (\tabref{tab:5-2}). Most (but not all) of those pro"=forms have similar"=looking sets consisting of a~proximal, a~distal, a~remote and an~indefinite/interrogative member (see \citealt[187--188]{haspelmath1993} for some strikingly similar sets in Lezgian).



\begin{table}[ht]
\caption{Correlations between pro"=forms}
\begin{tabularx}{\textwidth}{ Q Q Q Q Q }
\lsptoprule
&
Proximal
&
Distal
&
Remote
&
Indefinite/{\allowbreak}interrogative\\\midrule
Attributive &
\textit{anú, eenú} &
\textit{aṛó, eeṛó} &
\textit{eesó} &
\textit{khayí, ɡa} \\
Nominal &
\textit{anú, eenú} &
\textit{aṛó, eeṛó} &
\textit{eesó} &
\textit{koó, ɡubáa} \\
Location I &
\textit{índa} &
\textit{eeṛáa} &
\textit{eetáa} &
\textit{ɡóo} \\
Location II\footnote{It is only with a~following postposition that these forms have a~locational interpretation. Used alone they (ergatively) code the subject of a~perfective transitive clause.} &
\textit{aníi=} &
\textit{eeṛíi=} &
\textit{eetíi=} &
\textit{kíi=} \\
Source &
\textit{andóoi} &
\textit{eeṛáai} &
\textit{eetáai} &
\textit{ɡóoi(i)} \\
Quality &
&
&
\textit{eeteeṇú} &
\textit{kateeṇú} \\
Quantity &
&
&
\textit{eetí} &
\textit{katí} \\
Manner &
&
&
\textit{eendáa=} &
\textit{kanáa=} \\\lspbottomrule
\end{tabularx}
\label{tab:5-2}
\end{table}

Somewhat simplified, a~consonantal element (\textit{-n-, -ṛ-, -s-/-t-} or \textit{k-/ɡ-}) regularly indicates the deictic function. Additional dimensions are relevant to certain subsets, such as case, number, gender and emphasis for the adnominally and nominally used demonstratives, animacy for indefinite/interrogative pronouns, and spatial specification (primarily) for locational pro"=forms. For further discussion and examples, see \sectref{subsec:5-2-2}, \sectref{subsec:5-2-7}, \sectref{subsec:7-1-1}--\sectref{subsec:7-1-2}, \sectref{subsec:7-1-4}--\sectref{subsec:7-1-5}, \sectref{subsec:14-2-2}, \sectref{subsec:14-2-4} and \sectref{subsec:14-3-2}.

\subsection{Demonstratives and third person}
\label{subsec:5-2-2}


As already hinted at above, a~basic three"=way distance differentiation is used extensively with demonstratives and has at least to a~certain degree been extended to the realm of unstressed third"=person reference. I have chosen to regard all of these as demonstratives, of which many can and are indeed being used as third"=person pronouns, some more so than others. It is even typologically speaking problematic to make any clear"=cut distinction between personal pronouns and demonstratives, a~fact pointed out, among others, by \citet[206]{himmelmann1996} and \citet[123--124]{kibrik2011}, which is why it makes most sense to present them together. A distinction, however, that needs to be made within the category of demonstratives, is that between pronominal (or pro"=NPs) and adnominal demonstratives. As pointed out by \citet[206]{himmelmann1996}, pronominally used demonstratives tend, from a~typological perspective, to occur with a~lower frequency and in fewer contexts than adnominally used demonstratives. Since the pronominal demonstratives often are derived from the adnominal ones, the former are usually more complex. 



As can be seen in \tabref{tab:5-3}, the basic adnominal demonstratives occur in only two different forms in each deictic set, as compared to the much more complex system of pronominal demonstratives that we will discuss shortly. The only contrast is one between the demonstrative functioning attributively with a~masculine singular nominative referent vs. a~referent agreeing with its head in any other case, gender or number category. The forms are identical to the pronominally used masculine singular nominative and feminine singular nominative, respectively (see below), a~situation not uncommon in languages in general \citep[214]{himmelmann1996}. 


\begin{table}[ht]
\caption{Adnominal demonstratives}
\begin{tabularx}{\textwidth}{ l@{\hspace{30pt}} Q Q }
\lsptoprule
&
Nominative masculine singular &
Non"=nominative/{\allowbreak}Plural/{\allowbreak}Feminine\\\midrule
Proximal &
\textit{anú, eenú} (B \textit{hanú}) &
\textit{aní, eení} (B \textit{haní})\\
Distal &
\textit{eeṛó} (B \textit{haṛó}) &
\textit{eeṛé} (B \textit{haṛé})\\
Remote &
\textit{eesó} (B \textit{hasó}) &
\textit{eesé} (B \textit{hasé})\\\lspbottomrule
\end{tabularx}
\label{tab:5-3}
\end{table}



In example (\ref{ex:5-12}), the masculine noun \textit{xooṛá} `betrothal' first occurs in the nominative and is preceded by the adnominal demonstrative in its masculine singular nominative form, then it occurs in the genitive and is then preceded by the adnominal demonstrative in its other (non"=nominative, plural or feminine) form.

\ea
\label{ex:5-12}
\gll khayí dees-á {\ob}\textbf{eeṛó} \textbf{xooṛá}{\cb}\textsc{\textsubscript{\upshape nom}}  hensíl-u de {\ob}\textbf{eeṛé} \textbf{xooṛá-ii}{\cb}\textsc{\textsubscript{\upshape gen}} ṭeem-í ǰhaamatroó mooǰúd na hensíl-u
heentá{\dots} \\
which day-\textsc{obl} \textsc{dist.nom.msg} betrothal stay.\textsc{pfv"=msg} \textsc{pst}
\textsc{dist} betrothal-\textsc{gen} time-\textsc{obl} son"=in"=law present not stay.\textsc{pfv"=msg} would\\
\glt `If, on the day that betrothal took place, the son"=in"=law were not present at the time of that betrothal{\ldots}' (A:MAR116-7)
\z

The relevant categories for the pronominally used (or substantivised) demonstratives, presented in
\tabref{tab:5-4}, are number, case, gender and distance/visibility, the first three also found with Palula nouns.\footnote{Some of the forms do not occur in text data but are affirmed through direct elicitation as possible forms.} We can also differentiate between so"=called strong and weak forms of demonstratives (or at least for most of them). Initial \textit{ee-} in the strong forms in A (the ones displayed in \tabref{tab:5-4}) regularly corresponds to B \textit{ha-}. The pronouns in the weak remote sets of the demonstratives (\textit{so}, etc.), are the default choice and most frequently used as typical third person pronouns, with easily accessible discourse referents \citep[432--433]{diessel2006}, and could be said to stand particularly close functionally to the personal pronouns proper. 


\begin{table}[htp]
\caption{Pronominal demonstratives (Only markedly different B forms are cited in the table.)}
\begin{tabularx}{\textwidth}{ l l l Q Q Q P{23mm} }
\lsptoprule
&
&
&
\textsc{Nom} &
\textsc{Acc} &
\textsc{Obl} &
\textsc{Gen}\\\midrule
\textbf{Proximal} &
&
&
&
&
&
\\
\textsc{sg} &
\textsc{m} &
weak &
\textit{nu} &
\textit{nis} &
\textit{níi} &
\textit{nisíi} \\
&
&
strong &
\textit{anú, eenú} &
\textit{anís} &
\textit{aníi} &
\textit{anisíi} \\
&
\textsc{f} &
weak &
\textit{ni} &
\textit{nis} &
\textit{níi} &
\textit{nisíi} \\
&
&
strong &
\textit{aní, eení} &
\textit{anís} &
\textit{aníi} &
\textit{anisíi} \\
\textsc{pl} &
&
weak &
\textit{ni} &
\textit{niaám} &
\textit{niním} &
\textit{niníi} \\
&
&
strong &
\textit{aní, eení} &
\textit{aniaám} &
\textit{aniním} &
\textit{aniníi} (B~\textit{haninúme})\\
\textbf{Distal} &
&
&
&
&
&
\\
\textsc{sg} &
\textsc{m} &
weak &
\textit{lo, ṛo} &
\textit{las, ṛas} &
\textit{líi, ṛíi} &
\textit{lasíi, ṛasíi} \\
&
&
strong &
\textit{eeṛó, aṛó} &
\textit{eeṛás, aṛás} &
\textit{eeṛíi, aṛíi} &
\textit{eeṛasíi, aṛasíi} \\
&
\textsc{f} &
weak &
\textit{le, ṛe} &
\textit{las, ṛas} &
\textit{líi, ṛíi} &
\textit{lasíi, ṛasíi} \\
&
&
strong &
\textit{eeṛé, aṛé} &
\textit{eeṛás, aṛás} &
\textit{eeṛíi, aṛíi} &
\textit{eeṛasíi, aṛasíi} \\
\textsc{pl} &
&
weak &
\textit{le, ṛe} &
\textit{lanaám, ṛanaám} &
\textit{laním, ṛaním} &
\textit{laníi, ṛaníi} \\
&
&
strong &
\textit{eeṛé, aṛé} &
\textit{eeṛanaám, aṛanaám} &
\textit{eeṛaním} &
\textit{eeṛaníi} (B~\textit{haṛenúme})\\
\textbf{Remote} &
&
&
&
&
&
\\
\textsc{sg} &
\textsc{m} &
weak &
\textit{so} &
\textit{tas} &
\textit{tíi} &
\textit{tasíi} \\
&
&
strong &
\textit{eesó} &
\textit{eetás} &
\textit{eetíi} &
\textit{eetasíi} \\
&
\textsc{f} &
weak &
\textit{se} &
\textit{tas} &
\textit{tíi} &
\textit{tasíi} \\
&
&
strong &
\textit{eesé} &
\textit{eetás} &
\textit{eetíi} &
\textit{eetasíi} \\
\textsc{pl} &
&
weak &
\textit{se} &
\textit{tanaám} &
\textit{taním} &
\textit{taníi} (B~\textit{tenúme})\\
&
&
strong &
\textit{eesé} &
\textit{eetanaám} &
\textit{eetaním} &
\textit{eetaníi} (B~\textit{hatenúme})\\\lspbottomrule
\end{tabularx}
\label{tab:5-4}
\end{table}


The masculine--feminine gender distinction is found only in the singular nominative, compare (\ref{ex:5-13}) and (\ref{ex:5-14}), whereas masculine and feminine are syncretised in the plural, as is evident when comparing (\ref{ex:5-15}) with (\ref{ex:5-15b}). 


\begin{exe}
\ex
\label{ex:5-13}
%modified
\gll \textbf{so} aní kaafir"=aan-óom sanɡí madád th-áan-u \\
\textsc{3msg}.\textsc{nom} \textsc{prox} unbeliever-\textsc{pl"=obl.pl} with help do-\textsc{prs"=msg}\\
\glt `He is helping these unbelievers.' (A:BEZ053)

\ex
\label{ex:5-14}
%modified
\gll tasíi múuṛ-a wée ba \textbf{se} baḍíl-i de \\
\textsc{3sg}.\textsc{gen} lap-\textsc{obl} in \textsc{top} \textsc{3fsg}.\textsc{nom}
grow.\textsc{pfv-f} \textsc{pst}\\
\glt `She had grown up in his lap [i.e., in his house].' (A:PAS132)

\ex
\label{ex:5-15}
%modified
\gll \textbf{se} bhraawú bh-áan-a\\
\textsc{3pl}.\textsc{nom} brother.\textsc{pl} become-\textsc{prs"=mpl}\\
\glt `They become brothers.' (A:MIT011)

\ex
\label{ex:5-15b}
%modified
\gll kareé=ɡale \textbf{se} múṛ-im ta, asím  tenaám ḍhanɡéel-im\\
when=ever \textsc{3pl.nom} die.\textsc{pfv}-\textsc{fpl} \textsc{sub} \textsc{1pl.erg} \textsc{3pl.acc} bury.\textsc{pfv}-\textsc{fpl} \\
\glt `When they [the woman and her daughter] died, we buried them.'\newline (B:FOR037)
\end{exe}


The primary function of the distance differentiation is to distinguish three basic degrees of distance from the speaker/experiencer, where the proximal is used with referents close at hand, the distal with referents further removed from the speaker, and the remaining category, here labeled ``remote'', is used to refer to something or someone out of sight. 


The weak--strong opposition is related to accessibility. The strong forms tend to be picked for deictic or anaphoric use in order to keep track of less accessible discourse referents, or to contrast another referent or switch from one referent to another. They are also the ones used in relative"=correlative clauses and in most cases as adnominal demonstratives. In the example sentence (\ref{ex:5-16}), the speaker has for some time been talking about making doors when building a~house, but is here switching to talking about windows instead. The weak forms, on the other hand, are used almost exclusively non"=contrastively and to pick out the most natural or accessible referent in a~discourse. In example (\ref{ex:5-17}) the antecedent of the pronoun (the cupboard) is very close and there is no ambiguity in what is referred to.


\begin{exe}
\ex
\label{ex:5-16}
\gll seentá \textbf{eetás} samóol"=ii pahúrta théeba khiṛkí bi tarkaáṇ
khooǰa-áan-u\\
when \textsc{3sg}.\textsc{rem.acc} build.\textsc{pptc"=gen} after then window also carpenter ask-\textsc{prs"=msg} \\
\glt `Then when that has been built, the carpenter will ask about the window.' (A:HOW044)
\end{exe}


\begin{exe}
\ex
\label{ex:5-17}
\gll aalmaaríi bi muxtalíf dizeen-í yh-éend-i\\
cupboard.\textsc{gen} also different design-\textsc{pl} come-\textsc{prs-f}\\
\glt `A cupboard can also have different designs.' 

%modified
\gll \textbf{tasíi} xaaneé muxtalíf muxtalíf yh-áand-a \\
\textsc{3sg.gen} shelf.\textsc{pl} different different come-\textsc{prs"=mpl} \\
\glt `It can have many different kinds of shelves.' (A:HOW049-50)
\end{exe}

Thus, it is the weak form that functions most like a~third"=person pronoun. The contrast is similar to the distinction \citet[417--419]{givon2001a} makes between ``stressed independent pronouns'' and ``unstressed anaphoric pronouns''. Palula has also developed definite articles, formally identical to the nominative weak remote/non"=visible demonstratives (see \sectref{subsec:5-2-6}). This strong--weak continuum mirrors in many respects the relative anaphoric distance of \citet[419]{givon2001a}, as well as the grammaticalisation cline described by \citet[432]{diessel2006}. However, the absence of an~absolute distinction between demonstratives and third"=person pronouns across the board in Palula, and the presence of a~number of borderline cases, may very well reflect an~ongoing grammaticalisation \citep[213]{himmelmann1996}. The weak distal forms with \textit{ṛ} and \textit{l}, respectively, are mere pronunciation variants, of which only the \textit{l}-variant is phonologically possible in B. Nor is there anything indicating any significance in the alternation between the strong distal forms beginning with \textit{a-} and \textit{ee-}, respectively, but more detailed analysis is needed to be certain.



There is also another continuum that cuts across the entire paradigm. While the remote category is primarily a~phoric device \citep[131]{saxena2006}, with a~high frequency in, for instance, narrative discourse, and only secondarily functions deictically, the opposite holds for the proximal category, which is primarily deictic and only rarely functions phorically. The mid"=level category, the distal, is used extensively in both functions. An indirect effect of the differences in deictic scope is the ability to signal a~contrast in the presence--absence of the third"=person object or person in the speech situation. It is normally only when a~person or object is present in the speech situation that the proximal and distal terms are used (exophorically or anaphorically), whereas the remote assumes their absence. The same holds for a~distinction between narration and direct speech - especially crucial in oral story"=telling. Along with other quotative devices, the proximal and distal terms refer to the outside world as perceived by the quoted speaker, whereas the narrator (whether identical to the former or not) would use the corresponding remote terms to refer to the same entities.



\citet[204--205, 207--212]{schmidt2000} and \citet[134--136]{schmidtkohistani2001} describe similar effects obtained in the use of different sets of pronouns in some other Shina varieties. In Tileli there are two sets of third"=person pronouns, where one set signals visibility and first"=hand knowledge, whereas the other one codes what is invisible, unknown or indirectly inferred. In Kohistani Shina, a~similar discrimination can be made between information derived by visual means and information received by other means, through contrasting a~proximal set of demonstratives with a~remote set. 



Three particular functions (exophoric, anaphoric and discourse"=deictic) will be further discussed and exemplified below (\sectref{subsec:5-2-3}--\sectref{subsec:5-2-5}), following the pragmatic taxonomy of demonstratives suggested by \citet[205--254]{himmelmann1996} and discussed by \citet[432]{diessel2006}. I do not have any clear examples of recognitional use, a~fourth category; however, there is evidence for the development of a~definite, or ``anaphoric article'' \citep[486]{juvonen2006}, contrasting with a~similarly developing indefinite article. These two will be discussed in \sectref{subsec:5-2-6}. A secondary spatial specification will also be touched upon briefly in \sectref{subsec:5-2-7}.


\subsection{Exophoric use}
\label{subsec:5-2-3}

The exophoric, or situational, use orients the hearer in the outside world and points to an~entity in the speech situation. This could be said to be the basic or original use of the demonstrative. In Palula, it is primarily the proximal and distal levels that enter into this function, and it is almost without exception the strong forms that are being used, whether pronominal or adnominal.


The proximal term \textit{anú, aní} `this' points to something or someone close at hand. In its most typical use, it is adnominal while pointing to an~object, as in (\ref{ex:5-18}).

\begin{exe}
\ex
\label{ex:5-18}
%modified
\gll \textbf{aní} \textbf{kakaríi} kasíi thaní \\
\textsc{prox} skull whose \textsc{quot} \\
\glt `[He] said, ``Whose skull is this?''.' (A:WOM459)
\end{exe}
In (\ref{ex:5-19}), on the other hand, it refers pronominally to a~man who in the quoted speakers' presence has just defeated a~fierce bear, what \citet[222]{himmelmann1996} describes as reference in the narrated situation. In example (\ref{ex:5-20}), it points, in a~similar fashion, to a~boy who is threatened with being captured and taken away from his stepfather, who is the speaker within the narrative, and who is standing right next to the boy. In (\ref{ex:5-21}), the location is the same as where the speaker finds himself in at the speaking moment, although talking about times long ago, and is therefore an~instance of situational use in the actual utterance situation \citep[222]{himmelmann1996}.

\begin{exe}
\ex
\label{ex:5-19}
%modified
\gll \textbf{aníi} asíi nóo zindá thíil-u thaní \\
\textsc{3sg.prox.obl} \textsc{1pl.gen} name life do.\textsc{pfv"=msg} \textsc{quot} \\
\glt `They said, ``This one has saved our reputation.''' (A:BEW008)

\ex
\label{ex:5-20}
%modified
\gll \textbf{hanís} ɡhin-í háar"=ui típa ba \\
\textsc{3sg.prox.acc} take-\textsc{cv} take.away-\textsc{imp.pl} now \textsc{top} \\
\glt `Then, go ahead and take him away!' (B:ATI066)

\ex
\label{ex:5-21}
%modified
\gll \textbf{anú} \textbf{watán} áa zanɡál de \\
\textsc{3msg.prox.nom} land \textsc{idef} forest be.\textsc{pst} \\
\glt `This land was like a~forest.' (A:ANC006)
\end{exe}
The proximal term could also refer to temporal closeness, such as in (\ref{ex:5-22}), where it refers to the time of the utterance.

\begin{exe}
\ex
\label{ex:5-22}
\gll \textbf{aní} \textbf{dees-óom} atsareet-á wée qariibán čúur zára kušúni
hín-a \\
\textsc{prox} day-\textsc{obl.pl} Ashret-\textsc{obl} in about four thousand.\textsc{pl} households
be.\textsc{prs"=mpl} \\
\glt `In these days there are about 4,000 households in Ashret.'\footnote{It must be assumed that the speaker means `4,000' inhabitants, although he uses a~word normally used for `household, family'.} (A:PAS007)
\end{exe}
This stands in contrast to the past times that this particular narrative focuses on, which are elsewhere in this discourse referred to as \textit{eesé waxtíi} `of those times', thus using the remote term.

A simultaneous (and explanatory) reference to, for example, a~body part of a~character in a~narrative and the body part of the speaker, in (\ref{ex:5-23}) (accompanied by a~pointing gesture), is also done with the proximal term.


\begin{exe}
\ex
\label{ex:5-23}
\gll áak"=ii ta ḍheerdáṛ nikhéet-i áak"=ii ba \textbf{aní} \textbf{phiaaṛmaǰ-í} wée breéx nikhéet-i\\
one-\textsc{gen} \textsc{cntr} stomach.pain appear.\textsc{pfv-f} one-\textsc{gen} \textsc{top}
\textsc{prox} side-\textsc{obl} in pain appear.\textsc{pfv-f}\\
\glt `One of them felt pain in his stomach, and the other in this side.' (A:GHA059)
\end{exe}
The distal term \textit{aṛó} (B \textit{haṛó}) `that' points to something a~little farther away, but still fully visible and identifiable, as in (\ref{ex:5-24}), where it refers to something in the actual utterance situation, and in (\ref{ex:5-25}), where it refers to something in the narrated situation. 

\begin{exe}
\ex
\label{ex:5-24}
%modified
\gll \textbf{haṛó} kasée ɡhoóṣṭ \\
\textsc{3msg.dist.nom} whose house \\
\glt `Whose house is that [pointing]?' (B:DHN4839)

\ex
\label{ex:5-25}
%modified
\gll \textbf{aṛó} \textbf{míiš} thíi khaṇíit-u ɡa heentá ma tu the bíiḍ-i inaám d-úum \\
\textsc{dist.msg.nom} man \textsc{2sg.gen} hit.\textsc{pfv"=msg} what would \textsc{1}\textsc{sg.nom} 
\textsc{2sg.nom} to much-\textsc{f} gift give-\textsc{1}\textsc{sg} \\
\glt `If you hit that man, I will reward you richly.' (A:BEZ011)
\end{exe}

It is probably futile to try and give absolute measures of when the proximal term is used as compared to when the distal term is used. It is more a~matter of relative distance, as perceived by the speaker/experiencer in the situation. There are no apparent animacy restrictions on any of these terms when used exophorically, which seems to be the case with the corresponding terms in, for instance, Kohistani Shina \citep[135]{schmidtkohistani2001}. In as far as the remote term is used exophorically, it is in referring to an entity that is \textit{not} present or visible in the utterance situation. 


\subsection{Anaphoric use}
\label{subsec:5-2-4}

The anaphoric, or tracking, use is used to keep track of referents in discourse, and can been seen as being derived from the exophoric use by analogical extension, from pointing in the outside world to ``pointing'' to referents within the discourse. The remote is used particularly often for this, but the distal sets are also frequently used, whereas the use of the proximal seems highly limited in this regard. Depending on how easily accessible the referents are, either the strong or the weak form may be used. The more accessible, the more likely it is that the weak (unstressed) form is chosen, whereas the greater the need to refocus the hearer's attention, the more likely it is that the strong (stressed) form is selected. 


In a~passage, exemplified in (\ref{ex:5-26}), belonging to a~procedural discourse, the remote term is used, both in its weak and its strong form.
 
\ea
\label{ex:5-26}
\gll tsiip-áan-a tsiip-áan-a tsiip-áan-a tas díi \textbf{áa} \textbf{paṇáar-u}
 \textbf{šay} nikh-áand-u \textbf{tas} the man-áan-a iští \textbf{eetás} matíl-u
seentá \textbf{tasíi} bi ɡhiíṛ bh-áan-u\\
squeeze-\textsc{prs"=mpl} squeeze-\textsc{prs"=mpl} squeeze-\textsc{prs"=mpl} \textsc{3sg.acc} from
\textsc{idef} white-\textsc{msg} thing come.out-\textsc{prs"=msg} \textsc{3sg.acc}
to say-\textsc{prs"=mpl} ``ishti'' \textsc{3sg.rem.acc} churn.\textsc{pfv"=msg}
\textsc{condh} \textsc{3sg.gen} also butter become-\textsc{prs"=msg}\\
\glt `We keep squeezing, and from it comes out a~white part. We call that/it ``ishti''. When that has been
churned, it becomes butter.' (A:KEE040-1)
\z

Although the strong form tends to be used as an~anaphoric demonstrative~-- with a~refocusing or contrastive function \citep[432]{diessel2006}~-- and the weak tends to be used as a~personal pronoun, the distinction between the two is not all that clear and is deserving of further research.

When the distal term is used anaphorically (without any additional exophorical function), it is usually coreferential with a~main character or a~referent with special focus. Whereas the other referents, in an~exposé on the benefits of keeping goats, in (\ref{ex:5-27}), are referred to by the remote term, the goats are repeatedly~-- although not exclusively~-- referred to by the distal term (mostly in the singular). 

\ea
\label{ex:5-27}
\gll \textbf{eeṛé} \textbf{čhéeli} seetíl-i seentá /{\ldots}/ \textbf{las} eendáa=the saat"=eeṇḍeéu ki tariqá baándi wíi /{\ldots}/ \textbf{las} the beezáaya ṭoḍusóol-u seentá bi \textbf{le} xaraáp bh-éen-i\\
\textsc{dist} goat keep.\textsc{pfv-f} \textsc{condh} {} \textsc{3sg.dist.acc} like.this=do.\textsc{cv} keep-\textsc{oblg} \textsc{comp} method by water {} \textsc{3sg.dist.acc} to extra cut.\textsc{pfv"=msg} \textsc{condh} also \textsc{3fsg}.\textsc{dist.nom} spoiled become-\textsc{prs-f}\\
\glt `This goat should be cared for like this{\ldots} She should be given water at the proper time{\ldots} Also, if she is fed too much, it is not good for her.' (A:KEE012-4)
\z

\todo{typesetting only done up to this point}
When the term is used in this discourse, it is first in the strong form, but as it continues to be referred to, the weak form is used, thus functioning more like a~regular third"=person pronoun.

In a~particular historical account, of which (\ref{ex:5-28}) is part, two groups of people are referred to, and while the remote term is used for either of them when it is clear which one is intended, the distal is used at a~point when there is a~need for special emphasis or clarification.

\begin{exe}
\ex
\label{ex:5-28}
%modified
\gll ma díi xu \textbf{aṛanaám} mheeríl-a, uɡheeṇíi-a mheeríl-a\\
\textsc{1sg}.\textsc{nom} from but \textsc{3pl.}\textsc{dist.acc} kill.\textsc{pfv"=mpl} Pashtun-\textsc{pl} kill.\textsc{pfv"=mpl}\\
\glt `They [the people you gave me] were killed, the Pashtuns.' (A:BEZ116-7)\\
\end{exe}
When a~referent is present and close by in the speech situation (in (\ref{ex:5-29}), a~shepherd's flock of sheep and goats), it is referred to anaphorically with the promixal term.

\begin{exe}
\ex
\label{ex:5-29}
%modified
\gll \textbf{aniaám} wíi keé-na pila-áan-u thaní \\
\textsc{3pl.prox.acc} water why-\textsc{neg} make.drink-\textsc{prs"=msg} \textsc{quot}\\
\glt `Why not let them/these drink water.' (A:PAS070)
\end{exe}
The latter could alternatively be described as an~instance of the proximal term being used as (an unstressed) third"=person pronoun, since there is no competing antecedent, and the pronoun is not really used in order to refocus the listeners.

\subsection{Discourse"=deictic use}
\label{subsec:5-2-5}

The discourse"=deictic use has to do with entire propositions being referenced rather than with tracking individual discourse participants. Its function is to combine chunks of discourse \citep[432]{diessel2006}, and is thus more abstract in its scope. In this rather specialised function, it seems the distal term is used most of the time. It may be either anticipatory, as in (\ref{ex:5-30}), or refer back to a~rather long preceding chunk of discourse as in (\ref{ex:5-31}).

\begin{exe}
\ex
\label{ex:5-30}
%modified
\gll tartíb lasíi \textbf{eeṛé} ki...\\
method \textsc{3sg}.\textsc{dist.gen} \textsc{3fsg}.\textsc{dist.nom} \textsc{comp}\\
\glt `Its method is as follows{\ldots}.' (A:KEE024)

\ex
\label{ex:5-31}
%modified
\gll čhéelii phaaidá \textbf{eeṛó}\\
goat.\textsc{gen} benefit \textsc{3msg}.\textsc{dist.nom}\\
\glt `These are the benefits of the goat.' (A:KEE065)
\end{exe}

Although the typical use seems to be pronominal, it also occurs adnominally, as in (\ref{ex:5-32}). It usually occurs in its strong form, but it may occasionally be less emphasised, as in (\ref{ex:5-33}), where the weak form refers back to the preceding proposition, thus ``naming'' what has already been defined through discourse.

\begin{exe}
\ex
\label{ex:5-32}
%modified
\gll thée se míiš-a the \textbf{eeṛó} \textbf{qisá} páta de ki bhalaá áa čhaṭ-í baándi ta mar-áan-u\\
then \textsc{def} man-\textsc{obl} to \textsc{dist.msg.nom} story knowledge be.\textsc{pst} \textsc{comp} evil.spirit one shot-\textsc{obl} by \textsc{cntr} die-\textsc{prs"=msg}\\
\glt `Then the man remembered the saying that an~evil spirit dies by a~single shot.' (A:WOM675)
\end{exe}
\begin{exe}
\ex
\label{ex:5-33}
%modified
\gll \textbf{las} the asíi čoolaá man-áan-a dúula\\
\textsc{3sg}.\textsc{dist.acc} to \textsc{1pl.gen} language say-\textsc{prs"=mpl} ``duula''\\
\glt `In our language we call this ``duula'' [appr. proposal of marriage].'\newline (A:MAR026)
\end{exe}
If the demonstrative is immediately introducing a~chunk of discourse, the proximal term can also be used, as in (\ref{ex:5-34}), probably with the additional effect that the listener is invited into a~private exchange between a~husband and his wife. 

\begin{exe}
\ex
\label{ex:5-34}
%modified
\gll teeṇíi maǰí \textbf{eenú} \textbf{mašwará} thíil-u \\
\textsc{refl} among \textsc{prox.msg.nom} consultation do.\textsc{pfv"=msg}\\
\glt `Between themselves they made this plan{\ldots}' (A:WOM474)
\end{exe}

The same `plan' or `consultation' is a~little later, in (\ref{ex:5-35}), referred to with the distal term, then from the perspective of a~secret listener, standing outside the house where the exchange was taking place. The remote term does not seem to be frequently used discourse"=deictically. A possible example may be the one in (\ref{ex:5-36}), where the demonstrative refers to a~point in time in a~narrative, the `night' following the events described in the immediately preceding chunk of discourse. However, this rather specialised function, \citet[225]{himmelmann1996} defines as a~subtype of discourse"=deictic use.

\ea\label{ex:5-35}
%modified
\gll dharéndi ba áa bhalaá kúṛi=ee míiš-ii \textbf{eeṛó} \textbf{mašwará} ṣúṇ-a de \\
outside \textsc{top} \textsc{idef} evil.spirit woman=\textsc{cnj} man-\textsc{gen} \textsc{dist.msg.nom} consultation listen-\textsc{3sg} \textsc{pst} \\
\glt `Outside an~evil spirit was listening to that plan the wife and husband made.' (A:WOM632)
\z

\begin{exe}
\ex
\label{ex:5-36}
%modified
\gll \textbf{eesé} \textbf{róot-a} tíi se ǰinaazá the róota ḍipṭí thíil-i\\
\textsc{rem} night-\textsc{obl} \textsc{3sg.obl} \textsc{def} corpse to night-\textsc{obl} duty do.\textsc{pfv-f}\\
\glt `That night he held a~vigil for the corpse.' (A:WOM665)
\end{exe}

Sometimes, as in (\ref{ex:5-37}), there is also an~ambiguity as to whether the demonstrative refers anaphorically to a~particular discourse referent or to a~preceding chunk of discourse. Here, the planks as well as the totality are possible referents for the demonstrative.

\begin{exe}
\ex
\label{ex:5-37}
%modified
\gll aalmaarí wée ba páanǰ paaw-á ḍhínɡar bhit bhakúl-u laɡaiǰíl-u heentá \textbf{eesó} bíiḍ-u šóo bh-áan-u \\
cupboard into \textsc{top} five quarter-\textsc{pl} wood plank thick-\textsc{msg} be.put.\textsc{pfv"=msg} \textsc{condl} \textsc{3msg.rem.nom} much-\textsc{msg} good.\textsc{msg} become-\textsc{prs"=msg}\\
\glt `If five quarters of wooden planks are put into the cupboard, that will look very beautiful.'
(A:HOW051)
\end{exe}



\subsection{Article"=like uses}
\label{subsec:5-2-6}

Closely related to adnominally"=used demonstratives, and still possibly a~subcategory of them, are the forms \textit{so} and \textit{se} (formally the weak forms of the remote set), used for signalling definite or previously introduced and firmly established entities in the discourse. Their distribution (\tabref{tab:5-5}) is the same as that of the adnominal demonstratives: the form \textit{so} is used preposed to a~masculine singular nominative head, whereas \textit{se} is used with a~head in any other case, gender or number. 


\begin{table}[ht]
 \caption{Definite ``articles''}
\begin{tabularx}{\textwidth}{ Q Q }
\lsptoprule
Nominative masculine singular head &
Other (non"=nominative/plural/feminine) head\\\midrule
\textit{so} &
\textit{se} \\\lspbottomrule
\end{tabularx}
\label{tab:5-5}
\end{table}

Contrasting in definiteness and identifiability with \textit{so/se} used in this way, is the ``indefinite article'' \textit{áa} or \textit{áak} (B \textit{a} or \textit{ak}), derived from, and in one of its forms still homonymous with, the numeral \textit{áak} `one'. While \textit{áa} is used for introducing a~participant, \textit{áa míiš} `a (certain) man', \textit{so/se} is used in order to refer to or track an~already earlier introduced participant, \textit{so míiš} `the/that man'. The examples (\ref{ex:5-38})--(\ref{ex:5-41}) are all taken from one and the same story, where the two main characters are a~man and a (male) monster. At the very beginning of the story, in (\ref{ex:5-38}), the man is introduced, with a~full noun and the indefinite \textit{áa}.

\begin{exe}
\ex
\label{ex:5-38}
%modified
\gll muṣṭú zamanáii \textbf{áa} \textbf{míiš} de\\
before time.\textsc{gen} \textsc{idef} man be.\textsc{pst}\\
\glt `Once upon a~time, there was a~man.' (A:THA001)
\end{exe}

In the sentences that immediately follow (not included here), the man is referred to by
a~third"=person pronoun only (i.e., by the remote demonstrative, as described above), and we learn
that the man goes hunting, shoots a~deer and takes it to a~hut, where he prepares the meat and
starts eating. Then, in (\ref{ex:5-39}), the next participant shows up, the monster, which is also
introduced by a~full noun preceded by the indefinite \textit{áa} and a~descriptive attribute,
`hairy'.
\begin{exe}
\ex
\label{ex:5-39}
%modified
\gll tíi maǰí \textbf{áa} \textbf{ǰhaṭíl-u} \textbf{ṭhaaṭáaku} yhóol-u\\
\textsc{3sg.obl} on \textsc{idef} hairy-\textsc{msg} monster come.\textsc{pfv"=msg}\\
\glt `Meanwhile a~hairy monster came along.' (A:THA005)
\end{exe}

In the next sentence, in (\ref{ex:5-40}), the \textit{ṭhaaṭáaku} `monster' is referred to by a~full noun
only, without any determiner or attribute (which perhaps has some sort of intermediate status
between the indefinite and the definite), while \textit{míiš} `man', when he is referred to
again, is done so through the use of a~full noun marked with the definite \textit{so}.
\begin{exe}
\ex
\label{ex:5-40}
%modified
%modified
\gll \textbf{ṭhaaṭáaku} yhaí šíiṭi ačíit-u ta \textbf{so} \textbf{míiš} mhaás khóo de\\
monster come.\textsc{cv} inside enter.\textsc{pfv"=msg} \textsc{sub} \textsc{def.msg.nom} man meat eat.\textsc{3sg} \textsc{pst}\\
\glt `When the monster came inside, the man was eating.' (A:THA006)
\end{exe}

In the same way, when the monster is referred to in the following sentence, in (\ref{ex:5-41}), after the man is mentioned, he too is referred to by using a~full noun and the definite \textit{se} (that form because the head is in the oblique case).
\begin{exe}
\ex
\label{ex:5-41}
%modified
%modified
\gll théeba \textbf{se} \textbf{ṭhaaṭáak-a} bi \textbf{tas} sanɡí khainií široó thíil-u\\
then \textsc{def} monster-\textsc{obl} also \textsc{3sg.acc} with eat.\textsc{vn} start do.\textsc{pfv"=msg}\\
\glt `And the monster started eating with him.' (A:THA007)
\end{exe}

Having exemplified this use of the ``definite article'' primarily as a~tracking device, it should be pointed out that there are no clear and unambiguous examples of it being used as identifiable in the larger situation or associative"=anaphorically, two uses that are typical for definite articles, while not generally associated with demonstratives in general (\citealt[485]{juvonen2006}; \citealt[233]{himmelmann1996}). Therefore, since the use of adnominal \textit{so/se} is not dramatically different from the anaphoric use of adnominal demonstratives, an~alternative labelling of it would be ``anaphoric article'' \citep[486]{juvonen2006}, but it also comes close to what \citet[230--239]{himmelmann1996} includes in the recognitional use of demonstratives. What sets it apart from the (strong) anaphoric use (see \sectref{subsec:5-2-4}), however, and thus points to it being further grammaticalised towards a~more typical definite article interpretation as ``identifiable'' in general \citep[485]{juvonen2006}, is its frequent and near"=obligatory presence with already introduced and well"=established referents in a~discourse, its weak, thus phonetically reduced, form and its paradigmatic contrast with the indefinite article.


Along the same lines, the use of \textit{áa/áak} exemplied above does not set it apart as radically estranged from its use as a~numeral, but it could be said to have gone through an~initial grammaticalisation stage (as a~phonetically reduced presentative marker) towards a~full"=blown indefinite article \citep[486]{juvonen2006}.



The full system of reference and deixis, needless to say, is deserving of much more in"=depth research, and the sections above should be taken as a~preliminary analysis and suggestion as to what devices are being used in Palula.


\subsection{Spatial specification}
\label{subsec:5-2-7}

In addition to the basic differentiation between remote, distal and proximal, there are a~number of further spatially defined specifications that can be made. To the demonstratives (particularly in their adnominal use) of the distal set, certain spatial adverbs (\sectref{subsec:7-1-2}) can be added (or fused with the demonstratives) to indicate finer degrees of distance as well as vertical and horizontal position in relation to the speaker, as shown in \tabref{tab:5-6}.


\begin{table}[ht]
\caption{Secondary spatial specifications of distal demonstratives}
\begin{tabularx}{\textwidth}{ l@{\hspace{25pt}} l@{\hspace{25pt}} Q }
\lsptoprule
&
Nominative &
Non"=nominative/{\allowbreak}Plural/{\allowbreak}Feminine\\\midrule
`that/those down there' &
\textit{bhunaṛó} &
\textit{bhunaṛé} \\
`that/those straight up there' &
\textit{huṇḍaṛó} &
\textit{huṇḍaṛé} \\
`that/those up there' &
\textit{aǰaṛó} &
\textit{aǰaṛé} \\
`that/those over there' &
\textit{pharaṛó} &
\textit{pharaṛé} \\
`that/those far away over there' &
\textit{phaaraṛó} &
\textit{phaaraṛé} \\\lspbottomrule
\end{tabularx}
\label{tab:5-6}
\end{table}

\section{Possessive pronouns}
\label{sec:5-4}


What is sometimes referred to as a~category of distinct possessive pronouns were already introduced above as genitive case forms of the personal and demonstrative pronouns, and are exemplified in (\ref{ex:5-42}) and (\ref{ex:5-43}).

\begin{exe}
\ex
\label{ex:5-42}
%modified
\gll ma bhíiru ɡhin-í \textbf{thíi} ɡhooṣṭ-á the yh-úum\\
\textsc{1sg.nom} he.goat take-\textsc{cv} \textsc{2sg.gen} house-\textsc{obl} to com-\textsc{1sg}\\
\glt `I will come to your house and bring a~he"=goat.' (A:MIT013)

\ex
\label{ex:5-43}
%modified
\gll bíiḍ-u ɡáaḍ-u \textbf{tesée} dabdabá de\\
much-\textsc{msg} big-\textsc{msg} \textsc{3sg.gen} pomp be.\textsc{pst}\\
\glt `His power was great.' (B:ATI022)
\end{exe}

\section{Reflexive pronouns}
\label{sec:5-5}

There is one frequently used pronoun, \textit{teeṇíi,} that can be described as reflexive. It
occurs in this form only, as is evident from the examples (\ref{ex:5-44})--(\ref{ex:5-47}). Usually, but not
exclusively, it is the possessor in a~possessive construction, and its referent is then identical
with the clause subject (whether explicit or implicit).

\begin{exe}
\ex
\label{ex:5-44}
%modified
\gll se míiš-e kiraamát míi \textbf{teeṇíi} ac̣híi-am drhíṣṭ-i\\
\textsc{def} man-\textsc{gen} power \textsc{1sg.gen} \textsc{refl} eye-\textsc{obl.pl} see.\textsc{pfv-f}\\
\glt `I saw the man's power with my own eyes.' (B:ATI072)

\ex
\label{ex:5-45}
%modified
\gll \textbf{teeṇíi} ak putr kaarél thaní hatáa ɡal-í ba ɡáu \\
\textsc{refl} one son Carel \textsc{quot} there throw-\textsc{cv} \textsc{top} go.\textsc{pfv.msg}\\
\glt `[He] left his son, called Carel, there and left.' (B:ATI010-1)

\ex
\label{ex:5-46}
%modified
\gll se bhalaá se kúṛi the maníit-u ki \textbf{teeṇíi} banɡleé na širinɡá\\
\textsc{def} spirit \textsc{def} woman to say.\textsc{pfv-msg} \textsc{comp} \textsc{refl} bracelet.\textsc{pl} \textsc{neg} rattle.\textsc{imp.sg}\\
\glt `The spirit said to the woman, ``Don't rattle your bracelets!''' (A:WOM643)

\ex
\label{ex:5-47}
%modified
\gll ǰanǰ ɡúum \textbf{teeṇíi} sanɡí bíiḍ-a ba xálak-a ɡhin-í ɡúum\\
wedding.party go.\textsc{pfv.msg } \textsc{refl} with much-\textsc{mpl} \textsc{top} people-\textsc{pl} take-\textsc{cv} go.\textsc{pfv.msg}\\
\glt `He went to a~wedding party, taking a~lot of people with him.' (A:GHU008)
\end{exe}
However, when the reflexive occurs in the object position, as in (\ref{ex:5-48}), it seems necessary to use the construction \textit{teeṇíi zaán} `own self', where the reflexive pronoun remains in the possessive.
\begin{exe}
\ex
\label{ex:5-48}
%modified
\gll karáaṛ-a díi handáa=the \textbf{teeṇíi} \textbf{zaán} bač thíl-i\\
leopard-\textsc{obl} from this.way=do.\textsc{cv} \textsc{refl} self salvation do.\textsc{pfv-f}\\
\glt `This way he saved himself from the leopard.' (B:CLE381)
\end{exe}

\section{Reciprocal pronouns}
\label{sec:5-6}

The pronoun \textit{akaadúi} (literally `one' + a~segment \textit{aa} + `other') is reciprocal,
as illustrated in examples (\ref{ex:5-49}) and (\ref{ex:5-50}).

\ea 
\label{ex:5-49}
%modified
\gll dúu ǰáan-a \textbf{akaadúi} xox bhíl-a heentá\\
two person-\textsc{pl} \textsc{recp} liking become.\textsc{pfv"=mpl} \textsc{condl}\\
\glt `If two people would become fond of one another{\ldots}' (A:MIT010)

\ex
\label{ex:5-50}
%modified
\gll taníi kuṇaak-á \textbf{akaadúi} ɡaaḍbáabu lhookbáabu man-áan-a\\
\textsc{3pl.rem.gen} child-\textsc{pl} \textsc{recp} father's.older.brother father's.younger.brother say-\textsc{prs"=mpl} \\
\glt `Their children call one another uncle.' (A:MIT026)
\z

The finite verb agrees in the plural, and if the agent/direct object is a~pronoun that makes a~case differentiation between agent and direct object, it occurs in the accusative as in (\ref{ex:5-51}).
\begin{exe}
\ex
\label{ex:5-51}
%modified
\gll eetanaám \textbf{akaadúi} mheeríl-a\\
\textsc{3pl.rem.acc} \textsc{recp } kill.\textsc{pfv"=mpl}\\
\glt `They killed each other.' (A:MAB004)
\end{exe}

The reciprocal pronoun may, however, also occur in case"=inflected forms, such as the genitive form in (\ref{ex:5-52}). 
\begin{exe}
\ex
\label{ex:5-52}
%modified
\gll dhuimeém \textbf{akaaduéem-e} haalat-í khooǰéel-i\\
both.\textsc{obl} \textsc{recp.obl"=gen } condition-\textsc{pl} ask.\textsc{pfv-f}\\
\glt `The two inquired about each other's well"=being.' (B:FOY002)
\end{exe}

\section{Indefinite"=interrogative pronouns}
\label{sec:5-7}


The area of indefinite pronouns and their distribution needs further research, but there are strong indications that most, if not all, indefinite pronouns are also used as interrogative pronouns (as further exemplified in \sectref{subsec:14-2-2}). As mentioned above (\sectref{subsec:5-2-1}), the indefinite"=interrogative pronouns (\tabref{tab:5-iipro}) belong together with the demonstratives in the larger system of pro"=forms. An animacy distinction is also made here, compare (\ref{ex:5-53}) and (\ref{ex:5-54}), not otherwise part of the pronominal system.

\begin{table}[H]
\caption{Indefinite"=interrogative pronouns}
\begin{tabularx}{\textwidth}{ l l l }
\lsptoprule
\textit{koó} &
`who, someone, anyone' &
nominative (animate)\\
\textit{kaseé} &
`whom, etc.' &
accusative (animate)\\
\textit{kasíi} (B \textit{kasée}) &
`whose, etc.' &
genitive (animate)\\
\textit{kií} &
`who, etc.' &
oblique (animate)\\
\textit{ɡubáa} &
`what, etc.' &
inanimate \\
\textit{khayú, khayí} &
`which one, etc.' &
attributive (animate/inanimate)\\
\textit{ɡa} &
`what kind, etc.' &
attributive (inanimate)\\\lspbottomrule
\end{tabularx}
\label{tab:5-iipro}
\end{table}
 
\ea
\label{ex:5-53}
%modified
\gll ɡokhíi-a asaám the ḍanɡarík than-áan-a \textbf{koó} ba asaám the kaaláaṣ-a than-áan-a\\
Chitrali-\textsc{pl} \textsc{1pl.acc} to Dangarik call-\textsc{prs"=mpl} some \textsc{top} \textsc{1pl.acc} to Kalasha-\textsc{pl} call-\textsc{prs"=mpl}\\
\glt `The Chitralis call us Dangarik, and some consider us Kalasha.' (A:ANJ015)

\ex
\label{ex:5-54}
%modified
\gll tíi wíi-a wée \textbf{ɡubáa} šay dhríṣṭ-u hín-u\\
\textsc{3sg.obl} water-\textsc{obl} in some thing see.\textsc{pfv"=msg} be.\textsc{prs"=msg}\\
\glt `She has seen something in the water.' (A:SHY053)
\z

See also \sectref{subsec:14-3-2} for a~possible development of a~set of negative indefinite pronouns (\textit{ɡá=bi=na} `nothing', \textit{koó=bi=na} `nobody', \textit{ɡóo=bi=na} `nowhere'), based on the basic indefinite ones.

\section{Relative pronouns}
\label{sec:5-8}


There are no distinct relative pronouns in Palula. Instead, the demonstratives and the indefinite"=interrogative pronouns are used in relative constructions, or in the functional equivalents of relative clauses (see \sectref{sec:13-6}).