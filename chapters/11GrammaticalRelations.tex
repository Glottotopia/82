\chapter{Grammatical relations}
\label{chap:11}

Palula, like many related IA languages, displays an~intricate and quite complex relationship between the grammatical cases expressed, the particular case forms and agreement patterns available and the various functions a~noun phrase can have in a~given utterance \citep[230--231]{masica1991}. A complicating factor is the type of split ergativity displayed, related to aspect on the one hand and the nature of the NP on the other. The former is a~rather straightforward matter, with ergativity being a~feature only of perfective clauses, whereas the relationship between properties of the NP and ergativity is less transparent, with several different cutoff points, some of them less expected from a~typological standpoint. Word order is unmarkedly intransitive subject~-- verb and transitive subject~-- object~-- verb, but it allows for quite a~deal of pragmatic flexibility. As the two other factors weigh heaviest as far as grammatical relations and alignment are concerned, word order will not enter into the present discussion. This also goes along with the general observation that the presence of the other mechanisms in a~language correlates to a~relative flexibility in the basic word order \citep[15]{blake1994}.



Following \citet[6--8]{dixon1994} and \citet[110--116]{comrie1989}, I will be using the following abbreviations for the three grammatical relations (also syntactic primitive relations): S~-- intransitive subject, A~-- transitive subject, and O~-- transitive object.\footnote{P (for patient) in some of the typological literature \citep{comrie1989,croft1990}.\par } This means that a~verb with one core noun phrase is intransitive and the sole argument relation we refer to as S, a~verb with two core noun phrases is transitive and the two argument relations we call A and O, respectively: S-V\textit{itr} and A-O-V\textit{tr}.


\section{Verb agreement}
\label{sec:11-1}

As already described in \sectref{subsec:8-4-1}, grammatical relations can be reflected in the marking of the predicate itself, i.e. by verb agreement. In Palula, the verb can only display agreement with a~single argument of the clause. While agreement in person is limited to the non"=present imperfective (Future and Past Imperfective), the main type of agreement is one in gender and number, found in the perfective categories and in Present tense (see \sectref{sec:9-1}). 


However, as far as grammatical alignment is concerned, the dividing line goes between the perfective and all non"=perfective TMA categories: In the Perfective there is ergative verb agreement, and in the non"=perfective there is accusative verb agreement. 


\subsection{Accusative alignment}
\label{subsec:11-1-1}


In the non"=present imperfective (Past Imperfective in (\ref{ex:11-1}) and Future in (\ref{ex:11-2})), the verb always agrees in person in accordance with an~accusative alignment, i.e. with the subject, whether S (as in (\ref{ex:11-1})) or A (as in (\ref{ex:11-2})). 

\begin{exe}
\ex
\label{ex:11-1}
\glll praší phará [se] b-éen de \\
slope along \textsc{3pl.nom} go-\textsc{3pl} \textsc{pst} \\
{} {}  \textbf{S} \\
\glt `They were moving along the slope.' (B:AVA211)

\ex
\label{ex:11-2}
\glll [se zinaawur-aán] xu [ma] kh-óon \\
\textsc{def} beast-\textsc{pl} but \textsc{1sg.nom} eat-\textsc{3pl} \\
 \textbf{A} {} {}  \textbf{O} \\
\glt `The beasts will eat me.' (A:KAT059)
\end{exe}

The same accusative agreement pattern is displayed in the Present, as can be seen in examples (\ref{ex:11-3})--(\ref{ex:11-4}), although here in the form of number/gender agreement.

\begin{exe}
\ex
\label{ex:11-3}
\glll [ma] rhoošíia sóon-a the biáan-u \\
\textsc{1sg.nom} tomorrow pasture-\textsc{ob} to go.\textsc{prs-}\textsc{msg} \\
\textbf{S} \\
\glt `I am going to the (high) pastures tomorrow.' (A:SHY028)

\ex
\label{ex:11-4}
\glll [tu] [aniaám] keé-na khiáan-u \\
\textsc{2sg.nom} \textsc{3pl.prox.acc} why-\textsc{neg} eat.\textsc{prs-}\textsc{msg} \\
\textbf{A} \textbf{O} \\
\glt `Why don't you eat these?' (A:KAT067)
\end{exe}

\subsection{Ergative alignment}
\label{subsec:11-1-2}


When, on the other hand, any of the TMA categories based on the Perfective are used, the verb always agrees~-- in accordance with an~ergative alignment~-- with S, as in (\ref{ex:11-5}), or O as in (\ref{ex:11-6}) and (\ref{ex:11-7}), whereas it never agrees in gender and number with A.

\begin{exe}
\ex
\label{ex:11-5}
\glll [čhéeli] eetáa the ɡíi de ta \\
she.goat[\textsc{fsg}] there to go.\textsc{pfv.}\textsc{f} \textsc{pst} \textsc{prt} \\
 \textbf{S} \\
\glt `The goat had gone there.' (A:CAV026)

\ex
\label{ex:11-6}
\glll [ínc̣-a] [čhéeli] khéel-i \\
bear[\textsc{msg}]-\textsc{ob} she.goat[\textsc{fsg}] eat.\textsc{pfv-}\textsc{f} \\
\textbf{A} \textbf{O} \\
\glt `The bear ate the goat.' (A:PAS056)

\ex
\label{ex:11-7}
\glll [kúṛi] teeṇíi deec̣hinéeti ḍáḍi [ɡhúuṛu] nuuṭóol-u \\
woman[\textsc{fsg}] \textsc{refl} right side horse[\textsc{msg}] turn.\textsc{pfv-}\textbf{\textsc{msg}} \\
\textbf{A} {} {} {}  \textbf{O} \\
\glt `The woman turned the horse to the right.' (A:UXW028)
\end{exe}

\section{NP case differentiation}
\label{sec:11-2}

Three sub"=instances of case differentiation will be exemplified below as they relate to grammatical relations and alignment: inflectional case marking of nouns, pronominal case differentiation and dependent marking.


\subsection{Inflectional case marking}
\label{subsec:11-2-1}


As we saw in \sectref{sec:4-5}, the central case distinction made inflectionally is that between the nominative and the oblique cases. As for the relations we are interested in, the nominative is used for S, A and O alike in the non"=perfective categories, whereas in the Perfective, A, as in (\ref{ex:11-10}), is singled out as coded in the oblique case (\textit{kaṭamuš-á}) versus the nominatively coded\footnote{In the description of a~system with a~consistently ergative alignment, this would be referred to as absolutive, a~term I prefer not to use when describing the split ergative system of Palula.} S (\ref{ex:11-8}) and O (\ref{ex:11-9}), \textit{kaṭamúš}.

\begin{exe}
\ex
\label{ex:11-8}
\glll [kaṭamúš-\textsc{ø}] sóon-a dúši ɡúum  hín-u \\
Katamosh pasture-\textsc{ob} toward go.\textsc{pfv.msg}  be.\textsc{prs"=msg} \\
\textbf{S} \\
\glt `Katamosh set out for the (high) pastures.' (A:KAT010)

\ex
\label{ex:11-9}
\glll íṇc̣-a [kaṭamúš-\textsc{ø}] aamúuṣṭ-u hín-u \\
bear-\textsc{ob} Katamosh forget.\textsc{pfv"=msg} be.\textsc{prs"=msg} \\
 \textbf{A} \textbf{O} \\
\glt `The bear forgot about Katamosh.' (A:KAT140)

\ex
\label{ex:11-10}
\glll [kaṭamuš-á] ɡábina khóol-u hín-u \\
Katamosh-\textsc{ob} nothing eat.\textsc{pfv"=msg} be.\textsc{prs"=msg} \\
 \textbf{A} \textbf{O} \\
\glt `Katamosh didn't eat anything.' (A:KAT065)
\end{exe}

This, however, is a~somewhat simplified picture. In actual fact, not all nouns make the distinction between the nominative and the oblique, and some make it in the plural and not in the singular. The forms themselves occurring as morphological markers of ergativity also differ between nouns belonging to different declensions (see \sectref{sec:4-6}): 


\begin{table}[H]
\begin{tabularx}{\textwidth}{ | P{27mm} | Q | Q | l Q | }
\hline
\textbf{\textit{a}-declension} &
\textit{putr} `son' \textit{(m)} &
\textsc{sg} &
&
\textsc{pl}\\\cline{2-5}
&
S/O &
\textit{putr} & \ligrcell{~}
&
\ligrcell{\textit{putrá}} \\\cline{2-3}\cline{5-5}
&
A &
\multicolumn{1}{ l}{\ligrcell{\textit{putrá}}} &
\multicolumn{1}{ l |}{\ligrcell{~}} &
\textit{putróom} \\\hline
\end{tabularx}
\end{table}


\begin{table}[H]
\begin{tabularx}{\textwidth}{ | P{27mm} | Q | Q | l Q | }
\hline
\textbf{\textit{i}-declension} &
\textit{ɡhaáu} `cow' \textit{(f)} &
\textsc{sg} &
&
\textsc{pl}\\\cline{2-5}
&
S/O &
\textit{ɡhaáu} & \ligrcell{~}
&
\ligrcell{\textit{ɡheyí}} \\\cline{2-3}\cline{5-5}
&
A &
\multicolumn{1}{ l}{\ligrcell{\textit{ɡheyí}}} &
\multicolumn{1}{ l |}{\ligrcell{~}} &
\textit{ɡheyíim} \\\hline
\end{tabularx}
\end{table}

\begin{table}[H]
\begin{tabularx}{\textwidth}{ | P{27mm} | l | Q | Q |}
\hline
\textbf{\textit{m}-declension} &
\textit{méemi} `grandmother' \textit{(f)} &
\textsc{sg} &
\textsc{pl}\\\cline{2-4}
&
S/O/A &
\textit{méemi} &
\textit{méemim} \\\hline
\end{tabularx}
\end{table}

\begin{table}[H]
\begin{tabularx}{\textwidth}{ | P{27mm} | l | Q | Q |}
\hline
\textbf{\textit{aan}-declension (V-ending)} &
\textit{baačaá} `king' \textit{(m)} &
\textsc{sg} &
\textsc{pl}\\\cline{2-4}
&
S/O &
\ligrcell{~} &
\textit{baačaán} \\\cline{2-2}\cline{4-4}
&
A & \ligrcell{\raisebox{1.6ex}[1.6ex]{\textit{baačaá}}}
&
\textit{baačaanóom} \\\hline
\end{tabularx}
\end{table}

\begin{table}[H]
\begin{tabularx}{\textwidth}{ | P{27mm} | l | Q | Q |}
\hline
\textbf{\textit{aan}-declension (C-ending)} &
\textit{anɡreéz} `Brit' \textit{(m)} &
\textsc{sg} &
\textsc{pl}\\\cline{2-4}
&
S/O &
\textit{anɡreéz} &
\textit{anɡreezaán} \\\cline{2-4}
&
A &
\textit{anɡreezá} &
\textit{anɡreezaanóom} \\\hline
\end{tabularx}
\end{table}


Although there is a~form syncretism between the oblique singular and the nominative plural in the large \textit{a}- and \textit{i}-declensions, that does not distort the nominative"=oblique contrast \textit{per se}. Here, a~suffix \textit{-a} or \textit{-i} is the morphological reflex of ergativity in the singular and a~suffix -\textit{óom} or -\textit{íim} in the plural. 


In the \textit{m}-declension, on the other hand, there is no differentiating ergative case marking available at all. For some \textit{aan}-declension nouns, the case differentiation is neutralised in the singular but maintained in the plural, while for a~few others there are four distinct forms available: nominative singular, nominative plural, oblique singular and oblique plural. The \textit{ee}-declension nouns (not displayed here) largely make the same distinctions as \textit{a}- and \textit{i}-declension nouns.


With respect to frequency, the ``full contrast pattern'' represents the large majority of all Palula nouns (the \textit{a}-declension and the \textit{i}-declension together making up 70 per cent of all nouns), masculine as well as feminine, while the ``no contrast pattern'' is relatively small (about 16 per cent), comprising exclusively feminine nouns.


\subsection{Pronominal case differentiation}
\label{subsec:11-2-2}


The noun phrase slot could of course also be filled with a~pronoun, and here too we have different forms bearing a~relation to case. Apart from the singling out of A in the Perfective (\textit{asím} in (\ref{ex:11-12})), we also have pronominal forms particular to O (\textit{asaám} in (\ref{ex:11-13})) vis-à-vis S (\textit{be} in (\ref{ex:11-11})) and A (in the Perfective and the non"=perfective alike).

\begin{exe}
\ex
\label{ex:11-11}
\glll rhootašíia ba [be] ɡíia  \\
morning \textsc{prt} \textsc{1pl.nom} go.\textsc{pfv.pl}  \\
 {} {}  \textbf{S} \\
\glt `In the morning we left.' (A:GHA006)

\ex
\label{ex:11-12}
\glll [asím] ǰinaazá khaṣeel-í wheelíl-u de \\
\textsc{1pl.erg} corpse drag-\textsc{cv} take.down.\textsc{pfv"=msg} \textsc{pst}  \\
 \textbf{A} \textbf{O} \\
\glt `We dragged the corpse down.' (A:GHA044)

\ex
\label{ex:11-13}
\glll nu ba [asaám] mhaaranií the ukháat-u de \\
\textsc{3sg.prox.nom} \textsc{prt} \textsc{1pl.acc} kill.\textsc{nverb} to come.up.\textsc{pfv"=msg} \textsc{pst}  \\
{} {}   \textbf{O} \\
\glt `He has come up here to kill us.' (A:HUA071)
\end{exe}

This again is only part of the whole picture. Starting with the personal pronouns proper, these do not uniformly have the same number of forms or make the same distinctions formally, as can be seen in \tabref{tab:11-1}. 


\begin{table}[ht]
\caption{Personal pronouns and case differentiation in the Perfective}
\begin{tabularx}{.5\textwidth}{ l@{\hspace{15pt}} Q l@{\hspace{15pt}} : Q }
\lsptoprule
& O &
\multicolumn{1}{l}{S} &
 A\\\hline
\textsc{1sg} &
\multicolumn{2}{c:}{ \textit{ma}} &
 \textit{míi} \\
\textsc{2sg} &
\multicolumn{2}{c:}{ \textit{tu}} &
 \textit{thíi} \\
\textsc{1pl} &
 \textit{asaám} &
 \textit{be} &
 \textit{asím} \\
\textsc{2pl} &
 \textit{tusaám} &
 \textit{tus} &
 \textit{tusím} \\\lspbottomrule
\end{tabularx}
\label{tab:11-1}
\end{table}


The first- and second"=plural personal pronouns make a~three"=way distinction, with unique ergative forms, \textit{asím} and \textit{tusím}, whereas the first- and second"=singular only have two forms each (with nominative"=accusative neutralisation as well as an~ergative"=genitive neutralisation). 


The demonstratives, which are used as third"=person pronouns, see \tabref{tab:11-2}, show differentiation to the same extent as the plural personal pronouns, i.e. with a~three"=way case contrast. 


\begin{table}[ht]
\caption{Demonstrative case differentiation in the Perfective (only the remote set represented)}

\begin{tabularx}{.5\textwidth}{ l@{\hspace{15pt}} Q l@{\hspace{15pt}} : Q }
\lsptoprule
&
 O &
\multicolumn{1}{ l}{ S} &
 A\\\hline
\textsc{sg} &
 \textit{tas} &
 \textit{so/se} &
 \textit{tíi} \\
\textsc{pl} &
 \textit{tanaám} &
 \textit{se} &
 \textit{taním} \\\lspbottomrule
\end{tabularx}
\label{tab:11-2}
\end{table}


\subsection{Dependent marking}
\label{subsec:11-2-3}


Case marking is also relevant for dependents in the noun phrase, although it has a~much more limited scope (for details, see \sectref{sec:10-3}). There are two kinds of dependent agreement: a) determiner agreement and b) adjectival agreement.


Determiners occur in a~maximum of two forms, one of them occurring only with a~singular masculine noun head in the nominative, the other occurring elsewhere, i.e. with non"=nominative singular heads, feminine and plural heads. 


Adjectives of the inflecting category display three different forms to reflect properties of the noun head: one for a~singular masculine head in the nominative, another for non"=nominative singular and plural masculine head, and a~third for feminine heads, regardless of case or number. 


The agreement displayed by dependents is therefore not adding anything to the differentiation already made explicit by the case"=inflected head as far as case is concerned. 


\section{The split system summarised}
\label{sec:11-3}


Summarising the findings in \sectref{sec:11-1}--\sectref{sec:11-1}, we have two dimensions on which ergative vs. accusative alignment and their expressions depend in Palula. First and foremost, the presence of ergative alignment is aspectually determined. While accusative properties can be present regardless of the TMA category realised, it is in the Perfective only that an (additional) ergative pattern is found. A consistent correlation (see \figref{fig:11-1}) exists between perfective aspect and ergative verb agreement, and an~accusative verb agreement and non"=perfective categories. 

\begin{figure}[ht]
\begin{tabularx}{\textwidth}{ l@{\hspace{25pt}} C C C @{\hspace{25pt}}l }
\hline
Aspect &
 A &
 S &
 O &
Alignment\\\hline
Non"=perfective &
\ligrcell{~}
& \ligrcell{~}
&
&
Accusative\\
Perfective &
& \ligrcell{~}
& \ligrcell{~}
&
Ergative\\\hline
\end{tabularx}
\edcomm[inline]{Wäre das nicht besser als Tabelle?}
\caption{Correlations between aspect and alignment in verb agreement (shading represents verb agreement)}
\label{fig:11-1}
\end{figure}


Much less straightforward is the relationship between the nature of the NP and case marking. Even within the same aspectual category (the perfective) we have examples of non"=differentiation (ASO all the same as far as case marking is concerned), a~two"=way differentiation (A marked differently from S and O) as well as a~tripartite differentiation (A, S and O all distinguished by case marking). \tabref{tab:11-3} illustrates how case differentiation is displayed for four different categories of NPs in Palula: 

\begin{enumerate}
\item Pron1 are the first- and second"=person plural as well as all the third"=person pronouns; they display a~tripartite subsystem. 
\item Noun1 are all the nouns that make a~nominative/oblique distinction, and Pron2 are the first- and second"=singular pronouns; they display an~ergative subsystem. 
\item Noun2 are the nouns that do not make a~nominative/oblique distinction; they display a~neutral subsystem.
\end{enumerate}

\begin{table}[ht]
\caption{Morphologically realised case distinctions related to grammatical relations (The case marking below the dotted line applies in the Perfective only.)}

\begin{tabularx}{\textwidth}{ l@{\hspace{15pt}} l@{\hspace{15pt}} | Q | Q | Q }
\lsptoprule
&
\multicolumn{1}{ l}{Pron1} &
\multicolumn{1}{ l}{Noun1} &
\multicolumn{1}{ l}{Pron2} &
Noun2\\\hline
O &
Accusative &
\multirow{2}{*}{Nominative} &
Nominative &
Nominative\\\cline{2-2}
S &
Nominative &
&
&
\\\cdashline{2-5}
A &
Ergative/oblique &
Oblique &
Genitive &
\\\lspbottomrule
\end{tabularx}
\label{tab:11-3}
\end{table}


\section{Alignment and split features in the region and beyond}
\label{sec:11-4}


How do the features summarised above relate to those found in the surrounding region and in related languages?



As far as the presence of (morphological) ergativity is concerned, the situation in Palula is far from unique, neither among NIA languages \citep{deosharma2006} in the region nor beyond, but its manifestations and more precise characteristics are manifold and quite diverse in what \citet[250]{masica2001} describes as an ``ergative belt'' stretching from the north"=eastern part of the Subcontinent all the way to Caucasus, with modern Persian as one major exception in the middle of it. This belt includes Indo"=Aryan, Iranian and Tibeto"=Burman, as well as the isolate Burushaski and some of the language families represented in the Caucasus. 



While ergativity is conditional in Palula, ergative case marking is applied across the board in Burushaski and in the Shina varieties spoken adjacent to it (as far as case marking is concerned\footnote{As pointed out to me by Carla Radloff (pc), there is in Gilgit Shina a~clearcut split between accusatively aligned verb agreement and ergatively aligned case marking (invariably with \textit{-se} or \textit{-s}).}). Although the latter is due to substratum effects from Burushaski according to \citet[248]{masica2001}, another phenomenon, termed ``dual ergativity'' by \citet[213]{hookkoul2004}, is observed in certain Eastern (including Kohistani) Shina varieties, where there is a~TMA"=related (imperfective vs. perfective according to \citealt[51--53]{schmidtkohistani2008}alternation between ergativity markers of IA origin and an~ergativity marker supposedly of Tibetan origin (\citealt[214]{hookkoul2004}; \citealt[211]{bailey1924}). 



However, far more common in NIA languages as well as in Tibeto"=Burman, is some sort of TMA split between ergative patterns and accusative patterns \citep[248]{masica2001}, usually between perfective and non"=perfective tenses \citep[342--343]{masica1991}. This may be manifested, as in Palula, in the agreement of transitive verbs with O in the perfective along with a~distinctive case marking of A. That is the case in Urdu"=Hindi \citep[124]{schmidt1999} as well as in many of the other major NIA languages of the Subcontinent \citep[248]{masica2001}. Geographically closer to Palula, this is also observed for the Kohistani languages (\citealt[136]{baart1999a}; \citealt[34]{hallberghallberg1999}; \citealt[93--95]{lunsford2001}\footnote{In the case of Torwali, ergativity is also manifested in the future tense.}). While ergativity is seen in the case marking of A in other Shina varieties, verb agreement with O is not a~feature of Gilgiti or Kohistani Shina. Instead, as in neighbouring Dameli and Gawarbati (personal observations), the verb agrees (accusatively) with S or A and never with O, whether or not there are other manifestations of ergativity or accusativity.



A number of Iranian languages in the region also display split ergativity \citep{payne1980}, although with certain peculiarities (to which we will return shortly). For one, Pashto exemplifies a~tense split rather than an~aspectual split, with verb agreement with O and ergative case marking of A in past tenses, regardless of aspect (\citealt[4--5]{tegey1977}; \citealt[71--72]{lorenz1979}). In addition, a~class of intransitive verbs (expressing involuntary activity) also require an~ergatively marked A \citep[112]{babrakzai1999}, a~phenomenon also described by \citet[217]{hookkoul2004} for Indo"=Aryan Kashmiri.\footnote{This is also found in Urdu to a~limited extent \citep[168]{schmidt1999}.} 



While most languages within this so"=called ergative belt show some ergative features, there are nevertheless some where they are entirely absent. In the immediate vicinity of Palula, the most notable examples are Kalasha and Khowar \citep[41]{bashir1988}. While this absence is a~retention feature of Kalasha and Khowar, in some languages in other parts of the Subcontinent, such as Bengali, Oriya and Sinhalese, a~former ergative construction has probably been replaced only later by a~consistent accusative alignment \citep[343--344]{masica1991}.



A number of different patterns are observed in the region as far as case marking, case syncretism and various types of NP splits being realised. Those languages manifesting verb agreement with O in some TMA categories, also tend to case mark A distinctively vis-à-vis S and O, but there are also those languages that maintain a~tripartite S vs. O vs. A differentiation, if not for nouns, at least for the pronouns or a~subset of them. In Punjabi, there is a~shared \textsc{1sg} nominative/oblique form, whereas \textsc{2sg}, \textsc{1pl}, as well as \textsc{2pl} and the third"=person pronouns differentiate between these two cases \citep[229]{bhatia1993}. In Kalam Kohistani, the \textsc{1sg} and \textsc{2sg} make a~subject vs. object/oblique/agent distinction, the \textsc{1pl} and \textsc{2pl} a~subject/agent/object vs. oblique distinction, a~subject vs. agent vs. object/oblique distinction in the \textsc{3sg}, and a~subject/object/oblique vs. agent distinction in the \textsc{3pl} \citep[39]{baart1999a}. 



In neighbouring Dameli (\citealt{morgenstierne1942} and own observations), there is a~nominative vs. accusative/ergative differentiation\footnote{This rather unexpected grouping of A with O vs. S is sometimes referred to as ``double"=oblique'' \citep{payne1980}.} in first- and second"=singular as well as in plural, but where \textsc{1sg} and \textsc{1pl} nominative somewhat surprisingly have merged. For the demonstratives functioning as third"=person pronouns, the situation is further complicated by animacy distinctions. In Gawarbati (\citealt{morgenstierne1950} and personal observations), there seems to be an~almost complete nominative vs. accusative vs. ergative differentiation upheld in all persons and in singular and plural (with nouns and pronouns alike), but only in so far as the NPs are definite and occur in the perfective. However, due to the lack of a~more comprehensive study of the language sufficient to base any conclusions on, the analysis remains tentative. 


\section{The split system and possible explanations}
\label{sec:11-5}


The natural question to ask now is how the patterns in Palula can be related to split"=ergative typology. Of the four possible factors that \citet[70]{dixon1994} presents as conditioning such a~split, the two relevant ones here are: a) a~split conditioned by the semantic nature of the noun phrase, and b) a~split conditioned by the tense, aspect or mood of the clause. 



As far as the first factor is concerned, we have already seen that different Palula nouns behave differently when it comes to ergative case marking, and we have also seen that nouns and pronouns display different morphological systems. That in itself is hardly surprising or contrary to any typological tendencies, as languages that display some degree of ergativity usually do so only with a~subset of their noun phrases \citep[261]{garrett1990}. Depending on where the splits occur along a~Nominal Hierarchy \citep[84--85]{dixon1994}/NP Hierarchy \citep[262]{garrett1990}/Empathy Hierarchy \citep[626--627]{delancey1981}, we may make finer classifications of split types, regardless whether that classification can be seen as reflecting some fundamental functional properties of language \citep{delancey1981}, or mainly as a~product of diachronic developments \citep{garrett1990}: Pron 1\textsuperscript{st} {\textgreater} Pron 2\textsuperscript{nd} {\textgreater} Pron 3\textsuperscript{rd} or Dem {\textgreater} N Proper {\textgreater} N Human {\textgreater} N Animate {\textgreater} N Inanimate (after \citealt[85]{dixon1994}).



The way we are supposed to read this hierarchy is that the more to the left, the more likely that the noun phrase is an~agent (A), the more to the right, the more likely it is to be a~patient (P). That also carries over to ergative case marking: the less likely a~noun phrase is to be A, the more of a~need it has to be marked as such, which means that ergative case marking~-- when it occurs in a~language~-- spreads from the right, i.e. from inanimate nouns and ``upward'' (\citealt[645]{delancey1981}; \citealt[84]{dixon1994}; \citealt[262]{garrett1990}). The opposite is believed to hold true for accusative case marking, where spreading is from the left and ``downward''. If, for instance, pronouns and nouns happen to follow different systems of case marking, we would expect the former to distinguish between AS and O (accusatively), whereas the latter would distinguish between A and SO (ergatively). 



The cutoff point could theoretically be anywhere along this scale, but some points have been pointed out as more crucial (and possibly more commonly occurring in languages) than others. One is that between second- and third"=person pronouns, forming the boundary between SAPs (Speech Act Participants) and everyone/everything else \citep[644]{delancey1981}. Another one is between first- and second"=person, although the exact hierarchical relationship between 1 and 2 person is not generally agreed upon (\citealt[644]{delancey1981}; \citealt[88--90]{dixon1994}). Pointing to the formal divergence between pronominal and nominal morphology and the lower likelihood of analogical extensions to cross over from one to the other, the boundary between third"=person pronouns and nouns may equally well be regarded as the most crucial one \citep[286]{garrett1990}. 



As real language data confirms, there is not necessarily only a~simple split between accusatively marked noun phrases on the one hand and ergatively marked on the other. Instead there are various logical possibilities of the two systems overlapping or being inclusive of one or the other, or one of them being present while the other is absent \citep[109]{dixon1994}.



Interestingly, there are according to \citeauthor{dixon1994}, no known examples of ergative marking within an~otherwise accusative system, nor any language known where a~subset is marked accusatively, another one ergatively, while a~middle area is left where neither applies. This is where Palula may turn out to be typologically interesting. Not that it straightforwardly exemplifies any of these two ``missing'' types, but what we seem to have is a~kind of ``crossbreed'', where most NPs display ergative marking (A-SP) and some of them with an~added accusative marking (A-S-P), whereas another category of NPs lack such marking altogether, as to produce the following pattern:


\begin{table}[H]
\begin{tabularx}{\textwidth}{ C C C }
\hline
\dagrcell{~} &
\ligrcell{~} &\\\hline
 A-S-P &
 A-SP &
 ASP\\
\end{tabularx}
\authcomm[inline]{Is shading ok instead of hatched?}
\end{table}


Another problem is of course that these markings in Palula do not line up neatly with the Nominal Hierarchy mentioned above, or at least not with a~couple of the more crucial typological expectations connected with it, one having to do with accusative marking and the other with ergative marking. 



First, we learned that accusative marking spreads from left, and therefore at the very least
first"=person pronouns would be marked accusatively: ``Since first person is on the extreme left of
the hierarchy, it is the strongest candidate for accusative marking.'' \citep[89]{dixon1994} That is true
of \textsc{1pl} in Palula (S: \textit{be} `we', O: \textit{asaám} `us') but not of \textsc{1sg} (SO:
\textit{ma} `I, me'). The same split along the number category with 2 person (S: \textit{tus} `you 2\textsc{pl.nom}', O: \textit{tusaám} `you 2\textsc{pl.acc}'; but SO: \textit{tu} `you 2\textsc{sg.nom}'), even here with only the plural having accusative marking, even though the singular is as high on the animacy scale as the plural. Typologically speaking, more grammatical distinctions would be expected to crop up in the singular than in the plural \citep[92]{dixon1994}, but that is obviously the opposite in Palula as far as case distinctions are concerned. 



Second, we saw that ergativity normally spreads from right to left, and therefore at the very least inanimate nouns would be marked ergatively if the language displays any ergative marking. However, in Palula, the nouns that do not display any ergative marking at all (neither in singular or plural) belong to one particular inflectional class that exclusively contains feminine gender nouns, animate as well as inanimate, whereas the classes that do mark ergativity are equally inclusive of (mainly masculine) animate and inanimate nouns. More detailed research would be needed to be able to say anything more insightful about any possible correlation between the animacy of the nouns and the presence/absence of ergative marking. 



Even though we are able to give a~functional explanation to the overall NP split pattern in Palula~-- such as accusative marking being confined to pronouns, while ergative marking is found with pronouns as well as nouns~-- functionality fails to explain the exceptions mentioned above. As \citet{garrett1990} has shown in regard to the instrumental origin of ergative case"=marking in Anatolian and Gorokan, the answers may as well be found in specific diachronic developments~-- something that \citet[93]{dixon1994} points out as well. Even though we can be very specific as to the leftward direction of ergative spreading, \citeauthor{garrett1990} holds that the initial distribution of the marking in any one language is due to language"=internal factors (\citeyear[286]{garrett1990}). 



We would therefore have to look for possible sources of the present"=day Palula case inflections, especially ergative case marking, and how it might have spread throughout the Nominal Hierarchy by analogy or otherwise. Even though that in itself is well beyond the scope of this work, it should of course be pointed out that the history of ergativity in Palula must take the Indo"=Iranian \citep[263--264]{garrett1990} or Indo"=Aryan \citep[341--346]{masica1991} development at large into account, where the instrumental case seems to be an~important source for what came to be a~perfective agent marker \citep[13]{skalmowski1985}. In connection with that, it is interesting to note that the instrumental role in Palula sometimes is expressed with a~noun form \textit{-am/-íim/-óom}, i.e. a~form identical with the oblique plural (also with a~singular referent as in (\ref{ex:11-14})).

\begin{exe}
\ex
\label{ex:11-14}
\gll míiš-a ba tesée háat-a wée [thupíik"=am] ǰít-i \\
man-\textsc{ob} \textsc{prt} \textsc{3sg.gn} hand-\textsc{ob} into gun-\textsc{ins} shoot.\textsc{pfv-f} \\
\glt `Then the man shot through its hand with his gun.' (B:BEL317)
\end{exe}

Whether the source of the Palula ergative marking is to be found in this instrumental or the ergative"=oblique and the instrumental at some point formally started to merge, it is not wholly unlikely that it spread analogically throughout the nominal system. However, in the case of the \textit{-im/-íim} plural forms of our ``problematic'' inflectional class, a~neutralisation between the present plural form and an~earlier distinctly expressed instrumental is likely to have developed, which may also explain the apparent lack of a~distinctive ergative marking with this class of (feminine) nouns.



What then about the first- and second"=person singular pronouns showing less differentiation than the rest of the pronouns? Here again, we may have to resort to a~language"=specific diachronic development. \citet[208]{schmidt2000} and \citet[133]{schmidtkohistani2001} seem to hint that the now identical first- and second"=person singular nominative and oblique in Kohistani Shina once had their unique forms but have later merged. If that is the case, that may very well be a~development that can be traced back to the time before Palula split off from the Shina varieties spoken in the Indus Valley. 



Now, the second factor conditioning a~split, namely TMA, can be related in a~much more straightforward way to case marking typology. According to \citet[99, 101]{dixon1994}, ergative marking is always used for describing definite results or past/perfective events, if there is any kind of split conditioned by TMA. As with the semantic factor, \citet[110]{dixon1994} presents the possible types of split that are conditioned by TMA, pointing out that unlike the semantically conditioned splits we only find a~maximum of two kinds of marking in any one language. Palula is accordingly a~representative of an `all accusative, part ergative' language. In the perfective as well as in the imperfective, accusative case marking is present, while in the perfective, ergative case marking is added to the accusative marking.



The development and naturalness of this kind of split is once again the topic of an~ongoing discussion, with some scholars trying to motivate the split functionally, as, for example, \citet[639--647]{delancey1981}, who holds that this kind of split is a~result of the speaker's viewpoint being associated with the patient and the event's natural starting point with the agent, whereas in the imperfective both are associated with the agent. \citet[262--263]{garrett1990}, on the other hand, sees this split as the ordinary result of a~common diachronic reinterpretation process, in which a~passive verbal has been reinterpreted as an~active, transitive verb, and as a~kind of ``by"=product'' an~old instrumental case marking has become an~agent marker, sometimes analogically further spread throughout the morphology of the language. According to \citet[264]{garrett1990}, the correlation between ergative marking and perfectivity is therefore in itself not significant.



Another factor, although not really seen by \citet[94]{dixon1994} as a~kind of conditioning fully comparable with the two already mentioned, rather a~secondary phenomenon, has to do with whether a `free' or a `bound' form is being used. Two major kinds of morphology are involved in case marking: nominal (or case proper) and verbal (for cross"=referencing). The nominal morphology is the one here referred to as `free' and the verbal as `bound'. If both are being used in a~language, they may treat and group the clausal agent, subject and patient, respectively along the same lines, but the two may also be in conflict, i.e. go along with different principles. 



In Palula, both mechanisms are present, but they do not stand in conflict. We could rather see them as features supporting or reinforcing one another, sometimes disambiguing what would otherwise remain as an~ambiguity between the agent and the patient. The `bound' forms, i.e. the verbal agreement suffixes, stand in a~straightforward relationship to the aspectual split, so that in the imperfective agreement is always accusative, whereas in the perfective it is always ergative. The `free' forms, as we have already seen, may alternatively reflect an~absence of distinctions (ASO), an~ergative distinction (A-SO) or an~ergative"=cum"=accusative distinction (A-S-O), all depending on the other two factors, the semantics of the NP and the aspect of the utterance. 



\citeauthor{delancey1981}, who prefers to define ergativity solely on basis of transitive agent marking (\citeyear[628]{delancey1981}), points out that agreement corresponds to~-- while being independent of~-- NP case marking rather than to case"=role (\citeyear[631]{delancey1981}), and that the verb agrees with the unmarked NP if one is present. That could possibly be said of Palula as well, with the modification that the verb agrees with the NP with the least marked case within each aspectual category, which in our case would be the unmarked or the accusatively marked patient NP in the perfective, and the consistently unmarked agent NP in the imperfective.